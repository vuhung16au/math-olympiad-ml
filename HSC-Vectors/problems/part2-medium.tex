\begin{problem}[Line Intersection in 3D]
Find the intersection point of lines $\mathbf{r} = \begin{pmatrix} 3 \\ -1 \\ 7 \end{pmatrix} + \lambda \begin{pmatrix} 1 \\ 2 \\ 1 \end{pmatrix}$ and $\mathbf{r} = \begin{pmatrix} 1 \\ -6 \\ 2 \end{pmatrix} + \mu \begin{pmatrix} -2 \\ 1 \\ 3 \end{pmatrix}$.
\end{problem}

\begin{hint}
Set up three equations by equating components, then solve for parameters.
\end{hint}

\begin{solution}[Sketch]
Equating components: $3 + \lambda = 1 - 2\mu$, $-1 + 2\lambda = -6 + \mu$, $7 + \lambda = 2 + 3\mu$. From equations 1 and 3: $\lambda = -2\mu$ and $\lambda = -5 + 3\mu$. Solving: $\mu = 1$, $\lambda = -2$. Verify in equation 2. Intersection point: $(1, -5, 5)$.
\end{solution}

\vspace{1em}

\begin{problem}[Perpendicular Intersecting Lines]
Two lines intersect and are perpendicular. Find values of $p$ and $q$ for the line $\mathbf{r} = \begin{pmatrix} 1 \\ 0 \\ p \end{pmatrix} + \lambda \begin{pmatrix} 2 \\ q \\ -1 \end{pmatrix}$.
\end{problem}

\begin{hint}
Use perpendicularity condition (dot product = 0) and intersection condition (solve system).
\end{hint}

\begin{solution}[Sketch]
Direction vectors must be perpendicular: $\mathbf{d}_1 \cdot \mathbf{d}_2 = 0$. Additionally, equate position vectors at intersection and solve the resulting system for $p$ and $q$. The specific values depend on the second line equation (not fully shown in sample).
\end{solution}

\vspace{1em}

\begin{problem}[Tetrahedron Collinearity]
Given tetrahedron with vertices and specific vector relationships, prove collinearity of certain points using linear combinations of non-parallel vectors.
\end{problem}

\begin{hint}
Express vectors in terms of base vectors and use uniqueness of coefficients for non-parallel vectors.
\end{hint}

\begin{solution}[Sketch]
Show that if $\lambda \mathbf{a} + \mu \mathbf{b} = \vec{0}$ for non-parallel $\mathbf{a}, \mathbf{b}$, then $\lambda = \mu = 0$. Use this to prove position relationships, showing that $\overrightarrow{BL} = \frac{4}{7}\overrightarrow{BC}$ by equating coefficients.
\end{solution}

\vspace{1em}

\begin{problem}[Parallelogram Proof]
Show that quadrilateral $CDFE$ is a parallelogram given that $ABCD$ and $ABEF$ are parallelograms.
\end{problem}

\begin{hint}
Use the fact that opposite sides of parallelograms are equal vectors.
\end{hint}

\begin{solution}[Sketch]
From parallelogram $ABCD$: $\overrightarrow{AB} = \overrightarrow{DC}$. From parallelogram $ABEF$: $\overrightarrow{AB} = \overrightarrow{FE}$. Therefore $\overrightarrow{DC} = \overrightarrow{FE}$, which means $\overrightarrow{CD} = \overrightarrow{EF}$, proving $CDFE$ is a parallelogram.
\end{solution}

\vspace{1em}

\begin{problem}[Force Vector Analysis]
Force $\mathbf{F}_1$ has magnitude 12 N in direction $2\mathbf{i} - 2\mathbf{j} + \mathbf{k}$. Show $\mathbf{F}_1 = 8\mathbf{i} - 8\mathbf{j} + 4\mathbf{k}$, find resultant with $\mathbf{F}_2 = -6\mathbf{i} + 12\mathbf{j} + 4\mathbf{k}$, and compute work done.
\end{problem}

\begin{hint}
Find unit vector in given direction, multiply by magnitude, then add forces vectorially.
\end{hint}

\begin{solution}[Sketch]
Direction vector has magnitude $\sqrt{4+4+1} = 3$. Unit vector: $\frac{1}{3}(2\mathbf{i} - 2\mathbf{j} + \mathbf{k})$. Force: $12 \cdot \frac{1}{3}(2\mathbf{i} - 2\mathbf{j} + \mathbf{k}) = 8\mathbf{i} - 8\mathbf{j} + 4\mathbf{k}$. Resultant: $\mathbf{F}_3 = 2\mathbf{i} + 4\mathbf{j} + 8\mathbf{k}$. Work: $\mathbf{F}_3 \cdot \mathbf{d} = 2 + 4 + 16 = 22$.
\end{solution}

\vspace{1em}

\begin{problem}[Double Angle with Vectors]
For vectors $\mathbf{a} = \mathbf{i} + 2\mathbf{j} + 2\mathbf{k}$ and $\mathbf{b} = 2\mathbf{i} - 4\mathbf{j} + 4\mathbf{k}$ with acute angle $\theta$, find $\sin 2\theta$.
\end{problem}

\begin{hint}
Find $\cos\theta$ from dot product, then $\sin\theta$ from Pythagorean identity, and use $\sin 2\theta = 2\sin\theta\cos\theta$.
\end{hint}

\begin{solution}[Sketch]
$|\mathbf{a}| = 3$, $|\mathbf{b}| = 6$, $\mathbf{a} \cdot \mathbf{b} = 2$. Thus $\cos\theta = \frac{2}{18} = \frac{1}{9}$. Then $\sin\theta = \sqrt{1 - \frac{1}{81}} = \frac{4\sqrt{5}}{9}$. Therefore $\sin 2\theta = 2 \cdot \frac{4\sqrt{5}}{9} \cdot \frac{1}{9} = \frac{8\sqrt{5}}{81}$.
\end{solution}

\vspace{1em}

\begin{problem}[Vector Projection]
Find the integer value of $m$ such that the projection of $\mathbf{a} = 2\mathbf{i} - 3\mathbf{j} + \mathbf{k}$ onto $\mathbf{b} = \mathbf{i} + m\mathbf{j} - \mathbf{k}$ equals $-\frac{11}{18}\mathbf{b}$.
\end{problem}

\begin{hint}
Use projection formula: $\text{proj}_\mathbf{b}\mathbf{a} = \frac{\mathbf{a} \cdot \mathbf{b}}{|\mathbf{b}|^2}\mathbf{b}$.
\end{hint}

\begin{solution}[Sketch]
$\mathbf{a} \cdot \mathbf{b} = 2 - 3m - 1 = 1 - 3m$, $|\mathbf{b}|^2 = 1 + m^2 + 1 = m^2 + 2$. Set $\frac{1-3m}{m^2+2} = -\frac{11}{18}$. Cross-multiply: $18(1-3m) = -11(m^2+2)$, giving $11m^2 - 54m + 40 = 0$. Factor: $(11m-10)(m-4) = 0$. Since $m$ is an integer, $m = 4$.
\end{solution}

\vspace{1em}

\begin{problem}[Perpendicular Vectors Condition]
For $\mathbf{u} = -2\mathbf{i} - \mathbf{j} + 3\mathbf{k}$ and $\mathbf{v} = p\mathbf{i} + \mathbf{j} + 2\mathbf{k}$, find values of $p$ such that $\mathbf{u} - \mathbf{v}$ and $\mathbf{u} + \mathbf{v}$ are perpendicular.
\end{problem}

\begin{hint}
Use $(\mathbf{u} - \mathbf{v}) \cdot (\mathbf{u} + \mathbf{v}) = |\mathbf{u}|^2 - |\mathbf{v}|^2 = 0$.
\end{hint}

\begin{solution}[Sketch]
$|\mathbf{u}|^2 = 4 + 1 + 9 = 14$. $|\mathbf{v}|^2 = p^2 + 1 + 4 = p^2 + 5$. Setting equal: $14 = p^2 + 5 \Rightarrow p^2 = 9 \Rightarrow p = \pm 3$.
\end{solution}

\vspace{1em}

\begin{problem}[Parallel and Perpendicular Lines]
For lines with direction vectors $\begin{pmatrix} 3 \\ -3 \\ 3 \end{pmatrix}$ and $\begin{pmatrix} a-2 \\ -7 \\ 7 \end{pmatrix}$, find $a$ if they are: (a) parallel, (b) perpendicular.
\end{problem}

\begin{hint}
Parallel: direction vectors are scalar multiples. Perpendicular: dot product equals zero.
\end{hint}

\begin{solution}[Sketch]
(a) For parallel: $\frac{3}{a-2} = \frac{-3}{-7} = \frac{3}{7}$. Solving: $3 \cdot 7 = 3(a-2) \Rightarrow a = 9$. (b) For perpendicular: $3(a-2) + (-3)(-7) + 3(7) = 0 \Rightarrow 3a + 36 = 0 \Rightarrow a = -12$.
\end{solution}

\vspace{1em}

\begin{problem}[Angle BCD Using Dot Product]
Given position vectors $\mathbf{b} = \mathbf{i} - \mathbf{j} + 2\mathbf{k}$, $\mathbf{c} = 2\mathbf{i} - \mathbf{j} + \mathbf{k}$, $\mathbf{d} = a\mathbf{i} - 2\mathbf{j}$, and angle $BCD = \frac{\pi}{3}$, find $a$.
\end{problem}

\begin{hint}
Find vectors $\overrightarrow{CB}$ and $\overrightarrow{CD}$, use dot product formula with $\cos(\pi/3) = 1/2$.
\end{hint}

\begin{solution}[Sketch]
$\overrightarrow{CB} = -\mathbf{i} + \mathbf{k}$, $\overrightarrow{CD} = (a-2)\mathbf{i} - \mathbf{j} - \mathbf{k}$. Dot product: $-(a-2) - 1 = 1 - a$. Magnitudes: $|\overrightarrow{CB}| = \sqrt{2}$, $|\overrightarrow{CD}| = \sqrt{(a-2)^2 + 2}$. Equation: $1 - a = \frac{\sqrt{2}\sqrt{(a-2)^2+2}}{2}$. Squaring and solving: $a^2 = 4$, but checking sign constraint gives $a = -2$.
\end{solution}

\vspace{1em}

\begin{problem}[Direction Cosines Sum]
Prove that for a position vector making angles with coordinate axes, the sum of squares of direction cosines equals 1.
\end{problem}

\begin{hint}
Express direction cosines as ratios of components to magnitude.
\end{hint}

\begin{solution}[Sketch]
$\cos\alpha = \frac{a}{|\mathbf{r}|}$, $\cos\beta = \frac{b}{|\mathbf{r}|}$, $\cos\gamma = \frac{c}{|\mathbf{r}|}$. Sum: $\frac{a^2 + b^2 + c^2}{|\mathbf{r}|^2} = \frac{|\mathbf{r}|^2}{|\mathbf{r}|^2} = 1$.
\end{solution}

\vspace{1em}

\begin{problem}[Section Formula Proof]
Prove that if $C$ divides $AB$ in ratio $m:n$, then $\overrightarrow{OC} = \frac{m\mathbf{a} + n\mathbf{b}}{m+n}$ where $\mathbf{a} = \overrightarrow{OA}$, $\mathbf{b} = \overrightarrow{OB}$.
\end{problem}

\begin{hint}
Use $\overrightarrow{AC} = \frac{n}{m+n}\overrightarrow{AB}$ and vector addition.
\end{hint}

\begin{solution}[Sketch]
Since $\frac{CB}{AC} = \frac{m}{n}$, we have $\overrightarrow{AC} = \frac{n}{m+n}(\mathbf{b} - \mathbf{a})$. Then $\overrightarrow{OC} = \mathbf{a} + \overrightarrow{AC} = \mathbf{a} + \frac{n}{m+n}(\mathbf{b} - \mathbf{a}) = \frac{m\mathbf{a} + n\mathbf{b}}{m+n}$.
\end{solution}

\vspace{1em}

\begin{problem}[Skew or Intersecting Lines]
Determine whether lines $\mathbf{r} = \begin{pmatrix} 2 \\ -3 \\ 1 \end{pmatrix} + \lambda \begin{pmatrix} 1 \\ 4 \\ 1 \end{pmatrix}$ and $\mathbf{r} = \begin{pmatrix} -4 \\ 6 \\ -2 \end{pmatrix} + \mu \begin{pmatrix} 4 \\ -11 \\ 3 \end{pmatrix}$ are skew or intersecting.
\end{problem}

\begin{hint}
Check if direction vectors are parallel. If not, solve system to see if it's consistent.
\end{hint}

\begin{solution}[Sketch]
Direction vectors not parallel (check ratios). Solve system: from equations 1 and 3, find $\lambda = 6$, $\mu = 3$. Check equation 2: $4(6) + 11(3) = 57 \neq 9$. System inconsistent, so lines are skew.
\end{solution}

\vspace{1em}

\begin{problem}[Line Through Points, Intersection Check]
Line passes through $A(1,3,-2)$ and $B(2,-1,5)$. Does $C(3,4,9)$ lie on the line? Does it intersect another given line?
\end{problem}

\begin{hint}
Find direction vector, write parametric equations, check if $C$ satisfies them.
\end{hint}

\begin{solution}[Sketch]
Direction: $\overrightarrow{AB} = \mathbf{i} - 4\mathbf{j} + 7\mathbf{k}$. Line: $\mathbf{r} = \begin{pmatrix} 1 \\ 3 \\ -2 \end{pmatrix} + \lambda \begin{pmatrix} 1 \\ -4 \\ 7 \end{pmatrix}$. For $C$: $x$-component gives $\lambda = 2$, but $y$-component gives $-5 \neq 4$, so $C$ not on line. For intersection with second line, set up system and check consistency.
\end{solution}

\vspace{1em}

\begin{problem}[Line-Sphere Intersection Angle]
Find intersection points of line and sphere with center $O$ and radius $\sqrt{10}$, then find angle $\angle AOB$ between the two intersection points.
\end{problem}

\begin{hint}
Substitute parametric equations into sphere equation, solve quadratic for parameter values.
\end{hint}

\begin{solution}[Sketch]
Substitute line equations into $x^2 + y^2 + z^2 = 10$. Get quadratic in $\lambda$, solve to find $\lambda = 1$ and $\lambda = -2$. Points: $A(1,3,0)$ and $B(1,0,3)$. Find $\cos\theta = \frac{\overrightarrow{OA} \cdot \overrightarrow{OB}}{|\overrightarrow{OA}||\overrightarrow{OB}|} = \frac{1}{10}$, so $\theta \approx 84^\circ$.
\end{solution}

\vspace{1em}

\begin{problem}[Linear Combination of Vectors]
Express $\mathbf{u} = 5\mathbf{i} + 5\mathbf{j} + 5\mathbf{k}$ as $\lambda\mathbf{a} + \mu\mathbf{b} + \nu\mathbf{c}$ where $\mathbf{a}, \mathbf{b}, \mathbf{c}$ are given non-parallel vectors.
\end{problem}

\begin{hint}
Set up system of three equations by equating components, solve for $\lambda, \mu, \nu$.
\end{hint}

\begin{solution}[Sketch]
From component equations: $2\lambda + \mu - \nu = 5$, $3\lambda - \mu + 2\nu = 5$, $\lambda + 2\mu - \nu = 5$. Solve systematically: subtract equations to eliminate variables. Find $\lambda = \mu = \frac{15}{8}$, $\nu = \frac{5}{8}$.
\end{solution}

\vspace{1em}

\begin{problem}[Parallelogram Fourth Vertex]
Three vertices of parallelogram are $O(0,0,0)$, $A(2,2,1)$, $B(1,2,2)$. Find possible positions of fourth vertex.
\end{problem}

\begin{hint}
Consider three cases: which point is opposite to which, giving three parallelograms.
\end{hint}

\begin{solution}[Sketch]
Case 1: $C$ opposite to $O$: $\overrightarrow{OC} = \overrightarrow{OA} + \overrightarrow{OB} = (3,4,3)$. Case 2: $C$ opposite to $B$: $\overrightarrow{OC} = \overrightarrow{OA} - \overrightarrow{OB} = (1,0,-1)$. Case 3: $C$ opposite to $A$: $\overrightarrow{OC} = \overrightarrow{OB} - \overrightarrow{OA} = (-1,0,1)$.
\end{solution}

\vspace{1em}
