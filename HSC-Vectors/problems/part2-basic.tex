\begin{problem}[Vector to Cartesian Line Equation]
What is the Cartesian equation of the line $\mathbf{r} = \begin{pmatrix} 1 \\ 3 \end{pmatrix} + \lambda \begin{pmatrix} 2 \\ 4 \end{pmatrix}$?
\end{problem}

\begin{hint}
Eliminate the parameter $\lambda$ by expressing it from one equation and substituting into the other.
\end{hint}

\begin{solution}[Sketch]
Write parametric equations: $x = 1 + 2\lambda$, $y = 3 + 4\lambda$. From the first equation, $\lambda = \frac{x-1}{2}$. Substitute into second: $y = 3 + 4 \cdot \frac{x-1}{2} = 3 + 2(x-1) = 2x + 1$. Therefore $y = 2x + 1$ or $2x - y + 1 = 0$.
\end{solution}

\vspace{1em}

\begin{problem}[Sketch 3D Helix]
Sketch the curve described by $\mathbf{r}(t) = -5\cos t\,\mathbf{i} + 5\sin t\,\mathbf{j} + t\,\mathbf{k}$ for $0 \le t \le 4\pi$.
\end{problem}

\begin{hint}
The $x$ and $y$ components trace a circle while $z$ increases linearly with $t$.
\end{hint}

\begin{solution}[Sketch]
In the $xy$-plane, $x^2 + y^2 = 25\cos^2 t + 25\sin^2 t = 25$, which is a circle of radius 5. As $t$ increases from 0 to $4\pi$, the point moves around the circle twice while rising from $z=0$ to $z=4\pi$. This creates a helix spiraling upward around the $z$-axis.
\end{solution}

\vspace{1em}

\begin{problem}[Angle Between 3D Vectors]
Find the angle between vectors $\mathbf{a} = \begin{pmatrix} 2 \\ 0 \\ 4 \end{pmatrix}$ and $\mathbf{b} = \begin{pmatrix} -3 \\ 1 \\ 2 \end{pmatrix}$ to the nearest degree.
\end{problem}

\begin{hint}
Use $\cos \theta = \frac{\mathbf{a} \cdot \mathbf{b}}{|\mathbf{a}||\mathbf{b}|}$.
\end{hint}

\begin{solution}[Sketch]
$\mathbf{a} \cdot \mathbf{b} = (2)(-3) + (0)(1) + (4)(2) = -6 + 8 = 2$. $|\mathbf{a}| = \sqrt{4 + 16} = \sqrt{20} = 2\sqrt{5}$, $|\mathbf{b}| = \sqrt{9 + 1 + 4} = \sqrt{14}$. Therefore $\cos \theta = \frac{2}{2\sqrt{5}\sqrt{14}} = \frac{1}{\sqrt{70}}$, so $\theta = \cos^{-1}(1/\sqrt{70}) \approx 83^\circ$.
\end{solution}

\vspace{1em}

\begin{problem}[Unit Vector in Direction]
Given $\mathbf{u} = 2\mathbf{i} - 2\mathbf{j} - 5\mathbf{k}$, find a vector parallel to $\mathbf{u}$ with length 5 units.
\end{problem}

\begin{hint}
Find the unit vector in the direction of $\mathbf{u}$, then scale by 5.
\end{hint}

\begin{solution}[Sketch]
$|\mathbf{u}| = \sqrt{4 + 4 + 25} = \sqrt{33}$. Unit vector: $\hat{\mathbf{u}} = \frac{1}{\sqrt{33}}(2\mathbf{i} - 2\mathbf{j} - 5\mathbf{k})$. Required vector: $5\hat{\mathbf{u}} = \frac{5}{\sqrt{33}}(2\mathbf{i} - 2\mathbf{j} - 5\mathbf{k})$.
\end{solution}

\vspace{1em}

\begin{problem}[Distance to Plane and Axis]
Find the distance from point $P(1, 4, 3)$ to: (a) the $xz$-plane, (b) the $y$-axis.
\end{problem}

\begin{hint}
(a) Distance to $xz$-plane is the $y$-coordinate. (b) Distance to $y$-axis uses $x$ and $z$ coordinates.
\end{hint}

\begin{solution}[Sketch]
(a) The $xz$-plane has equation $y = 0$, so distance is $|y| = |4| = 4$. (b) The closest point on the $y$-axis is $(0, 4, 0)$, so distance is $\sqrt{1^2 + 3^2} = \sqrt{10}$.
\end{solution}

\vspace{1em}

\begin{problem}[Unit Vector Perpendicular to Two Vectors]
Find a unit vector perpendicular to both $\mathbf{u} = \mathbf{i} + \mathbf{j}$ and $\mathbf{v} = \mathbf{i} + \mathbf{k}$, where $\mathbf{i}$, $\mathbf{j}$, and $\mathbf{k}$ are the standard unit vectors along the $x$-, $y$-, and $z$-axes respectively.
\end{problem}

\begin{hint}
Use the cross product $\mathbf{u} \times \mathbf{v}$, then normalize.
\end{hint}

\begin{solution}[Sketch]
$\mathbf{u} \times \mathbf{v} = \begin{vmatrix} \mathbf{i} & \mathbf{j} & \mathbf{k} \\ 1 & 1 & 0 \\ 1 & 0 & 1 \end{vmatrix} = \mathbf{i}(1) - \mathbf{j}(1) + \mathbf{k}(-1) = \mathbf{i} - \mathbf{j} - \mathbf{k}$. Magnitude: $|\mathbf{w}| = \sqrt{1 + 1 + 1} = \sqrt{3}$. Unit vector: $\frac{1}{\sqrt{3}}(\mathbf{i} - \mathbf{j} - \mathbf{k})$.
\end{solution}

\vspace{1em}

\begin{problem}[Point on Line]
Find $a$ and $b$ such that $(a, 1, b)$ lies on the line through $(0, 2, 1)$ and $(2, 7, 4)$.
\end{problem}

\begin{hint}
Find parametric equations using direction vector $(2, 5, 3)$, then solve for the parameter when $y = 1$.
\end{hint}

\begin{solution}[Sketch]
Direction vector: $(2, 5, 3)$. Parametric equations: $x = 2t$, $y = 2 + 5t$, $z = 1 + 3t$. When $y = 1$: $2 + 5t = 1 \Rightarrow t = -1/5$. Therefore $a = 2(-1/5) = -2/5$ and $b = 1 + 3(-1/5) = 2/5$.
\end{solution}

\vspace{1em}

\begin{problem}[Equal Magnitude Perpendicular Vectors]
Prove that if $\mathbf{u}$ and $\mathbf{v}$ are non-zero vectors of equal magnitude, then $\mathbf{u} - \mathbf{v}$ is perpendicular to $\mathbf{u} + \mathbf{v}$.
\end{problem}

\begin{hint}
Show that $(\mathbf{u} - \mathbf{v}) \cdot (\mathbf{u} + \mathbf{v}) = 0$.
\end{hint}

\begin{solution}[Sketch]
Expand the dot product: $(\mathbf{u} - \mathbf{v}) \cdot (\mathbf{u} + \mathbf{v}) = \mathbf{u} \cdot \mathbf{u} - \mathbf{v} \cdot \mathbf{v} = |\mathbf{u}|^2 - |\mathbf{v}|^2$. Since $|\mathbf{u}| = |\mathbf{v}|$, this equals 0, proving perpendicularity.
\end{solution}

\vspace{1em}

\begin{problem}[Direction Cosines]
Prove that $\cos^2\alpha + \cos^2\beta + \cos^2\gamma = 1$ for a position vector $\mathbf{r} = a\mathbf{i} + b\mathbf{j} + c\mathbf{k}$ making angles $\alpha, \beta, \gamma$ with the positive $x$-, $y$-, and $z$-axes respectively. (The quantities $\cos\alpha$, $\cos\beta$, and $\cos\gamma$ are called the \emph{direction cosines} of the vector $\mathbf{r}$.)
\end{problem}

\begin{hint}
Use dot products with unit vectors $\mathbf{i}, \mathbf{j}, \mathbf{k}$ to find the direction cosines.
\end{hint}

\begin{solution}[Sketch]
$\cos\alpha = \frac{a}{|\mathbf{r}|}$, $\cos\beta = \frac{b}{|\mathbf{r}|}$, $\cos\gamma = \frac{c}{|\mathbf{r}|}$. Therefore $\cos^2\alpha + \cos^2\beta + \cos^2\gamma = \frac{a^2 + b^2 + c^2}{|\mathbf{r}|^2} = \frac{|\mathbf{r}|^2}{|\mathbf{r}|^2} = 1$.
\end{solution}

\vspace{1em}

\begin{problem}[Line Intersection by Components]
Find the intersection point of lines $\mathbf{r} = \begin{pmatrix} 3 \\ -1 \\ 7 \end{pmatrix} + \lambda_1 \begin{pmatrix} 1 \\ 2 \\ 1 \end{pmatrix}$ and $\mathbf{r} = \begin{pmatrix} 3 \\ -6 \\ 2 \end{pmatrix} + \lambda_2 \begin{pmatrix} -2 \\ 1 \\ 3 \end{pmatrix}$.
\end{problem}

\begin{hint}
Equate components and solve the system for $\lambda_1$ and $\lambda_2$.
\end{hint}

\begin{solution}[Sketch]
Equating: $3 + \lambda_1 = 3 - 2\lambda_2$, $-1 + 2\lambda_1 = -6 + \lambda_2$, $7 + \lambda_1 = 2 + 3\lambda_2$. From first equation: $\lambda_1 = -2\lambda_2$. Substitute into second: $-1 - 4\lambda_2 = -6 + \lambda_2 \Rightarrow \lambda_2 = 1$, so $\lambda_1 = -2$. Check third equation: $7 - 2 = 5 = 2 + 3$ $\checkmark$. Intersection point: $(1, -5, 5)$.
\end{solution}

\vspace{1em}

\begin{problem}[Multiple Choice: Cartesian Equation]
What is the Cartesian equation of $\mathbf{r} = \begin{pmatrix} 1 \\ 3 \end{pmatrix} + \lambda \begin{pmatrix} -2 \\ 4 \end{pmatrix}$? 

Options: \\
A. $2y + x = 7$, \\
B. $y - 2x = -5$, \\
C. $y + 2x = 5$, \\ 
D. $2y - x = -1$.
\end{problem}

\begin{hint}
Eliminate $\lambda$ from $x = 1 - 2\lambda$ and $y = 3 + 4\lambda$.
\end{hint}

\begin{solution}[Sketch]
From $x = 1 - 2\lambda$, get $\lambda = \frac{1-x}{2}$. Substitute: $y = 3 + 4 \cdot \frac{1-x}{2} = 3 + 2(1-x) = 5 - 2x$, giving $y + 2x = 5$. Answer: C.
\end{solution}

\vspace{1em}

\begin{problem}[Closest Point on Line to Origin]
Find the point on line $\mathbf{r} = \begin{pmatrix} 1 \\ 4 \\ 6 \end{pmatrix} + \lambda \begin{pmatrix} 2 \\ 1 \\ 1 \end{pmatrix}$ closest to the origin. That is, find the value of $\lambda$ such that the distance from the point $\mathbf{r}(\lambda)$ to the origin $O(0, 0, 0)$ is minimized.
\end{problem}

\begin{hint}
The closest point occurs when $\mathbf{r}$ is perpendicular to the direction vector.
\end{hint}

\begin{solution}[Sketch]
Let $\mathbf{r} = \begin{pmatrix} 1 + 2\lambda \\ 4 + \lambda \\ 6 + \lambda \end{pmatrix}$. For closest point: $\mathbf{r} \cdot \begin{pmatrix} 2 \\ 1 \\ 1 \end{pmatrix} = 0$. This gives $2(1+2\lambda) + (4+\lambda) + (6+\lambda) = 0 \Rightarrow 6\lambda + 12 = 0 \Rightarrow \lambda = -2$. Point: $(-3, 2, 4)$.
\end{solution}

\vspace{1em}

\begin{problem}[Unit Vector Perpendicular to Two]
Find a unit vector perpendicular to $\begin{bmatrix} 3 \\ -2 \\ 1 \end{bmatrix}$ and $\begin{bmatrix} 1 \\ 4 \\ -1 \end{bmatrix}$.
\end{problem}

\begin{hint}
Compute the cross product and normalize the result.
\end{hint}

\begin{solution}[Sketch]
Cross product: $\mathbf{a} \times \mathbf{b} = \begin{bmatrix} (-2)(-1) - (1)(4) \\ (1)(1) - (3)(-1) \\ (3)(4) - (-2)(1) \end{bmatrix} = \begin{bmatrix} -2 \\ 4 \\ 14 \end{bmatrix}$. Magnitude: $\sqrt{4 + 16 + 196} = 6\sqrt{6}$. Unit vector: $\frac{1}{6\sqrt{6}}\begin{bmatrix} -2 \\ 4 \\ 14 \end{bmatrix} = \frac{1}{3\sqrt{6}}\begin{bmatrix} -1 \\ 2 \\ 7 \end{bmatrix}$.
\end{solution}

\vspace{1em}

\begin{problem}[Perpendicular Dot Product Proof]
Prove that for non-zero vectors $\mathbf{a}, \mathbf{b}$, the equation $(\mathbf{a} + \mathbf{b}) \cdot (\mathbf{a} + \mathbf{b}) = |\mathbf{a}|^2 + |\mathbf{b}|^2$ holds only if $\mathbf{a} \perp \mathbf{b}$.
\end{problem}

\begin{hint}
Expand the left side and compare with the right side.
\end{hint}

\begin{solution}[Sketch]
Expanding: $(\mathbf{a} + \mathbf{b}) \cdot (\mathbf{a} + \mathbf{b}) = |\mathbf{a}|^2 + 2\mathbf{a} \cdot \mathbf{b} + |\mathbf{b}|^2$. Comparing with RHS: $|\mathbf{a}|^2 + 2\mathbf{a} \cdot \mathbf{b} + |\mathbf{b}|^2 = |\mathbf{a}|^2 + |\mathbf{b}|^2 \Rightarrow 2\mathbf{a} \cdot \mathbf{b} = 0 \Rightarrow \mathbf{a} \cdot \mathbf{b} = 0$, which means $\mathbf{a} \perp \mathbf{b}$.
\end{solution}

\vspace{1em}
