% Part 2 Problem 05 (Easy): Mersenne Prime Contrapositive
% Source: samples/06.tex
% Technique: Proof by Contrapositive
% Topic: Number Theory/Primes

\begin{problem}[Easy]
Prove that for all integers $n \geq 3$, if $2^n - 1$ is prime, then $n$ cannot be even.
\end{problem}

\begin{hint}
Use proof by contrapositive. Instead of proving "$P \implies Q$", prove "$\neg Q \implies \neg P$".

That is, prove: "If $n$ is even (and $n \geq 3$), then $2^n - 1$ is composite."

Write $n = 2k$ and factor $2^{2k} - 1$ as a difference of squares.
\end{hint}

\begin{solutionsketch}
\textbf{Proof by Contrapositive:}

We prove the contrapositive: If $n$ is even (with $n \geq 3$), then $2^n - 1$ is composite.

Assume $n$ is even. Then $n = 2k$ for some integer $k$.

Since $n \geq 3$ and $n$ is even, we have $n \geq 4$, so $k \geq 2$.

Substitute:
\[
2^n - 1 = 2^{2k} - 1 = (2^k)^2 - 1^2
\]

Factor as difference of squares:
\[
2^n - 1 = (2^k - 1)(2^k + 1)
\]

Since $k \geq 2$:
\begin{itemize}
\item $2^k \geq 4$, so $2^k - 1 \geq 3 > 1$
\item $2^k \geq 4$, so $2^k + 1 \geq 5 > 1$
\end{itemize}

Both factors are integers strictly greater than $1$, so their product is composite.

Therefore, $2^n - 1$ is composite.

\vspace{0.3cm}

By contrapositive, if $2^n - 1$ is prime, then $n$ cannot be even. $\qed$

\vspace{0.3cm}

\textit{Note: This is why Mersenne primes have the form $2^p - 1$ where $p$ itself is prime (though not all such numbers are prime).}
\end{solutionsketch}
