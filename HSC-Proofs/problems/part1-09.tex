% Part 1, Problem 9: Divisibility by 4 rule - Direct Proof (Biconditional, Multi-part)
% Medium

\begin{problem}[Medium]
Prove that a number is divisible by 4 if and only if the last two digits form a number divisible by 4.
\end{problem}

\begin{solution}
\textbf{Biconditional Proof (If and Only If)}

This requires proving both directions:

\textbf{Forward Direction} ($\implies$): If $N$ is divisible by 4, then its last two digits form a number divisible by 4.

\textbf{Reverse Direction} ($\impliedby$): If the last two digits form a number divisible by 4, then $N$ is divisible by 4.

\vspace{0.5em}
\textbf{Setup:} Express any integer $N$ as:
\[ N = 100A + L \]
where $A$ is the number formed by all digits except the last two, and $L$ is the two-digit number formed by the last two digits ($0 \leq L \leq 99$).

\vspace{0.5em}
\textbf{Forward Direction Proof:}

Assume $4 \mid N$, so $N \equiv 0 \pmod{4}$.

Then:
\begin{align*}
    100A + L &\equiv 0 \pmod{4}
\end{align*}

Since $100 = 4 \times 25$, we have $100A \equiv 0 \pmod{4}$.

Therefore:
\begin{align*}
    0 + L &\equiv 0 \pmod{4} \\
    L &\equiv 0 \pmod{4}
\end{align*}

Thus $4 \mid L$. \hfill $\square$

\vspace{0.5em}
\textbf{Reverse Direction Proof:}

Assume $4 \mid L$, so $L \equiv 0 \pmod{4}$.

Since $100 = 4 \times 25$, we have $100A \equiv 0 \pmod{4}$.

Therefore:
\begin{align*}
    N = 100A + L &\equiv 0 + 0 \pmod{4} \\
    N &\equiv 0 \pmod{4}
\end{align*}

Thus $4 \mid N$. \hfill $\square$

\vspace{0.5em}
\textbf{Conclusion:}

Since both directions are proven, we conclude:
\[ N \text{ is divisible by } 4 \iff \text{last two digits of } N \text{ divisible by } 4 \]
\end{solution}

\begin{takeaways}
\begin{itemize}
\item \textbf{Iff Proof Structure:} Must prove both ($\implies$) and ($\impliedby$) directions independently
\item \textbf{Digit Representation:} Writing $N = 100A + L$ separates last two digits for modular analysis
\item \textbf{Key Observation:} Since $100 \equiv 0 \pmod{4}$, divisibility of $N$ by 4 depends only on last two digits
\item \textbf{Generalization:} Same technique proves divisibility rules for powers of 2 (e.g., rule for 8 uses last three digits)
\end{itemize}
\end{takeaways}
