% Part 1, Problem 10: Parity of m^2-n^2 - Direct Proof (Biconditional, Multi-part)
% Medium

\begin{problem}[Medium]
Prove that if $m, n$ are integers, then $m^2 - n^2$ is even if and only if at least one of $(m+n)$ and $(m-n)$ is even.
\end{problem}

\begin{solution}
\textbf{Biconditional Proof}

\textbf{Step 1: Factor the expression}

\[ m^2 - n^2 = (m-n)(m+n) \]

This factorization will be used in both directions.

\vspace{0.5em}
\textbf{Forward Direction} ($\implies$): If $m^2 - n^2$ is even, then at least one of $(m+n)$ or $(m-n)$ is even.

\begin{proof}
Since $m^2 - n^2$ is even, the product $(m-n)(m+n)$ is even.

A product of two integers is even if and only if at least one factor is even.

Therefore, at least one of $(m-n)$ or $(m+n)$ must be even. \qed
\end{proof}

\vspace{0.5em}
\textbf{Reverse Direction} ($\impliedby$): If at least one of $(m+n)$ or $(m-n)$ is even, then $m^2 - n^2$ is even.

\begin{proof}
We consider two cases:

\textbf{Case 1:} $(m+n)$ is even.

Write $(m+n) = 2k$ for some integer $k$.

Then:
\[ m^2 - n^2 = (m-n)(m+n) = (m-n)(2k) = 2k(m-n) \]

Since $k(m-n)$ is an integer, $m^2 - n^2$ is even.

\textbf{Case 2:} $(m-n)$ is even.

Write $(m-n) = 2j$ for some integer $j$.

Then:
\[ m^2 - n^2 = (m-n)(m+n) = (2j)(m+n) = 2j(m+n) \]

Since $j(m+n)$ is an integer, $m^2 - n^2$ is even.

In both cases, $m^2 - n^2$ is even. \qed
\end{proof}

\vspace{0.5em}
\textbf{Conclusion:}

Both directions proven, therefore:
\[ m^2 - n^2 \text{ is even} \iff \text{at least one of } (m+n), (m-n) \text{ is even} \]
\end{solution}

\begin{takeaways}
\begin{itemize}
\item \textbf{Factorization First:} Factoring $m^2 - n^2 = (m-n)(m+n)$ immediately connects to parity of factors
\item \textbf{Product Parity:} Product even $\iff$ at least one factor even (fundamental parity property)
\item \textbf{Case Analysis:} Reverse direction handles two cases (either factor even) separately
\item \textbf{Note on Parity:} Actually, $(m+n)$ and $(m-n)$ always have same parity (both even or both odd), so "at least one even" is equivalent to "both even"
\end{itemize}
\end{takeaways}
