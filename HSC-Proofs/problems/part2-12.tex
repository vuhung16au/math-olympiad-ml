% Part 2 Problem 12 (Hard): Pythagorean triple parity constraint
% Source: samples/23.tex
% Technique: Proof by Contradiction
% Topic: Parity/Number Theory

\begin{problem}[Hard]
Prove that there is no Pythagorean triple $(a, b, c)$ where $a$ and $b$ (the two smallest numbers) are both even and $c$ (the largest number) is odd.
\end{problem}

\begin{hint}
Use proof by contradiction. Assume such a triple exists with $a = 2k$ and $b = 2m$ (both even) and $c$ odd.

Substitute into the Pythagorean equation $a^2 + b^2 = c^2$ and analyze the parity of $c^2$. What can you conclude about the parity of $c$?
\end{hint}

\begin{solutionsketch}
\textbf{Proof by Contradiction:}

Assume there exists a Pythagorean triple $(a, b, c)$ where:
\begin{itemize}
\item $a^2 + b^2 = c^2$
\item $a$ and $b$ are both even
\item $c$ is odd
\end{itemize}

Since $a$ and $b$ are even, write $a = 2k$ and $b = 2m$ for integers $k, m$.

\vspace{0.3cm}

\textbf{Substitute into Pythagorean equation:}
\begin{align*}
(2k)^2 + (2m)^2 &= c^2 \\
4k^2 + 4m^2 &= c^2 \\
4(k^2 + m^2) &= c^2
\end{align*}

\vspace{0.3cm}

\textbf{Analyze parity of $c^2$:}

The equation shows $c^2 = 4(k^2 + m^2)$, which is a multiple of $4$.

In particular, $c^2$ is divisible by $2$, so $c^2$ is \textbf{even}.

\vspace{0.3cm}

\textbf{Derive parity of $c$:}

If $c$ were odd, then $c = 2n+1$ for some integer $n$, and:
\[
c^2 = (2n+1)^2 = 4n^2 + 4n + 1 = 2(2n^2 + 2n) + 1
\]
This shows $c^2$ would be \textbf{odd}, contradicting that $c^2$ is even.

Therefore, $c$ must be \textbf{even}.

\vspace{0.3cm}

\textbf{Contradiction:}

We derived that $c$ must be even, which contradicts our assumption that $c$ is odd.

Hence, no such Pythagorean triple exists. $\qed$
\end{solutionsketch}
