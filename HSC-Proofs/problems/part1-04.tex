% Part 1, Problem 4: Contrapositive logic - Direct Proof (Logical Analysis)
% Easy

\begin{problem}[Easy]
Consider the statement: ``For all integers $n$, if $n$ is a multiple of 6, then $n$ is a multiple of 2.''

Which of the following is the contrapositive of this statement?

\begin{enumerate}[label=\Alph*.]
    \item There exists an integer $n$ such that $n$ is a multiple of 6 and not a multiple of 2.
    \item There exists an integer $n$ such that $n$ is a multiple of 2 and not a multiple of 6.
    \item For all integers $n$, if $n$ is not a multiple of 2, then $n$ is not a multiple of 6.
    \item For all integers $n$, if $n$ is not a multiple of 6, then $n$ is not a multiple of 2.
\end{enumerate}
\end{problem}

\begin{solution}
\textbf{Logical Analysis of Contrapositive}

\textbf{Step 1: Identify the logical structure}

The original statement has the form:
\[ \forall n \in \mathbb{Z}, \quad P(n) \implies Q(n) \]

Where:
\begin{itemize}
    \item $P(n)$: ``$n$ is a multiple of 6''
    \item $Q(n)$: ``$n$ is a multiple of 2''
\end{itemize}

\textbf{Step 2: Apply contrapositive definition}

The contrapositive of $P \implies Q$ is $\neg Q \implies \neg P$.

Key: The universal quantifier (``for all'') remains unchanged.

\textbf{Step 3: Negate each part}

\begin{itemize}
    \item $\neg Q(n)$: ``$n$ is NOT a multiple of 2''
    \item $\neg P(n)$: ``$n$ is NOT a multiple of 6''
\end{itemize}

\textbf{Step 4: Construct the contrapositive}

\[ \forall n \in \mathbb{Z}, \quad \neg Q(n) \implies \neg P(n) \]

In words: ``For all integers $n$, if $n$ is not a multiple of 2, then $n$ is not a multiple of 6.''

This matches \textbf{Option C}.

\textbf{Analysis of incorrect options:}
\begin{itemize}
    \item \textbf{Option A:} Negation of original ($\exists n: P(n) \land \neg Q(n)$), not contrapositive
    \item \textbf{Option B:} Negation of converse
    \item \textbf{Option D:} Inverse ($\neg P \implies \neg Q$), not contrapositive
\end{itemize}

\textbf{Answer:} \boxed{\text{C}}
\end{solution}

\begin{takeaways}
\begin{itemize}
\item \textbf{Contrapositive Form:} $P \implies Q$ has contrapositive $\neg Q \implies \neg P$ (swap and negate both parts)
\item \textbf{Logical Equivalence:} A statement and its contrapositive are logically equivalent (same truth value)
\item \textbf{Quantifiers Unchanged:} Universal quantifier (``for all'') stays when forming contrapositive
\item \textbf{Common Errors:} Inverse ($\neg P \implies \neg Q$) and converse ($Q \implies P$) are NOT equivalent to original
\end{itemize}
\end{takeaways}
