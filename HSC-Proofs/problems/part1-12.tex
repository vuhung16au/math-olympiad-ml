% Part 1, Problem 12: Nested radical induction - Mathematical Induction
% Hard

\begin{problem}[Hard]
The numbers $a_n$, for integers $n \geq 1$, are defined as:
\begin{align*}
    a_1 &= \sqrt{2} \\
    a_2 &= \sqrt{2 + \sqrt{2}} \\
    a_3 &= \sqrt{2 + \sqrt{2 + \sqrt{2}}}, \text{ and so on.}
\end{align*}

These numbers satisfy the relation $a_{n+1}^2 = 2 + a_n$, for $n \geq 1$. (Do NOT prove this.)

Use mathematical induction to prove that $a_n = 2\cos\left(\frac{\pi}{2^{n+1}}\right)$ for all integers $n \geq 1$.
\end{problem}

\begin{solution}
\textbf{Proof by Mathematical Induction}

\textbf{Base Case} ($n=1$):

LHS: $a_1 = \sqrt{2}$ (given)

RHS: $2\cos\left(\frac{\pi}{2^{1+1}}\right) = 2\cos\left(\frac{\pi}{4}\right) = 2 \cdot \frac{\sqrt{2}}{2} = \sqrt{2}$

Since LHS = RHS, the statement holds for $n=1$. \checkmark

\vspace{0.5em}
\textbf{Inductive Hypothesis:}

Assume the statement is true for $n=k$, where $k$ is a positive integer:
\[ a_k = 2\cos\left(\frac{\pi}{2^{k+1}}\right) \]

\vspace{0.5em}
\textbf{Inductive Step:}

We must prove the statement for $n=k+1$:
\[ a_{k+1} = 2\cos\left(\frac{\pi}{2^{k+2}}\right) \]

Using the recurrence relation $a_{k+1}^2 = 2 + a_k$ and the inductive hypothesis:
\begin{align*}
    a_{k+1}^2 &= 2 + 2\cos\left(\frac{\pi}{2^{k+1}}\right) \\
    &= 2\left[1 + \cos\left(\frac{\pi}{2^{k+1}}\right)\right]
\end{align*}

Apply the double angle identity: $1 + \cos(2\theta) = 2\cos^2(\theta)$.

Let $2\theta = \frac{\pi}{2^{k+1}}$, so $\theta = \frac{\pi}{2^{k+2}}$:
\begin{align*}
    a_{k+1}^2 &= 2 \cdot 2\cos^2\left(\frac{\pi}{2^{k+2}}\right) \\
    &= 4\cos^2\left(\frac{\pi}{2^{k+2}}\right)
\end{align*}

Taking square roots:
\[ a_{k+1} = \pm 2\cos\left(\frac{\pi}{2^{k+2}}\right) \]

Since $n \geq 1$, we have $0 < \frac{\pi}{2^{k+2}} < \frac{\pi}{2}$, so $\cos\left(\frac{\pi}{2^{k+2}}\right) > 0$.

Also, $a_n$ is defined as nested square roots of positive numbers, so $a_{k+1} > 0$.

Therefore:
\[ a_{k+1} = 2\cos\left(\frac{\pi}{2^{k+2}}\right) \]

This completes the inductive step. \checkmark

\vspace{0.5em}
\textbf{Conclusion:}

By mathematical induction, $a_n = 2\cos\left(\frac{\pi}{2^{n+1}}\right)$ for all integers $n \geq 1$.

\hfill $\square$
\end{solution}

\begin{takeaways}
\begin{itemize}
\item \textbf{Induction with Recurrence:} Use given recurrence relation in inductive step to relate $a_{k+1}$ to $a_k$
\item \textbf{Double Angle Formula:} Identity $1 + \cos(2\theta) = 2\cos^2(\theta)$ is key to converting sum to square
\item \textbf{Sign Consideration:} Must justify taking positive square root using domain/range analysis
\item \textbf{Angle Halving:} Pattern shows $a_n$ relates to $\cos$ of successively halved angles ($\pi/4, \pi/8, \pi/16, \ldots$)
\end{itemize}
\end{takeaways}
