% Part 2 Problem 08 (Medium): Divisibility by 9 iff sum of digits divisible by 9
% Source: samples/21.tex
% Technique: Direct Proof (Biconditional)
% Topic: Divisibility Rules/Modular Arithmetic

\begin{problem}[Medium]
Prove that a three-digit number is divisible by $9$ if and only if the sum of its digits is divisible by $9$.
\end{problem}

\begin{hint}
Let $N = 100a + 10b + c$ where $a, b, c$ are the digits.

Rewrite this as $N = 9(11a + b) + (a + b + c)$ to establish the relationship between $N$ and the digit sum.

Then prove both directions of the biconditional using this relationship.
\end{hint}

\begin{solutionsketch}
Let $N = 100a + 10b + c$ be a three-digit number with digits $a, b, c$.

Let $S = a + b + c$ be the sum of digits.

\vspace{0.3cm}

\textbf{Key Relationship:}

Rewrite $N$:
\begin{align*}
N &= 100a + 10b + c \\
&= (99a + a) + (9b + b) + c \\
&= 9(11a + b) + (a + b + c) \\
&= 9(11a + b) + S
\end{align*}

Therefore: $N - S = 9(11a + b)$, which means $N \equiv S \pmod{9}$.

\vspace{0.3cm}

\textbf{Direction 1 ($\Rightarrow$):} If $9|N$, then $9|S$.

If $N \equiv 0 \pmod{9}$, then from $N \equiv S \pmod{9}$, we get $S \equiv 0 \pmod{9}$.

Thus $9|S$.

\vspace{0.3cm}

\textbf{Direction 2 ($\Leftarrow$):} If $9|S$, then $9|N$.

If $S \equiv 0 \pmod{9}$, then from $N \equiv S \pmod{9}$, we get $N \equiv 0 \pmod{9}$.

Thus $9|N$.

\vspace{0.3cm}

Both directions proven, so the biconditional holds. $\qed$

\vspace{0.3cm}

\textit{Note: This proof generalizes to all positive integers and divisibility by 9. The key is the modular relationship $N \equiv S \pmod{9}$.}
\end{solutionsketch}
