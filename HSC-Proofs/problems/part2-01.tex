% Part 2 Problem 01 (Easy): a^3 - a + 1 is always odd
% Source: samples/13.tex
% Technique: Direct Proof or Proof by Cases
% Topic: Parity

\begin{problem}[Easy]
Prove that the expression $a^3 - a + 1$ is odd for all positive integer values of $a$.
\end{problem}

\begin{hint}
Consider factoring $a^3 - a$ as a product of consecutive integers. What can you say about the parity of any product of consecutive integers?

Alternatively, try proof by cases: analyze when $a$ is even and when $a$ is odd separately.
\end{hint}

\begin{solutionsketch}
\textbf{Method 1 (Factorization - Elegant):}

Factor the expression:
\begin{align*}
a^3 - a + 1 &= a(a^2 - 1) + 1 \\
&= a(a-1)(a+1) + 1 \\
&= (a-1)a(a+1) + 1
\end{align*}

The product $(a-1)a(a+1)$ is the product of three consecutive integers. In any set of consecutive integers, at least one must be even, so the product is even. Let $(a-1)a(a+1) = 2k$ for some integer $k$.

Therefore, $a^3 - a + 1 = 2k + 1$, which is odd by definition.

\vspace{0.5cm}

\textbf{Method 2 (Cases):}

\textit{Case 1:} If $a = 2m$ (even), then $a^3 - a + 1 = 8m^3 - 2m + 1 = 2(4m^3 - m) + 1$ (odd).

\textit{Case 2:} If $a = 2m+1$ (odd), then expanding shows $a^3 - a + 1 = 2(\text{integer}) + 1$ (odd).

In both cases, the expression is odd. $\qed$
\end{solutionsketch}
