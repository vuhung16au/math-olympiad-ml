% Part 1, Problem 7: Modular powers - Proof by Cases (Multi-part)
% Medium

\begin{problem}[Medium]
\begin{enumerate}[label=(\alph*)]
    \item Given $a$ is integral and not divisible by 5, prove the remainder when $a^2$ is divided by 5 is either 1 or 4.
    
    \item Hence, given that $a, b$ are integral and not divisible by 5, prove that $a^4 - b^4$ is divisible by 5.
\end{enumerate}
\end{problem}

\begin{solution}
\textbf{Part (a): Proof by Cases}

Since $a$ is not divisible by 5, we have $a \not\equiv 0 \pmod{5}$.

By the division algorithm, $a$ must be congruent to 1, 2, 3, or 4 modulo 5.

We check each case:

\textbf{Case 1:} $a \equiv 1 \pmod{5}$
\[ a^2 \equiv 1^2 \equiv 1 \pmod{5} \]

\textbf{Case 2:} $a \equiv 2 \pmod{5}$
\[ a^2 \equiv 2^2 \equiv 4 \pmod{5} \]

\textbf{Case 3:} $a \equiv 3 \pmod{5}$
\[ a^2 \equiv 3^2 \equiv 9 \equiv 4 \pmod{5} \]

\textbf{Case 4:} $a \equiv 4 \pmod{5}$
\[ a^2 \equiv 4^2 \equiv 16 \equiv 1 \pmod{5} \]

In all cases, $a^2 \equiv 1$ or $4 \pmod{5}$.

Therefore, the remainder when $a^2$ is divided by 5 is either 1 or 4. \hfill $\square$

\vspace{1em}

\textbf{Part (b): Using Part (a)}

From part (a), for any integer $x$ not divisible by 5: $x^2 \equiv 1$ or $4 \pmod{5}$.

Consider $a^4 = (a^2)^2$:
\begin{itemize}
    \item If $a^2 \equiv 1 \pmod{5}$, then $(a^2)^2 \equiv 1^2 \equiv 1 \pmod{5}$
    \item If $a^2 \equiv 4 \pmod{5}$, then $(a^2)^2 \equiv 4^2 \equiv 16 \equiv 1 \pmod{5}$
\end{itemize}

Thus $a^4 \equiv 1 \pmod{5}$ for any integer $a$ not divisible by 5.

Similarly, $b^4 \equiv 1 \pmod{5}$.

Therefore:
\begin{align*}
    a^4 - b^4 &\equiv 1 - 1 \pmod{5} \\
    &\equiv 0 \pmod{5}
\end{align*}

Hence $5 \mid (a^4 - b^4)$. \hfill $\square$
\end{solution}

\begin{takeaways}
\begin{itemize}
\item \textbf{Systematic Case Analysis:} For $5 \nmid a$, check all residues $a \equiv 1, 2, 3, 4 \pmod{5}$ systematically
\item \textbf{Squaring Congruences:} If $a \equiv b \pmod{m}$, then $a^2 \equiv b^2 \pmod{m}$
\item \textbf{``Hence'' Strategy:} Part (b) builds directly on part (a)'s result; apply it twice to get $a^4 \equiv b^4 \equiv 1 \pmod{5}$
\item \textbf{Fermat's Little Theorem Preview:} Result $a^4 \equiv 1 \pmod{5}$ for $\gcd(a,5)=1$ is special case of FLT
\end{itemize}
\end{takeaways}
