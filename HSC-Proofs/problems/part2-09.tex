% Part 2 Problem 09 (Medium): Complex numbers geometry (multi-part)
% Source: samples/16.tex
% Technique: Direct Proof (Geometric/Algebraic)
% Topic: Complex Numbers

\begin{problem}[Medium]
Consider the region $\mathcal{R}$ in the complex plane defined by $|z - (2\sqrt{3} + 2i)| \leq 2$.

\begin{enumerate}[label=(\alph*)]
\item Describe the region $\mathcal{R}$ geometrically (what shape? center? radius?).
\item Find the maximum and minimum values of $\arg(z)$ for $z \in \mathcal{R}$, where $-\pi < \arg(z) \leq \pi$.
\item Find the complex number $z$ associated with the maximum argument in part (b) in the form $x + iy$.
\end{enumerate}
\end{problem}

\begin{hint}
(a) The inequality $|z - w| \leq r$ represents a disk (filled circle) in the complex plane.

(b) Find $\arg(C)$ where $C = 2\sqrt{3} + 2i$ is the center. The max/min arguments occur at tangent lines from the origin to the circle. Use the right triangle formed by the origin, center, and tangent point.

(c) The point with maximum argument lies on the ray from $O$ with angle $\arg_{\max}$ found in (b), at distance $\sqrt{|OC|^2 - r^2}$ from origin.
\end{hint}

\begin{solutionsketch}
\textbf{(a) Geometric Description:}

The region is a closed disk (circle plus interior) with:
\begin{itemize}
\item Center: $C = 2\sqrt{3} + 2i$ (which is $(2\sqrt{3}, 2)$ in Cartesian coordinates)
\item Radius: $r = 2$
\end{itemize}

Distance from origin to center: $|OC| = \sqrt{(2\sqrt{3})^2 + 2^2} = \sqrt{12 + 4} = 4$.

Since $|OC| = 4 > r = 2$, the origin lies outside the circle.

\vspace{0.3cm}

\textbf{(b) Max and Min Arguments:}

The argument of the center is:
\[
\arg(C) = \arctan\left(\frac{2}{2\sqrt{3}}\right) = \arctan\left(\frac{1}{\sqrt{3}}\right) = \frac{\pi}{6}
\]

The tangent lines from $O$ to the circle create a right triangle with:
\begin{itemize}
\item Hypotenuse: $|OC| = 4$
\item Opposite side (radius): $r = 2$
\end{itemize}

The angle $\theta$ from $OC$ to the tangent satisfies:
\[
\sin \theta = \frac{r}{|OC|} = \frac{2}{4} = \frac{1}{2} \implies \theta = \frac{\pi}{6}
\]

Therefore:
\begin{align*}
\arg_{\max} &= \frac{\pi}{6} + \frac{\pi}{6} = \frac{\pi}{3} \\
\arg_{\min} &= \frac{\pi}{6} - \frac{\pi}{6} = 0
\end{align*}

\vspace{0.3cm}

\textbf{(c) Complex Number for Max Argument:}

The tangent point distance from origin:
\[
|OP| = \sqrt{|OC|^2 - r^2} = \sqrt{16 - 4} = \sqrt{12} = 2\sqrt{3}
\]

The point lies on the ray with argument $\pi/3$:
\begin{align*}
z &= 2\sqrt{3} \cdot e^{i\pi/3} = 2\sqrt{3}\left(\cos\frac{\pi}{3} + i\sin\frac{\pi}{3}\right) \\
&= 2\sqrt{3}\left(\frac{1}{2} + i\frac{\sqrt{3}}{2}\right) \\
&= \sqrt{3} + 3i
\end{align*}

\textbf{Answer:} $z = \sqrt{3} + 3i$. $\qed$
\end{solutionsketch}
