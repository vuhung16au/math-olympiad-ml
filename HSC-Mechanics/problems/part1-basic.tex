% Part 1: Basic Problems - Detailed Solutions
% Problems: 39, 05, 30, 16, 48

% Problem from samples/39.tex
\begin{problem}
A particle is moving along a straight line. Initially its displacement is at $x = 1$, its velocity is $v = 2$ and its acceleration is $a = 4$.

\vspace{0.5cm}

\noindent Which equation could describe the motion of the particle?

\begin{enumerate}[label=\Alph*.]
    \item $v = 2\sin(x-1) + 2$
    \item $v = 2 + 4\log_e x$
    \item $v^2 = 4(x^2 - 2)$
    \item $v^2 = x^2 + 2x + 4$
\end{enumerate}
\end{problem}

\begin{solution}
\textbf{Approach:} We need to test each option using the given initial conditions: $x = 1$, $v = 2$, and $a = 4$. Since the options provide velocity as a function of displacement $v = f(x)$, we use the acceleration formula:
\[
a = v\frac{dv}{dx}
\]

\textbf{Testing Option A:} $v = 2\sin(x-1) + 2$

First, check velocity at $x=1$:
\[ v = 2\sin(1-1) + 2 = 2\sin(0) + 2 = 0 + 2 = 2 \quad \checkmark \]

Now check acceleration. Find $\frac{dv}{dx}$:
\[ \frac{dv}{dx} = 2\cos(x-1) \]

At $x=1$:
\[ \frac{dv}{dx} = 2\cos(0) = 2 \]

Calculate acceleration using $a = v\frac{dv}{dx}$:
\[ a = (2)(2) = 4 \quad \checkmark \]

Both conditions are satisfied!

\vspace{0.5cm}

\textit{Verification of other options for completeness:}

\textbf{Option B:} $v = 2 + 4\log_e x$
\begin{itemize}
    \item At $x=1$: $v = 2 + 4(0) = 2$ \quad \checkmark
    \item $\frac{dv}{dx} = \frac{4}{x}$. At $x=1$: $\frac{dv}{dx} = 4$
    \item $a = v\frac{dv}{dx} = 2 \cdot 4 = 8$ \quad \texttimes \quad (Should be 4, not 8)
\end{itemize}

\textbf{Option C:} $v^2 = 4(x^2 - 2)$
\begin{itemize}
    \item At $x=1$: $v^2 = 4(1 - 2) = -4$
    \item Since $v^2$ cannot be negative, this is physically impossible. \quad \texttimes
\end{itemize}

\textbf{Option D:} $v^2 = x^2 + 2x + 4$
\begin{itemize}
    \item At $x=1$: $v^2 = 1 + 2 + 4 = 7$, so $v = \pm\sqrt{7} \neq 2$ \quad \texttimes
\end{itemize}

\textbf{Answer:} \boxed{\text{A}}
\end{solution}

\newpage

% Problem from samples/05.tex
\begin{problem}
The acceleration of a particle is given by $\ddot{x} = 32x(x^2 + 3)$, where $x$ is the displacement of the particle from a fixed-point $O$ after $t$ seconds, in metres. Initially the particle is at $O$ and has a velocity of $12 \, \text{m s}^{-1}$ in the negative direction.

\begin{enumerate}
    \item[(i)] Show that the velocity of the particle is given by $v = -4(x^2 + 3)$.
    \item[(ii)] Find the time taken for the particle to travel $3$ metres from the origin.
\end{enumerate}
\end{problem}

\begin{solution}
\textbf{(i) Finding the velocity function}

\textbf{Approach:} We have acceleration as a function of displacement. We use the identity:
\[ \ddot{x} = \frac{d}{dx}\left(\frac{1}{2}v^2\right) \]

Given: $\ddot{x} = 32x(x^2 + 3) = 32x^3 + 96x$

Applying the identity:
\begin{align*}
    \frac{d}{dx}\left(\frac{1}{2}v^2\right) &= 32x^3 + 96x
\end{align*}

Integrate both sides with respect to $x$:
\begin{align*}
    \frac{1}{2}v^2 &= \int (32x^3 + 96x) \, dx \\
    &= 32 \cdot \frac{x^4}{4} + 96 \cdot \frac{x^2}{2} + C \\
    &= 8x^4 + 48x^2 + C
\end{align*}

Multiply by 2:
\[ v^2 = 16x^4 + 96x^2 + 2C \]

\textbf{Apply initial conditions:} At $t=0$, $x=0$ and $v = -12$ (negative direction):
\begin{align*}
    (-12)^2 &= 16(0)^4 + 96(0)^2 + 2C \\
    144 &= 2C \\
    C &= 72
\end{align*}

Therefore:
\[ v^2 = 16x^4 + 96x^2 + 144 \]

Factor the right side:
\begin{align*}
    v^2 &= 16(x^4 + 6x^2 + 9) \\
    &= 16(x^2 + 3)^2
\end{align*}

Take the square root:
\[ v = \pm 4(x^2 + 3) \]

Since the particle moves in the negative direction initially and $x^2 + 3$ is always positive (meaning $v$ maintains constant sign), we take the negative case:
\[ \boxed{v = -4(x^2 + 3)} \]

\vspace{0.5cm}

\textbf{(ii) Time to travel 3 metres from the origin}

Since $v = -4(x^2+3) < 0$, the particle moves in the negative direction. To travel 3 metres from the origin means moving from $x = 0$ to $x = -3$.

We use $v = \frac{dx}{dt}$:
\begin{align*}
    \frac{dx}{dt} &= -4(x^2 + 3) \\
    \frac{dt}{dx} &= \frac{-1}{4(x^2 + 3)} \\
    dt &= \frac{-1}{4(x^2 + 3)} \, dx
\end{align*}

Integrate from $x = 0$ to $x = -3$:
\[ t = \int_{0}^{-3} \frac{-1}{4(x^2 + 3)} \, dx \]

Factor out the constant:
\[ t = -\frac{1}{4} \int_{0}^{-3} \frac{1}{x^2 + 3} \, dx \]

Use the standard integral $\int \frac{1}{a^2 + x^2} \, dx = \frac{1}{a}\tan^{-1}\left(\frac{x}{a}\right)$ with $a = \sqrt{3}$:
\begin{align*}
    t &= -\frac{1}{4} \left[ \frac{1}{\sqrt{3}} \tan^{-1}\left(\frac{x}{\sqrt{3}}\right) \right]_{0}^{-3} \\
    &= -\frac{1}{4\sqrt{3}} \left[ \tan^{-1}\left(\frac{-3}{\sqrt{3}}\right) - \tan^{-1}(0) \right] \\
    &= -\frac{1}{4\sqrt{3}} \left[ \tan^{-1}(-\sqrt{3}) - 0 \right]
\end{align*}

Since $\tan^{-1}(-\sqrt{3}) = -\frac{\pi}{3}$:
\begin{align*}
    t &= -\frac{1}{4\sqrt{3}} \left( -\frac{\pi}{3} \right) \\
    &= \frac{\pi}{12\sqrt{3}}
\end{align*}

Rationalize the denominator:
\[ t = \frac{\pi}{12\sqrt{3}} \cdot \frac{\sqrt{3}}{\sqrt{3}} = \frac{\pi\sqrt{3}}{36} \text{ seconds} \]

\textbf{Answer:} \boxed{t = \frac{\pi\sqrt{3}}{36} \text{ seconds}}
\end{solution}

\newpage

% Problem from samples/30.tex
\begin{problem}
A particle is projected from the origin with initial velocity $u$ to pass through a point $(a,b)$. Prove that there are two possible trajectories if:
$$ (u^2 - gb)^2 > g^2(a^2 + b^2) $$
Assume no air resistance.
\end{problem}

\begin{solution}
\textbf{Approach:} We'll derive the trajectory equation, substitute the target point $(a,b)$, and analyze when there are two distinct projection angles.

\textbf{Step 1: Establish the equations of motion}

Let the angle of projection be $\theta$ to the horizontal. The equations of motion are:
\begin{align}
    x &= ut \cos\theta \label{eq:x} \\
    y &= ut \sin\theta - \frac{1}{2}gt^2 \label{eq:y}
\end{align}

\textbf{Step 2: Derive the trajectory equation}

From equation \eqref{eq:x}, solve for $t$:
$$ t = \frac{x}{u \cos\theta} $$

Substitute into equation \eqref{eq:y}:
\begin{align*}
    y &= u\left( \frac{x}{u \cos\theta} \right) \sin\theta - \frac{1}{2}g \left( \frac{x}{u \cos\theta} \right)^2 \\
    &= x \tan\theta - \frac{gx^2}{2u^2 \cos^2\theta}
\end{align*}

Using the identity $\sec^2\theta = 1 + \tan^2\theta$:
\begin{equation}
    y = x \tan\theta - \frac{gx^2}{2u^2} (1 + \tan^2\theta) \label{eq:traj}
\end{equation}

\textbf{Step 3: Substitute the target point $(a,b)$}

Since the particle passes through $(a,b)$, substitute $x=a$ and $y=b$:
$$ b = a \tan\theta - \frac{ga^2}{2u^2} (1 + \tan^2\theta) $$

\textbf{Step 4: Form a quadratic in $\tan\theta$}

Let $T = \tan\theta$. Multiply by $2u^2$:
\begin{align*}
    2u^2b &= 2u^2a T - ga^2(1 + T^2) \\
    2u^2b &= 2u^2a T - ga^2 - ga^2 T^2
\end{align*}

Rearrange into standard form:
$$ ga^2 T^2 - 2u^2a T + (ga^2 + 2u^2b) = 0 $$

\textbf{Step 5: Analyze the discriminant}

For two distinct trajectories, the quadratic must have two distinct real roots, so the discriminant $\Delta > 0$.

With $A = ga^2$, $B = -2u^2a$, $C = ga^2 + 2u^2b$:
\begin{align*}
    \Delta &= B^2 - 4AC \\
    &= (-2u^2a)^2 - 4(ga^2)(ga^2 + 2u^2b) \\
    &= 4u^4a^2 - 4g^2a^4 - 8u^2g a^2 b
\end{align*}

For $\Delta > 0$ (assuming $a \neq 0$), divide by $4a^2$:
$$ u^4 - g^2a^2 - 2u^2gb > 0 $$

\textbf{Step 6: Complete the square}

Rearrange:
$$ u^4 - 2u^2gb - g^2a^2 > 0 $$

Add and subtract $g^2b^2$:
\begin{align*}
    u^4 - 2u^2gb + g^2b^2 - g^2b^2 - g^2a^2 &> 0 \\
    (u^2 - gb)^2 - g^2(a^2 + b^2) &> 0 \\
    (u^2 - gb)^2 &> g^2(a^2 + b^2)
\end{align*}

This is the required condition. \qed

\textbf{Physical interpretation:} The condition ensures that the initial kinetic energy $(u^2)$ minus the potential energy gained reaching height $b$ (represented by $gb$) is sufficiently large compared to the total "distance" to the target.
\end{solution}

\newpage

% Problem from samples/16.tex
\begin{problem}
Two model airplanes race around a circular course, with the second airplane taking off $T$ seconds after the first plane. Their position vectors are:
\[
\vect{r}_1(t) = \sin t \vect{i} + \cos t \vect{j} + \sin t \vect{k}
\]
and
\[
\vect{r}_2(t) = \sin(2t - \alpha) \vect{i} + \cos(2t - \alpha) \vect{j} + \sin(2t - \alpha) \vect{k}
\]
where time is measured in seconds from when the first airplane took off. They collide when they have both completed one and a half laps. Find $T$ given the first plane takes 20 seconds to complete one lap.
\end{problem}

\begin{solution}
\textbf{Approach:} We'll determine the angular frequencies, calculate lap times, find when collision occurs, and solve for the delay $T$.

\textbf{Step 1: Determine angular frequencies}

The angular frequency $\omega$ is the coefficient of $t$ in the trigonometric arguments.

For the first plane:
\begin{itemize}
    \item Argument: $t$
    \item Angular frequency: $\omega_1 = 1$
\end{itemize}

For the second plane:
\begin{itemize}
    \item Argument: $(2t - \alpha)$
    \item Angular frequency: $\omega_2 = 2$
\end{itemize}

Since $\omega_2 = 2\omega_1$, the second plane travels twice as fast as the first.

\textbf{Step 2: Calculate periods}

The period (time to complete one lap) is $P = \frac{2\pi}{\omega}$.

For the first plane:
\[ P_1 = 20 \text{ seconds (given)} \]

For the second plane (traveling twice as fast):
\[ P_2 = \frac{P_1}{2} = \frac{20}{2} = 10 \text{ seconds} \]

\textbf{Step 3: Time of collision}

Both planes complete 1.5 laps when they collide.

For the first plane (starting at $t=0$):
\[ t_{\text{collision}} = 1.5 \times P_1 = 1.5 \times 20 = 30 \text{ seconds} \]

For the second plane:
\[ \text{Flight time}_2 = 1.5 \times P_2 = 1.5 \times 10 = 15 \text{ seconds} \]

\textbf{Step 4: Solve for $T$}

The second plane takes off $T$ seconds after the first. Therefore:
\[ t_{\text{collision}} = T + \text{Flight time}_2 \]

Substituting:
\begin{align*}
    30 &= T + 15 \\
    T &= 30 - 15 \\
    T &= 15
\end{align*}

\textbf{Answer:} \boxed{T = 15 \text{ seconds}}

\textbf{Verification:} First plane at $t=30$: completes $\frac{30}{20} = 1.5$ laps \checkmark \\
Second plane takes off at $t=15$, flies for 15 seconds, completing $\frac{15}{10} = 1.5$ laps \checkmark
\end{solution}

\newpage

% Problem from samples/48.tex
\begin{problem}
A particle is moving vertically in a resistive medium under the influence of gravity. The resistive force is proportional to the velocity of the particle.

The initial speed of the particle is NOT zero.

Which of the following statements about the motion of the particle is always true?

\begin{enumerate}[label=\textbf{\Alph*.}]
    \item If the particle is initially moving downwards, then its speed will increase.
    \item If the particle is initially moving downwards, then its speed will decrease.
    \item If the particle is initially moving upwards, then its speed will eventually approach a terminal speed.
    \item If the particle is initially moving upwards, then its speed will not eventually approach a terminal speed.
\end{enumerate}
\end{problem}

\begin{solution}
\textbf{Approach:} Analyze the motion in two cases: downward and upward initial motion.

\textbf{Setup:}
\begin{itemize}
    \item Mass: $m$, Velocity: $v$, Gravity: $g$
    \item Resistive force: $R = kv$ (proportional to velocity)
    \item Terminal velocity (when resistance balances gravity): $v_T = \frac{mg}{k}$
\end{itemize}

\textbf{Case 1: Particle initially moving downwards}

Let downward be positive. Newton's Second Law:
\[ ma = mg - kv \]
\[ a = g - \frac{k}{m}v \]

\begin{center}
\begin{tikzpicture}
    \draw[fill=black] (0,0) circle (0.1);
    \draw[->, thick, bookpurple] (0,0) -- (0,-1.5) node[right] {$mg$};
    \draw[->, thick, bookpurple] (0,0) -- (0,1.0) node[right] {$kv$};
    \node at (2,0) {Motion: Down $\downarrow$};
\end{tikzpicture}
\end{center}

Analysis:
\begin{itemize}
    \item If $v_0 < v_T$: then $mg > kv_0$, so $a > 0$. Particle speeds up toward $v_T$.
    \item If $v_0 > v_T$: then $mg < kv_0$, so $a < 0$. Particle slows down toward $v_T$.
\end{itemize}

Since we don't know whether $v_0 < v_T$ or $v_0 > v_T$, neither statement \textbf{A} nor \textbf{B} is always true.

\textbf{Case 2: Particle initially moving upwards}

Let upward be positive. Both gravity and resistance oppose motion:
\[ ma = -mg - kv \]
\[ a = -\left(g + \frac{k}{m}v\right) < 0 \]

\begin{center}
\begin{tikzpicture}
    \draw[fill=black] (0,0) circle (0.1);
    \draw[->, thick, bookpurple] (0,0) -- (0,-1.5) node[right] {$mg$};
    \draw[->, thick, bookpurple] (0,0) -- (0,-0.8) node[left] {$kv$};
    \draw[->, dashed, bookred] (0.5, 0) -- (0.5, 1) node[right] {$v$};
    \node at (2,0) {Motion: Up $\uparrow$};
\end{tikzpicture}
\end{center}

Analysis:
\begin{itemize}
    \item Acceleration is always negative, so the particle decelerates.
    \item The particle will momentarily come to rest ($v=0$) at maximum height.
    \item Once at rest, gravity causes it to fall downward.
    \item Once falling, it enters Case 1 (downward motion).
    \item As shown in Case 1, a falling particle approaches terminal velocity $v_T$.
\end{itemize}

Therefore, regardless of initial upward speed, the particle will eventually approach terminal velocity.

\textbf{Conclusion:} Statement \textbf{C} is always true.

\textbf{Answer:} \boxed{\text{C}}

\textbf{Key insight:} The critical difference is that upward motion must eventually reverse (becoming downward), at which point terminal velocity is approached. Downward motion behavior depends on initial speed relative to terminal velocity.
\end{solution}
