% Part 1: Basic Problems - Detailed Solutions
% Problems: 39, 05, 30, 16, 48

% Problem from samples/39.tex
\begin{problem}
A particle is moving along a straight line. Initially its displacement is at $x = 1$, its velocity is $v = 2$ and its acceleration is $a = 4$.

\vspace{0.5cm}

\noindent Which equation could describe the motion of the particle?

\begin{enumerate}[label=\Alph*.]
    \item $v = 2\sin(x-1) + 2$
    \item $v = 2 + 4\log_e x$
    \item $v^2 = 4(x^2 - 2)$
    \item $v^2 = x^2 + 2x + 4$
\end{enumerate}
\end{problem}

\begin{solution}
\textbf{Approach:} We need to test each option using the given initial conditions: $x = 1$, $v = 2$, and $a = 4$. Since the options provide velocity as a function of displacement $v = f(x)$, we use the acceleration formula:
\[
a = v\frac{dv}{dx}
\]

\textbf{Testing Option A:} $v = 2\sin(x-1) + 2$

First, check velocity at $x=1$:
\[ v = 2\sin(1-1) + 2 = 2\sin(0) + 2 = 0 + 2 = 2 \quad \checkmark \]

Now check acceleration. Find $\frac{dv}{dx}$:
\[ \frac{dv}{dx} = 2\cos(x-1) \]

At $x=1$:
\[ \frac{dv}{dx} = 2\cos(0) = 2 \]

Calculate acceleration using $a = v\frac{dv}{dx}$:
\[ a = (2)(2) = 4 \quad \checkmark \]

Both conditions are satisfied!

\vspace{0.5cm}

\textit{Verification of other options for completeness:}

\textbf{Option B:} $v = 2 + 4\log_e x$
\begin{itemize}
    \item At $x=1$: $v = 2 + 4(0) = 2$ \quad \checkmark
    \item $\frac{dv}{dx} = \frac{4}{x}$. At $x=1$: $\frac{dv}{dx} = 4$
    \item $a = v\frac{dv}{dx} = 2 \cdot 4 = 8$ \quad \texttimes \quad (Should be 4, not 8)
\end{itemize}

\textbf{Option C:} $v^2 = 4(x^2 - 2)$
\begin{itemize}
    \item At $x=1$: $v^2 = 4(1 - 2) = -4$
    \item Since $v^2$ cannot be negative, this is physically impossible. \quad \texttimes
\end{itemize}

\textbf{Option D:} $v^2 = x^2 + 2x + 4$
\begin{itemize}
    \item At $x=1$: $v^2 = 1 + 2 + 4 = 7$, so $v = \pm\sqrt{7} \neq 2$ \quad \texttimes
\end{itemize}

\textbf{Answer:} \boxed{\text{A}}
\end{solution}

\begin{takeaways}
\item \textbf{Acceleration Formula:} For velocity as function of displacement, use $a = v\frac{dv}{dx}$ to test motion equations
\item \textbf{Initial Condition Testing:} Always verify given initial values ($x$, $v$, $a$) in candidate equations before calculating derivatives
\item \textbf{Multiple Choice Strategy:} Test easier conditions (like initial velocity) before computing acceleration to eliminate options efficiently
\item \textbf{Sign Convention:} Remember acceleration and velocity can have different signs (e.g., $v=2 > 0$ and $a=4 > 0$)
\end{takeaways}

\newpage

% Problem from samples/05.tex
\begin{problem}
The acceleration of a particle is given by $\ddot{x} = 32x(x^2 + 3)$, where $x$ is the displacement of the particle from a fixed-point $O$ after $t$ seconds, in metres. Initially the particle is at $O$ and has a velocity of $12 \, \text{m s}^{-1}$ in the negative direction.

\begin{enumerate}
    \item[(i)] Show that the velocity of the particle is given by $v = -4(x^2 + 3)$.
    \item[(ii)] Find the time taken for the particle to travel $3$ metres from the origin.
\end{enumerate}
\end{problem}

\begin{solution}
\textbf{(i)} Use $\ddot{x} = \frac{d}{dx}\left(\frac{1}{2}v^2\right)$ with $\ddot{x} = 32x^3 + 96x$:
\[ \frac{1}{2}v^2 = \int (32x^3 + 96x) \, dx = 8x^4 + 48x^2 + C \]

At $t=0$: $x=0$, $v = -12$ gives $144 = 2C$, so $C = 72$. Thus $v^2 = 16x^4 + 96x^2 + 144 = 16(x^2 + 3)^2$. Taking negative root (particle moves in negative direction): $\boxed{v = -4(x^2 + 3)}$

\textbf{(ii)} From $\frac{dx}{dt} = -4(x^2 + 3)$, we have $dt = \frac{-dx}{4(x^2 + 3)}$. Integrating from $x = 0$ to $x = -3$:
\[ t = -\frac{1}{4} \int_{0}^{-3} \frac{dx}{x^2 + 3} = -\frac{1}{4\sqrt{3}} \left[ \tan^{-1}\left(\frac{x}{\sqrt{3}}\right) \right]_{0}^{-3} = -\frac{1}{4\sqrt{3}} \left( -\frac{\pi}{3} \right) = \frac{\pi\sqrt{3}}{36} \text{ seconds} \]
\end{solution}

\begin{takeaways}
\item \textbf{Energy-Based Integration:} For $\ddot{x} = f(x)$, use identity $\ddot{x} = \frac{d}{dx}\left(\frac{1}{2}v^2\right)$ to integrate with respect to displacement
\item \textbf{Factoring Perfect Squares:} Recognize $16x^4 + 96x^2 + 144 = 16(x^2+3)^2$ to simplify velocity expressions
\item \textbf{Direction from Sign:} Negative velocity coefficient indicates motion in negative direction; sign of $v$ remains constant if expression maintains sign
\item \textbf{Arctangent Integrals:} For $\int \frac{1}{x^2+a^2}dx$, use standard form $\frac{1}{a}\tan^{-1}\left(\frac{x}{a}\right)$ with $a = \sqrt{3}$ here
\end{takeaways}

\newpage

% Problem from samples/30.tex
\begin{problem}
A particle is projected from the origin with initial velocity $u$ to pass through a point $(a,b)$. Prove that there are two possible trajectories if:
$$ (u^2 - gb)^2 > g^2(a^2 + b^2) $$
Assume no air resistance.
\end{problem}

\begin{solution}
From $x = ut\cos\theta$ and $y = ut\sin\theta - \frac{1}{2}gt^2$, eliminate $t = \frac{x}{u\cos\theta}$ to get trajectory:
\[ y = x\tan\theta - \frac{gx^2}{2u^2}(1 + \tan^2\theta) \]

At point $(a,b)$: $b = a\tan\theta - \frac{ga^2}{2u^2}(1 + \tan^2\theta)$. Let $T = \tan\theta$ and multiply by $2u^2$:
\[ ga^2 T^2 - 2u^2a T + (ga^2 + 2u^2b) = 0 \]

For two distinct trajectories, discriminant $\Delta > 0$:
\[ \Delta = 4u^4a^2 - 4ga^2(ga^2 + 2u^2b) = 4a^2(u^4 - g^2a^2 - 2u^2gb) > 0 \]

Dividing by $4a^2$ and completing the square (add/subtract $g^2b^2$):
\[ u^4 - 2u^2gb - g^2a^2 > 0 \implies (u^2-gb)^2 - g^2b^2 - g^2a^2 > 0 \implies (u^2-gb)^2 > g^2(a^2+b^2) \qed \]
\end{solution}

\begin{takeaways}
\item \textbf{Trajectory Derivation:} Eliminate time from parametric equations $x = ut\cos\theta$, $y = ut\sin\theta - \frac{1}{2}gt^2$ to get trajectory
\item \textbf{Quadratic in $\tan\theta$:} Substituting target point creates quadratic equation; discriminant determines number of solutions
\item \textbf{Completing the Square:} Transform discriminant condition from $u^4 - 2u^2gb - g^2a^2 > 0$ by adding/subtracting $g^2b^2$
\item \textbf{Physical Meaning:} Two trajectories exist when initial energy $(u^2 - gb)^2$ exceeds geometric constraint $g^2(a^2+b^2)$
\end{takeaways}

\newpage

% Problem from samples/16.tex
\begin{problem}
Two model airplanes race around a circular course, with the second airplane taking off $T$ seconds after the first plane. Their position vectors are:
\[
\vect{r}_1(t) = \sin t \vect{i} + \cos t \vect{j} + \sin t \vect{k}
\]
and
\[
\vect{r}_2(t) = \sin(2t - \alpha) \vect{i} + \cos(2t - \alpha) \vect{j} + \sin(2t - \alpha) \vect{k}
\]
where time is measured in seconds from when the first airplane took off. They collide when they have both completed one and a half laps. Find $T$ given the first plane takes 20 seconds to complete one lap.
\end{problem}

\begin{solution}
\textbf{Approach:} We'll determine the angular frequencies, calculate lap times, find when collision occurs, and solve for the delay $T$.

\textbf{Step 1: Determine angular frequencies}

The angular frequency $\omega$ is the coefficient of $t$ in the trigonometric arguments.

For the first plane:
\begin{itemize}
    \item Argument: $t$
    \item Angular frequency: $\omega_1 = 1$
\end{itemize}

For the second plane:
\begin{itemize}
    \item Argument: $(2t - \alpha)$
    \item Angular frequency: $\omega_2 = 2$
\end{itemize}

Since $\omega_2 = 2\omega_1$, the second plane travels twice as fast as the first.

\textbf{Step 2: Calculate periods}

The period (time to complete one lap) is $P = \frac{2\pi}{\omega}$.

For the first plane:
\[ P_1 = 20 \text{ seconds (given)} \]

For the second plane (traveling twice as fast):
\[ P_2 = \frac{P_1}{2} = \frac{20}{2} = 10 \text{ seconds} \]

\textbf{Step 3: Time of collision}

Both planes complete 1.5 laps when they collide.

For the first plane (starting at $t=0$):
\[ t_{\text{collision}} = 1.5 \times P_1 = 1.5 \times 20 = 30 \text{ seconds} \]

For the second plane:
\[ \text{Flight time}_2 = 1.5 \times P_2 = 1.5 \times 10 = 15 \text{ seconds} \]

\textbf{Step 4: Solve for $T$}

The second plane takes off $T$ seconds after the first. Therefore:
\[ t_{\text{collision}} = T + \text{Flight time}_2 \]

Substituting:
\begin{align*}
    30 &= T + 15 \\
    T &= 30 - 15 \\
    T &= 15
\end{align*}

\textbf{Answer:} \boxed{T = 15 \text{ seconds}}

\textbf{Verification:} First plane at $t=30$: completes $\frac{30}{20} = 1.5$ laps \checkmark \\
Second plane takes off at $t=15$, flies for 15 seconds, completing $\frac{15}{10} = 1.5$ laps \checkmark
\end{solution}

\begin{takeaways}
\item \textbf{Angular Frequency:} From argument coefficient in $\sin(\omega t)$ or $\cos(\omega t)$, identify angular frequency $\omega$ directly
\item \textbf{Relative Speed:} If $\omega_2 = 2\omega_1$, second object travels twice as fast (completes laps in half the time)
\item \textbf{Period Formula:} Relationship $P = \frac{2\pi}{\omega}$ connects period to angular frequency for circular/periodic motion
\item \textbf{Time Accounting:} For delayed starts, collision time = start delay + flight time of delayed object
\end{takeaways}

\newpage

% Problem from samples/48.tex
\begin{problem}
A particle is moving vertically in a resistive medium under the influence of gravity. The resistive force is proportional to the velocity of the particle.

The initial speed of the particle is NOT zero.

Which of the following statements about the motion of the particle is always true?

\begin{enumerate}[label=\textbf{\Alph*.}]
    \item If the particle is initially moving downwards, then its speed will increase.
    \item If the particle is initially moving downwards, then its speed will decrease.
    \item If the particle is initially moving upwards, then its speed will eventually approach a terminal speed.
    \item If the particle is initially moving upwards, then its speed will not eventually approach a terminal speed.
\end{enumerate}
\end{problem}

\begin{solution}
With resistive force $R = kv$ and terminal velocity $v_T = \frac{mg}{k}$ (when $mg = kv$):

\textbf{Case 1: Downward motion} ($a = g - \frac{k}{m}v$)
\begin{itemize}
    \item If $v_0 < v_T$: $a > 0$, particle speeds up toward $v_T$
    \item If $v_0 > v_T$: $a < 0$, particle slows down toward $v_T$
\end{itemize}
Since initial speed relative to $v_T$ is unknown, statements A and B are not always true.

\textbf{Case 2: Upward motion} ($a = -g - \frac{k}{m}v < 0$ always)

Particle decelerates, reaches maximum height, reverses to downward motion, then approaches $v_T$ (Case 1). Therefore statement \textbf{C} is always true. $\boxed{\text{C}}$
\end{solution}

\begin{takeaways}
\item \textbf{Terminal Velocity:} Occurs when $mg = kv$, giving $v_T = \frac{mg}{k}$ (resistance balances gravity)
\item \textbf{Downward Motion Analysis:} If $v < v_T$, particle accelerates; if $v > v_T$, particle decelerates toward terminal velocity
\item \textbf{Upward Motion:} Both gravity and resistance oppose motion, so $a = -(g + \frac{k}{m}v) < 0$ always
\item \textbf{Motion Reversal:} Upward-moving particle must stop, reverse, then fall and approach terminal velocity
\item \textbf{"Always True" Questions:} Test all cases systematically; reject options that fail in any valid scenario
\end{takeaways}
