% Part 1: Medium Problems - Detailed Solutions
% Problems: 03, 09, 06, 11, 28

% Problem from samples/03.tex
\begin{problem}
A particle of mass $1 \text{ kg}$ is projected from the origin with speed $40 \text{ m s}^{-1}$ at an angle $30^\circ$ to the horizontal plane.

\begin{enumerate}[label=(\roman*)]
    \item Use the information above to show that the initial velocity of the particle is $\v{v}(0) = \begin{pmatrix} 20\sqrt{3} \\ 20 \end{pmatrix}$.
\end{enumerate}

The forces acting on the particle are gravity and air resistance. The air resistance is proportional to the velocity vector with a constant of proportionality $4$. Let the acceleration due to gravity be $10 \text{ m s}^{-2}$.

The position vector of the particle, at time $t$ seconds after the particle is projected, is $\v{r}(t)$ and the velocity vector is $\v{v}(t)$.

\begin{enumerate}[label=(\roman*), resume]
    \item Show that $\v{v}(t) = \begin{pmatrix} 20\sqrt{3}e^{-4t} \\[6pt] \dfrac{45}{2}e^{-4t} - \dfrac{5}{2} \end{pmatrix}$.
    
    \item Show that $\v{r}(t) = \begin{pmatrix} 5\sqrt{3}\left(1 - e^{-4t}\right) \\[6pt] \dfrac{45}{8}\left(1 - e^{-4t}\right) - \dfrac{5}{2}t \end{pmatrix}$.
    
    \item The graphs $y = 1 - e^{-4x}$ and $y = \frac{4x}{9}$ are given in the diagram. Using the diagram, find the horizontal range of the particle, giving your answer rounded to one decimal place. (Note: The intersection occurs at $x_0 \approx 2.25$.)
\end{enumerate}
\end{problem}

\begin{solution}
\textbf{(i) Initial velocity vector}

Given: speed $V = 40 \text{ m s}^{-1}$, angle $\theta = 30^\circ$

The initial velocity components are:
\begin{align*}
    \v{v}(0) &= \begin{pmatrix} 40 \cos 30^\circ \\ 40 \sin 30^\circ \end{pmatrix} \\
    &= \begin{pmatrix} 40 \times \frac{\sqrt{3}}{2} \\ 40 \times \frac{1}{2} \end{pmatrix} \\
    &= \begin{pmatrix} 20\sqrt{3} \\ 20 \end{pmatrix} \quad \text{(shown)}
\end{align*}

\vspace{0.5cm}

\textbf{(ii) Velocity vector $\v{v}(t)$}

\textbf{Setup:} Using Newton's Second Law with $m=1$, $g=10$, and resistance force $4\v{v}$:
\[
m\v{a} = m\v{g} - 4\v{v} \implies \dot{\v{v}} = \begin{pmatrix} 0 \\ -10 \end{pmatrix} - 4\v{v}
\]

Separating into components:
\[
\ddot{x} = -4\dot{x} \quad \text{and} \quad \ddot{y} = -10 - 4\dot{y}
\]

\textbf{Horizontal component ($\dot{x}$):}

The differential equation is:
\[ \frac{d\dot{x}}{dt} = -4\dot{x} \]

This is a first-order linear ODE with solution:
\[ \dot{x} = A e^{-4t} \]

Applying initial condition $\dot{x}(0) = 20\sqrt{3}$:
\[ 20\sqrt{3} = A e^{0} = A \]

Therefore:
\[ \dot{x} = 20\sqrt{3}e^{-4t} \]

\textbf{Vertical component ($\dot{y}$):}

The differential equation is:
\[ \frac{d\dot{y}}{dt} + 4\dot{y} = -10 \]

This is a first-order linear ODE. Using integrating factor $I(t) = e^{4t}$:
\begin{align*}
    \frac{d}{dt}(\dot{y} e^{4t}) &= -10 e^{4t} \\
    \dot{y} e^{4t} &= \int -10 e^{4t} dt = -\frac{10}{4}e^{4t} + C = -\frac{5}{2}e^{4t} + C \\
    \dot{y} &= -\frac{5}{2} + Ce^{-4t}
\end{align*}

Applying initial condition $\dot{y}(0) = 20$:
\begin{align*}
    20 &= -\frac{5}{2} + C \\
    C &= 20 + \frac{5}{2} = \frac{40 + 5}{2} = \frac{45}{2}
\end{align*}

Therefore:
\[ \dot{y} = \frac{45}{2}e^{-4t} - \frac{5}{2} \]

Combining both components:
\[ \boxed{\v{v}(t) = \begin{pmatrix} 20\sqrt{3}e^{-4t} \\[4pt] \frac{45}{2}e^{-4t} - \frac{5}{2} \end{pmatrix}} \quad \text{(shown)} \]

\vspace{0.5cm}

\textbf{(iii) Position vector $\v{r}(t)$}

We integrate the velocity components, starting from the origin at $t=0$.

\textbf{Horizontal position ($x$):}
\begin{align*}
    x(t) &= \int 20\sqrt{3}e^{-4t} dt = 20\sqrt{3} \cdot \frac{e^{-4t}}{-4} + C_1 = -5\sqrt{3}e^{-4t} + C_1
\end{align*}

Applying $x(0) = 0$:
\[ 0 = -5\sqrt{3} + C_1 \implies C_1 = 5\sqrt{3} \]

Therefore:
\[ x(t) = 5\sqrt{3}(1 - e^{-4t}) \]

\textbf{Vertical position ($y$):}
\begin{align*}
    y(t) &= \int \left( \frac{45}{2}e^{-4t} - \frac{5}{2} \right) dt \\
    &= \frac{45}{2} \cdot \frac{e^{-4t}}{-4} - \frac{5}{2}t + C_2 \\
    &= -\frac{45}{8}e^{-4t} - \frac{5}{2}t + C_2
\end{align*}

Applying $y(0) = 0$:
\[ 0 = -\frac{45}{8} + C_2 \implies C_2 = \frac{45}{8} \]

Therefore:
\[ y(t) = \frac{45}{8}(1 - e^{-4t}) - \frac{5}{2}t \]

Combining both components:
\[ \boxed{\v{r}(t) = \begin{pmatrix} 5\sqrt{3}(1 - e^{-4t}) \\[4pt] \frac{45}{8}(1 - e^{-4t}) - \frac{5}{2}t \end{pmatrix}} \quad \text{(shown)} \]

\vspace{0.5cm}

\textbf{(iv) Horizontal range}

The particle hits the ground when $y(t) = 0$ (for $t > 0$):
\[ \frac{45}{8}(1 - e^{-4t}) - \frac{5}{2}t = 0 \]

Multiply by 8:
\begin{align*}
    45(1 - e^{-4t}) - 20t &= 0 \\
    45(1 - e^{-4t}) &= 20t \\
    1 - e^{-4t} &= \frac{20t}{45} = \frac{4t}{9}
\end{align*}

This equation $1 - e^{-4t} = \frac{4t}{9}$ represents the intersection of the two given graphs. From the diagram, this occurs at $t \approx 2.25$ seconds.

To find the horizontal range, substitute $t = 2.25$ into $x(t)$:

Using the relationship $1 - e^{-4t} = \frac{4t}{9}$ at $t = 2.25$:
\begin{align*}
    \text{Range} &= x(2.25) = 5\sqrt{3}\left( \frac{4(2.25)}{9} \right) \\
    &= 5\sqrt{3} \left( \frac{9}{9} \right) \\
    &= 5\sqrt{3} \\
    &\approx 8.660 \text{ metres}
\end{align*}

\textbf{Answer:} \boxed{8.7 \text{ metres}} (to 1 decimal place)

\textbf{Key insight:} The exponential decay of velocity due to air resistance means the projectile doesn't follow a simple parabolic path. The problem cleverly provides a graphical solution to the transcendental equation.
\end{solution}

\begin{takeaways}
\begin{itemize}
\item \textbf{Vector Force Equation:} Air resistance $4\v{v}$ opposes motion; Newton's law gives $\dot{\v{v}} = \v{g} - 4\v{v}$ for unit mass
\item \textbf{Separating Components:} Solve horizontal ($\ddot{x} = -4\dot{x}$) and vertical ($\ddot{y} = -10 - 4\dot{y}$) equations independently
\item \textbf{Integrating Factor Method:} For $\frac{d\dot{y}}{dt} + 4\dot{y} = -10$, multiply by $e^{4t}$ to get $\frac{d}{dt}(\dot{y}e^{4t}) = -10e^{4t}$
\item \textbf{Graphical Solutions:} Transcendental equations like $1-e^{-4t} = \frac{4t}{9}$ often require graphical or numerical methods
\item \textbf{Integration Constants:} Apply initial position conditions after integrating velocity to find position functions
\end{itemize}
\end{takeaways}

\newpage

% Problem from samples/09.tex
\begin{problem}
A particle of unit mass moves horizontally in a straight line. It experiences a resistive force proportional to $v^2$, where $v \text{ m s}^{-1}$ is the speed of the particle, so that the acceleration is given by $-kv^2$.

Initially the particle is at the origin and has a velocity of $40 \text{ m s}^{-1}$ to the right. After the particle has moved $15 \text{ m}$ to the right, its velocity is $10 \text{ m s}^{-1}$ (to the right).

\begin{enumerate}[(i)]
    \item Show that $v = 40e^{-kx}$.
    \item Show that $k = \frac{\ln 4}{15}$.
    \item At what time will the particle's velocity be $30 \text{ m s}^{-1}$ to the right?
\end{enumerate}
\end{problem}

\begin{solution}
\textbf{(i) Finding velocity as a function of displacement}

Given: $\ddot{x} = -kv^2$

To relate $v$ and $x$, we use the identity:
\[ \ddot{x} = v\frac{dv}{dx} \]

Substituting:
\begin{align*}
    v\frac{dv}{dx} &= -kv^2 \\
    \frac{dv}{dx} &= -kv \quad (\text{dividing by } v, \text{ assuming } v \neq 0)
\end{align*}

This is a separable first-order ODE:
\begin{align*}
    \frac{dv}{v} &= -k \, dx \\
    \int \frac{1}{v} \, dv &= \int -k \, dx \\
    \ln |v| &= -kx + C
\end{align*}

Apply initial conditions: when $x=0$, $v=40$:
\[ \ln 40 = -k(0) + C \implies C = \ln 40 \]

Substituting back:
\begin{align*}
    \ln v &= -kx + \ln 40 \\
    \ln v - \ln 40 &= -kx \\
    \ln\left(\frac{v}{40}\right) &= -kx \\
    \frac{v}{40} &= e^{-kx} \\
    v &= 40e^{-kx}
\end{align*}

\boxed{v = 40e^{-kx}} \quad \text{(shown)}

\vspace{0.5cm}

\textbf{(ii) Finding the constant $k$}

We're given that when $x=15$, $v=10$. Substitute into the equation from part (i):
\begin{align*}
    10 &= 40e^{-k(15)} \\
    \frac{10}{40} &= e^{-15k} \\
    \frac{1}{4} &= e^{-15k}
\end{align*}

Take natural logarithm of both sides:
\begin{align*}
    \ln\left(\frac{1}{4}\right) &= -15k \\
    -\ln 4 &= -15k \\
    k &= \frac{\ln 4}{15}
\end{align*}

\boxed{k = \frac{\ln 4}{15}} \quad \text{(shown)}

\vspace{0.5cm}

\textbf{(iii) Finding time when $v=30$}

To find time, we use $\ddot{x} = \frac{dv}{dt}$:
\[ \frac{dv}{dt} = -kv^2 \]

Separate variables:
\begin{align*}
    \frac{dv}{v^2} &= -k \, dt \\
    \int \frac{1}{v^2} \, dv &= \int -k \, dt
\end{align*}

Integrate from initial state ($t=0$, $v=40$) to the state when $v=30$ (at time $t=T$):
\begin{align*}
    \int_{40}^{30} v^{-2} \, dv &= \int_{0}^{T} -k \, dt \\
    \left[ -v^{-1} \right]_{40}^{30} &= [-kt]_{0}^{T} \\
    \left( -\frac{1}{30} \right) - \left( -\frac{1}{40} \right) &= -kT \\
    -\frac{1}{30} + \frac{1}{40} &= -kT
\end{align*}

Find common denominator for the left side:
\begin{align*}
    \frac{-4 + 3}{120} &= -kT \\
    -\frac{1}{120} &= -kT \\
    T &= \frac{1}{120k}
\end{align*}

Substitute $k = \frac{\ln 4}{15}$:
\begin{align*}
    T &= \frac{1}{120 \cdot \frac{\ln 4}{15}} \\
    &= \frac{15}{120 \ln 4} \\
    &= \frac{1}{8 \ln 4}
\end{align*}

\textbf{Answer:} \boxed{T = \frac{1}{8 \ln 4} \text{ seconds}}

Alternatively: $T = \frac{1}{8 \ln 4} = \frac{1}{16 \ln 2} \approx 0.090$ seconds

\textbf{Key insight:} Quadratic resistance leads to exponential decay in velocity with position, but requires integration of $1/v^2$ to find time. The particle asymptotically approaches rest but never actually stops in finite time.
\end{solution}

\begin{takeaways}
\begin{itemize}
\item \textbf{Quadratic Resistance Form:} Acceleration $\ddot{x} = -kv^2$ leads to velocity-displacement relationship through $v\frac{dv}{dx} = -kv^2$
\item \textbf{Exponential Velocity Decay:} Separating variables gives $\frac{dv}{v} = -kdx$, leading to $v = v_0 e^{-kx}$
\item \textbf{Finding Constants:} Use given conditions (here: $v=10$ at $x=15$, $v=40$ at $x=0$) to determine resistance coefficient $k$
\item \textbf{Time Integration:} Converting to time requires $\frac{dv}{v^2} = -kdt$, yielding $\int v^{-2}dv = [-v^{-1}]$ form
\item \textbf{Asymptotic Behavior:} With quadratic resistance, particle approaches rest asymptotically but never reaches $v=0$ in finite time
\end{itemize}
\end{takeaways}

\newpage

% Problem from samples/06.tex
\begin{problem}
A particle of mass $1$ kg is projected from the origin with a speed of $50 \text{ m s}^{-1}$, at an angle of $\theta$ below the horizontal into a resistive medium.

The position of the particle $t$ seconds after projection is $(x, y)$, and the velocity of the particle at that time is $\uvec{v} = \begin{pmatrix} \dot{x} \\ \dot{y} \end{pmatrix}$.

The resistive force, $\uvec{R}$, is proportional to the velocity of the particle, so that $\uvec{R} = -k\uvec{v}$, where $k$ is a positive constant.

Taking the acceleration due to gravity to be $10 \text{ m s}^{-2}$, and the upwards vertical direction to be positive, the acceleration of the particle at time $t$ is given by:
$$
\uvec{a} = \begin{pmatrix} -k\dot{x} \\ -k\dot{y} - 10 \end{pmatrix}. \quad (\text{Do NOT prove this.})
$$

Derive the Cartesian equation of the motion of the particle, given $\sin \theta = \frac{3}{5}$.
\end{problem}

\begin{solution}
\textbf{Step 1: Initial conditions}

Given: speed $V = 50 \text{ m s}^{-1}$, angle $\theta$ below horizontal, $\sin \theta = \frac{3}{5}$

From the Pythagorean identity with the 3-4-5 triangle:
$$ \cos \theta = \frac{4}{5} $$

Initial velocity components at $t=0$:
\begin{align*}
    \dot{x}(0) &= 50 \cos \theta = 50 \left(\frac{4}{5}\right) = 40 \\
    \dot{y}(0) &= -50 \sin \theta = -50 \left(\frac{3}{5}\right) = -30 \quad \text{(negative, downward)}
\end{align*}

Initial position: $x(0) = 0$, $y(0) = 0$

\textbf{Step 2: Horizontal motion}

Equation: $\ddot{x} = -k\dot{x}$

Integrating:
\[ \dot{x} = C_1 e^{-kt} \]

With $\dot{x}(0) = 40$:
\[ \dot{x} = 40e^{-kt} \]

Integrating again:
\[ x = \int 40e^{-kt} \, dt = -\frac{40}{k}e^{-kt} + C_2 \]

With $x(0) = 0$:
\[ 0 = -\frac{40}{k} + C_2 \implies C_2 = \frac{40}{k} \]

Therefore:
\begin{equation}
    x = \frac{40}{k}(1 - e^{-kt}) \label{eq:x_pos}
\end{equation}

From equation \eqref{eq:x_pos}, we can express $e^{-kt}$:
\begin{equation}
    e^{-kt} = 1 - \frac{kx}{40} \label{eq:exp_kt}
\end{equation}

\textbf{Step 3: Vertical motion}

Equation: $\ddot{y} = -k\dot{y} - 10$

Rearranging:
\[ \frac{d\dot{y}}{dt} = -(k\dot{y} + 10) \]

Separate variables:
\begin{align*}
    \frac{d\dot{y}}{k\dot{y} + 10} &= -dt \\
    \int \frac{1}{k\dot{y} + 10} \, d\dot{y} &= \int -1 \, dt \\
    \frac{1}{k} \ln|k\dot{y} + 10| &= -t + C_3
\end{align*}

With $\dot{y}(0) = -30$:
\[ C_3 = \frac{1}{k} \ln|k(-30) + 10| = \frac{1}{k} \ln|10 - 30k| \]

Therefore:
\begin{align*}
    \ln\left|\frac{k\dot{y} + 10}{10 - 30k}\right| &= -kt \\
    k\dot{y} + 10 &= (10 - 30k)e^{-kt} \\
    \dot{y} &= \frac{10 - 30k}{k}e^{-kt} - \frac{10}{k} \\
    \dot{y} &= \left(\frac{10}{k} - 30\right)e^{-kt} - \frac{10}{k}
\end{align*}

Integrating to find $y$:
\begin{align*}
    y &= \int \left[ \left(\frac{10}{k} - 30\right)e^{-kt} - \frac{10}{k} \right] dt \\
    &= \left(\frac{10}{k} - 30\right) \cdot \frac{e^{-kt}}{-k} - \frac{10t}{k} + C_4 \\
    &= -\frac{1}{k}\left(\frac{10}{k} - 30\right)e^{-kt} - \frac{10t}{k} + C_4
\end{align*}

With $y(0) = 0$:
\[ C_4 = \frac{1}{k}\left(\frac{10}{k} - 30\right) \]

Therefore:
\begin{equation}
    y = \frac{1}{k}\left(\frac{10}{k} - 30\right)(1 - e^{-kt}) - \frac{10t}{k} \label{eq:y_pos}
\end{equation}

\textbf{Step 4: Derive Cartesian equation}

From equation \eqref{eq:x_pos}: $1 - e^{-kt} = \frac{kx}{40}$

Also, from $e^{-kt} = 1 - \frac{kx}{40}$, taking logarithm:
\[ -kt = \ln\left(1 - \frac{kx}{40}\right) \implies t = -\frac{1}{k}\ln\left(1 - \frac{kx}{40}\right) \]

Substitute into equation \eqref{eq:y_pos}:
\begin{align*}
    y &= \frac{1}{k}\left(\frac{10}{k} - 30\right) \cdot \frac{kx}{40} + \frac{10}{k^2}\ln\left(1 - \frac{kx}{40}\right) \\
    &= \frac{x}{40}\left(\frac{10}{k} - 30\right) + \frac{10}{k^2}\ln\left(1 - \frac{kx}{40}\right) \\
    &= \frac{x}{4k} - \frac{3x}{4} + \frac{10}{k^2}\ln\left(1 - \frac{kx}{40}\right)
\end{align*}

\textbf{Final Answer:}
$$ \boxed{y = \frac{x}{4}\left(\frac{1}{k} - 3\right) + \frac{10}{k^2}\ln\left(1 - \frac{kx}{40}\right)} $$

\textbf{Key insight:} Linear resistance in a vector field requires solving coupled differential equations. The logarithmic term in the trajectory equation is characteristic of exponential decay in velocity components.
\end{solution}

\begin{takeaways}
\begin{itemize}
\item \textbf{Initial Velocity Components:} For angle $\theta$ below horizontal, $\dot{x}(0) = V\cos\theta$ (positive), $\dot{y}(0) = -V\sin\theta$ (negative)
\item \textbf{Exponential Motion Solution:} Linear resistance $\ddot{x} = -k\dot{x}$ yields $\dot{x} = Ae^{-kt}$ and $x = \frac{A}{k}(1-e^{-kt})$
\item \textbf{Non-homogeneous ODE:} For $\ddot{y} = -k\dot{y} - 10$, separate variables $\frac{d\dot{y}}{k\dot{y}+10} = -dt$ to integrate
\item \textbf{Eliminating Time:} Express $e^{-kt}$ from one equation, then substitute into the other to eliminate $t$
\item \textbf{Logarithmic Trajectories:} Linear drag creates logarithmic terms in Cartesian trajectory equations
\end{itemize}
\end{takeaways}

\newpage

% Problem from samples/11.tex
\begin{problem}
Two particles, $A$ and $B$, each have mass $1 \text{ kg}$ and are in a medium that exerts a resistance to motion equal to $kv$, where $k > 0$ and $v$ is the velocity of any particle. Both particles maintain vertical trajectories.

The acceleration due to gravity is $g \text{ m s}^{-2}$, where $g > 0$.

The two particles are simultaneously projected towards each other with the same speed, $v_0 \text{ m s}^{-1}$, where $0 < v_0 < \frac{g}{k}$.

The particle $A$ is initially $d$ metres directly above particle $B$, where $d < \frac{2v_0}{k}$.

\textbf{Find the time taken for the particles to meet.}
\end{problem}

\begin{solution}
\textbf{Setup and coordinate system}

Let origin be at particle $B$'s initial position. Upward direction is positive.

\textbf{Forces and equation of motion:}

For mass $m=1$, Newton's Second Law gives:
\begin{equation}
    \ddot{x} = -g - k\dot{x} \label{eq:motion_gen}
\end{equation}

This applies to both particles.

\textbf{Particle B (projected upwards):}

Initial conditions: $x_B(0) = 0$, $\dot{x}_B(0) = v_0$

Solve equation \eqref{eq:motion_gen} for velocity:
\begin{align*}
    \frac{d\dot{x}}{dt} &= -(g + k\dot{x}) \\
    \frac{d\dot{x}}{g + k\dot{x}} &= -dt \\
    \frac{1}{k} \ln(g + k\dot{x}) &= -t + C_1
\end{align*}

With $\dot{x}(0) = v_0$:
\[ C_1 = \frac{1}{k} \ln(g + kv_0) \]

Therefore:
\begin{align*}
    \ln\left(\frac{g + k\dot{x}}{g + kv_0}\right) &= -kt \\
    \dot{x}_B(t) &= \frac{1}{k} \left[ (g + kv_0)e^{-kt} - g \right]
\end{align*}

Integrate for position:
\begin{align*}
    x_B(t) &= \int \left( \frac{g+kv_0}{k}e^{-kt} - \frac{g}{k} \right) dt \\
    &= -\frac{g+kv_0}{k^2}e^{-kt} - \frac{gt}{k} + C_2
\end{align*}

With $x_B(0) = 0$:
\[ C_2 = \frac{g+kv_0}{k^2} \]

Therefore:
\begin{equation}
    x_B(t) = \frac{g+kv_0}{k^2}(1 - e^{-kt}) - \frac{gt}{k} \label{eq:pos_B}
\end{equation}

\textbf{Particle A (projected downwards):}

Initial conditions: $x_A(0) = d$, $\dot{x}_A(0) = -v_0$

Following the same procedure with $\dot{x}(0) = -v_0$:
\[ C_3 = \frac{1}{k} \ln(g - kv_0) \]

Note: $g - kv_0 > 0$ since $v_0 < \frac{g}{k}$.

Therefore:
\[ \dot{x}_A(t) = \frac{1}{k} \left[ (g - kv_0)e^{-kt} - g \right] \]

Integrating:
\begin{align*}
    x_A(t) &= -\frac{g-kv_0}{k^2}e^{-kt} - \frac{gt}{k} + C_4
\end{align*}

With $x_A(0) = d$:
\[ C_4 = d + \frac{g-kv_0}{k^2} \]

Therefore:
\begin{equation}
    x_A(t) = d + \frac{g-kv_0}{k^2}(1 - e^{-kt}) - \frac{gt}{k} \label{eq:pos_A}
\end{equation}

\textbf{Finding meeting time:}

Particles meet when $x_A(t) = x_B(t)$. Equating \eqref{eq:pos_A} and \eqref{eq:pos_B}:
\begin{align*}
    d + \frac{g-kv_0}{k^2}(1 - e^{-kt}) - \frac{gt}{k} &= \frac{g+kv_0}{k^2}(1 - e^{-kt}) - \frac{gt}{k}
\end{align*}

The $-\frac{gt}{k}$ terms cancel:
\begin{align*}
    d + \frac{g-kv_0}{k^2}(1 - e^{-kt}) &= \frac{g+kv_0}{k^2}(1 - e^{-kt}) \\
    d &= \left[ \frac{g+kv_0 - (g-kv_0)}{k^2} \right] (1 - e^{-kt}) \\
    d &= \frac{2kv_0}{k^2} (1 - e^{-kt}) \\
    d &= \frac{2v_0}{k} (1 - e^{-kt})
\end{align*}

Solve for $t$:
\begin{align*}
    \frac{kd}{2v_0} &= 1 - e^{-kt} \\
    e^{-kt} &= 1 - \frac{kd}{2v_0} = \frac{2v_0 - kd}{2v_0} \\
    -kt &= \ln\left( \frac{2v_0 - kd}{2v_0} \right) \\
    t &= -\frac{1}{k} \ln\left( \frac{2v_0 - kd}{2v_0} \right)
\end{align*}

Using logarithm property $-\ln(a/b) = \ln(b/a)$:

\textbf{Answer:} \boxed{t = \frac{1}{k} \ln\left( \frac{2v_0}{2v_0 - kd} \right) \text{ seconds}}

\textbf{Verification:} The condition $d < \frac{2v_0}{k}$ ensures $2v_0 - kd > 0$, so the argument of the logarithm is positive and greater than 1, giving $t > 0$.

\textbf{Key insight:} Even though both particles experience the same resistance law, their different initial velocities lead to different position functions. The symmetry in the problem allows the $gt/k$ terms to cancel, simplifying the solution.
\end{solution}

\begin{takeaways}
\begin{itemize}
\item \textbf{Identical Force Laws:} Both particles satisfy $\ddot{x} = -g - k\dot{x}$, but initial conditions differ
\item \textbf{First-Order Linear ODE:} Equation $\frac{d\dot{x}}{dt} = -(g+k\dot{x})$ solved by separating variables: $\frac{d\dot{x}}{g+k\dot{x}} = -dt$
\item \textbf{Logarithmic Integration:} Yields $\frac{1}{k}\ln(g+k\dot{x}) = -t + C$, leading to $\dot{x}(t) = \frac{1}{k}[(g+kv_0)e^{-kt} - g]$
\item \textbf{Cancellation Symmetry:} When finding meeting point, identical terms (like $-\frac{gt}{k}$) cancel, simplifying algebra
\item \textbf{Constraint Interpretation:} Condition $v_0 < \frac{g}{k}$ ensures $g-kv_0 > 0$; condition $d < \frac{2v_0}{k}$ ensures positive time
\end{itemize}
\end{takeaways}

\newpage

% Problem from samples/28.tex
\begin{problem}
A particle of unit mass moves in a straight line against a resistance numerically equal to $v+v^3$, where $v$ is its velocity. Initially the particle is at the origin and is travelling with velocity $Q$, where $Q > 0$.

\begin{enumerate}[label=(\alph*)]
    \item Show that the velocity is related to the displacement by the formula:
    \[ x = \tan^{-1}\left(\frac{Q-v}{1+Qv}\right) \]
    \item Show that the elapsed time when the particle is travelling with velocity $v$ is given by:
    \[ t = \frac{1}{2} \ln \frac{Q^2(1+v^2)}{v^2(1+Q^2)} \]
    \item Find $v^2$ as a function of $t$.
    \item Find the limiting value of $v$ and $x$ as $t \to \infty$.
\end{enumerate}
\end{problem}

\begin{solution}
\textbf{Setup:}

Given: mass $m=1$, resistance $R = v + v^3 = v(1+v^2)$, initial conditions: $x(0)=0$, $v(0)=Q$

Equation of motion:
\[ \ddot{x} = -(v+v^3) = -v(1+v^2) \]

\textbf{(a) Velocity and displacement}

Use $\ddot{x} = v\frac{dv}{dx}$:
\begin{align*}
    v \frac{dv}{dx} &= -v(1+v^2) \\
    \frac{dv}{dx} &= -(1+v^2) \quad \text{(dividing by } v \neq 0\text{)} \\
    \frac{dv}{1+v^2} &= -dx
\end{align*}

Integrate both sides:
\begin{align*}
    \int \frac{1}{1+v^2} \, dv &= \int -1 \, dx \\
    \tan^{-1} v &= -x + C
\end{align*}

Apply initial conditions ($x=0$, $v=Q$):
\[ \tan^{-1} Q = C \]

Therefore:
\[ x = \tan^{-1} Q - \tan^{-1} v \]

Using the inverse tangent subtraction identity:
\[ \tan^{-1} A - \tan^{-1} B = \tan^{-1} \left( \frac{A-B}{1+AB} \right) \]

\boxed{x = \tan^{-1} \left( \frac{Q-v}{1+Qv} \right)} \quad \text{(shown)}

\vspace{0.5cm}

\textbf{(b) Velocity and time}

Use $\ddot{x} = \frac{dv}{dt}$:
\begin{align*}
    \frac{dv}{dt} &= -v(1+v^2) \\
    \frac{dv}{v(1+v^2)} &= -dt
\end{align*}

Use partial fractions: $\frac{1}{v(1+v^2)} = \frac{A}{v} + \frac{Bv+C}{1+v^2}$

Solving: $A=1$, $B=-1$, $C=0$

Therefore:
\[ \frac{1}{v(1+v^2)} = \frac{1}{v} - \frac{v}{1+v^2} \]

Integrate:
\begin{align*}
    \int \left( \frac{1}{v} - \frac{v}{1+v^2} \right) dv &= \int -1 \, dt \\
    \ln v - \frac{1}{2} \ln(1+v^2) &= -t + C_2 \\
    \frac{1}{2} \ln \left( \frac{v^2}{1+v^2} \right) &= -t + C_2
\end{align*}

Apply initial conditions ($t=0$, $v=Q$):
\[ C_2 = \frac{1}{2} \ln \left( \frac{Q^2}{1+Q^2} \right) \]

Therefore:
\begin{align*}
    t &= C_2 - \frac{1}{2} \ln \left( \frac{v^2}{1+v^2} \right) \\
    &= \frac{1}{2} \ln \left( \frac{Q^2}{1+Q^2} \cdot \frac{1+v^2}{v^2} \right)
\end{align*}

\boxed{t = \frac{1}{2} \ln \frac{Q^2(1+v^2)}{v^2(1+Q^2)}} \quad \text{(shown)}

\vspace{0.5cm}

\textbf{(c) $v^2$ as a function of $t$}

From part (b):
\begin{align*}
    2t &= \ln \left( \frac{Q^2(1+v^2)}{v^2(1+Q^2)} \right) \\
    e^{2t} &= \frac{Q^2}{1+Q^2} \cdot \frac{1+v^2}{v^2} \\
    e^{2t} &= \frac{Q^2}{1+Q^2} \left( \frac{1}{v^2} + 1 \right)
\end{align*}

Let $K = \frac{Q^2}{1+Q^2}$:
\begin{align*}
    e^{2t} &= K \left( \frac{1}{v^2} + 1 \right) \\
    \frac{e^{2t}}{K} - 1 &= \frac{1}{v^2} \\
    \frac{1}{v^2} &= \frac{e^{2t} - K}{K} \\
    v^2 &= \frac{K}{e^{2t} - K}
\end{align*}

Substitute back $K = \frac{Q^2}{1+Q^2}$:
\begin{align*}
    v^2 &= \frac{\frac{Q^2}{1+Q^2}}{e^{2t} - \frac{Q^2}{1+Q^2}}
\end{align*}

Multiply numerator and denominator by $(1+Q^2)$:

\boxed{v^2 = \frac{Q^2}{(1+Q^2)e^{2t} - Q^2}}

\vspace{0.5cm}

\textbf{(d) Limiting values as $t \to \infty$}

\textbf{Limiting velocity:}
\begin{align*}
    \lim_{t \to \infty} v^2 &= \lim_{t \to \infty} \frac{Q^2}{(1+Q^2)e^{2t} - Q^2} \\
    &= \frac{Q^2}{\infty} = 0
\end{align*}

Therefore: \boxed{\lim_{t \to \infty} v = 0}

\textbf{Limiting displacement:}

From part (a), as $v \to 0$:
\begin{align*}
    \lim_{t \to \infty} x &= \lim_{v \to 0} \tan^{-1}\left(\frac{Q-v}{1+Qv}\right) \\
    &= \tan^{-1}\left(\frac{Q-0}{1+0}\right) \\
    &= \tan^{-1}(Q)
\end{align*}

\boxed{\lim_{t \to \infty} x = \tan^{-1}(Q)}

\textbf{Key insight:} The cubic resistance term $v^3$ dominates at high velocities, causing rapid deceleration. As $t \to \infty$, the particle approaches rest at a finite distance $\tan^{-1}(Q)$ from the origin, demonstrating that strong resistance can prevent infinite displacement.
\end{solution}

\begin{takeaways}
\begin{itemize}
\item \textbf{Combined Resistance:} Force $R = v + v^3 = v(1+v^2)$ combines linear and cubic terms
\item \textbf{Velocity-Displacement:} Using $v\frac{dv}{dx} = -v(1+v^2)$ simplifies to $\frac{dv}{1+v^2} = -dx$
\item \textbf{Arctangent Relationship:} Integration $\int \frac{1}{1+v^2}dv = \tan^{-1}v$ leads to $x = \tan^{-1}Q - \tan^{-1}v$
\item \textbf{Partial Fractions for Time:} $\frac{1}{v(1+v^2)} = \frac{1}{v} - \frac{v}{1+v^2}$ enables integration for time relation
\item \textbf{Exponential from Logarithm:} From $\frac{1}{2}\ln\frac{v^2}{1+v^2} = -t + C$, solve for $v^2$ as function of $t$
\item \textbf{Finite Limiting Distance:} Strong cubic resistance causes particle to stop at $x = \tan^{-1}(Q)$ as $t \to \infty$
\end{itemize}
\end{takeaways}
