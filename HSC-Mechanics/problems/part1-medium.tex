% Part 1: Medium Problems - Detailed Solutions
% Problems: 03, 09, 06, 11, 28

% Problem from samples/03.tex
\begin{problem}
A particle of mass $1 \text{ kg}$ is projected from the origin with speed $40 \text{ m s}^{-1}$ at an angle $30^\circ$ to the horizontal plane.

\begin{enumerate}[label=(\roman*)]
    \item Use the information above to show that the initial velocity of the particle is $\v{v}(0) = \begin{pmatrix} 20\sqrt{3} \\ 20 \end{pmatrix}$.
\end{enumerate}

The forces acting on the particle are gravity and air resistance. The air resistance is proportional to the velocity vector with a constant of proportionality $4$. Let the acceleration due to gravity be $10 \text{ m s}^{-2}$.

The position vector of the particle, at time $t$ seconds after the particle is projected, is $\v{r}(t)$ and the velocity vector is $\v{v}(t)$.

\begin{enumerate}[label=(\roman*), resume]
    \item Show that $\v{v}(t) = \begin{pmatrix} 20\sqrt{3}e^{-4t} \\[6pt] \dfrac{45}{2}e^{-4t} - \dfrac{5}{2} \end{pmatrix}$.
    
    \item Show that $\v{r}(t) = \begin{pmatrix} 5\sqrt{3}\left(1 - e^{-4t}\right) \\[6pt] \dfrac{45}{8}\left(1 - e^{-4t}\right) - \dfrac{5}{2}t \end{pmatrix}$.
    
    \item The graphs $y = 1 - e^{-4x}$ and $y = \frac{4x}{9}$ are given in the diagram. Using the diagram, find the horizontal range of the particle, giving your answer rounded to one decimal place. (Note: The intersection occurs at $x_0 \approx 2.25$.)
\end{enumerate}
\end{problem}

\begin{solution}
\textbf{(i)} Given $V = 40 \text{ m s}^{-1}$, $\theta = 30^\circ$:
\[
\v{v}(0) = \begin{pmatrix} 40 \cos 30^\circ \\ 40 \sin 30^\circ \end{pmatrix} = \begin{pmatrix} 40 \cdot \frac{\sqrt{3}}{2} \\ 40 \cdot \frac{1}{2} \end{pmatrix} = \begin{pmatrix} 20\sqrt{3} \\ 20 \end{pmatrix} \quad \text{(shown)}
\]

\textbf{(ii)} Newton's law with $m=1$, $g=10$, resistance $4\v{v}$ gives $\dot{\v{v}} = \begin{pmatrix} 0 \\ -10 \end{pmatrix} - 4\v{v}$, so $\ddot{x} = -4\dot{x}$ and $\ddot{y} = -10 - 4\dot{y}$.

\textbf{Horizontal:} $\frac{d\dot{x}}{dt} = -4\dot{x} \implies \dot{x} = Ae^{-4t}$. With $\dot{x}(0) = 20\sqrt{3}$: $\dot{x} = 20\sqrt{3}e^{-4t}$.

\textbf{Vertical:} $\frac{d\dot{y}}{dt} + 4\dot{y} = -10$. Using integrating factor $e^{4t}$: $\frac{d}{dt}(\dot{y}e^{4t}) = -10e^{4t} \implies \dot{y}e^{4t} = -\frac{5}{2}e^{4t} + C \implies \dot{y} = -\frac{5}{2} + Ce^{-4t}$.
With $\dot{y}(0) = 20$: $C = \frac{45}{2}$, so $\dot{y} = \frac{45}{2}e^{-4t} - \frac{5}{2}$.

\[
\boxed{\v{v}(t) = \begin{pmatrix} 20\sqrt{3}e^{-4t} \\[4pt] \frac{45}{2}e^{-4t} - \frac{5}{2} \end{pmatrix}} \quad \text{(shown)}
\]

\textbf{(iii)} Integrating velocity from $t=0$:

\textbf{Horizontal:} $x(t) = \int 20\sqrt{3}e^{-4t} dt = -5\sqrt{3}e^{-4t} + C_1$. With $x(0)=0$: $C_1 = 5\sqrt{3}$, so $x(t) = 5\sqrt{3}(1-e^{-4t})$.

\textbf{Vertical:} $y(t) = \int \left(\frac{45}{2}e^{-4t} - \frac{5}{2}\right) dt = -\frac{45}{8}e^{-4t} - \frac{5}{2}t + C_2$. With $y(0)=0$: $C_2 = \frac{45}{8}$, so $y(t) = \frac{45}{8}(1-e^{-4t}) - \frac{5}{2}t$.

\[
\boxed{\v{r}(t) = \begin{pmatrix} 5\sqrt{3}(1 - e^{-4t}) \\[4pt] \frac{45}{8}(1 - e^{-4t}) - \frac{5}{2}t \end{pmatrix}} \quad \text{(shown)}
\]

\textbf{(iv)} Ground impact when $y(t)=0$: $\frac{45}{8}(1-e^{-4t}) - \frac{5}{2}t = 0 \implies 45(1-e^{-4t}) = 20t \implies 1-e^{-4t} = \frac{4t}{9}$.

From diagram, intersection at $t \approx 2.25$. Range: $x(2.25) = 5\sqrt{3}\left(\frac{4(2.25)}{9}\right) = 5\sqrt{3}(1) = 5\sqrt{3} \approx 8.660$ metres.

\textbf{Answer:} \boxed{8.7 \text{ metres}}
\end{solution}

\begin{takeaways}
\begin{itemize}
\item \textbf{Vector Force Equation:} Air resistance $4\v{v}$ opposes motion; Newton's law gives $\dot{\v{v}} = \v{g} - 4\v{v}$ for unit mass
\item \textbf{Separating Components:} Solve horizontal ($\ddot{x} = -4\dot{x}$) and vertical ($\ddot{y} = -10 - 4\dot{y}$) equations independently
\item \textbf{Integrating Factor Method:} For $\frac{d\dot{y}}{dt} + 4\dot{y} = -10$, multiply by $e^{4t}$ to get $\frac{d}{dt}(\dot{y}e^{4t}) = -10e^{4t}$
\item \textbf{Graphical Solutions:} Transcendental equations like $1-e^{-4t} = \frac{4t}{9}$ often require graphical or numerical methods
\item \textbf{Integration Constants:} Apply initial position conditions after integrating velocity to find position functions
\end{itemize}
\end{takeaways}

\newpage

% Problem from samples/09.tex
\begin{problem}
A particle of unit mass moves horizontally in a straight line. It experiences a resistive force proportional to $v^2$, where $v \text{ m s}^{-1}$ is the speed of the particle, so that the acceleration is given by $-kv^2$.

Initially the particle is at the origin and has a velocity of $40 \text{ m s}^{-1}$ to the right. After the particle has moved $15 \text{ m}$ to the right, its velocity is $10 \text{ m s}^{-1}$ (to the right).

\begin{enumerate}[(i)]
    \item Show that $v = 40e^{-kx}$.
    \item Show that $k = \frac{\ln 4}{15}$.
    \item At what time will the particle's velocity be $30 \text{ m s}^{-1}$ to the right?
\end{enumerate}
\end{problem}

\begin{solution}
\textbf{(i)} Given $\ddot{x} = -kv^2$, use $\ddot{x} = v\frac{dv}{dx}$:
\[
v\frac{dv}{dx} = -kv^2 \implies \frac{dv}{dx} = -kv \implies \frac{dv}{v} = -kdx \implies \ln|v| = -kx + C
\]
With $v(0)=40$: $C = \ln 40$, so $\ln v = -kx + \ln 40 \implies \ln\left(\frac{v}{40}\right) = -kx \implies v = 40e^{-kx}$.

\boxed{v = 40e^{-kx}} \quad \text{(shown)}

\textbf{(ii)} With $v=10$ at $x=15$:
\[
10 = 40e^{-15k} \implies \frac{1}{4} = e^{-15k} \implies \ln\left(\frac{1}{4}\right) = -15k \implies k = \frac{\ln 4}{15}
\]

\boxed{k = \frac{\ln 4}{15}} \quad \text{(shown)}

\textbf{(iii)} Using $\frac{dv}{dt} = -kv^2$, separate variables: $\frac{dv}{v^2} = -kdt$.

Integrate from $(t=0, v=40)$ to $(t=T, v=30)$:
\[
\int_{40}^{30} v^{-2}dv = \int_0^T -kdt \implies \left[-v^{-1}\right]_{40}^{30} = -kT \implies -\frac{1}{30} + \frac{1}{40} = -kT \implies \frac{-1}{120} = -kT
\]

Thus $T = \frac{1}{120k} = \frac{1}{120 \cdot \frac{\ln 4}{15}} = \frac{15}{120\ln 4} = \frac{1}{8\ln 4}$.

\textbf{Answer:} \boxed{T = \frac{1}{8 \ln 4} \text{ seconds}} $\approx 0.090$ seconds
\end{solution}

\begin{takeaways}
\begin{itemize}
\item \textbf{Quadratic Resistance Form:} Acceleration $\ddot{x} = -kv^2$ leads to velocity-displacement relationship through $v\frac{dv}{dx} = -kv^2$
\item \textbf{Exponential Velocity Decay:} Separating variables gives $\frac{dv}{v} = -kdx$, leading to $v = v_0 e^{-kx}$
\item \textbf{Finding Constants:} Use given conditions (here: $v=10$ at $x=15$, $v=40$ at $x=0$) to determine resistance coefficient $k$
\item \textbf{Time Integration:} Converting to time requires $\frac{dv}{v^2} = -kdt$, yielding $\int v^{-2}dv = [-v^{-1}]$ form
\item \textbf{Asymptotic Behavior:} With quadratic resistance, particle approaches rest asymptotically but never reaches $v=0$ in finite time
\end{itemize}
\end{takeaways}

\newpage

% Problem from samples/06.tex
\begin{problem}
A particle of mass $1$ kg is projected from the origin with a speed of $50 \text{ m s}^{-1}$, at an angle of $\theta$ below the horizontal into a resistive medium.

The position of the particle $t$ seconds after projection is $(x, y)$, and the velocity of the particle at that time is $\uvec{v} = \begin{pmatrix} \dot{x} \\ \dot{y} \end{pmatrix}$.

The resistive force, $\uvec{R}$, is proportional to the velocity of the particle, so that $\uvec{R} = -k\uvec{v}$, where $k$ is a positive constant.

Taking the acceleration due to gravity to be $10 \text{ m s}^{-2}$, and the upwards vertical direction to be positive, the acceleration of the particle at time $t$ is given by:
$$
\uvec{a} = \begin{pmatrix} -k\dot{x} \\ -k\dot{y} - 10 \end{pmatrix}. \quad (\text{Do NOT prove this.})
$$

Derive the Cartesian equation of the motion of the particle, given $\sin \theta = \frac{3}{5}$.
\end{problem}

\begin{solution}
Given $V = 50 \text{ m s}^{-1}$, $\sin\theta = \frac{3}{5}$ below horizontal: $\cos\theta = \frac{4}{5}$, so $\dot{x}(0) = 40$, $\dot{y}(0) = -30$, $x(0)=y(0)=0$.

\textbf{Horizontal motion:} $\ddot{x} = -k\dot{x} \implies \dot{x} = 40e^{-kt} \implies x = \frac{40}{k}(1-e^{-kt})$, so $e^{-kt} = 1 - \frac{kx}{40}$.

\textbf{Vertical motion:} $\ddot{y} = -k\dot{y} - 10$. Separating: $\frac{d\dot{y}}{k\dot{y}+10} = -dt \implies \frac{1}{k}\ln|k\dot{y}+10| = -t + C_3$.

With $\dot{y}(0)=-30$: $C_3 = \frac{1}{k}\ln|10-30k|$, so $\ln\left|\frac{k\dot{y}+10}{10-30k}\right| = -kt \implies \dot{y} = \left(\frac{10}{k}-30\right)e^{-kt} - \frac{10}{k}$.

Integrating: $y = -\frac{1}{k}\left(\frac{10}{k}-30\right)e^{-kt} - \frac{10t}{k} + C_4$. With $y(0)=0$: $C_4 = \frac{1}{k}\left(\frac{10}{k}-30\right)$, thus
\[
y = \frac{1}{k}\left(\frac{10}{k}-30\right)(1-e^{-kt}) - \frac{10t}{k}
\]

\textbf{Eliminate $t$:} From $e^{-kt} = 1-\frac{kx}{40}$: $t = -\frac{1}{k}\ln\left(1-\frac{kx}{40}\right)$. Substituting and using $1-e^{-kt} = \frac{kx}{40}$:
\[
y = \frac{1}{k}\left(\frac{10}{k}-30\right) \cdot \frac{kx}{40} + \frac{10}{k^2}\ln\left(1-\frac{kx}{40}\right) = \frac{x}{4k} - \frac{3x}{4} + \frac{10}{k^2}\ln\left(1-\frac{kx}{40}\right)
\]

\textbf{Answer:} $\boxed{y = \frac{x}{4}\left(\frac{1}{k} - 3\right) + \frac{10}{k^2}\ln\left(1 - \frac{kx}{40}\right)}$
\end{solution}

\begin{takeaways}
\begin{itemize}
\item \textbf{Initial Velocity Components:} For angle $\theta$ below horizontal, $\dot{x}(0) = V\cos\theta$ (positive), $\dot{y}(0) = -V\sin\theta$ (negative)
\item \textbf{Exponential Motion Solution:} Linear resistance $\ddot{x} = -k\dot{x}$ yields $\dot{x} = Ae^{-kt}$ and $x = \frac{A}{k}(1-e^{-kt})$
\item \textbf{Non-homogeneous ODE:} For $\ddot{y} = -k\dot{y} - 10$, separate variables $\frac{d\dot{y}}{k\dot{y}+10} = -dt$ to integrate
\item \textbf{Eliminating Time:} Express $e^{-kt}$ from one equation, then substitute into the other to eliminate $t$
\item \textbf{Logarithmic Trajectories:} Linear drag creates logarithmic terms in Cartesian trajectory equations
\end{itemize}
\end{takeaways}

\newpage

% Problem from samples/11.tex
\begin{problem}
Two particles, $A$ and $B$, each have mass $1 \text{ kg}$ and are in a medium that exerts a resistance to motion equal to $kv$, where $k > 0$ and $v$ is the velocity of any particle. Both particles maintain vertical trajectories.

The acceleration due to gravity is $g \text{ m s}^{-2}$, where $g > 0$.

The two particles are simultaneously projected towards each other with the same speed, $v_0 \text{ m s}^{-1}$, where $0 < v_0 < \frac{g}{k}$.

The particle $A$ is initially $d$ metres directly above particle $B$, where $d < \frac{2v_0}{k}$.

\textbf{Find the time taken for the particles to meet.}
\end{problem}

\begin{solution}
Origin at $B$'s initial position, upward positive. Both particles satisfy $\ddot{x} = -g - k\dot{x}$ with $m=1$.

\textbf{Particle B:} $x_B(0)=0$, $\dot{x}_B(0)=v_0$. 

Separating: $\frac{d\dot{x}}{g+k\dot{x}} = -dt \implies \frac{1}{k}\ln(g+k\dot{x}) = -t + C_1$. With $\dot{x}(0)=v_0$: $C_1 = \frac{1}{k}\ln(g+kv_0)$, so $\dot{x}_B = \frac{1}{k}[(g+kv_0)e^{-kt} - g]$.

Integrating with $x_B(0)=0$: $x_B(t) = \frac{g+kv_0}{k^2}(1-e^{-kt}) - \frac{gt}{k}$.

\textbf{Particle A:} $x_A(0)=d$, $\dot{x}_A(0)=-v_0$. 

Similarly: $\dot{x}_A = \frac{1}{k}[(g-kv_0)e^{-kt} - g]$ and $x_A(t) = d + \frac{g-kv_0}{k^2}(1-e^{-kt}) - \frac{gt}{k}$.

\textbf{Meeting time:} Set $x_A(t) = x_B(t)$. The $-\frac{gt}{k}$ terms cancel:
\[
d + \frac{g-kv_0}{k^2}(1-e^{-kt}) = \frac{g+kv_0}{k^2}(1-e^{-kt}) \implies d = \frac{2kv_0}{k^2}(1-e^{-kt}) = \frac{2v_0}{k}(1-e^{-kt})
\]

Thus $e^{-kt} = 1 - \frac{kd}{2v_0} = \frac{2v_0-kd}{2v_0} \implies t = -\frac{1}{k}\ln\left(\frac{2v_0-kd}{2v_0}\right) = \frac{1}{k}\ln\left(\frac{2v_0}{2v_0-kd}\right)$.

\textbf{Answer:} $\boxed{t = \frac{1}{k}\ln\left(\frac{2v_0}{2v_0-kd}\right) \text{ seconds}}$
\end{solution}

\begin{takeaways}
\begin{itemize}
\item \textbf{Identical Force Laws:} Both particles satisfy $\ddot{x} = -g - k\dot{x}$, but initial conditions differ
\item \textbf{First-Order Linear ODE:} Equation $\frac{d\dot{x}}{dt} = -(g+k\dot{x})$ solved by separating variables: $\frac{d\dot{x}}{g+k\dot{x}} = -dt$
\item \textbf{Logarithmic Integration:} Yields $\frac{1}{k}\ln(g+k\dot{x}) = -t + C$, leading to $\dot{x}(t) = \frac{1}{k}[(g+kv_0)e^{-kt} - g]$
\item \textbf{Cancellation Symmetry:} When finding meeting point, identical terms (like $-\frac{gt}{k}$) cancel, simplifying algebra
\item \textbf{Constraint Interpretation:} Condition $v_0 < \frac{g}{k}$ ensures $g-kv_0 > 0$; condition $d < \frac{2v_0}{k}$ ensures positive time
\end{itemize}
\end{takeaways}

\newpage

% Problem from samples/28.tex
\begin{problem}
A particle of unit mass moves in a straight line against a resistance numerically equal to $v+v^3$, where $v$ is its velocity. Initially the particle is at the origin and is travelling with velocity $Q$, where $Q > 0$.

\begin{enumerate}[label=(\alph*)]
    \item Show that the velocity is related to the displacement by the formula:
    \[ x = \tan^{-1}\left(\frac{Q-v}{1+Qv}\right) \]
    \item Show that the elapsed time when the particle is travelling with velocity $v$ is given by:
    \[ t = \frac{1}{2} \ln \frac{Q^2(1+v^2)}{v^2(1+Q^2)} \]
    \item Find $v^2$ as a function of $t$.
    \item Find the limiting value of $v$ and $x$ as $t \to \infty$.
\end{enumerate}
\end{problem}

\begin{solution}
Given $m=1$, resistance $R = v(1+v^2)$, ICs: $x(0)=0$, $v(0)=Q$. Equation of motion: $\ddot{x} = -v(1+v^2)$.

\textbf{(a)} Using $v\frac{dv}{dx} = -v(1+v^2) \implies \frac{dv}{1+v^2} = -dx$, integrate: $\tan^{-1}v = -x + C$. With $x(0)=0$, $v(0)=Q$: $C = \tan^{-1}Q$, so $x = \tan^{-1}Q - \tan^{-1}v$. Apply identity $\tan^{-1}A - \tan^{-1}B = \tan^{-1}\left(\frac{A-B}{1+AB}\right)$:

\boxed{x = \tan^{-1}\left(\frac{Q-v}{1+Qv}\right)} \quad \text{(shown)}

\textbf{(b)} Using $\frac{dv}{dt} = -v(1+v^2) \implies \frac{dv}{v(1+v^2)} = -dt$. Partial fractions: $\frac{1}{v(1+v^2)} = \frac{1}{v} - \frac{v}{1+v^2}$. Integrate: $\ln v - \frac{1}{2}\ln(1+v^2) = -t + C_2 \implies \frac{1}{2}\ln\left(\frac{v^2}{1+v^2}\right) = -t + C_2$. With $t=0$, $v=Q$: $C_2 = \frac{1}{2}\ln\left(\frac{Q^2}{1+Q^2}\right)$. Thus:

\boxed{t = \frac{1}{2}\ln\frac{Q^2(1+v^2)}{v^2(1+Q^2)}} \quad \text{(shown)}

\textbf{(c)} From part (b): $2t = \ln\left(\frac{Q^2(1+v^2)}{v^2(1+Q^2)}\right) \implies e^{2t} = \frac{Q^2}{1+Q^2}\left(\frac{1}{v^2}+1\right)$. Let $K=\frac{Q^2}{1+Q^2}$: $\frac{e^{2t}}{K}-1 = \frac{1}{v^2} \implies v^2 = \frac{K}{e^{2t}-K}$. Substituting $K$ and multiplying by $(1+Q^2)$:

\boxed{v^2 = \frac{Q^2}{(1+Q^2)e^{2t} - Q^2}}

\textbf{(d)} As $t \to \infty$: $\lim_{t\to\infty} v^2 = \lim_{t\to\infty}\frac{Q^2}{(1+Q^2)e^{2t}-Q^2} = 0$, so \boxed{\lim_{t\to\infty} v = 0}

From part (a), as $v \to 0$: $\lim_{t\to\infty} x = \tan^{-1}\left(\frac{Q-0}{1+0}\right) = \tan^{-1}(Q)$, so \boxed{\lim_{t\to\infty} x = \tan^{-1}(Q)}

The cubic resistance $v^3$ dominates at high velocities. The particle stops at finite distance $\tan^{-1}(Q)$ as $t \to \infty$.
\end{solution}

\begin{takeaways}
\begin{itemize}
\item \textbf{Combined Resistance:} Force $R = v + v^3 = v(1+v^2)$ combines linear and cubic terms
\item \textbf{Velocity-Displacement:} Using $v\frac{dv}{dx} = -v(1+v^2)$ simplifies to $\frac{dv}{1+v^2} = -dx$
\item \textbf{Arctangent Relationship:} Integration $\int \frac{1}{1+v^2}dv = \tan^{-1}v$ leads to $x = \tan^{-1}Q - \tan^{-1}v$
\item \textbf{Partial Fractions for Time:} $\frac{1}{v(1+v^2)} = \frac{1}{v} - \frac{v}{1+v^2}$ enables integration for time relation
\item \textbf{Exponential from Logarithm:} From $\frac{1}{2}\ln\frac{v^2}{1+v^2} = -t + C$, solve for $v^2$ as function of $t$
\item \textbf{Finite Limiting Distance:} Strong cubic resistance causes particle to stop at $x = \tan^{-1}(Q)$ as $t \to \infty$
\end{itemize}
\end{takeaways}
