% Part 1: Advanced Problems - Detailed Solutions
% Problems: 10, 21, 22, 13, 56

% Problem from samples/10.tex
\begin{problem}
A bar magnet is held vertically. An object that is repelled by the magnet is to be dropped from directly above the magnet and will maintain a vertical trajectory. Let $x$ be the distance of the object above the magnet.

\begin{center}
    \begin{tikzpicture}[scale=0.8]
        % Draw Magnet
        \draw[thick] (-0.5, -3) rectangle (0.5, -1);
        \draw (-0.5, -2) -- (0.5, -2);
        \node at (0, -1.5) {N};
        \node at (0, -2.5) {S};

        % Draw dashed lines for height x
        \draw[dashed] (-1.5, 2) -- (0, 2);
        \draw[dashed] (-1.5, -1) -- (0, -1);

        % Draw dimension line x
        \draw[<->] (-1.2, 2) -- (-1.2, -1) node[midway, left] {$x$};

        % Draw Object (Ball)
        \shade[ball color=gray!50] (0, 2) circle (0.4);

        % Draw Force Arrows
        \draw[-latex, thick, bookpurple] (0, 2.4) -- (0, 3.5) node[right] {$\dfrac{27g}{x^3}$};
        \draw[-latex, thick, bookpurple] (0, 1.6) -- (0, 0.8) node[right] {$g$};
    \end{tikzpicture}
\end{center}

The object is subject to acceleration due to gravity, $g$, and an acceleration due to the magnet $\frac{27g}{x^3}$, so that the total acceleration of the object is given by:
\[
    a = \frac{27g}{x^3} - g
\]
The object is released from rest at $x = 6$.

\begin{enumerate}
    \item[(i)] Show that $v^2 = g\left(\frac{51}{4} - 2x - \frac{27}{x^2}\right)$.
    \item[(ii)] Find where the object next comes to rest, giving your answer correct to 1 decimal place.
\end{enumerate}
\end{problem}

\begin{solution}
\textbf{(i)} Given $a = 27gx^{-3} - g$, use $\frac{d}{dx}\left(\frac{v^2}{2}\right) = a$. Integrate: $\frac{v^2}{2} = \int(27gx^{-3} - g)dx = -\frac{27g}{2x^2} - gx + C \implies v^2 = -\frac{27g}{x^2} - 2gx + K$. With $v=0$ at $x=6$: $0 = -\frac{27g}{36} - 12g + K = -\frac{3g}{4} - 12g + K \implies K = \frac{51g}{4}$. Thus:

\boxed{v^2 = g\left(\frac{51}{4} - 2x - \frac{27}{x^2}\right)} \quad \text{(shown)}

\textbf{(ii)} Set $v^2=0$: $\frac{51}{4} - 2x - \frac{27}{x^2} = 0$. Multiply by $4x^2$: $51x^2 - 8x^3 - 108 = 0 \implies 8x^3 - 51x^2 + 108 = 0$. Factor with root $x=6$: $(x-6)(8x^2-3x-18)=0$. Quadratic formula: $x = \frac{3 \pm \sqrt{9+576}}{16} = \frac{3 \pm \sqrt{585}}{16} \approx \frac{3 \pm 24.187}{16}$. Positive root: $x = \frac{27.187}{16} \approx 1.699$.

\textbf{Answer:} \boxed{x \approx 1.7 \text{ units}}

Inverse cube repulsion weak at $x=6$; gravity dominates initially. As object approaches magnet, repulsion strengthens until equilibrium at $x \approx 1.7$.
\end{solution}

\begin{takeaways}
\item \textbf{Inverse Cube Force:} Acceleration $a = \frac{27g}{x^3} - g$ combines repulsive magnetic force and gravity
\item \textbf{Energy Integration:} Use $\frac{d}{dx}\left(\frac{v^2}{2}\right) = a$ to integrate acceleration with respect to position
\item \textbf{Integration of Powers:} $\int x^{-3}dx = \frac{x^{-2}}{-2} = -\frac{1}{2x^2}$ is key for inverse cube terms
\item \textbf{Initial Conditions:} "Released from rest" means $v=0$ at starting position; use to find integration constant
\item \textbf{Solving Cubics:} Factor known root $(x-6)$ from cubic, then solve remaining quadratic using formula
\item \textbf{Physical Constraints:} Reject negative solutions ($x > 0$); select root representing next rest point
\end{takeaways}

\newpage

% Problem from samples/21.tex
\begin{problem}
An object of mass $1 \text{ kg}$ is projected vertically upwards with an initial velocity of $u \text{ m/s}$. It experiences air resistance of magnitude $kv^2$ newtons where $v$ is the velocity of the object, in m/s, and $k$ is a positive constant. The height of the object above its starting point is $x$ metres. The time since projection is $t$ seconds and acceleration due to gravity is $g \text{ m/s}^2$.

\begin{enumerate}[label=(\roman*)]
    \item Show that the time for the object to reach its maximum height is 
    \[
    \frac{1}{\sqrt{gk}}\arctan\left(u\sqrt{\frac{k}{g}}\right) \text{ seconds.}
    \]
    \item Find an expression for the maximum height reached by the object, in terms of $k$, $g$ and $u$.
\end{enumerate}
\end{problem}


\begin{solution}
Upward positive, $m=1$. Forces: gravity $-g$, air resistance $-kv^2$. Newton's law: $\ddot{x} = -(g + kv^2)$.

\begin{center}
\begin{tikzpicture}
    % Ground
    \draw[thick] (-1,0) -- (1,0);
    \node[below] at (0,0) {Ground ($x=0$)};
    
    % Particle
    \filldraw [black] (0,2) circle (2pt);
    \node[right] at (0.1,2) {$m=1$};
    
    % Velocity vector
    \draw[->, thick, blue] (0,2.2) -- (0,3) node[right] {$v$ (up)};
    
    % Force vectors
    \draw[->, thick, bookpurple] (0,1.8) -- (0,1) node[right] {$mg$};
    \draw[->, thick, bookpurple] (-0.2,1.8) -- (-0.2,0.8) node[left] {$kv^2$};
    
    % Axis
    \draw[->, dashed] (-2,0) -- (-2,3) node[left] {$x (+)$};
\end{tikzpicture}
\end{center}

\textbf{(i)} Using $\frac{dv}{dt} = -(g+kv^2) \implies \frac{dv}{g+kv^2} = -dt$. Integrate from $(t=0, v=u)$ to $(t=T, v=0)$: $\int_0^u \frac{dv}{g+kv^2} = T$. Factor: $T = \frac{1}{k}\int_0^u \frac{dv}{(\sqrt{g/k})^2 + v^2} = \frac{1}{k} \cdot \sqrt{\frac{k}{g}} \left[\arctan\left(v\sqrt{\frac{k}{g}}\right)\right]_0^u = \frac{1}{\sqrt{gk}}\arctan\left(u\sqrt{\frac{k}{g}}\right)$.

\boxed{T = \frac{1}{\sqrt{gk}} \arctan\left(u\sqrt{\frac{k}{g}}\right) \text{ seconds}} \quad \text{(shown)}

\textbf{(ii)} Using $v\frac{dv}{dx} = -(g+kv^2) \implies \frac{v\,dv}{g+kv^2} = -dx$. Integrate from $(x=0, v=u)$ to $(x=H, v=0)$: $\int_0^u \frac{v\,dv}{g+kv^2} = H$. Substitute $w = g+kv^2$, $dw = 2kv\,dv$: $H = \int_g^{g+ku^2} \frac{1}{2k}\frac{dw}{w} = \frac{1}{2k}\ln\left(\frac{g+ku^2}{g}\right) = \frac{1}{2k}\ln\left(1+\frac{ku^2}{g}\right)$.

\textbf{Answer:} \boxed{H = \frac{1}{2k} \ln\left(1 + \frac{k}{g}u^2\right) \text{ metres}}

For weak resistance ($ku^2 \ll g$), $\ln(1+x) \approx x$ gives $H \approx \frac{u^2}{2g}$ (no-resistance case).
\end{solution}

\begin{takeaways}
\item \textbf{Quadratic Air Resistance:} Upward motion has $\ddot{x} = -(g + kv^2) = -g(1 + \frac{v^2}{k/g})$ combining gravity and drag
\item \textbf{Arctangent Integral:} $\int \frac{1}{a^2+x^2}dx = \frac{1}{a}\arctan\left(\frac{x}{a}\right)$ with $a = \sqrt{g/k}$ gives time to max height
\item \textbf{Velocity-Displacement:} Use $v\frac{dv}{dx} = -(g+kv^2)$ to relate velocity and position
\item \textbf{Logarithmic Integration:} Substitution $w = g + kv^2$ transforms $\int \frac{v\,dv}{g+kv^2}$ to $\frac{1}{2k}\int \frac{dw}{w} = \frac{1}{2k}\ln w$
\item \textbf{Limiting Behavior:} For weak resistance ($ku^2 \ll g$), logarithm approximates to linear, recovering classical result
\item \textbf{Terminal Velocity:} Downward motion reaches equilibrium at $v_T = \sqrt{g/k}$ when drag balances gravity
\end{takeaways}

\newpage

% Problem from samples/22.tex
\begin{problem}
A particle is undergoing simple harmonic motion with period $\frac{\pi}{3}$. The central point of motion of the particle is at $x = \sqrt{3}$. When $t = 0$ the particle has its maximum displacement of $2\sqrt{3}$ from the central point of motion.

\vspace{0.5cm}
\noindent Find an equation for the displacement, $x$, of the particle in terms of $t$.
\end{problem}

\begin{solution}
SHM general form with shifted center: $x(t) = C + A\cos(nt + \alpha)$ where $C$ is center, $A$ is amplitude, $n$ is angular frequency, $\alpha$ is phase.

Given: $T = \frac{\pi}{3} \implies n = \frac{2\pi}{T} = 6$; center $C = \sqrt{3}$; amplitude $A = 2\sqrt{3}$ (max displacement from center); starts at maximum $\implies \alpha = 0$.

\textbf{Answer:} \boxed{x = \sqrt{3} + 2\sqrt{3}\cos(6t)} or \boxed{x = \sqrt{3}(1 + 2\cos(6t))}

Verify: $t=0 \implies x=3\sqrt{3}$ (max); $t=\frac{\pi}{12} \implies x=\sqrt{3}$ (center); $t=\frac{\pi}{6} \implies x=-\sqrt{3}$ (min); $t=\frac{\pi}{3} \implies x=3\sqrt{3}$ (period complete).
\end{solution}

\begin{takeaways}
\item \textbf{SHM General Form:} $x(t) = C + A\cos(nt + \alpha)$ where $C$ is center, $A$ is amplitude, $n$ is angular frequency
\item \textbf{Period to Frequency:} From period $T = \frac{\pi}{3}$, calculate $n = \frac{2\pi}{T} = 6$
\item \textbf{Maximum Displacement:} "Maximum displacement from center" is the amplitude ($A = 2\sqrt{3}$), not absolute position
\item \textbf{Phase Constant:} Starting at maximum means $\cos(\alpha) = 1$, so $\alpha = 0$ (or $2\pi k$)
\item \textbf{Verification Strategy:} Check multiple time values to confirm equation matches all features (extrema, center crossings, period)
\item \textbf{Alternative Form:} Can factor common terms: $x = \sqrt{3}(1 + 2\cos(6t))$
\end{takeaways}

\newpage

% Problem from samples/13.tex
\begin{problem}
The point $P$ is $4$ metres to the right of the origin $O$ on a straight line.

\vspace{0.5em}
\noindent A particle is released from rest at $P$ and moves along the straight line in simple harmonic motion about $O$, with period $8\pi$ seconds.

\vspace{0.5em}
\noindent After $2\pi$ seconds, another particle is released from rest at $P$ and also moves along this straight line in simple harmonic motion about $O$, with period $8\pi$ seconds.

\vspace{0.5em}
\noindent Find when and where the two particles first collide.
\end{problem}

\begin{solution}
SHM from rest at extremity: $x(t) = a\cos(nt)$ where $a$ is amplitude, $n$ is angular frequency.

Given: $a=4$, $T=8\pi \implies n = \frac{2\pi}{T} = \frac{1}{4}$.

Particle 1 (released $t=0$): $x_1(t) = 4\cos\left(\frac{t}{4}\right)$. Particle 2 (released $t=2\pi$): $x_2(t) = 4\cos\left(\frac{t-2\pi}{4}\right)$ for $t \geq 2\pi$.

Collision when $x_1(t)=x_2(t)$: $\cos\left(\frac{t}{4}\right) = \cos\left(\frac{t}{4}-\frac{\pi}{2}\right) = \sin\left(\frac{t}{4}\right) \implies \tan\left(\frac{t}{4}\right) = 1 \implies \frac{t}{4} = \frac{\pi}{4}, \frac{5\pi}{4}, \ldots \implies t = \pi, 5\pi, 9\pi, \ldots$

First valid solution (for $t \geq 2\pi$): $t=5\pi$. Position: $x = 4\cos\left(\frac{5\pi}{4}\right) = 4\left(-\frac{1}{\sqrt{2}}\right) = -2\sqrt{2}$.

\textbf{Answer:} \boxed{t = 5\pi \text{ seconds}}, \boxed{x = -2\sqrt{2} \text{ metres}}

Time delay $2\pi = T/4$ (quarter period). Particle 1 completes 5/8 period, Particle 2 completes 3/8 period when they meet.
\end{solution}

\begin{takeaways}
\item \textbf{SHM from Rest at Extreme:} Starting from rest at maximum displacement gives $x(t) = a\cos(nt)$ (no phase shift)
\item \textbf{Time-Shifted Motion:} For particle starting at $t = t_0$, replace $t$ with $(t-t_0)$ in the equation
\item \textbf{Period and Angular Frequency:} From $T = 8\pi$, get $n = \frac{2\pi}{T} = \frac{1}{4}$
\item \textbf{Collision Condition:} Set positions equal: $a\cos(nt_1) = a\cos(n(t_1-t_0))$
\item \textbf{Cosine Identity:} Use $\cos(\theta - \frac{\pi}{2}) = \sin(\theta)$ to simplify collision equation
\item \textbf{Periodic Solutions:} $\tan(\frac{t}{4}) = 1$ gives $t = \pi, 5\pi, 9\pi, \ldots$ (multiples of period difference)
\item \textbf{Validity Check:} Ensure collision time satisfies constraints (e.g., second particle must exist: $t \geq 2\pi$)
\end{takeaways}

\newpage

% Problem from samples/56.tex
\begin{problem}
A particle of mass $m$ is attracted towards the origin by a force of magnitude $\frac{\mu m}{x^2}$ for $x \neq 0$, where the distance from the origin is $x$ and $\mu$ is a positive constant.

\begin{enumerate}[label=\textbf{\roman*.}]
    \item Prove that $\displaystyle \frac{d}{dx}\left[\sqrt{bx - x^2} + \frac{b}{2}\cos^{-1}\left(\frac{2x-b}{b}\right)\right] = -\sqrt{\frac{x}{b-x}}$ for $x \ge 0$.
    \item If the particle starts at rest at a distance $b$ to the right of the origin, show that its velocity $v$ is given by $\displaystyle v^2 = 2\mu \left(\frac{b-x}{bx}\right)$.
    \item Find the time required for the particle to reach a point halfway towards the origin.
\end{enumerate}
\end{problem}

\begin{solution}
\begin{center}
\begin{tikzpicture}
    % Axis
    \draw[->, thick] (-1,0) -- (6,0) node[right] {$x$};
    \draw[thick] (0,-0.2) -- (0,0.2) node[above] {$O$};
    
    % Points
    \draw[thick] (5,-0.2) -- (5,0.2) node[above] {$x=b$ (Start, $v=0$)};
    \draw[thick] (2.5,-0.2) -- (2.5,0.2) node[above] {$x=b/2$};
    
    % Particle
    \filldraw [blue] (3.5,0) circle (3pt) node[above right, blue] {$m$};
    \draw[->, thick, bookpurple] (3.5,0.3) -- (2.5,0.3) node[midway, above] {$F$};
    
    % Label motion
    \node at (3.5, -0.5) {Motion towards Origin};
\end{tikzpicture}
\end{center}

\textbf{(i)} Let $y = \sqrt{bx - x^2} + \frac{b}{2}\cos^{-1}\left(\frac{2x-b}{b}\right)$. Differentiate: $\frac{dy}{dx} = \frac{b-2x}{2\sqrt{bx-x^2}}$ (first term) and with $u = \frac{2x-b}{b}$, $\frac{d}{dx}\left[\frac{b}{2}\cos^{-1}(u)\right] = -\frac{b}{2} \cdot \frac{2/b}{\sqrt{1-u^2}} = -\frac{1}{\sqrt{1-(2x-b)^2/b^2}} = -\frac{b}{2\sqrt{bx-x^2}}$ (second term). Combine: $\frac{dy}{dx} = \frac{b-2x-b}{2\sqrt{bx-x^2}} = \frac{-2x}{2\sqrt{x(b-x)}} = -\sqrt{\frac{x}{b-x}}$.

\boxed{\frac{d}{dx}\left[\sqrt{bx - x^2} + \frac{b}{2}\cos^{-1}\left(\frac{2x-b}{b}\right)\right] = -\sqrt{\frac{x}{b-x}}} \quad \text{(proven)}

\textbf{(ii)} Attractive force: $F = -\frac{\mu m}{x^2} \implies \ddot{x} = -\frac{\mu}{x^2}$. Use $\frac{d}{dx}(\frac{v^2}{2}) = \ddot{x}$: integrate $\frac{v^2}{2} = \int -\mu x^{-2}dx = \frac{\mu}{x} + C$. With $v=0$ at $x=b$: $C = -\frac{\mu}{b}$. Thus $\frac{v^2}{2} = \mu\left(\frac{1}{x}-\frac{1}{b}\right) = \mu\frac{b-x}{bx}$.

\boxed{v^2 = 2\mu \left(\frac{b-x}{bx}\right)} \quad \text{(shown)}

\textbf{(iii)} From (ii), $v = \frac{dx}{dt} = -\sqrt{2\mu\frac{b-x}{bx}}$ (negative for decreasing $x$). Rearrange: $dt = -\sqrt{\frac{b}{2\mu}}\sqrt{\frac{x}{b-x}}dx$. Integrate from $x=b$ to $x=b/2$: $T = \sqrt{\frac{b}{2\mu}}\int_{b/2}^b \sqrt{\frac{x}{b-x}}dx$. From part (i), the antiderivative is $-\left[\sqrt{bx-x^2} + \frac{b}{2}\cos^{-1}\left(\frac{2x-b}{b}\right)\right]$. Evaluate at $x=b/2$: $\sqrt{b^2/4} + \frac{b}{2}\cos^{-1}(0) = \frac{b}{2} + \frac{b\pi}{4}$; at $x=b$: $0$. Thus $T = \sqrt{\frac{b}{2\mu}}\left(\frac{b}{2} + \frac{b\pi}{4}\right) = \sqrt{\frac{b}{2\mu}} \cdot \frac{b(2+\pi)}{4}$.

\textbf{Answer:} \boxed{T = \frac{(2 + \pi)\sqrt{b^3}}{4\sqrt{2\mu}} \text{ seconds}}

Inverse square force weakens with distance; time $\propto \sqrt{b^3/\mu}$ shows nonlinear scaling.
\end{solution}

\begin{takeaways}
\item \textbf{Inverse Square Law:} Force $F = -\frac{\mu m}{x^2}$ attracts toward origin; acceleration $\ddot{x} = -\frac{\mu}{x^2}$
\item \textbf{Chain Rule for Derivatives:} Part (i) requires product rule, chain rule, and derivatives of $\cos^{-1}$: $\frac{d}{dx}\cos^{-1}(u) = \frac{-u'}{\sqrt{1-u^2}}$
\item \textbf{Energy-Based Integration:} Use $\frac{d}{dx}(\frac{v^2}{2}) = -\frac{\mu}{x^2}$ and integrate: $\frac{v^2}{2} = \frac{\mu}{x} + C$
\item \textbf{Provided Antiderivatives:} Part (i) proves the antiderivative of $-\sqrt{\frac{x}{b-x}}$, used directly in part (iii)
\item \textbf{Time Integration Setup:} From $v = \frac{dx}{dt}$, rearrange to $dt = \frac{dx}{v}$ and integrate over displacement range
\item \textbf{Evaluating Complex Integrals:} Use given antiderivative formula, carefully evaluate at endpoints with proper substitution
\item \textbf{Scaling Analysis:} Time $\propto \sqrt{b^3/\mu}$ shows nonlinear dependence on initial distance
\end{takeaways}
