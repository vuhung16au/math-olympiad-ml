\begin{problem}[Sum of First $n$ Odd Numbers]
Prove that $1 + 3 + 5 + \cdots + (2n-1) = n^2$ for all integers $n \ge 1$.
\end{problem}

\begin{hint}
The $(n+1)$-st odd number is $2(n+1)-1 = 2n+1$. Add this to the hypothesis $1+3+\cdots+(2n-1) = n^2$ and factor the result.
\end{hint}

\begin{solution}[Sketch]
Base case: $n=1$ gives $1 = 1^2$. For the inductive step, assume the sum of the first $k$ odd numbers equals $k^2$. Then the sum of the first $k+1$ odd numbers is $k^2 + (2k+1) = k^2 + 2k + 1 = (k+1)^2$.
\end{solution}

\begin{problem}[Divisibility of $4^n - 1$ by $3$]
Prove that $4^n - 1$ is divisible by $3$ for every integer $n \ge 1$.
\end{problem}

\begin{hint}
Note that $4 \equiv 1 \pmod{3}$, so $4^{k+1} - 1 = 4 \cdot 4^k - 1 = 4(4^k - 1) + 3$.
\end{hint}

\begin{solution}[Sketch]
Base case: $4^1 - 1 = 3$ is divisible by $3$. If $4^k - 1$ is divisible by $3$, then $4^{k+1} - 1 = 4 \cdot 4^k - 1 = 4(4^k - 1) + 3$, which is the sum of two multiples of $3$.
\end{solution}

\begin{problem}[Sum of First $n$ Squares]
Prove that 

$$\sum_{k=1}^{n} k^2 = \frac{n(n+1)(2n+1)}{6}$$ 
for all integers $n \ge 1$.

\end{problem}

\begin{hint}
Add $(k+1)^2$ to both sides of the hypothesis and show the result simplifies to $\frac{(k+1)(k+2)(2k+3)}{6}$.
\end{hint}

\begin{solution}[Sketch]
Base case: $1^2 = 1 = \frac{1 \cdot 2 \cdot 3}{6}$. Assume the formula holds for $n=k$. Then $\sum_{r=1}^{k+1} r^2 = \frac{k(k+1)(2k+1)}{6} + (k+1)^2 = \frac{k(k+1)(2k+1) + 6(k+1)^2}{6} = \frac{(k+1)(2k^2+7k+6)}{6} = \frac{(k+1)(k+2)(2k+3)}{6}$.
\end{solution}

\begin{problem}[Geometric Series]
Prove that 

$$\sum_{k=0}^{n} r^k = \frac{r^{n+1}-1}{r-1}$$  
for all integers $n \ge 0$ and $r \neq 1$.

\end{problem}

\begin{hint}
Multiply both sides of the hypothesis by $r$ and compare with the $(n+1)$ case to derive a telescoping relationship.
\end{hint}

\begin{solution}[Sketch]
Base case: $r^0 = 1 = \frac{r-1}{r-1}$. Assume the formula holds for $n=k$. Then $\sum_{j=0}^{k+1} r^j = \frac{r^{k+1}-1}{r-1} + r^{k+1} = \frac{r^{k+1}-1 + r^{k+1}(r-1)}{r-1} = \frac{r^{k+2}-1}{r-1}$.
\end{solution}

\begin{problem}[Bernoulli's Inequality]
Prove that $(1+x)^n \ge 1 + nx$ for all real numbers $x > -1$ and integers $n \ge 1$.
\end{problem}

\begin{hint}
Multiply the hypothesis $(1+x)^k \ge 1 + kx$ by $(1+x)$ and use the fact that $x^2 \ge 0$ and $kx^2 \ge 0$ when $k \ge 1$.
\end{hint}

\begin{solution}[Sketch]
Base case: $(1+x)^1 = 1+x$. Assume $(1+x)^k \ge 1+kx$ for some $k \ge 1$. Then $(1+x)^{k+1} = (1+x)(1+x)^k \ge (1+x)(1+kx) = 1 + kx + x + kx^2 = 1 + (k+1)x + kx^2 \ge 1 + (k+1)x$ since $kx^2 \ge 0$.
\end{solution}

\begin{problem}[De Moivre's Formula (Conjugate Form)]
Prove by induction that for all integers $n \ge 1$:
\[(\cos \theta - i \sin \theta)^n = \cos(n\theta) - i \sin(n\theta)\]
\end{problem}

\begin{hint}
Use the standard angle addition formulas: $\cos(A+B) = \cos A \cos B - \sin A \sin B$ and $\sin(A+B) = \sin A \cos B + \cos A \sin B$.
\end{hint}

\begin{solution}[Sketch]
Base case: $n=1$ gives $\cos \theta - i \sin \theta = \cos \theta - i \sin \theta$. Assume the formula holds for $n=k$. Then $(\cos \theta - i \sin \theta)^{k+1} = (\cos k\theta - i \sin k\theta)(\cos \theta - i \sin \theta) = \cos k\theta \cos \theta - \sin k\theta \sin \theta - i(\sin k\theta \cos \theta + \cos k\theta \sin \theta) = \cos(k+1)\theta - i \sin(k+1)\theta$.
\end{solution}

\begin{problem}[Power Inequality $2^n \ge n^2 - 2$]
Prove by induction that $2^n \ge n^2 - 2$ for all integers $n \ge 3$.
\end{problem}

\begin{hint}
Show that $2 \cdot k^2 - 2 \ge (k+1)^2 - 2$ for $k \ge 3$, which simplifies to $k^2 - 2k - 1 \ge 0$.
\end{hint}

\begin{solution}[Sketch]
Base case: $n=3$ gives $2^3 = 8 \ge 7 = 3^2 - 2$. Assume $2^k \ge k^2 - 2$ for some $k \ge 3$. Then $2^{k+1} = 2 \cdot 2^k \ge 2(k^2-2) = 2k^2 - 4$. We need $2k^2 - 4 \ge (k+1)^2 - 2 = k^2 + 2k - 1$, which simplifies to $k^2 - 2k - 3 \ge 0$ or $(k-3)(k+1) \ge 0$. This holds for all $k \ge 3$.
\end{solution}
