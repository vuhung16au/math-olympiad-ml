\begin{problem}[Telescoping Harmonic Sum]
Show that for every integer $n \ge 1$,
\[
\sum_{k=1}^{n} \frac{1}{k(k+1)} = \frac{n}{n+1}.
\]
\end{problem}

\begin{solution}
\textbf{Base case ($n=1$).} We have $\frac{1}{1\cdot 2} = \frac12$ and $\frac{1}{1+1} = \frac12$, so the statement is true for $n=1$.

\textbf{Inductive step.} Assume the statement holds for some $n=k$, i.e.
\[
\sum_{r=1}^{k} \frac{1}{r(r+1)} = \frac{k}{k+1}.
\]
Then
\begin{align*}
\sum_{r=1}^{k+1} \frac{1}{r(r+1)}
&= \underbrace{\sum_{r=1}^{k} \frac{1}{r(r+1)}}_{\frac{k}{k+1}} + \frac{1}{(k+1)(k+2)} \\
&= \frac{k}{k+1} + \frac{1}{(k+1)(k+2)} \\
&= \frac{k(k+2)+1}{(k+1)(k+2)} = \frac{k^2+2k+1}{(k+1)(k+2)} \\
&= \frac{(k+1)^2}{(k+1)(k+2)} = \frac{k+1}{k+2}.
\end{align*}
Therefore the identity is true for $n=k+1$, so by induction it holds for all $n \ge 1$.
\end{solution}

\begin{takeaways}
Partial fractions $\frac{1}{k(k+1)} = \frac{1}{k}-\frac{1}{k+1}$ produce telescoping sums, \\
so the algebra in the inductive step stays simple and transparent.
\end{takeaways}

\begin{problem}[Divisibility of $n^3+2n$]
Prove that $n^3 + 2n$ is divisible by $12$ for every even integer $n$.
\end{problem}

\begin{solution}
Let $P(n)$ be the statement for the even integer $n$. Write $n=2m$ for some integer $m$.

\textbf{Base case ($n=2$).} $2^3 + 2\cdot2 = 8 + 4 = 12$, which is divisible by $12$.

\textbf{Inductive step.} Assume $P(2m)$ holds, so $2m$ even implies $ (2m)^3 + 2(2m) = 8m^3 + 4m$ is a multiple of $12$. Consider the next even integer $2(m+1)$:
\begin{align*}
(2m+2)^3 + 2(2m+2)
&= 8(m+1)^3 + 4(m+1) \\
&= 8(m^3 + 3m^2 + 3m + 1) + 4m + 4 \\
&= (8m^3 + 4m) + 24m^2 + 24m + 12.
\end{align*}
By the hypothesis, $8m^3 + 4m$ is divisible by $12$, and each remaining term has factor $12$. Hence the whole expression is divisible by $12$, completing the induction.
\end{solution}

\begin{takeaways}
Factoring $n$ as $2m$ keeps arithmetic simple and highlights the repeated factor of $12$.
\end{takeaways}

\begin{problem}[Nine divides $7^n + 2^n$]
Show that $7^n + 2^n$ is divisible by $9$ for every positive odd integer $n$.
\end{problem}

\begin{solution}
\textbf{Base case ($n=1$).} $7^1 + 2^1 = 9$, divisible by $9$.

\textbf{Inductive step.} Suppose $n=2k+1$ is odd and $7^{2k+1}+2^{2k+1}$ is divisible by $9$. Consider $n+2=2(k+1)+1$, the next odd integer:
\begin{align*}
7^{2k+3} + 2^{2k+3}
&= 7^2\cdot 7^{2k+1} + 2^2 \cdot 2^{2k+1} \\
&= 49\cdot 7^{2k+1} + 4\cdot 2^{2k+1} \\
\ &= 45\cdot 7^{2k+1} + 4\bigl(7^{2k+1} + 2^{2k+1}\bigr).
\end{align*}
Both terms are multiples of $9$: the first is obvious, and the second is $4$ times the inductive hypothesis (which is already a multiple of $9$). Therefore $7^{2k+3}+2^{2k+3}$ is divisible by $9$, completing the induction on odd integers.
\end{solution}

\begin{takeaways}
Working modulo $9$ turns the inductive step into a one-line parity check because $7 \equiv -2 \pmod 9$.
\end{takeaways}

\begin{problem}[Triangular Numbers]
Given $T_1 = 1$ and $T_n = T_{n-1} + n$ for $n \ge 2$, prove that $T_n = \frac{n(n+1)}{2}$ for every $n \ge 1$.
\end{problem}

\begin{solution}
\textbf{Base case ($n=1$).} $T_1 = 1$ and $\frac{1(1+1)}{2} = 1$, so the formula holds.

\textbf{Inductive step.} Assume $T_k = \frac{k(k+1)}{2}$ for some $k \ge 1$. Then
\[
T_{k+1} = T_k + (k+1) = \frac{k(k+1)}{2} + (k+1) = \frac{k(k+1) + 2(k+1)}{2} = \frac{(k+1)(k+2)}{2}.
\]
Therefore the closed form holds for $k+1$, completing the induction.
\end{solution}

\begin{takeaways}
Most recurrence relations in Extension~2 can be guessed and then confirmed with induction if you carefully add the new term in the inductive step.
\end{takeaways}

\begin{problem}[Polygon Interior Angle Sum]
Prove that the sum of the interior angles of an $n$-gon is $(n-2)\times 180^\circ$ for all integers $n \ge 3$.
\end{problem}

\begin{solution}
\textbf{Base case ($n=3$).} A triangle has interior angles summing to $180^\circ$, which equals $(3-2)\cdot 180^\circ$.

\textbf{Inductive step.} Assume every $k$-gon ($k\ge 3$) has interior angle sum $(k-2)\cdot 180^\circ$. Take a $(k+1)$-gon. Draw a diagonal from one vertex to split it into a triangle and a $k$-gon. The $k$-gon contributes $(k-2)\cdot 180^\circ$ and the triangle contributes $180^\circ$, totaling $(k-2)\cdot 180^\circ + 180^\circ = (k-1)\cdot 180^\circ$, as required.
\end{solution}

\begin{takeaways}
Breaking polygons into a $k$-gon plus a triangle is a classic structural induction idea: reduce a statement for $k+1$ objects to the known case for $k$.
\end{takeaways}
