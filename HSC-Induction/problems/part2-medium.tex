\begin{problem}[Sum of Cubes]
Prove that 

$$\sum_{k=1}^{n} k^3 = \left(\frac{n(n+1)}{2}\right)^2$$ for all integers $n \ge 1$.

\end{problem}

\begin{hint}
Note that the right side is the square of the sum of the first $n$ positive integers. Add $(k+1)^3$ to the hypothesis and show it equals $\left(\frac{(k+1)(k+2)}{2}\right)^2$.
\end{hint}

\begin{solution}[Sketch]
Base case: $1^3 = 1 = \left(\frac{1 \cdot 2}{2}\right)^2$. Assume the formula holds for $n=k$. Then $\sum_{r=1}^{k+1} r^3 = \left(\frac{k(k+1)}{2}\right)^2 + (k+1)^3 = \frac{k^2(k+1)^2 + 4(k+1)^3}{4} = \frac{(k+1)^2(k^2+4k+4)}{4} = \left(\frac{(k+1)(k+2)}{2}\right)^2$.
\end{solution}

\begin{problem}[Tower Inequality]
Prove that $2^n > n^2$ for all integers $n \ge 5$.
\end{problem}

\begin{hint}
For the inductive step, show that $2 \cdot k^2 > (k+1)^2$ when $k \ge 5$ by expanding and simplifying.
\end{hint}

\begin{solution}[Sketch]
Base case: $2^5 = 32 > 25 = 5^2$. Assume $2^k > k^2$ for some $k \ge 5$. Then $2^{k+1} = 2 \cdot 2^k > 2k^2$. We need $2k^2 \ge (k+1)^2 = k^2 + 2k + 1$, which simplifies to $k^2 \ge 2k + 1$ or $k^2 - 2k - 1 \ge 0$. This holds for $k \ge 5$ since $25 - 10 - 1 = 14 > 0$.
\end{solution}

\begin{problem}[Inequality with Factorials]
Prove that $n! > 2^n$ for all integers $n \ge 4$,
where $n! = n \times (n-1) \times (n-2) \times \cdots \times 2 \times 1$.
\end{problem}

\begin{hint}
In the inductive step, multiply the hypothesis by $(k+1)$ and show that $(k+1) \cdot 2^k > 2^{k+1}$ when $k \ge 4$.
\end{hint}

\begin{solution}[Sketch]
Base case: $4! = 24 > 16 = 2^4$. Assume $k! > 2^k$ for some $k \ge 4$. Then $(k+1)! = (k+1) \cdot k! > (k+1) \cdot 2^k$. Since $k+1 \ge 5 > 2$, we have $(k+1) \cdot 2^k > 2 \cdot 2^k = 2^{k+1}$.
\end{solution}

\begin{problem}[Postage Stamp Problem]
Prove that any integer $n \ge 12$ can be expressed as $n = 4a + 5b$ for some non-negative integers $a$ and $b$.
\end{problem}

\begin{hint}
Use strong induction with base cases $n=12,13,14,15$. For $n \ge 16$, express $n = (n-4) + 4$ and apply the hypothesis to $n-4 \ge 12$.
\end{hint}

\begin{solution}[Sketch]
Base cases: $12 = 4(3) + 5(0)$, $13 = 4(2) + 5(1)$, $14 = 4(1) + 5(2)$, $15 = 4(0) + 5(3)$. For $k \ge 16$, by hypothesis $k-4 = 4a + 5b$ for some $a,b \ge 0$. Then $k = (k-4) + 4 = 4(a+1) + 5b$.
\end{solution}

\begin{problem}[Inequality $3^n > n^3$]
Prove that $3^n > n^3$ for all integers $n \ge 4$.
\end{problem}

\begin{hint}
Show that $3k^3 > (k+1)^3$ for $k \ge 4$ by expanding $(k+1)^3$ and verifying $3k^3 - k^3 - 3k^2 - 3k - 1 > 0$.
\end{hint}

\begin{solution}[Sketch]
Base case: $3^4 = 81 > 64 = 4^3$. Assume $3^k > k^3$ for some $k \ge 4$. Then $3^{k+1} = 3 \cdot 3^k > 3k^3$. We need $3k^3 > (k+1)^3 = k^3 + 3k^2 + 3k + 1$, which simplifies to $2k^3 > 3k^2 + 3k + 1$. For $k=4$: $2(64) = 128 > 48+12+1 = 61$. Since both sides grow with $k$, the inequality holds for all $k \ge 4$.
\end{solution}

\begin{problem}[Factorial Inequality $(2n)! \ge 2^n (n!)^2$]
Prove that $(2n)! \ge 2^n (n!)^2$ for all positive integers $n$.
\end{problem}

\begin{hint}
For the inductive step, express $(2(k+1))!$ as $(2k+2)(2k+1) \cdot (2k)!$ and show that $(2k+2)(2k+1) \ge 2(k+1)^2$ by expanding and simplifying.
\end{hint}

\begin{solution}[Sketch]
Base case: $n=1$ gives $(2 \cdot 1)! = 2 \ge 2^1(1!)^2 = 2$. Assume $(2k)! \ge 2^k(k!)^2$ for some $k \ge 1$. Then $(2(k+1))! = (2k+2)! = (2k+2)(2k+1)(2k)! \ge (2k+2)(2k+1) \cdot 2^k(k!)^2$. We need $(2k+2)(2k+1) \ge 2(k+1)^2$, or $(2k+2)(2k+1) \ge 2(k^2+2k+1)$. Expanding: $4k^2+6k+2 \ge 2k^2+4k+2$, which simplifies to $2k^2+2k \ge 0$, true for all $k \ge 1$. Thus $(2(k+1))! \ge 2^{k+1}((k+1)!)^2$.
\end{solution}

\begin{problem}[Exponential Inequality $4^n - 1 - 7n > 0$]
Use mathematical induction to prove that $4^n - 1 - 7n > 0$ for all integers $n \ge 2$.
\end{problem}

\begin{hint}
For the inductive step, show that $4(4^k - 1 - 7k) > 4^{k+1} - 1 - 7(k+1)$ reduces to $3 \cdot 4^k + 21k > 7$, which is clearly true for $k \ge 2$.
\end{hint}

\begin{solution}[Sketch]
Base case: $n=2$ gives $4^2 - 1 - 14 = 1 > 0$. Assume $4^k - 1 - 7k > 0$ for some $k \ge 2$. Then $4^{k+1} - 1 - 7(k+1) = 4 \cdot 4^k - 1 - 7k - 7 = 4(4^k - 1 - 7k) + 3 \cdot 4^k + 21k$. Since $4^k - 1 - 7k > 0$ by hypothesis and $3 \cdot 4^k + 21k > 0$, we have $4^{k+1} - 1 - 7(k+1) > 0$.
\end{solution}

\begin{problem}[Weighted Geometric Sum]
Prove by induction that for integers $n \ge 1$:
\[1 + 2\left(\frac{1}{2}\right) + 3\left(\frac{1}{2}\right)^2 + \cdots + n\left(\frac{1}{2}\right)^{n-1} = 4 - \frac{n+2}{2^{n-1}}\]
\end{problem}

\begin{hint}
Add the $(k+1)$-th term $(k+1)\left(\frac{1}{2}\right)^k$ to the hypothesis and show it simplifies to $4 - \frac{k+3}{2^k}$.
\end{hint}

\begin{solution}[Sketch]
Base case: $n=1$ gives $1 = 4 - 3$. Assume the formula holds for $n=k$. Then the sum up to $k+1$ is $4 - \frac{k+2}{2^{k-1}} + (k+1)\left(\frac{1}{2}\right)^k = 4 - \frac{2(k+2)}{2^k} + \frac{k+1}{2^k} = 4 - \frac{2k+4-k-1}{2^k} = 4 - \frac{k+3}{2^k}$.
\end{solution}


