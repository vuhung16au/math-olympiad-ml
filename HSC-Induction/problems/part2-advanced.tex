\begin{problem}[Binomial Theorem]
Prove that $(a+b)^n = \sum_{k=0}^{n} \binom{n}{k} a^{n-k} b^k$ for all integers $n \ge 0$ and real numbers $a,b$.
\end{problem}

\begin{hint}
Use Pascal's identity $\binom{n+1}{k} = \binom{n}{k-1} + \binom{n}{k}$ to expand $(a+b)^{n+1} = (a+b)(a+b)^n$.
\end{hint}

\begin{solution}[Sketch]
Base case: $(a+b)^0 = 1 = \binom{0}{0}a^0b^0$. Assume the formula holds for $n$. Then $(a+b)^{n+1} = (a+b)\sum_{k=0}^{n}\binom{n}{k}a^{n-k}b^k = \sum_{k=0}^{n}\binom{n}{k}a^{n+1-k}b^k + \sum_{k=0}^{n}\binom{n}{k}a^{n-k}b^{k+1}$. Reindex and apply Pascal's identity to combine terms.
\end{solution}

\begin{problem}[Fermat's Little Theorem]
For any prime $p$ and integer $a$, prove that $a^p \equiv a \pmod{p}$.
\end{problem}

\begin{hint}
Use induction on $a$. Show $(k+1)^p - (k+1) \equiv k^p - k \pmod{p}$ by expanding with the binomial theorem and noting that $\binom{p}{j}$ is divisible by $p$ for $1 \le j \le p-1$.
\end{hint}

\begin{solution}[Sketch]
Base case: $0^p \equiv 0 \pmod{p}$. Assume $k^p \equiv k \pmod{p}$. Then $(k+1)^p = k^p + \binom{p}{1}k^{p-1} + \cdots + \binom{p}{p-1}k + 1$. Since $p \mid \binom{p}{j}$ for $1 \le j \le p-1$, we have $(k+1)^p \equiv k^p + 1 \equiv k + 1 \pmod{p}$ by the hypothesis.
\end{solution}

% --- New Problem: Symmetric Sum Inequality ---
\begin{problem}[Symmetric Sum Inequality]
Let $n$ be a positive integer and $a_1, a_2, \dots, a_n$ be $n$ positive real numbers. Prove by mathematical induction that:
\[
\left( a_1 + a_2 + \dots + a_n \right) \left( \frac{1}{a_1} + \frac{1}{a_2} + \dots + \frac{1}{a_n} \right) \geq n^2
\]
for all $n \geq 1$.
\end{problem}

\begin{hint}
Use induction on $n$. For the inductive step, expand $\left( \sum_{i=1}^{k+1} a_i \right) \left( \sum_{i=1}^{k+1} \frac{1}{a_i} \right)$ and apply the inequality $x + \frac{1}{x} \geq 2$ for $x > 0$ to each pair $\frac{a_i}{a_{k+1}}$ and $\frac{a_{k+1}}{a_i}$.
\end{hint}

\begin{solution}[Sketch]
Base case: $n=1$. $(a_1)\left(\frac{1}{a_1}\right) = 1 \geq 1^2$.

Inductive hypothesis: Assume true for $n=k$, i.e., $\left( \sum_{i=1}^k a_i \right) \left( \sum_{i=1}^k \frac{1}{a_i} \right) \geq k^2$.

Inductive step: For $n=k+1$,
\begin{align*}
\left( \sum_{i=1}^{k+1} a_i \right) \left( \sum_{i=1}^{k+1} \frac{1}{a_i} \right) &= \left( S_k + a_{k+1} \right) \left( R_k + \frac{1}{a_{k+1}} \right) \\
&= S_k R_k + S_k \cdot \frac{1}{a_{k+1}} + a_{k+1} \cdot R_k + 1 \\
&= S_k R_k + 1 + \sum_{i=1}^k \left( \frac{a_i}{a_{k+1}} + \frac{a_{k+1}}{a_i} \right)
\end{align*}
By the hypothesis, $S_k R_k \geq k^2$. For each $i$, $\frac{a_i}{a_{k+1}} + \frac{a_{k+1}}{a_i} \geq 2$. Thus,
\begin{align*}
	ext{LHS}_{k+1} &\geq k^2 + 1 + 2k = (k+1)^2
\end{align*}
So the result holds for $n=k+1$.

Conclusion: By induction, the inequality holds for all $n \geq 1$.
\end{solution}

\begin{problem}[Symmetric Sum Inequality]
Let $n$ be a positive integer and $a_1, a_2, \dots, a_n$ be $n$ positive real numbers. Prove by mathematical induction that:
\[
\left( a_1 + a_2 + \dots + a_n \right) \left( \frac{1}{a_1} + \frac{1}{a_2} + \dots + \frac{1}{a_n} \right) \geq n^2
\]
for all $n \geq 1$.
\end{problem}

\begin{hint}
Use induction on $n$. For the inductive step, expand $\left( \sum_{i=1}^{k+1} a_i \right) \left( \sum_{i=1}^{k+1} \frac{1}{a_i} \right)$ and apply the inequality $x + \frac{1}{x} \geq 2$ for $x > 0$ to each pair $\frac{a_i}{a_{k+1}}$ and $\frac{a_{k+1}}{a_i}$.
\end{hint}

\begin{solution}[Sketch]
Base case: $n=1$. $(a_1)\left(\frac{1}{a_1}\right) = 1 \geq 1^2$.

Inductive hypothesis: Assume true for $n=k$, i.e., $\left( \sum_{i=1}^k a_i \right) \left( \sum_{i=1}^k \frac{1}{a_i} \right) \geq k^2$.

Inductive step: For $n=k+1$,
\begin{align*}
\left( \sum_{i=1}^{k+1} a_i \right) \left( \sum_{i=1}^{k+1} \frac{1}{a_i} \right) &= \left( S_k + a_{k+1} \right) \left( R_k + \frac{1}{a_{k+1}} \right) \\
&= S_k R_k + S_k \cdot \frac{1}{a_{k+1}} + a_{k+1} \cdot R_k + 1 \\
&= S_k R_k + 1 + \sum_{i=1}^k \left( \frac{a_i}{a_{k+1}} + \frac{a_{k+1}}{a_i} \right)
\end{align*}
By the hypothesis, $S_k R_k \geq k^2$. For each $i$, $\frac{a_i}{a_{k+1}} + \frac{a_{k+1}}{a_i} \geq 2$. Thus,
\begin{align*}
	ext{LHS}_{k+1} &\geq k^2 + 1 + 2k = (k+1)^2
\end{align*}
So the result holds for $n=k+1$.

Conclusion: By induction, the inequality holds for all $n \geq 1$.
\end{solution}

\begin{problem}[Tower of Hanoi]
The Tower of Hanoi is a classic puzzle involving three pegs and $n$ disks of different sizes, all initially stacked in order of decreasing size on one peg. The objective is to move the entire stack to another peg, moving only one disk at a time and never placing a larger disk on top of a smaller one.
Prove that the minimum number of moves required to transfer $n$ disks in the Tower of Hanoi puzzle is $2^n - 1$.
\end{problem}

\begin{hint}
To move $k+1$ disks, first move the top $k$ disks to the auxiliary peg (requiring $T_k$ moves), then move the largest disk (1 move), then move the $k$ disks to the destination (requiring $T_k$ moves again). This gives $T_{k+1} = 2T_k + 1$.
\end{hint}

\begin{solution}[Sketch]
Base case: $T_1 = 1 = 2^1 - 1$. Assume $T_k = 2^k - 1$. To move $k+1$ disks: move $k$ disks to auxiliary peg ($2^k-1$ moves), move largest disk (1 move), move $k$ disks to destination ($2^k-1$ moves). Total: $T_{k+1} = 2(2^k-1) + 1 = 2^{k+1} - 1$.
\end{solution}

\begin{problem}[Vandermonde's Identity]
Prove that $\binom{m+n}{r} = \sum_{k=0}^{r} \binom{m}{k}\binom{n}{r-k}$ for non-negative integers $m,n,r$.
\end{problem}

\begin{hint}
Use induction on $n$. Apply Pascal's identity $\binom{n+1}{j} = \binom{n}{j-1} + \binom{n}{j}$ to split the right-hand side.
\end{hint}

\begin{solution}[Sketch]
Induct on $n$ with $m,r$ fixed. Base case $n=0$: both sides equal $\binom{m}{r}$. Assume the identity holds for $n$. Then $\binom{m+n+1}{r} = \binom{m+n}{r-1} + \binom{m+n}{r}$. Apply the hypothesis to both terms and rearrange using $\binom{n}{j-1} + \binom{n}{j} = \binom{n+1}{j}$ to obtain the desired sum.
\end{solution}

\begin{problem}[Wilson's Theorem]
Prove that $(p-1)! \equiv -1 \pmod{p}$ for every prime $p$.
\end{problem}

\begin{hint}
For $p>2$, pair each $a \in \{2,3,\ldots,p-2\}$ with its inverse $a^{-1} \pmod{p}$. The only elements that are their own inverses are $1$ and $p-1$.
\end{hint}

\begin{solution}[Sketch]
Base case: $p=2$ gives $(2-1)! = 1 \equiv -1 \pmod{2}$. For prime $p>2$, in the product $(p-1)! = 1 \cdot 2 \cdots (p-1)$, each $a \in \{2,\ldots,p-2\}$ pairs with its inverse $a^{-1}$ (where $a \neq a^{-1}$), contributing $aa^{-1} \equiv 1$. Only $1$ and $p-1$ are self-inverse, so $(p-1)! \equiv 1 \cdot (p-1) \equiv -1 \pmod{p}$.
\end{solution}

\begin{problem}[Recursive Sequence with Surds]
A sequence $a_n$ is defined by $a_n = 2a_{n-1} + a_{n-2}$ for $n \ge 2$ with $a_0 = a_1 = 2$. Use mathematical induction to prove that:
\[a_n = (1+\sqrt{2})^n + (1-\sqrt{2})^n \quad \text{for all } n \ge 0\]
\end{problem}

\begin{hint}
Use strong induction. For the base cases verify $n=0$ and $n=1$. Then show $(1+\sqrt{2})^{k+1} + (1-\sqrt{2})^{k+1} = 2[(1+\sqrt{2})^k + (1-\sqrt{2})^k] + [(1+\sqrt{2})^{k-1} + (1-\sqrt{2})^{k-1}]$ using the fact that $(1 \pm \sqrt{2})$ satisfy $x^2 = 2x + 1$.
\end{hint}

\begin{solution}[Sketch]
Base cases: $a_0 = 2 = 1 + 1$, $a_1 = 2 = (1+\sqrt{2}) + (1-\sqrt{2})$. Assume the formula holds for $n=k-1$ and $n=k$. Note that $\alpha = 1+\sqrt{2}$ and $\beta = 1-\sqrt{2}$ both satisfy $x^2 = 2x + 1$. Then $a_{k+1} = 2a_k + a_{k-1} = 2(\alpha^k + \beta^k) + (\alpha^{k-1} + \beta^{k-1}) = \alpha^{k-1}(2\alpha + 1) + \beta^{k-1}(2\beta + 1) = \alpha^{k-1} \cdot \alpha^2 + \beta^{k-1} \cdot \beta^2 = \alpha^{k+1} + \beta^{k+1}$.
\end{solution}

\begin{problem}[Sum of Cosines]
It is given that $2\cos A \sin B = \sin(A+B) - \sin(A-B)$. Prove by induction that for integers $n \ge 1$:
\[\cos \theta + \cos 3\theta + \cdots + \cos(2n-1)\theta = \frac{\sin 2n\theta}{2\sin\theta}\]
\end{problem}

\begin{hint}
Use the product-to-sum identity to show that $\sin 2(k+1)\theta = \sin 2k\theta + 2\sin\theta \cos(2k+1)\theta$.
\end{hint}

\begin{solution}[Sketch]
Base case: $n=1$ gives $\cos \theta = \frac{\sin 2\theta}{2\sin\theta} = \frac{2\sin\theta\cos\theta}{2\sin\theta} = \cos\theta$. Assume $\sum_{r=1}^{k} \cos(2r-1)\theta = \frac{\sin 2k\theta}{2\sin\theta}$. Then $\sum_{r=1}^{k+1} \cos(2r-1)\theta = \frac{\sin 2k\theta}{2\sin\theta} + \cos(2k+1)\theta$. Using $2\cos(2k+1)\theta \sin\theta = \sin(2k+2)\theta - \sin 2k\theta$, we get $\cos(2k+1)\theta = \frac{\sin 2(k+1)\theta - \sin 2k\theta}{2\sin\theta}$. Thus the sum equals $\frac{\sin 2(k+1)\theta}{2\sin\theta}$.
\end{solution}

\begin{problem}[Nested Radicals]
The numbers $a_n$, for integers $n \ge 1$, are defined as $a_1 = \sqrt{2}$, $a_2 = \sqrt{2+\sqrt{2}}$, $a_3 = \sqrt{2+\sqrt{2+\sqrt{2}}}$ and so on. These numbers satisfy the relation $a_{n+1}^2 = 2 + a_n$. Use mathematical induction to prove that:
\[a_n = 2\cos\frac{\pi}{2^{n+1}} \quad \text{for all integers } n \ge 1\]
\end{problem}

\begin{hint}
Use the half-angle formula: $\cos\frac{\theta}{2} = \sqrt{\frac{1+\cos\theta}{2}}$ to show that $2\cos\frac{\pi}{2^{k+2}}$ satisfies the recurrence when squared.
\end{hint}

\begin{solution}[Sketch]
Base case: $a_1 = \sqrt{2} = 2\cos\frac{\pi}{4}$. Assume $a_k = 2\cos\frac{\pi}{2^{k+1}}$. Then $a_{k+1}^2 = 2 + a_k = 2 + 2\cos\frac{\pi}{2^{k+1}} = 2\left(1 + \cos\frac{\pi}{2^{k+1}}\right) = 4\cos^2\frac{\pi}{2^{k+2}}$ using the double-angle formula $1 + \cos\theta = 2\cos^2\frac{\theta}{2}$. Since $a_{k+1} > 0$, we have $a_{k+1} = 2\cos\frac{\pi}{2^{k+2}}$.
\end{solution}

\begin{problem}[Cosecant Sum Formula]
Use mathematical induction to prove that for all $n \ge 1$:
\[\sum_{r=1}^{n}\csc(2^r x) = \cot x - \cot(2^n x)\]
\end{problem}

\begin{hint}
Use the identity $\cot x - \cot 2x = \csc 2x$ to simplify the $(k+1)$-th term.
\end{hint}

\begin{solution}[Sketch]
Base case: $n=1$ gives $\csc 2x = \frac{1}{\sin 2x} = \frac{1}{2\sin x \cos x} = \frac{1}{2\sin x}\cdot\frac{1}{\cos x} = \cot x - \cot 2x$ (using $\cot x - \cot 2x = \frac{\cos x}{\sin x} - \frac{\cos 2x}{\sin 2x} = \frac{2\cos x \sin x - \cos 2x \sin x}{\sin 2x \sin x} = \frac{\sin 2x}{\sin 2x \sin x} = \csc 2x$). Assume $\sum_{r=1}^{k}\csc(2^r x) = \cot x - \cot(2^k x)$. Then $\sum_{r=1}^{k+1}\csc(2^r x) = \cot x - \cot(2^k x) + \csc(2^{k+1}x) = \cot x - [\cot(2^k x) - \csc(2^{k+1}x)] = \cot x - \cot(2^{k+1}x)$.
\end{solution}

\begin{problem}[Arctangent Sum]
Use mathematical induction to prove that for all positive integers $n$:
\[\sum_{j=1}^{n}\tan^{-1}\left(\frac{1}{2j^2}\right) = \tan^{-1}\left(\frac{n}{n+1}\right)\]
\end{problem}

\begin{hint}
Use the arctangent addition formula: $\tan^{-1}a + \tan^{-1}b = \tan^{-1}\left(\frac{a+b}{1-ab}\right)$ when $ab < 1$. Show that $\frac{1}{2(k+1)^2} + \frac{k}{k+1} = \frac{k+1}{k+2}$ after applying the formula.
\end{hint}

\begin{solution}[Sketch]
Base case: $n=1$ gives $\tan^{-1}\frac{1}{2} = \tan^{-1}\frac{1}{2}$. Assume $\sum_{j=1}^{k}\tan^{-1}\left(\frac{1}{2j^2}\right) = \tan^{-1}\left(\frac{k}{k+1}\right)$. Then $\sum_{j=1}^{k+1}\tan^{-1}\left(\frac{1}{2j^2}\right) = \tan^{-1}\left(\frac{k}{k+1}\right) + \tan^{-1}\left(\frac{1}{2(k+1)^2}\right)$. Using the addition formula with $a = \frac{k}{k+1}$ and $b = \frac{1}{2(k+1)^2}$: $\frac{a+b}{1-ab} = \frac{\frac{k}{k+1} + \frac{1}{2(k+1)^2}}{1 - \frac{k}{2(k+1)^3}} = \frac{2k(k+1)+1}{2(k+1)^2 - k} = \frac{2k^2+2k+1}{2k^2+4k+2-k} = \frac{2k^2+2k+1}{2k^2+3k+2} = \frac{(k+1)(2k+1)}{(k+2)(2k+1)} = \frac{k+1}{k+2}$.
\end{solution}
