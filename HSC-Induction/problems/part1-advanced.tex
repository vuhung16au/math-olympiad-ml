% Inserted problem from samples2/05.tex
\begin{problem}[Recursive sequence via $f(x)=2^x+2^{-x}$]
Let the function $f(x)$ be defined by 

$$f(x)=2^x+2^{-x}$$ for all real $x$.

Consider the sequence $(u_n)$ defined by
\[
u_1=\tfrac{5}{2},\qquad u_{n+1}=\sqrt{2+u_n}\quad(n\ge1).
\]

\begin{enumerate}[label=(\roman*)]
  \item Show that $f(x)^2=f(2x)+2$, and deduce that
  \[f\left(\frac{x}{2}\right)=\sqrt{2+f(x)}\quad(\forall x\in\mathbb{R}).\]
  \item Use induction to prove that for every integer $n\ge1$,
  \[u_n=f\left(\frac{1}{2^{n-1}}\right).
  \]
  \item Find the exact value of
  \[\lim_{n\to\infty} 2^n\sqrt{u_n-2}.\]
  \item Evaluate the telescoping product limit
  \[\lim_{n\to\infty}\frac{u_1u_2\cdots u_n}{2^n}.
  \]
\end{enumerate}
\end{problem}

\begin{hint}
\begin{itemize}
  \item Expand $(2^x+2^{-x})^2$ and take the positive square root (note $f>0$).
  \item For induction, set $x=2^{1-k}$ when passing from $k$ to $k+1$ and use part (i).
  \item Write $u_n-2=(2^{t}-2^{-t})^2$ for a small $t$ and use the limit $\lim_{t\to0}\frac{2^t-1}{t}=\ln2$.
  \item Multiply the product by the conjugate factors $2^{x}-2^{-x}$ to telescope.
\end{itemize}
\end{hint}

\begin{solution}
Part (i): Expand to get $[f(x)]^2=2^{2x}+2^{-2x}+2=f(2x)+2$. Replace $x$ by $x/2$ and take the positive root to obtain the identity claimed.

Part (ii): Base case $u_1=f(1)=2+2^{-1}=5/2$. If $u_k=f(2^{1-k})$ then
\[u_{k+1}=\sqrt{2+u_k}=\sqrt{2+f(2^{1-k})}=f\left(\tfrac{2^{1-k}}{2}\right)=f(2^{-k}),\]
so the statement holds by induction.

Part (iii): Put $h_n=1/2^{n-1}$ so $u_n=2^{h_n}+2^{-h_n}$. Then
\[\sqrt{u_n-2}=2^{h_n/2}-2^{-h_n/2},\quad \text{where }h_n/2=1/2^n.
\]
Thus
\[2^n\sqrt{u_n-2}=\frac{2^t-2^{-t}}{t}\Big|_{t=1/2^n}\to (\ln2)-(-\ln2)=2\ln2.
\]

Part (iv): Let $x_k=1/2^{k-1}$ and set $g_k=2^{x_k}-2^{-x_k}$. Note
\[u_k g_k=2^{2x_k}-2^{-2x_k}=g_{k-1}.
\]
Multiplying for $k=1,\dots,n$ gives $P_n g_n=g_0$ where $P_n=\prod_{k=1}^n u_k$ and $g_0=2^{2}-2^{-2}=15/4$. Hence
\[\frac{P_n}{2^n}=\frac{15/4}{2^n(2^{x_n}-2^{-x_n})}.
\]
With $t=x_n=1/2^{n-1}$ we have $2^n(2^t-2^{-t})\to 2\cdot2\ln2=4\ln2$, so the limit equals $\dfrac{15}{16\ln2}$.
\end{solution}

\begin{takeaways}
\begin{itemize}
  \item What happens when $u_1 < 2$? Can we use trigonometric functions instead of exponentials?
  \item When $u_1 = 2$, the sequence is constant: $u_n = 2$ for all $n$ and we call the point a fixed point.
  \item Link this problem to the hyperbolic cosine function $\cosh x = \frac{e^x + e^{-x}}{2}$.
  \item Recognise when sequences stem from a two-term exponential identity such as $2^x+2^{-x}$ (a hyperbolic/cosh-like form).
  \item Telescoping products often require multiplying by conjugates; identify the right companion factor.
  \item Convert limits with $n\to\infty$ and exponentials to standard derivative-form limits via substitution $t\to0$.
  \item Using L'Hopital's rule can help evaluate tricky limits.
  
    L'Hopital's rule states that if $\lim_{x \to a} f(x) = 0$ and $\lim_{x \to a} g(x) = 0$, and the derivatives $f'(x)$ and $g'(x)$ are continuous near $a$ with $g'(x) \neq 0$ for $x \neq a$, then
        \[
        \lim_{x \to a} \frac{f(x)}{g(x)} = \lim_{x \to a} \frac{f'(x)}{g'(x)}
        \]
    provided the limit on the right exists.

\end{itemize}
\end{takeaways}


% \begin{remark}
%     L'Hopital's rule states that if $\lim_{x \to a} f(x) = 0$ and $\lim_{x \to a} g(x) = 0$, and the derivatives $f'(x)$ and $g'(x)$ are continuous near $a$ with $g'(x) \neq 0$ for $x \neq a$, then
% \[
% \lim_{x \to a} \frac{f(x)}{g(x)} = \lim_{x \to a} \frac{f'(x)}{g'(x)}
% \]
% provided the limit on the right exists.

% \end{remark}

\begin{problem}[De Moivre's Theorem]
Let $z = r(\cos\theta + i\sin\theta)$ with $r>0$. Prove by induction that for every integer $n \ge 1$,
\[
z^n = r^n (\cos n\theta + i \sin n\theta).
\]
\end{problem}

\begin{solution}
\textbf{Base case ($n=1$).} Trivial because $z^1 = r(\cos\theta + i\sin\theta)$.

\textbf{Inductive step.} Assume $z^k = r^k(\cos k\theta + i\sin k\theta)$. Then
\begin{align*}
z^{k+1}
&= z \cdot z^{k}
= r(\cos\theta + i\sin\theta)\cdot r^k(\cos k\theta + i\sin k\theta) \\
&= r^{k+1}\bigl[(\cos\theta\cos k\theta - \sin\theta\sin k\theta) + i(\sin\theta\cos k\theta + \cos\theta\sin k\theta)\bigr] \\
&= r^{k+1}\bigl[\cos((k+1)\theta) + i\sin((k+1)\theta)\bigr],
\end{align*}
using the addition formulas for sine and cosine. The statement follows by induction.
\end{solution}

\begin{takeaways}
Complex-number induction proofs nearly always hinge on angle addition formulas once you factor out magnitudes.
\end{takeaways}

\begin{problem}[Tiling a Defective $2^n \times 2^n$ Board]
Prove that any $2^n \times 2^n$ chessboard with one square missing can be completely tiled with L-shaped trominoes for all integers $n \ge 1$.
\end{problem}

\begin{solution}
\textbf{Base case ($n=1$).} A $2\times 2$ board with one square removed exactly matches a single L-shaped tromino.

\textbf{Inductive step.} Assume the claim is true for $n=k$: every $2^k \times 2^k$ board with one missing square can be tiled. Consider a board of size $2^{k+1}\times 2^{k+1}$. Divide it into four quadrants, each of size $2^k \times 2^k$. One quadrant already has the missing square. Place a single L-shaped tromino at the center to cover one square from each of the other three quadrants, effectively creating an artificial “missing” square inside each. Now each quadrant is a $2^k \times 2^k$ board with one square missing, so by the inductive hypothesis all four quadrants can be tiled. Thus the entire board can be tiled, completing the induction.
\end{solution}

\begin{takeaways}
Structural induction typically requires creating the same “defect” in each subproblem so the hypothesis applies uniformly.
\end{takeaways}

\begin{problem}[Evaluating $I_n$ from a Recurrence]
Given $I_0 = 1$ and the recurrence
\[
I_n = \frac{2n}{2n+1} I_{n-1} \quad (n \ge 1),
\]
show that
\[
I_n = \frac{2^{2n}(n!)^2}{(2n+1)!}
\]
for every $n \ge 0$.
\end{problem}

\begin{solution}
\textbf{Base case ($n=0$).} The closed form gives $I_0 = \frac{2^{0}(0!)^2}{1!} = 1$, matching the definition.

\textbf{Inductive step.} Assume the result holds for $n=k$, so $I_k = \frac{2^{2k}(k!)^2}{(2k+1)!}$. Then
\begin{align*}
I_{k+1}
&= \frac{2(k+1)}{2(k+1)+1} I_k
= \frac{2(k+1)}{2k+3} \cdot \frac{2^{2k}(k!)^2}{(2k+1)!} \\
&= \frac{2^{2k+1}(k+1)(k!)^2}{(2k+3)(2k+1)!}
= \frac{2^{2k+2}((k+1)!)^2}{(2k+3)!}.
\end{align*}
The last step follows because $(k+1)(k!)^2 = (k+1)! \cdot k!$ and $(2k+3)! = (2k+3)(2k+2)(2k+1)!$. Therefore the closed form holds for $k+1$, and by induction it is valid for all $n$.
\end{solution}

\begin{takeaways}
Products such as $\frac{2n}{2n+1}$ often telescope into factorials; writing several terms explicitly helps you guess the factorial pattern to confirm by induction.
\end{takeaways}

\begin{problem}[Integer Coefficients in $\int_0^1 x^n e^x \, dx$]
Let 

$$I_n = \int_0^{1} x^n e^x \, dx$$. 

Show by induction that there exist integers $a_n$ and $b_n$ with $I_n = a_n + b_n e$ for every $n \ge 0$.
\end{problem}

\begin{solution}
Integrate by parts with $u = x^n$ and $dv = e^x dx$ to obtain, for $n \ge 1$,
\[
I_n = [x^n e^x]_0^1 - n \int_0^1 x^{n-1} e^x dx = e - n I_{n-1}.
\]

\textbf{Base case ($n=0$).} $I_0 = \int_0^1 e^x dx = e - 1$, so choose $(a_0,b_0) = (-1,1)$.

\textbf{Inductive step.} Suppose $I_{n-1} = a_{n-1} + b_{n-1} e$ with integers $a_{n-1}, b_{n-1}$. Then
\[
I_n = e - n(a_{n-1} + b_{n-1} e) = (-n a_{n-1}) + (1 - n b_{n-1}) e.
\]
Both coefficients are integers, so set $a_n = -n a_{n-1}$ and $b_n = 1 - n b_{n-1}$. Therefore $I_n$ always takes the form $a_n + b_n e$ with integers $a_n,b_n$.
\end{solution}

\begin{takeaways}
The recurrence $I_n = e - n I_{n-1}$ keeps the structure “integer plus integer times $e$,” so tracking the coefficients directly is an efficient inductive strategy.
\end{takeaways}

\begin{problem}[Closed Form for $J_n$]
Let $J_n = \displaystyle \int_{0}^{1} x^n e^{-x}\,dx$ with $J_0 = 1 - \frac{1}{e}$. Show that for every $n \ge 0$,
\[
J_n = n! - \frac{n!}{e}\sum_{r=0}^{n} \frac{1}{r!}.
\]
\end{problem}

\begin{solution}
We have the recurrence $J_n = n J_{n-1} - \frac{1}{e}$ for $n \ge 1$.

\textbf{Base case ($n=0$).} The right-hand side yields $0! - \frac{0!}{e}\cdot 1 = 1 - \frac{1}{e}$, which equals $J_0$.

\textbf{Inductive step.} Assume $J_k = k! - \frac{k!}{e}\sum_{r=0}^{k} \frac{1}{r!}$. Then
\begin{align*}
J_{k+1}
&= (k+1)J_k - \frac{1}{e} \\
&= (k+1)\left(k! - \frac{k!}{e}\sum_{r=0}^{k} \frac{1}{r!}\right) - \frac{1}{e} \\
&= (k+1)! - \frac{(k+1)!}{e}\sum_{r=0}^{k} \frac{1}{r!} - \frac{1}{e}.
\end{align*}
Observe that $\frac{1}{e} = \frac{(k+1)!}{e}\cdot \frac{1}{(k+1)!}$. Therefore
\[
J_{k+1} = (k+1)! - \frac{(k+1)!}{e}\left(\sum_{r=0}^{k} \frac{1}{r!} + \frac{1}{(k+1)!}\right)
= (k+1)! - \frac{(k+1)!}{e}\sum_{r=0}^{k+1} \frac{1}{r!}.
\]
This matches the claimed closed form, completing the induction.
\end{solution}

\begin{takeaways}
Recurrences derived from integration by parts usually reveal factorial patterns; substituting the hypothesis and carefully aligning summations keep the algebra manageable.
\end{takeaways}
