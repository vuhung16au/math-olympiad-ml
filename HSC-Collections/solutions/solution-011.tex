\subsection{Solution to Problem 11: Vectors and Complex Numbers}

\begin{solution}
\textbf{(i)} Since $\vec{OM} = \frac{1}{2}(\mathbf{a} + \mathbf{b})$, we can write:
\[
\vec{OM} = \frac{1}{2}\mathbf{a} + \frac{1}{2}\mathbf{b} = \mathbf{a} + \frac{1}{2}(\mathbf{b} - \mathbf{a})
\]
This shows $M$ lies on the line through $A$ (when parameter = 0) and $B$ (when parameter = 1).

\textbf{(ii)} We have $\vec{OG} = \frac{1}{3}(\mathbf{a} + \mathbf{b} + \mathbf{c}) = \frac{2}{3} \cdot \frac{1}{2}(\mathbf{a} + \mathbf{b}) + \frac{1}{3}\mathbf{c} = \frac{2}{3}\vec{OM} + \frac{1}{3}\mathbf{c}$.

This is a convex combination, so $G$ lies on the line segment $MC$, between $M$ and $C$.

\textbf{(iii)} Suppose $\frac{1}{3}(x+w+z) = (xwz)^{1/3}$ for some cube root. Then $|\frac{1}{3}(x+w+z)| = |(xwz)^{1/3}| = 1$.

But by the triangle inequality: $|\frac{1}{3}(x+w+z)| \leq \frac{1}{3}(|x|+|w|+|z|) = 1$.

Equality occurs only if $x$, $w$, $z$ all have the same argument. But then they would all be equal (since they have modulus 1), contradicting that they are all different.

Therefore, $\frac{1}{3}(x+w+z)$ is never a cube root of $xwz$.
\end{solution}

\begin{takeaways}
\begin{itemize}
    \item Midpoint: $\frac{1}{2}(\mathbf{a}+\mathbf{b})$ lies on line $AB$
    \item Centroid: $\frac{1}{3}(\mathbf{a}+\mathbf{b}+\mathbf{c})$ lies between midpoint and third vertex
    \item Triangle inequality: equality requires collinear vectors with same direction
\end{itemize}
\end{takeaways}

