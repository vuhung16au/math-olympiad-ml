\subsection{Solution to Problem 13: Circle and Cosine Function}

\begin{solution}
At point $P(a, b)$, we have $b = \cos(ka)$ and the point lies on a circle centered at origin, so $a^2 + b^2 = r^2$ for some radius $r$.

The slope of the radius vector is $b/a$. The slope of the tangent to $y = \cos(kx)$ at $x = a$ is $-k\sin(ka)$.

Since they are perpendicular: $\frac{b}{a} \cdot (-k\sin(ka)) = -1$, so $kb\sin(ka) = a$.

Since $b = \cos(ka)$, we get: $k\cos(ka)\sin(ka) = a$, or $\frac{k}{2}\sin(2ka) = a$.

Also, $a^2 + \cos^2(ka) = r^2$.

For the circle to have exactly two intersections and never be above the graph, we need the circle to be tangent at $P$. This requires $r^2 = a^2 + b^2 = a^2 + \cos^2(ka)$.

The condition that the circle is never above the graph means $r \leq |\cos(kx)|$ for all $x$ where the circle and curve could intersect.

For $k \leq 1$, the period is $\geq 2\pi$, and the geometry doesn't work. For $k > 1$, we can have the required configuration. Therefore $k > 1$.
\end{solution}

\begin{takeaways}
\begin{itemize}
    \item Perpendicular condition: slopes multiply to $-1$
    \item Use $b = \cos(ka)$ and the circle equation $a^2 + b^2 = r^2$
    \item Analyze the geometry to determine constraint on $k$
\end{itemize}
\end{takeaways}

