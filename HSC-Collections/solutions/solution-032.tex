\subsection{Solution to Problem 32: Parallelogram Geometry}

\begin{solution}
\textbf{(i)} In parallelogram $OPQR$, we have $\vec{OQ} = \vec{OP} + \vec{OR} = \mathbf{p} + \mathbf{r}$.

Since $S$ is the midpoint of $QR$: $\vec{OS} = \vec{OR} + \frac{1}{2}\vec{RQ} = \mathbf{r} + \frac{1}{2}(\mathbf{p} + \mathbf{r} - \mathbf{r}) = \mathbf{r} + \frac{1}{2}\mathbf{p}$.

Point $T$ lies on both $PR$ and $OS$.

On $PR$: $\vec{OT} = \mathbf{r} + \lambda(\mathbf{p} - \mathbf{r}) = (1-\lambda)\mathbf{r} + \lambda\mathbf{p}$ for some $\lambda$.

On $OS$: $\vec{OT} = \mu\left(\mathbf{r} + \frac{1}{2}\mathbf{p}\right) = \mu\mathbf{r} + \frac{\mu}{2}\mathbf{p}$ for some $\mu$.

Equating: $(1-\lambda)\mathbf{r} + \lambda\mathbf{p} = \mu\mathbf{r} + \frac{\mu}{2}\mathbf{p}$.

So: $1-\lambda = \mu$ and $\lambda = \frac{\mu}{2}$.

Substituting: $1 - \frac{\mu}{2} = \mu$, so $1 = \frac{3\mu}{2}$, giving $\mu = \frac{2}{3}$ and $\lambda = \frac{1}{3}$.

Therefore: $\vec{OT} = \frac{2}{3}\mathbf{r} + \frac{1}{3}\mathbf{p}$.

\textbf{(ii)} On line $PR$: $\vec{OT} = \frac{1}{3}\mathbf{p} + \frac{2}{3}\mathbf{r}$.

This means $T$ divides $PR$ in the ratio $PT : TR = \frac{2}{3} : \frac{1}{3} = 2 : 1$.
\end{solution}

\begin{takeaways}
\begin{itemize}
    \item Parametric form: Express points on lines using parameters
    \item Intersection: Equate parametric forms to find intersection point
    \item Section formula: Use to determine division ratio
\end{itemize}
\end{takeaways}

