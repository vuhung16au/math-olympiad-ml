\subsection{Solution to Problem 15: Integer Equation with Large Exponents}

\begin{solution}
We show that $(n + 1)^{41} - 79n^{40} = 2$ has no integer solutions by considering the equation modulo 2.

\textbf{Case 1:} $n$ is even.

Then $n+1$ is odd, so $(n+1)^{41} \equiv 1 \pmod{2}$ (odd to any power is odd).

Also, $n^{40}$ is even (since $n$ is even), so $79n^{40} \equiv 0 \pmod{2}$.

Therefore: $(n+1)^{41} - 79n^{40} \equiv 1 - 0 = 1 \pmod{2}$.

\textbf{Case 2:} $n$ is odd.

Then $n+1$ is even, so $(n+1)^{41} \equiv 0 \pmod{2}$.

Also, $n^{40}$ is odd (since $n$ is odd), so $79n^{40} \equiv 1 \pmod{2}$ (odd times odd is odd).

Therefore: $(n+1)^{41} - 79n^{40} \equiv 0 - 1 = -1 \equiv 1 \pmod{2}$.

In both cases, the left-hand side is congruent to $1 \pmod{2}$, but the right-hand side is $2 \equiv 0 \pmod{2}$.

This is a contradiction. Therefore, there is no integer $n$ satisfying the equation.
\end{solution}

\begin{takeaways}
\begin{itemize}
    \item Modular arithmetic: Check equations modulo small primes to find contradictions
    \item Parity: Modulo 2 often gives quick contradictions for integer equations
    \item Fermat's Little Theorem: For prime $p$ and $\gcd(a,p)=1$, $a^{p-1} \equiv 1 \pmod{p}$
\end{itemize}
\end{takeaways}

