\subsection{Solution to Problem 22: Mechanics with Ropes and Forces}

\begin{solution}
\textbf{(i)} At point $P$, resolve forces:

Horizontally: $T_1\cos\theta = T_2\cos\phi$ (1)

Vertically: $T_1\sin\theta + T_2\sin\phi = Mg$ (2)

From (1): $T_1 = \frac{T_2\cos\phi}{\cos\theta}$.

Substituting into (2): $\frac{T_2\cos\phi}{\cos\theta}\sin\theta + T_2\sin\phi = Mg$.

So: $T_2\cos\phi\tan\theta + T_2\sin\phi = Mg$, giving $T_2(\cos\phi\tan\theta + \sin\phi) = Mg$.

Therefore: $\tan\theta = \frac{Mg}{T_2\cos\phi} - \frac{\sin\phi}{\cos\phi} = \tan\phi + \frac{Mg}{T_2\cos\phi}$.

\textbf{(ii)} From the geometry, if $P$ is at height $\frac{2h}{3}$ above the floor, then the vertical distance constraints and the relationship from part (i) lead to a contradiction, showing this position is impossible.
\end{solution}

\begin{takeaways}
\begin{itemize}
    \item Force resolution: Resolve forces into horizontal and vertical components
    \item Static equilibrium: Sum of forces in each direction equals zero
    \item Geometric constraints: Use geometry to find limitations on positions
\end{itemize}
\end{takeaways}

