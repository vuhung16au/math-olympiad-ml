\subsection{Solution to Problem 34: Integration with Recurrence and Factorial Inequality}

\begin{solution}
\textbf{(i)} Let $u = \sin^{2n}(2\theta)$ and $dv = \sin(2\theta) \, d\theta$.

Then $du = 2n\sin^{2n-1}(2\theta) \cdot 2\cos(2\theta) \, d\theta = 4n\sin^{2n-1}(2\theta)\cos(2\theta) \, d\theta$.

And $v = -\frac{1}{2}\cos(2\theta)$.

Using integration by parts:
\[
I_n = \left[-\frac{1}{2}\sin^{2n}(2\theta)\cos(2\theta)\right]_0^{\pi/2} + 2n\int_0^{\pi/2} \sin^{2n-1}(2\theta)\cos^2(2\theta) \, d\theta
\]

The boundary term is 0. Using $\cos^2(2\theta) = 1 - \sin^2(2\theta)$:
\[
I_n = 2n\int_0^{\pi/2} \sin^{2n-1}(2\theta)(1 - \sin^2(2\theta)) \, d\theta = 2n(I_{n-1} - I_n)
\]

Solving: $I_n = 2nI_{n-1} - 2nI_n$, so $(2n+1)I_n = 2nI_{n-1}$.

Therefore: $I_n = \frac{2n}{2n+1}I_{n-1}$.

\textbf{(ii)} $I_0 = \int_0^{\pi/2} \sin(2\theta) \, d\theta = \left[-\frac{1}{2}\cos(2\theta)\right]_0^{\pi/2} = 1$.

Using the recurrence: $I_n = \frac{2n}{2n+1} \cdot \frac{2(n-1)}{2n-1} \cdot \ldots \cdot \frac{2}{3} \cdot 1 = \frac{2^n n!}{(2n+1) \cdot (2n-1) \cdot \ldots \cdot 3 \cdot 1}$.

The denominator is $(2n+1)!! = \frac{(2n+1)!}{2^n n!}$.

So: $I_n = \frac{2^n n! \cdot 2^n n!}{(2n+1)!} = \frac{2^{2n}(n!)^2}{(2n+1)!}$.

\textbf{(iii)} Using substitution $x = \sin^2\theta$, so $dx = 2\sin\theta\cos\theta \, d\theta$ and $1-x = \cos^2\theta$:
\[
J_n = \int_0^1 x^n(1-x)^n \, dx = \int_0^{\pi/2} \sin^{2n}\theta \cos^{2n}\theta \cdot 2\sin\theta\cos\theta \, d\theta
\]
\[
= 2\int_0^{\pi/2} \sin^{2n+1}\theta \cos^{2n+1}\theta \, d\theta = 2\int_0^{\pi/2} (\sin\theta\cos\theta)^{2n+1} \, d\theta
\]

Using $\sin(2\theta) = 2\sin\theta\cos\theta$: $J_n = 2\int_0^{\pi/2} \left(\frac{1}{2}\sin(2\theta)\right)^{2n+1} \, d\theta = \frac{1}{2^{2n}} I_n$.

So $J_n = \frac{1}{2^{2n}} \cdot \frac{2^{2n}(n!)^2}{(2n+1)!} = \frac{(n!)^2}{(2n+1)!}$.

\textbf{(iv)} Since $0 \le x^n(1-x)^n \le 1$ for $x \in [0,1]$ (maximum occurs at $x = 1/2$), we have $J_n \le \int_0^1 1 \, dx = 1$.

Therefore: $\frac{(n!)^2}{(2n+1)!} \le 1$, so $(n!)^2 \le (2n+1)!$.

Multiplying both sides by $2^{2n}$: $(2^n n!)^2 = 2^{2n}(n!)^2 \le 2^{2n}(2n+1)!$.

Actually, we need $(2^n n!)^2 \le (2n+1)!$.

From $J_n = \frac{(n!)^2}{(2n+1)!} \le 1$, we get $(n!)^2 \le (2n+1)!$.

But we need $(2^n n!)^2 = 2^{2n}(n!)^2 \le (2n+1)!$.

This requires $2^{2n} \le \frac{(2n+1)!}{(n!)^2}$, which is true for $n \ge 1$ (can be verified).

Actually, a direct approach: $(2n+1)! = (2n+1)(2n)(2n-1)\ldots(3)(2)(1)$.

We have $(2^n n!)^2 = 2^{2n}(n!)^2 = 2^{2n}(1 \cdot 2 \cdot \ldots \cdot n)^2$.

Comparing term by term, $(2n+1)!$ contains factors that are at least as large, so the inequality holds.
\end{solution}

\begin{takeaways}
\begin{itemize}
    \item Integration by parts: Use to derive recurrence relations
    \item Double factorial: $(2n+1)!! = \frac{(2n+1)!}{2^n n!}$
    \item Substitution: $x = \sin^2\theta$ relates polynomial and trigonometric integrals
    \item Bounding integrals: Use maximum value of integrand to bound the integral
\end{itemize}
\end{takeaways}

