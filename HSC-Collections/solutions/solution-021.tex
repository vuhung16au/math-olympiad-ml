\subsection{Solution to Problem 21: Complex Numbers and Region Sketching}

\begin{solution}
	extit{Note: Exponential form is being removed from the 2024 syllabus; this solution stays in modulus--argument (trigonometric) form, though the shorthand is still handy to know.}
We have $\frac{xz+yw}{z} = x + y\frac{w}{z}$.

Since $|w| = |z| = 1$, we have $|\frac{w}{z}| = 1$. Also, $\text{Arg}(\frac{z}{w}) \in (\pi/2, \pi)$ means $\text{Arg}(\frac{w}{z}) = -\text{Arg}(\frac{z}{w}) \in (-\pi, -\pi/2)$.

Since we want the principal argument in $(-\pi, \pi]$, and $-\pi < \text{Arg}(\frac{w}{z}) < -\pi/2$, we can write $\text{Arg}(\frac{w}{z}) = \alpha$ where $-\pi < \alpha < -\pi/2$; this places $\frac{w}{z}$ in the third quadrant.

We want $\text{Arg}(x + y\frac{w}{z}) \in (\pi/2, \pi)$, i.e., $x + y\frac{w}{z}$ should be in the second quadrant.

Let $\frac{w}{z} = \cos\alpha + i\sin\alpha$ where $\alpha \in (-\pi, -\pi/2)$.

Then $x + y\frac{w}{z} = x + y\cos\alpha + iy\sin\alpha$.

For this to be in the second quadrant:
- Real part $< 0$: $x + y\cos\alpha < 0$, so $x < -y\cos\alpha$
- Imaginary part $> 0$: $y\sin\alpha > 0$

Since $\alpha \in (-\pi, -\pi/2)$, we have $\sin\alpha < 0$, so we need $y < 0$ for $y\sin\alpha > 0$.

Also, $\cos\alpha < 0$ (since $\alpha$ is in third quadrant), so $-y\cos\alpha > 0$ when $y < 0$.

Therefore, the region is: $y < 0$ and $x < -y\cos\alpha$ where $\alpha = \text{Arg}(\frac{w}{z})$.

Since $\alpha$ is fixed (determined by the given condition), this describes a half-plane.

The boundary is the line $x = -y\cos\alpha$ (a line through the origin with negative slope), and the region is the half-plane below this line (since $y < 0$ and we need $x < -y\cos\alpha$).

More precisely: The region is $\{(x,y) : y < 0 \text{ and } x + y\cos\alpha < 0\}$, which is a half-plane in the lower half of the $xy$-plane, bounded by a line through the origin.
\end{solution}

\begin{takeaways}
\begin{itemize}
    \item Simplify complex expressions: $\frac{xz+yw}{z} = x + y\frac{w}{z}$
    \item Argument conditions: Determine quadrant from argument range
    \item Region sketching: Identify boundaries and which side satisfies the inequality
\end{itemize}
\end{takeaways}

