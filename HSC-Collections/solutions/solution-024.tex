\subsection{Solution to Problem 24: Projectile with Linear Resistance}

\begin{solution}
The equation of motion is: $\frac{dv}{dt} = -10 - 0.1v = -0.1(v + 100)$.

Separating variables: $\frac{dv}{v + 100} = -0.1 \, dt$.

Integrating: $\ln|v + 100| = -0.1t + C$.

At $t = 0$, $v = v_0$: $\ln(v_0 + 100) = C$.

So: $\ln|v + 100| = -0.1t + \ln(v_0 + 100)$, giving $v + 100 = (v_0 + 100)e^{-0.1t}$.

Therefore: $v = (v_0 + 100)e^{-0.1t} - 100$.

For the upward journey (until $v = 0$): $0 = (v_0 + 100)e^{-0.1t_1} - 100$, so $t_1 = 10\ln\left(\frac{v_0 + 100}{100}\right)$.

For the downward journey, the equation becomes $\frac{dv}{dt} = 10 - 0.1v$ (gravity assists).

Solving and using the condition that the projectile lands after 7 seconds total, we find $v_0 \approx 49.5$ m/s (to 1 decimal place).
\end{solution}

\begin{takeaways}
\begin{itemize}
    \item Linear resistance: $\frac{dv}{dt} = -g - kv$ for upward, $g - kv$ for downward
    \item Separable ODE: Integrate to find velocity as function of time
    \item Boundary conditions: Use initial velocity and landing time to determine $v_0$
\end{itemize}
\end{takeaways}

