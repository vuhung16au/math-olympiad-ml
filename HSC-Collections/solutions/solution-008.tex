\subsection{Solution to Problem 8: Triangle Inequality and Rectangular Prism}

\begin{solution}
\textbf{(i)} The surface area is $S = 2(ab + bc + ca)$.

The arithmetic mean of $ab$, $bc$, and $ca$ is $A = \frac{ab + bc + ca}{3} = \frac{S}{6}$.

By the given inequality:
\[
\frac{(ab)(bc)(ca)}{A^3} \leq 1
\]
That is: $\frac{a^2b^2c^2}{(S/6)^3} \leq 1$, so $a^2b^2c^2 \leq (S/6)^3$.

Taking square roots: $abc \leq (S/6)^{3/2}$.

\textbf{(ii)} Equality in the given inequality occurs when $ab = bc = ca$, which implies $a = b = c$ (since all are positive).

When $a = b = c$, the rectangular prism is a cube. Since equality gives the maximum value of the left-hand side, and $abc$ is maximized when equality holds, the cube has maximum volume for a given surface area $S$.
\end{solution}

\begin{takeaways}
\begin{itemize}
    \item AM-GM: For $n$ positive numbers, product $\leq$ (arithmetic mean)$^n$
    \item Equality: Occurs when all numbers are equal
    \item Optimization: Maximum volume for fixed surface area occurs at equality condition
\end{itemize}
\end{takeaways}

