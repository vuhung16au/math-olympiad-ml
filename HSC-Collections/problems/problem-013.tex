\subsection{Problem 13: Circle and Cosine Function}

\begin{problem}
Consider the function $y=\cos(kx)$ where $k>0$. The value of $k$ has been chosen so that a circle can be drawn, centred at the origin, which has exactly two points of intersection with the graph of the function and so that the circle is never above the graph of the function.

The point $P(a, b)$ is the point of intersection in the first quadrant, so $a>0$ and $b>0$, as shown in the diagram below.

\vspace{0.5em}
\noindent
\begin{center}
\includegraphics[width=0.7\textwidth]{images/13-01.png}
\end{center}
\vspace{0.3em}

The vector joining the origin to the point $P(a, b)$ is perpendicular to the tangent to the graph of the function at that point. (Do NOT prove this.)

Show that $k>1$.
\end{problem}

\begin{hint}
At point $P(a,b)$, the radius vector is perpendicular to the tangent. The slope of the radius is $b/a$, and the slope of the tangent is $-k\sin(ka)$. Use the perpendicularity condition and the fact that $P$ lies on both the circle and the curve.
\end{hint}

