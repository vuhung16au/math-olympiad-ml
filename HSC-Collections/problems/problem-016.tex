\subsection{Problem 16: Projectile with Quadratic Resistance}

\begin{problem}
A projectile of mass $M$ kg is launched vertically upwards from the origin with an initial speed $v_{0}$ m s$^{-1}$. The acceleration due to gravity is $g$ m s$^{-2}$.

The projectile experiences a resistive force of magnitude $kMv^{2}$ newtons, where $k$ is a positive constant and $v$ is the speed of the projectile at time $t$ seconds.

\begin{enumerate}
    \item[(i)] The maximum height reached by the particle is $H$ metres. Show that
    \[
    H=\frac{1}{2k}\ln\left(\frac{kv_{0}^{2}+g}{g}\right).
    \]
    
    \item[(ii)] When the projectile lands on the ground, its speed is $v_{1}$ m s$^{-1}$, where $v_{1}$ is less than the magnitude of the terminal velocity. Show that $g(v_{0}^{2}-v_{1}^{2})=kv_{0}^{2}v_{1}^{2}$.
\end{enumerate}
\end{problem}

\begin{hint}
For part (i), use $a = v \frac{dv}{dx}$ and integrate. The resistive force opposes motion, so the acceleration is $-g - kv^2$ on the way up. For part (ii), consider the energy or use the same technique for the downward motion.
\end{hint}

