\subsection{Problem 23: Simple Harmonic Motion}

\begin{problem}
A piston moves with simple harmonic motion between a maximum height of 0.17 m and a minimum height of 0.05 m.

The mass of the piston is 0.8 kg. The piston completes 40 cycles per second.

What is the resultant force on the piston, in newtons, that produces the maximum acceleration of the piston? Give your answer correct to the nearest newton.
\end{problem}

\begin{hint}
For SHM, the maximum acceleration occurs at the extremes. Use $a_{\max} = \omega^2 A$ where $\omega = 2\pi f$ and $A$ is the amplitude. Then use $F = ma$.
\end{hint}

\subsection{Solution to Problem 23: Simple Harmonic Motion}

\begin{solution}
The amplitude is $A = \frac{0.17 - 0.05}{2} = 0.06$ m.

The frequency is $f = 40$ Hz, so $\omega = 2\pi f = 80\pi$ rad/s.

For SHM, the maximum acceleration is $a_{\max} = \omega^2 A = (80\pi)^2 \times 0.06 = 6400\pi^2 \times 0.06 = 384\pi^2$ m/s$^2$.

The maximum force is $F_{\max} = ma_{\max} = 0.8 \times 384\pi^2 = 307.2\pi^2 \approx 3032$ N.

To the nearest newton: $F_{\max} \approx 3032$ N.
\end{solution}

\begin{takeaways}
\begin{itemize}
    \item SHM amplitude: $A = \frac{\text{max} - \text{min}}{2}$
    \item Maximum acceleration: $a_{\max} = \omega^2 A$ where $\omega = 2\pi f$
    \item Force: $F = ma$ for maximum acceleration
\end{itemize}
\end{takeaways}

