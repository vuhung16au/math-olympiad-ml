\subsection{Problem 11: Vectors and Complex Numbers}

\begin{problem}
Consider the three vectors $\mathbf{a}=\vec{OA}$, $\mathbf{b}=\vec{OB}$ and $\mathbf{c}=\vec{OC}$, where $O$ is the origin and the points $A$, $B$ and $C$ are all different from each other and the origin.

The point $M$ is the point such that $\frac{1}{2}(\mathbf{a}+\mathbf{b}) = \vec{OM}$.

\begin{enumerate}
    \item[(i)] Show that $M$ lies on the line passing through $A$ and $B$.
    
    \item[(ii)] The point $G$ is the point such that $\frac{1}{3}(\mathbf{a}+\mathbf{b}+\mathbf{c}) = \vec{OG}$. Show that $G$ lies on the line passing through $M$ and $C$, and lies between $M$ and $C$.
    
    \item[(iii)] The complex numbers $x$, $w$ and $z$ are all different and all have modulus 1. Using part (ii), or otherwise, show that $\frac{1}{3}(x+w+z)$ is never a cube root of $xwz$.
\end{enumerate}
\end{problem}

\begin{hint}
For part (i), express $M$ as a linear combination of $A$ and $B$. For part (ii), show that $G$ can be written as a weighted combination of $M$ and $C$. For part (iii), use the geometric interpretation: if $\frac{1}{3}(x+w+z)$ were a cube root of $xwz$, what would that imply about the positions on the unit circle?
\end{hint}

