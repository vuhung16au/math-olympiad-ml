\subsection{Problem 22: Mechanics with Ropes and Forces}

\begin{problem}
A machine $P$ of mass $M$ kg is being lifted using two ropes. Rope $T_1$ is attached to the ceiling $C$ and rope $T_2$ is attached to the floor $F$. The ropes make angles $\theta$ and $\phi$ with the horizontal respectively, as shown in the diagram below.

\vspace{0.5em}
\noindent
\begin{center}
\includegraphics[width=0.7\textwidth]{images/22-01.png}
\end{center}
\vspace{0.3em}

\begin{enumerate}
    \item[(i)] By considering horizontal and vertical components of the forces at $P$, show that
    \[
    \tan \theta = \tan \phi + \frac{Mg}{T_2 \cos \phi}.
    \]
    
    \item[(ii)] Hence, or otherwise, show that the point $P$ cannot be lifted to a position $\frac{2h}{3}$ metres above the floor.
\end{enumerate}
\end{problem}

\begin{hint}
For part (i), resolve forces horizontally and vertically at point $P$. For part (ii), use the relationship from part (i) and consider the geometry constraints involving the heights and angles.
\end{hint}

