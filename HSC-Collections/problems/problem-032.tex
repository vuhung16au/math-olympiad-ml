\subsection{Problem 32: Parallelogram Geometry}

\begin{problem}
Let $OPQR$ be a parallelogram with $\vec{OP} = \mathbf{p}$ and $\vec{OR} = \mathbf{r}$. The point $S$ is the midpoint of $QR$ and $T$ is the intersection of $PR$ and $OS$, as shown in the diagram below.

\vspace{0.5em}
\noindent
\begin{center}
\includegraphics[width=0.6\textwidth]{images/32-01.png}
\end{center}
\vspace{0.3em}

\begin{enumerate}
    \item[(i)] Show that $\vec{OT} = \frac{2}{3}\mathbf{r} + \frac{1}{3}\mathbf{p}$.
    
    \item[(ii)] Using part (i), or otherwise, prove that $T$ is the point that divides the interval $PR$ in the ratio 2:1.
\end{enumerate}
\end{problem}

\begin{hint}
For part (i), express $T$ as a point on both lines $PR$ and $OS$ using parameters, then equate. For part (ii), show that $\vec{PT} = 2\vec{TR}$ or use the section formula.
\end{hint}

\subsection{Solution to Problem 32: Parallelogram Geometry}

\begin{solution}
\textbf{(i)} In parallelogram $OPQR$, we have $\vec{OQ} = \vec{OP} + \vec{OR} = \mathbf{p} + \mathbf{r}$.

Since $S$ is the midpoint of $QR$: $\vec{OS} = \vec{OR} + \frac{1}{2}\vec{RQ} = \mathbf{r} + \frac{1}{2}(\mathbf{p} + \mathbf{r} - \mathbf{r}) = \mathbf{r} + \frac{1}{2}\mathbf{p}$.

Point $T$ lies on both $PR$ and $OS$.

On $PR$: $\vec{OT} = \mathbf{r} + \lambda(\mathbf{p} - \mathbf{r}) = (1-\lambda)\mathbf{r} + \lambda\mathbf{p}$ for some $\lambda$.

On $OS$: $\vec{OT} = \mu\left(\mathbf{r} + \frac{1}{2}\mathbf{p}\right) = \mu\mathbf{r} + \frac{\mu}{2}\mathbf{p}$ for some $\mu$.

Equating: $(1-\lambda)\mathbf{r} + \lambda\mathbf{p} = \mu\mathbf{r} + \frac{\mu}{2}\mathbf{p}$.

So: $1-\lambda = \mu$ and $\lambda = \frac{\mu}{2}$.

Substituting: $1 - \frac{\mu}{2} = \mu$, so $1 = \frac{3\mu}{2}$, giving $\mu = \frac{2}{3}$ and $\lambda = \frac{1}{3}$.

Therefore: $\vec{OT} = \frac{2}{3}\mathbf{r} + \frac{1}{3}\mathbf{p}$.

\textbf{(ii)} On line $PR$: $\vec{OT} = \frac{1}{3}\mathbf{p} + \frac{2}{3}\mathbf{r}$.

This means $T$ divides $PR$ in the ratio $PT : TR = \frac{2}{3} : \frac{1}{3} = 2 : 1$.
\end{solution}

\begin{takeaways}
\begin{itemize}
    \item Parametric form: Express points on lines using parameters
    \item Intersection: Equate parametric forms to find intersection point
    \item Section formula: Use to determine division ratio
\end{itemize}
\end{takeaways}

