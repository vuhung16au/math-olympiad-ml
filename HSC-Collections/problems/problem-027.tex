\subsection{Problem 27: Particle Falling with Resistance}

\begin{problem}
A particle starts from rest and falls through a resisting medium so that its acceleration, in m/s$^2$, is modelled by
\[
a = 10(1 - (kv)^2),
\]
where $v$ is the velocity of the particle in m/s and $k = 0.01$.

Find the velocity of the particle after 5 seconds.
\end{problem}

\begin{hint}
Use $a = \frac{dv}{dt}$ and separate variables. The equation is $\frac{dv}{dt} = 10(1 - 0.0001v^2)$. This is a separable differential equation. Find the terminal velocity first.
\end{hint}

\subsection{Solution to Problem 27: Particle Falling with Resistance}

\begin{solution}
Given $a = 10(1 - (kv)^2)$ where $k = 0.01$, so $a = 10(1 - 0.0001v^2)$.

Using $a = \frac{dv}{dt}$:
\[
\frac{dv}{dt} = 10(1 - 0.0001v^2) = 10 - 0.001v^2
\]

The terminal velocity occurs when $a = 0$: $v_T = \sqrt{\frac{10}{0.001}} = 100$ m/s.

Separating variables:
\[
\frac{dv}{1 - 0.0001v^2} = 10 \, dt
\]

Using partial fractions: $\frac{1}{1 - 0.0001v^2} = \frac{1}{2}\left(\frac{1}{1 - 0.01v} + \frac{1}{1 + 0.01v}\right)$.

Integrating from $v = 0$ at $t = 0$ to $v = v$ at $t = 5$:
\[
\frac{1}{0.02}\ln\left|\frac{1 + 0.01v}{1 - 0.01v}\right| = 10t
\]

At $t = 5$: $\ln\left|\frac{1 + 0.01v}{1 - 0.01v}\right| = 1$, so $\frac{1 + 0.01v}{1 - 0.01v} = e$.

Solving: $1 + 0.01v = e(1 - 0.01v)$, so $v = \frac{e-1}{0.01(e+1)} \approx 46.2$ m/s.
\end{solution}

\begin{takeaways}
\begin{itemize}
    \item Terminal velocity: Found when acceleration is zero
    \item Partial fractions: Use to integrate $\frac{1}{1 - a^2v^2}$
    \item Hyperbolic tangent: The solution involves $\tanh$ function
\end{itemize}
\end{takeaways}

