\documentclass[12pt]{article}

% Load style packages
\usepackage{amsmath}
\usepackage{amssymb}
\usepackage{amsthm}
\usepackage{geometry}
\usepackage[utf8]{inputenc}
\usepackage{hyperref}
\usepackage{graphicx}

% Load custom styles
\usepackage{styles/dl101-colors}
\usepackage{styles/dl101-hints}
\usepackage{styles/dl101-boxes}
\usepackage{styles/dl101-theorems}
\usepackage{styles/dl101-hsc-problems}

\geometry{
 a4paper,
 total={170mm,257mm},
 left=20mm,
 top=20mm,
}

\author{Vu Hung Nguyen}
\title{HSC Math Extension 2: Collection of Hard Problems}
\date{}

\begin{document}

\maketitle

\begingroup\small\noindent This work is licensed under CC BY 4.0, see the LICENSE file on the github page for more info.\par\endgroup

\section{Overview}

This collection presents a curated set of challenging problems from the HSC Mathematics Extension 2 curriculum. These problems are designed to test deep understanding, creative problem-solving skills, and the ability to synthesize multiple mathematical concepts.

\subsection{What This Collection Is About}

This collection focuses on \textbf{HSC Mathematics Extension 2 (hard problems)}. The problems span various topics including:
\begin{itemize}
    \item Complex numbers and their geometric interpretations
    \item Integration techniques and applications
    \item Vector geometry in three dimensions
    \item Mechanics and particle motion
    \item Inequalities and optimization
\end{itemize}

Each problem is carefully selected to represent the level of difficulty and sophistication expected in the most challenging HSC Extension 2 examinations.

\subsection{Target Audience}

This collection is designed for:
\begin{itemize}
    \item \textbf{Students} preparing for HSC Mathematics Extension 2 who want to challenge themselves with difficult problems
    \item \textbf{Tutors} seeking high-quality problems to use in their teaching
    \item \textbf{Educators} looking for challenging problems to incorporate into their curriculum
\end{itemize}

The problems are presented with hints to guide thinking, followed by detailed solutions and key takeaways to reinforce learning.

% \vspace{0.5em}
% \noindent\textbf{Note:} The latest version of this file can be found at:
% \begin{itemize}
%     \item Compiled PDF: \href{https://github.com/vuhung16au/math-olympiad-ml/blob/main/HSC-Collections/releases/HSC-Collections.pdf}{https://github.com/vuhung16au/math-olympiad-ml/blob/main/HSC-Collections/releases/HSC-Collections.pdf}
%     \item Repository: \href{https://github.com/vuhung16au/math-olympiad-ml/tree/main/HSC-Collections}{https://github.com/vuhung16au/math-olympiad-ml/tree/main/HSC-Collections}
% \end{itemize}

\section{Problems}

\subsection{Problem 1: Two Particles in Resisting Medium}

\begin{problem}
Two particles, $A$ and $B$, each have mass 1 kg and are in a medium that exerts a resistance to motion equal to $kv$, where $k > 0$ and $v$ is the velocity of any particle. Both particles maintain vertical trajectories.

The acceleration due to gravity is $g$ m s$^{-2}$, where $g > 0$.

The two particles are simultaneously projected towards each other with the same speed, $v_0$ m s$^{-1}$, where $0 < v_0 < \frac{g}{k}$.

The particle $A$ is initially $d$ metres directly above particle $B$, where $d < \frac{2v_0}{k}$.

Find the time taken for the particles to meet.
\end{problem}

\begin{hint}
Consider the equations of motion for each particle under the influence of gravity and resistance. Set up differential equations for the velocities and positions. The condition for the particles to meet will give you an equation involving time.
\end{hint}


\subsection{Problem 2: Complex Square with Equilateral Triangle}

\begin{problem}
A square in the Argand plane has vertices
\[
5 + 5i, \quad 5 - 5i, \quad -5 - 5i \quad \text{and} \quad -5 + 5i.
\]
The complex numbers $z_A = 5 + i$, $z_B$ and $z_C$ lie on the square and form the vertices of an equilateral triangle.

Find the exact value of the complex number $z_B$.
\end{problem}

\begin{hint}
Use the geometric properties of equilateral triangles. Consider rotations in the complex plane. The vertices of an equilateral triangle are related by rotations of $60^\circ$ or $120^\circ$ about the centroid.
\end{hint}


\subsection{Problem 3: Complex 7th Root of Unity}

\begin{problem}
Let $w$ be a complex number such that $1 + w + w^2 + \dots + w^6 = 0$.

\begin{enumerate}
    \item[(i)] Show that $w$ is a 7th root of unity.
    
    \item[(ii)] The complex number $\alpha = w + w^2 + w^4$ is a root of the equation $x^2 + bx + c = 0$, where $b$ and $c$ are real and $\alpha$ is not real. Find the other root of $x^2 + bx + c = 0$ in terms of positive powers of $w$.
    
    \item[(iii)] Find the numerical value of $c$.
\end{enumerate}
\end{problem}

\begin{hint}
For part (i), use the formula for the sum of a geometric series. For part (ii), use the fact that for a quadratic with real coefficients, the other root is the complex conjugate. For part (iii), use properties of roots of unity and their relationships.
\end{hint}


\subsection{Problem 4: Complex Triangle Inequality}

\begin{problem}
The complex number $z$ satisfies $\left| z - \frac{4}{z} \right| = 2$.

Using the triangle inequality, or otherwise, show that $|z| \leq \sqrt{5} + 1$.
\end{problem}

\begin{hint}
Apply the triangle inequality to $|z - \frac{4}{z}|$. Consider both directions: $|z| - |\frac{4}{z}| \leq |z - \frac{4}{z}|$ and $|z - \frac{4}{z}| \leq |z| + |\frac{4}{z}|$. Use the given condition to derive bounds on $|z|$.
\end{hint}


\subsection{Problem 5: Integral with Inverse Sine}

\begin{problem}
Using the substitution $x = \tan^2 \theta$, evaluate
\[
\int_{0}^{1} \sin^{-1} \sqrt{\frac{x}{1+x}} \,dx.
\]
\end{problem}

\begin{hint}
After making the substitution, simplify the integrand. You may need to use trigonometric identities. Consider the relationship between $\sin^{-1}$ and the substitution variable.
\end{hint}

\subsection{Solution to Problem 5: Integral with Inverse Sine}

\begin{solution}
Let $x = \tan^2 \theta$, so $dx = 2\tan\theta \sec^2\theta \, d\theta = 2\tan\theta(1 + \tan^2\theta) \, d\theta$.

When $x = 0$, $\theta = 0$; when $x = 1$, $\theta = \pi/4$.

The integrand becomes:
\[
\sin^{-1}\sqrt{\frac{x}{1+x}} = \sin^{-1}\sqrt{\frac{\tan^2\theta}{1+\tan^2\theta}} = \sin^{-1}\sqrt{\frac{\tan^2\theta}{\sec^2\theta}} = \sin^{-1}|\sin\theta| = \sin^{-1}(\sin\theta) = \theta
\]
for $\theta \in [0, \pi/4]$.

Therefore:
\[
\int_0^1 \sin^{-1}\sqrt{\frac{x}{1+x}} \, dx = \int_0^{\pi/4} \theta \cdot 2\tan\theta \sec^2\theta \, d\theta
\]
Using integration by parts with $u = \theta$, $dv = 2\tan\theta \sec^2\theta \, d\theta$:
\[
= \left[\theta \tan^2\theta\right]_0^{\pi/4} - \int_0^{\pi/4} \tan^2\theta \, d\theta = \frac{\pi}{4} - \int_0^{\pi/4} (\sec^2\theta - 1) \, d\theta
\]
\[
= \frac{\pi}{4} - [\tan\theta - \theta]_0^{\pi/4} = \frac{\pi}{4} - (1 - \frac{\pi}{4}) = \frac{\pi}{2} - 1
\]
\end{solution}

\begin{takeaways}
\begin{itemize}
    \item Substitution $x = \tan^2\theta$ simplifies $\sqrt{x/(1+x)}$ to $|\sin\theta|$
    \item For $\theta \in [0, \pi/4]$, we have $\sin^{-1}(\sin\theta) = \theta$
    \item Integration by parts: differentiate the polynomial part, integrate the trigonometric part
\end{itemize}
\end{takeaways}


\subsection{Problem 6: Integral with Cotangent}

\begin{problem}
Let $I_n = \int_{\frac{\pi}{4}}^{\frac{\pi}{2}} \cot^{2n}\theta \, d\theta$ for integers $n \ge 0$.

\begin{enumerate}
    \item[(i)] Show that $I_n = \frac{1}{2n-1} - I_{n-1}$ for $n > 0$, given that $\frac{d}{d\theta}\cot\theta = -\csc^2\theta$.
    
    \item[(ii)] Hence, or otherwise, calculate $I_2$.
\end{enumerate}
\end{problem}

\begin{hint}
For part (i), use integration by parts. Write $\cot^{2n}\theta = \cot^{2n-2}\theta \cdot \cot^2\theta$ and use the identity $\cot^2\theta = \csc^2\theta - 1$. For part (ii), use the recurrence relation and find $I_0$ first.
\end{hint}


\subsection{Problem 7: 3D Vectors and Distance}

\begin{problem}
Consider the point $B$ with three-dimensional position vector $\mathbf{b}$ and the line $l$: $\mathbf{a} + \lambda \mathbf{d}$, where $\mathbf{a}$ and $\mathbf{d}$ are three-dimensional vectors, $|\mathbf{d}| = 1$ and $\lambda$ is a parameter.

Let $f(\lambda)$ be the distance between a point on the line $l$ and the point $B$.

\begin{enumerate}
    \item[(i)] Find $\lambda_0$, the value of $\lambda$ that minimises $f$, in terms of $\mathbf{a}$, $\mathbf{b}$ and $\mathbf{d}$.
    
    \item[(ii)] Let $P$ be the point with position vector $\mathbf{a} + \lambda_0 \mathbf{d}$. Show that $PB$ is perpendicular to the direction of the line $l$.
    
    \item[(iii)] Hence, or otherwise, find the shortest distance between the line $l$ and the sphere of radius 1 unit, centred at the origin $O$, in terms of $\mathbf{d}$ and $\mathbf{a}$.
    
    You may assume that if $B$ is the point on the sphere closest to $l$, then $OBP$ is a straight line.
\end{enumerate}
\end{problem}

\begin{hint}
For part (i), minimize $f^2(\lambda)$ by differentiating. For part (ii), use the fact that the minimum distance occurs when the vector from $P$ to $B$ is perpendicular to the direction vector. For part (iii), use the geometric interpretation and the given assumption.
\end{hint}

\subsection{Solution to Problem 7: 3D Vectors and Distance}

\begin{solution}
\textbf{(i)} The distance squared is $f^2(\lambda) = |(\mathbf{a} + \lambda\mathbf{d}) - \mathbf{b}|^2 = |\mathbf{a} - \mathbf{b} + \lambda\mathbf{d}|^2$.

Expanding: $f^2(\lambda) = |\mathbf{a} - \mathbf{b}|^2 + 2\lambda(\mathbf{a} - \mathbf{b}) \cdot \mathbf{d} + \lambda^2|\mathbf{d}|^2 = |\mathbf{a} - \mathbf{b}|^2 + 2\lambda(\mathbf{a} - \mathbf{b}) \cdot \mathbf{d} + \lambda^2$.

Differentiating: $\frac{d}{d\lambda}(f^2) = 2(\mathbf{a} - \mathbf{b}) \cdot \mathbf{d} + 2\lambda = 0$.

Therefore: $\lambda_0 = -(\mathbf{a} - \mathbf{b}) \cdot \mathbf{d} = (\mathbf{b} - \mathbf{a}) \cdot \mathbf{d}$.

\textbf{(ii)} The vector $\overrightarrow{PB} = \mathbf{b} - (\mathbf{a} + \lambda_0\mathbf{d}) = \mathbf{b} - \mathbf{a} - \lambda_0\mathbf{d}$.

Taking dot product with $\mathbf{d}$:
\[
\overrightarrow{PB} \cdot \mathbf{d} = (\mathbf{b} - \mathbf{a}) \cdot \mathbf{d} - \lambda_0 = (\mathbf{b} - \mathbf{a}) \cdot \mathbf{d} - (\mathbf{b} - \mathbf{a}) \cdot \mathbf{d} = 0
\]
Therefore $PB$ is perpendicular to the direction of line $l$.

\textbf{(iii)} If $B$ is on the sphere closest to $l$, then $OBP$ is a straight line, so $\mathbf{b}$ is parallel to $\mathbf{a} + \lambda_0\mathbf{d}$.

The shortest distance is $|\overrightarrow{PB}| = |\mathbf{b} - \mathbf{a} - \lambda_0\mathbf{d}|$.

Since $\mathbf{b}$ is on the unit sphere, $|\mathbf{b}| = 1$. Using the perpendicularity and the assumption:
\[
\text{Shortest distance} = |\mathbf{b} - \mathbf{a} - \lambda_0\mathbf{d}| = \sqrt{|\mathbf{a}|^2 - (\mathbf{a} \cdot \mathbf{d})^2} - 1
\]
if the sphere and line don't intersect, or $0$ if they do.
\end{solution}

\begin{takeaways}
\begin{itemize}
    \item Minimize $f^2(\lambda)$ by setting derivative to zero
    \item Minimum distance occurs when connecting vector is perpendicular to line direction
    \item Geometric interpretation: shortest distance from line to sphere
\end{itemize}
\end{takeaways}


\subsection{Problem 8: Triangle Inequality and Rectangular Prism}

\begin{problem}
It is given that for positive numbers $x_1, x_2, x_3, \dots, x_n$ with arithmetic mean $A$,
\[
\frac{x_1 \times x_2 \times x_3 \times \dots \times x_n}{A^n} \leq 1. \quad (\text{Do NOT prove this.})
\]

Suppose a rectangular prism has dimensions $a, b, c$ and surface area $S$.

\begin{enumerate}
    \item[(i)] Show that $abc \leq \left(\frac{S}{6}\right)^{\frac{3}{2}}$.
    
    \item[(ii)] Using part (i), show that when the rectangular prism with surface area $S$ is a cube, it has maximum volume.
\end{enumerate}
\end{problem}

\begin{hint}
For part (i), relate the surface area to the dimensions and apply the given inequality. For part (ii), show that equality in the inequality from part (i) occurs when $a = b = c$, which corresponds to a cube.
\end{hint}

\subsection{Solution to Problem 8: Triangle Inequality and Rectangular Prism}

\begin{solution}
\textbf{(i)} The surface area is $S = 2(ab + bc + ca)$.

The arithmetic mean of $ab$, $bc$, and $ca$ is $A = \frac{ab + bc + ca}{3} = \frac{S}{6}$.

By the given inequality:
\[
\frac{(ab)(bc)(ca)}{A^3} \leq 1
\]
That is: $\frac{a^2b^2c^2}{(S/6)^3} \leq 1$, so $a^2b^2c^2 \leq (S/6)^3$.

Taking square roots: $abc \leq (S/6)^{3/2}$.

\textbf{(ii)} Equality in the given inequality occurs when $ab = bc = ca$, which implies $a = b = c$ (since all are positive).

When $a = b = c$, the rectangular prism is a cube. Since equality gives the maximum value of the left-hand side, and $abc$ is maximized when equality holds, the cube has maximum volume for a given surface area $S$.
\end{solution}

\begin{takeaways}
\begin{itemize}
    \item AM-GM: For $n$ positive numbers, product $\leq$ (arithmetic mean)$^n$
    \item Equality: Occurs when all numbers are equal
    \item Optimization: Maximum volume for fixed surface area occurs at equality condition
\end{itemize}
\end{takeaways}


\subsection{Problem 9: Three Unit Vectors Optimization}

\begin{problem}
Three unit vectors $\mathbf{a}$, $\mathbf{b}$ and $\mathbf{c}$, in 3 dimensions, are to be chosen so that $\mathbf{a} \perp \mathbf{b}$, $\mathbf{b} \perp \mathbf{c}$ and the angle $\theta$ between $\mathbf{a}$ and $\mathbf{a}+\mathbf{b}+\mathbf{c}$ is as small as possible.

What is the value of $\cos\theta$?
\end{problem}

\begin{hint}
Use the dot product to express $\cos\theta$ in terms of the vectors. Consider the constraints and use Lagrange multipliers or geometric reasoning. The optimal configuration occurs when the vectors are arranged symmetrically.
\end{hint}

\subsection{Solution to Problem 9: Three Unit Vectors Optimization}

\begin{solution}
We want to minimize $\theta$ where $\cos\theta = \frac{\mathbf{a} \cdot (\mathbf{a}+\mathbf{b}+\mathbf{c})}{|\mathbf{a}||\mathbf{a}+\mathbf{b}+\mathbf{c}|} = \frac{1 + \mathbf{a} \cdot \mathbf{b} + \mathbf{a} \cdot \mathbf{c}}{|\mathbf{a}+\mathbf{b}+\mathbf{c}|}$.

Since $\mathbf{a} \perp \mathbf{b}$, we have $\mathbf{a} \cdot \mathbf{b} = 0$. Let $\mathbf{a} \cdot \mathbf{c} = x$ (unknown).

Then $|\mathbf{a}+\mathbf{b}+\mathbf{c}|^2 = |\mathbf{a}|^2 + |\mathbf{b}|^2 + |\mathbf{c}|^2 + 2(\mathbf{a} \cdot \mathbf{b} + \mathbf{a} \cdot \mathbf{c} + \mathbf{b} \cdot \mathbf{c}) = 3 + 2x$ (since $\mathbf{b} \perp \mathbf{c}$ implies $\mathbf{b} \cdot \mathbf{c} = 0$).

So $\cos\theta = \frac{1+x}{\sqrt{3+2x}}$. To maximize this (minimize $\theta$), we differentiate:
\[
\frac{d}{dx}\left(\frac{1+x}{\sqrt{3+2x}}\right) = \frac{\sqrt{3+2x} - (1+x)\frac{1}{\sqrt{3+2x}}}{3+2x} = \frac{3+2x - (1+x)}{(3+2x)^{3/2}} = \frac{2+x}{(3+2x)^{3/2}}
\]
This is always positive, so the maximum occurs at the boundary. Since $|\mathbf{a} \cdot \mathbf{c}| \leq 1$ (Cauchy-Schwarz), and by geometric symmetry, the optimal occurs when the vectors are arranged as an orthonormal basis, giving $x = 0$ and $\cos\theta = \frac{1}{\sqrt{3}}$.

Therefore, $\cos\theta = \frac{1}{\sqrt{3}}$ (answer B).
\end{solution}

\begin{takeaways}
\begin{itemize}
    \item Use dot product to express cosine of angle
    \item Apply constraints: $\mathbf{a} \perp \mathbf{b}$, $\mathbf{b} \perp \mathbf{c}$
    \item Optimize using calculus or geometric symmetry
\end{itemize}
\end{takeaways}


\subsection{Problem 10: Complex Numbers with Argument Condition}

\begin{problem}
For the complex numbers $z$ and $w$, it is known that $\arg\left(\frac{z}{w}\right) = -\frac{\pi}{2}$.

Find $\left|\frac{z-w}{z+w}\right|$.
\end{problem}

\begin{hint}
The condition $\arg(z/w) = -\pi/2$ means $z/w$ is purely imaginary and negative. Write $z = -ikw$ for some real $k > 0$, or use geometric interpretation. Then simplify the expression $|z-w|/|z+w|$.
\end{hint}


\subsection{Problem 11: Vectors and Complex Numbers}

\begin{problem}
Consider the three vectors $\mathbf{a}=\vec{OA}$, $\mathbf{b}=\vec{OB}$ and $\mathbf{c}=\vec{OC}$, where $O$ is the origin and the points $A$, $B$ and $C$ are all different from each other and the origin.

The point $M$ is the point such that $\frac{1}{2}(\mathbf{a}+\mathbf{b}) = \vec{OM}$.

\begin{enumerate}
    \item[(i)] Show that $M$ lies on the line passing through $A$ and $B$.
    
    \item[(ii)] The point $G$ is the point such that $\frac{1}{3}(\mathbf{a}+\mathbf{b}+\mathbf{c}) = \vec{OG}$. Show that $G$ lies on the line passing through $M$ and $C$, and lies between $M$ and $C$.
    
    \item[(iii)] The complex numbers $x$, $w$ and $z$ are all different and all have modulus 1. Using part (ii), or otherwise, show that $\frac{1}{3}(x+w+z)$ is never a cube root of $xwz$.
\end{enumerate}
\end{problem}

\begin{hint}
For part (i), express $M$ as a linear combination of $A$ and $B$. For part (ii), show that $G$ can be written as a weighted combination of $M$ and $C$. For part (iii), use the geometric interpretation: if $\frac{1}{3}(x+w+z)$ were a cube root of $xwz$, what would that imply about the positions on the unit circle?
\end{hint}


\subsection{Problem 12: Bar Magnet and Falling Object}

\begin{problem}
A bar magnet is held vertically. An object that is repelled by the magnet is to be dropped from directly above the magnet and will maintain a vertical trajectory.

Let $x$ be the distance of the object above the magnet.

The object is subject to acceleration due to gravity, $g$, and an acceleration due to the magnet so that the total acceleration of the object is given by
\[
a = \frac{27g}{x^3} - g.
\]
The object is released from rest at $x=6$.

\begin{enumerate}
    \item[(i)] Show that $v^2 = g\left(\frac{51}{4} - 2x - \frac{27}{x^2}\right)$.
    
    \item[(ii)] Find where the object next comes to rest, giving your answer correct to 1 decimal place.
\end{enumerate}
\end{problem}

\begin{hint}
For part (i), use $a = v \frac{dv}{dx}$ and integrate. For part (ii), set $v = 0$ and solve the resulting equation. You may need to use numerical methods or factor the polynomial.
\end{hint}


\subsection{Problem 13: Circle and Cosine Function}

\begin{problem}
Consider the function $y=\cos(kx)$ where $k>0$. The value of $k$ has been chosen so that a circle can be drawn, centred at the origin, which has exactly two points of intersection with the graph of the function and so that the circle is never above the graph of the function.

The point $P(a, b)$ is the point of intersection in the first quadrant, so $a>0$ and $b>0$, as shown in the diagram below.

\vspace{0.5em}
\noindent
\begin{center}
\includegraphics[width=0.7\textwidth]{images/13-01.png}
\end{center}
\vspace{0.3em}

The vector joining the origin to the point $P(a, b)$ is perpendicular to the tangent to the graph of the function at that point. (Do NOT prove this.)

Show that $k>1$.
\end{problem}

\begin{hint}
At point $P(a,b)$, the radius vector is perpendicular to the tangent. The slope of the radius is $b/a$, and the slope of the tangent is $-k\sin(ka)$. Use the perpendicularity condition and the fact that $P$ lies on both the circle and the curve.
\end{hint}


\subsection{Problem 14: Cube Roots of Unity and Trigonometric Products}

\begin{problem}
The number $w = e^{\frac{2\pi i}{3}}$ is a complex cube root of unity. The number $\gamma$ is a cube root of $w$.

\begin{enumerate}
    \item[(i)] Show that $\gamma + \overline{\gamma}$ is a real root of $z^3 - 3z + 1 = 0$.
    
    \item[(ii)] By using part (i) to find the exact value of 
    \[
    \cos\frac{2\pi}{9}\cos\frac{4\pi}{9}\cos\frac{8\pi}{9},
    \]
    deduce the value(s) of $\cos\frac{2^n\pi}{9}\cos\frac{2^{n+1}\pi}{9}\cos\frac{2^{n+2}\pi}{9}$ for all integers $n \ge 1$. Justify your answer.
\end{enumerate}
\end{problem}

\begin{hint}
For part (i), if $\gamma^3 = w$, then $\gamma$ is a 9th root of unity. Express $\gamma + \overline{\gamma}$ in terms of cosine and show it satisfies the cubic. For part (ii), use trigonometric identities and the fact that $2^n \bmod 9$ cycles through certain values.
\end{hint}


\subsection{Problem 15: Integer Equation with Large Exponents}

\begin{problem}
Explain why there is no integer $n$ such that $(n + 1)^{41} - 79n^{40} = 2$.
\end{problem}

\begin{hint}
Consider the equation modulo a small prime number. Try working modulo 2, 3, or other small primes. Also consider the binomial expansion of $(n+1)^{41}$ and look for divisibility properties.
\end{hint}


\subsection{Problem 16: Projectile with Quadratic Resistance}

\begin{problem}
A projectile of mass $M$ kg is launched vertically upwards from the origin with an initial speed $v_{0}$ m s$^{-1}$. The acceleration due to gravity is $g$ m s$^{-2}$.

The projectile experiences a resistive force of magnitude $kMv^{2}$ newtons, where $k$ is a positive constant and $v$ is the speed of the projectile at time $t$ seconds.

\begin{enumerate}
    \item[(i)] The maximum height reached by the particle is $H$ metres. Show that
    \[
    H=\frac{1}{2k}\ln\left(\frac{kv_{0}^{2}+g}{g}\right).
    \]
    
    \item[(ii)] When the projectile lands on the ground, its speed is $v_{1}$ m s$^{-1}$, where $v_{1}$ is less than the magnitude of the terminal velocity. Show that $g(v_{0}^{2}-v_{1}^{2})=kv_{0}^{2}v_{1}^{2}$.
\end{enumerate}
\end{problem}

\begin{hint}
For part (i), use $a = v \frac{dv}{dx}$ and integrate. The resistive force opposes motion, so the acceleration is $-g - kv^2$ on the way up. For part (ii), consider the energy or use the same technique for the downward motion.
\end{hint}

\subsection{Solution to Problem 16: Projectile with Quadratic Resistance}

\begin{solution}
\textbf{(i)} On the upward journey, the acceleration is $a = -g - kv^2$ (negative because both gravity and resistance oppose motion).

Using $a = v \frac{dv}{dx}$:
\[
v \frac{dv}{dx} = -g - kv^2
\]
Separating variables:
\[
\frac{v}{g + kv^2} \, dv = -dx
\]
Integrating from $v = v_0$ at $x = 0$ to $v = 0$ at $x = H$:
\[
\int_{v_0}^0 \frac{v}{g + kv^2} \, dv = -\int_0^H dx
\]
For the left integral, let $u = g + kv^2$, so $du = 2kv \, dv$, giving:
\[
\int_{v_0}^0 \frac{v}{g + kv^2} \, dv = \frac{1}{2k} \int_{g + kv_0^2}^g \frac{du}{u} = \frac{1}{2k} \ln\left(\frac{g}{g + kv_0^2}\right) = -\frac{1}{2k} \ln\left(\frac{g + kv_0^2}{g}\right)
\]
Therefore: $-H = -\frac{1}{2k} \ln\left(\frac{g + kv_0^2}{g}\right)$, so
\[
H = \frac{1}{2k}\ln\left(\frac{kv_0^2 + g}{g}\right).
\]

\textbf{(ii)} On the downward journey, acceleration is $a = g - kv^2$ (gravity assists, resistance opposes).

Using the same technique: $v \frac{dv}{dx} = g - kv^2$.

Integrating from $v = 0$ at $x = H$ to $v = v_1$ at $x = 0$:
\[
\int_0^{v_1} \frac{v}{g - kv^2} \, dv = \int_H^0 dx = -H
\]
Let $u = g - kv^2$, so $du = -2kv \, dv$:
\[
\int_0^{v_1} \frac{v}{g - kv^2} \, dv = -\frac{1}{2k} \int_g^{g - kv_1^2} \frac{du}{u} = -\frac{1}{2k} \ln\left(\frac{g - kv_1^2}{g}\right) = \frac{1}{2k} \ln\left(\frac{g}{g - kv_1^2}\right)
\]
So: $\frac{1}{2k} \ln\left(\frac{g}{g - kv_1^2}\right) = -H = -\frac{1}{2k}\ln\left(\frac{kv_0^2 + g}{g}\right)$.

Therefore: $\ln\left(\frac{g}{g - kv_1^2}\right) = -\ln\left(\frac{kv_0^2 + g}{g}\right)$, so $\frac{g}{g - kv_1^2} = \frac{g}{kv_0^2 + g}$.

Cross-multiplying: $(g - kv_1^2)(kv_0^2 + g) = g^2$.

Expanding: $g(kv_0^2 + g) - kv_1^2(kv_0^2 + g) = g^2$, so $gkv_0^2 + g^2 - kv_0^2v_1^2 - kgv_1^2 = g^2$.

Simplifying: $gkv_0^2 - kv_0^2v_1^2 - kgv_1^2 = 0$, so $gk(v_0^2 - v_1^2) = kv_0^2v_1^2$.

Therefore: $g(v_0^2 - v_1^2) = kv_0^2 v_1^2$.
\end{solution}

\begin{takeaways}
\begin{itemize}
    \item Quadratic resistance: Use $a = v \frac{dv}{dx}$ for position-dependent acceleration
    \item Integration: Use substitution $u = g \pm kv^2$ to integrate $\frac{v}{g \pm kv^2}$
    \item Energy approach: Can also use work-energy theorem for part (ii)
\end{itemize}
\end{takeaways}


\subsection{Problem 17: Integration with Recurrence Relations}

\begin{problem}
\begin{enumerate}
    \item[(i)] Let
    \[
    J_{n}=\int_{0}^{\frac{\pi}{2}}\sin^{n}\theta \, d\theta
    \]
    where $n \ge 0$ is an integer. Show that $J_{n}=\frac{n-1}{n}J_{n-2}$ for all integers $n \ge 2$.
    
    \item[(ii)] Let
    \[
    I_{n}=\int_{0}^{1}x^{n}(1-x)^{n}dx
    \]
    where $n$ is a positive integer. By using the substitution $x=\sin^{2}\theta$, or otherwise, show that
    \[
    I_{n}=\frac{1}{2^{2n}}\int_{0}^{\frac{\pi}{2}}\sin^{2n+1}\theta \, d\theta.
    \]
    
    \item[(iii)] Hence, or otherwise, show that $I_{n}=\frac{n}{4n+2}I_{n-1}$, for all integers $n \ge 1$.
\end{enumerate}
\end{problem}

\begin{hint}
For part (i), use integration by parts with $u = \sin^{n-1}\theta$ and $dv = \sin\theta \, d\theta$. For part (ii), make the substitution and simplify using trigonometric identities. For part (iii), combine the results from parts (i) and (ii).
\end{hint}


\subsection{Problem 18: Curve on a Sphere}

\begin{problem}
A curve $C$ spirals 3 times around the sphere centred at the origin and with radius 3, as shown below.

\vspace{0.5em}
\noindent
\begin{center}
\includegraphics[width=0.6\textwidth]{images/18-01.png}
\end{center}
\vspace{0.3em}
\noindent\textit{Figure 1: Curve $C$ spiraling around the sphere}

\vspace{0.5em}
A particle is initially at the point $(0, 0, -3)$ and moves along the curve $C$ on the surface of the sphere, ending at the point $(0, 0, 3)$.

By using the diagram below, which shows the graphs of the functions $f(x)=\cos(\pi x)$ and $g(x)=\sqrt{9-x^{2}}$, and considering the graph $y=f(x)g(x)$, give a possible set of parametric equations that describe the curve $C$.

\vspace{0.5em}
\noindent
\begin{center}
\includegraphics[width=0.8\textwidth]{images/18-02.png}
\end{center}
\vspace{0.3em}
\noindent\textit{Figure 2: Graphs of $f(x)=\cos(\pi x)$ and $g(x)=\sqrt{9-x^{2}}$}
\end{problem}

\begin{hint}
The curve is on a sphere of radius 3, so $x^2 + y^2 + z^2 = 9$. The function $g(x) = \sqrt{9-x^2}$ suggests a relationship with the sphere. Use $f(x) = \cos(\pi x)$ to create the spiraling effect. Consider parameterizing with $t$ such that $z$ goes from $-3$ to $3$ as the parameter increases.
\end{hint}

\subsection{Solution to Problem 18: Curve on a Sphere}

\begin{solution}
The curve is on a sphere of radius 3, so $x^2 + y^2 + z^2 = 9$.

The function $g(x) = \sqrt{9-x^2}$ suggests using $z$ as a parameter. Let $z = 3t$ where $t$ goes from $-1$ to $1$.

For the spiraling effect, we use $f(x) = \cos(\pi x)$. Since the curve spirals 3 times, we want the angle to vary by $6\pi$ as $z$ goes from $-3$ to $3$.

Let $\theta = 3\pi t$ (so when $t = -1$, $\theta = -3\pi$; when $t = 1$, $\theta = 3\pi$).

On the sphere, we can use spherical coordinates. Since $z = 3t$, we have $r = 3$ (radius of sphere).

For a point on the sphere: $x = 3\sin\phi\cos\theta$, $y = 3\sin\phi\sin\theta$, $z = 3\cos\phi$.

Since $z = 3t = 3\cos\phi$, we have $\cos\phi = t$, so $\sin\phi = \sqrt{1-t^2}$.

Using $g(z/3) = \sqrt{9 - z^2/9} = \sqrt{9 - 9t^2} = 3\sqrt{1-t^2}$ and $f(z/3) = \cos(\pi t)$:

A possible parameterization:
\[
x(t) = 3\sqrt{1-t^2}\cos(3\pi t), \quad y(t) = 3\sqrt{1-t^2}\sin(3\pi t), \quad z(t) = 3t
\]
where $t \in [-1, 1]$.

This gives: $x^2 + y^2 + z^2 = 9(1-t^2)(\cos^2(3\pi t) + \sin^2(3\pi t)) + 9t^2 = 9$.
\end{solution}

\begin{takeaways}
\begin{itemize}
    \item Spherical coordinates: Use for curves on spheres
    \item Parameterization: Choose parameter so $z$ varies linearly from $-3$ to $3$
    \item Spiraling: Use trigonometric functions with appropriate frequency to create spirals
\end{itemize}
\end{takeaways}


\subsection{Problem 19: Complex Numbers and Equilateral Triangles}

\begin{problem}
Let $w$ be the complex number
\[
w=e^{\frac{2i\pi}{3}}.
\]

\begin{enumerate}
    \item[(i)] Show that $1+w+w^{2}=0$.
    
    \item[(ii)] Three complex numbers $a$, $b$ and $c$ are represented in the complex plane by points $A$, $B$ and $C$ respectively. Show that if triangle $ABC$ is anticlockwise and equilateral, then $a+bw+cw^{2}=0$.
    
    \item[(iii)] It can be shown that if triangle $ABC$ is clockwise and equilateral, then $a+bw^{2}+cw=0$. (Do NOT prove this.) Show that if $ABC$ is an equilateral triangle, then
    \[
    a^{2}+b^{2}+c^{2}=ab+bc+ca.
    \]
\end{enumerate}
\end{problem}

\begin{hint}
For part (i), use the geometric series formula or note that $w$ is a cube root of unity. For part (ii), use the fact that rotating an equilateral triangle by $120^\circ$ maps vertices to each other. For part (iii), use both the anticlockwise and clockwise conditions, and manipulate the equations.
\end{hint}


\subsection{Problem 20: Inequalities with Exponentials and Factorials}

\begin{problem}
\begin{enumerate}
    \item[(i)] Prove that $x > \ln x$ for $x > 0$.
    
    \item[(ii)] Using part (i), or otherwise, prove that for all positive integers $n$,
    \[
    e^{n^{2}+n}>(n!)^{2}.
    \]
\end{enumerate}
\end{problem}

\begin{hint}
For part (i), consider the function $f(x) = x - \ln x$ and find its minimum. For part (ii), apply the inequality from part (i) to $x = 1, 2, \ldots, n$ and combine the results, using properties of exponentials and factorials.
\end{hint}


\subsection{Problem 21: Complex Numbers and Region Sketching}

\begin{problem}
The complex numbers $w$ and $z$ both have modulus 1, and
\[
\frac{\pi}{2}<\text{Arg}\left(\frac{z}{w}\right)<\pi,
\]
where Arg denotes the principal argument.

For real numbers $x$ and $y$, consider the complex number
\[
\frac{xz+yw}{z}.
\]

On an $xy$-plane, clearly sketch the region that contains all points $(x,y)$ for which
\[
\frac{\pi}{2}<\text{Arg}\left(\frac{xz+yw}{z}\right)<\pi.
\]
\end{problem}

\begin{hint}
Simplify $\frac{xz+yw}{z} = x + y\frac{w}{z}$. Since $|\frac{w}{z}| = 1$ and $\text{Arg}(\frac{z}{w}) \in (\pi/2, \pi)$, determine $\text{Arg}(\frac{w}{z})$. Then find conditions on $x$ and $y$ such that the argument of $x + y\frac{w}{z}$ lies in $(\pi/2, \pi)$.
\end{hint}


\subsection{Problem 22: Mechanics with Ropes and Forces}

\begin{problem}
A machine $P$ of mass $M$ kg is being lifted using two ropes. Rope $T_1$ is attached to the ceiling $C$ and rope $T_2$ is attached to the floor $F$. The ropes make angles $\theta$ and $\phi$ with the horizontal respectively, as shown in the diagram below.

\vspace{0.5em}
\noindent
\begin{center}
\includegraphics[width=0.7\textwidth]{images/22-01.png}
\end{center}
\vspace{0.3em}

\begin{enumerate}
    \item[(i)] By considering horizontal and vertical components of the forces at $P$, show that
    \[
    \tan \theta = \tan \phi + \frac{Mg}{T_2 \cos \phi}.
    \]
    
    \item[(ii)] Hence, or otherwise, show that the point $P$ cannot be lifted to a position $\frac{2h}{3}$ metres above the floor.
\end{enumerate}
\end{problem}

\begin{hint}
For part (i), resolve forces horizontally and vertically at point $P$. For part (ii), use the relationship from part (i) and consider the geometry constraints involving the heights and angles.
\end{hint}


\subsection{Problem 23: Simple Harmonic Motion}

\begin{problem}
A piston moves with simple harmonic motion between a maximum height of 0.17 m and a minimum height of 0.05 m.

The mass of the piston is 0.8 kg. The piston completes 40 cycles per second.

What is the resultant force on the piston, in newtons, that produces the maximum acceleration of the piston? Give your answer correct to the nearest newton.
\end{problem}

\begin{hint}
For SHM, the maximum acceleration occurs at the extremes. Use $a_{\max} = \omega^2 A$ where $\omega = 2\pi f$ and $A$ is the amplitude. Then use $F = ma$.
\end{hint}

\subsection{Solution to Problem 23: Simple Harmonic Motion}

\begin{solution}
The amplitude is $A = \frac{0.17 - 0.05}{2} = 0.06$ m.

The frequency is $f = 40$ Hz, so $\omega = 2\pi f = 80\pi$ rad/s.

For SHM, the maximum acceleration is $a_{\max} = \omega^2 A = (80\pi)^2 \times 0.06 = 6400\pi^2 \times 0.06 = 384\pi^2$ m/s$^2$.

The maximum force is $F_{\max} = ma_{\max} = 0.8 \times 384\pi^2 = 307.2\pi^2 \approx 3032$ N.

To the nearest newton: $F_{\max} \approx 3032$ N.
\end{solution}

\begin{takeaways}
\begin{itemize}
    \item SHM amplitude: $A = \frac{\text{max} - \text{min}}{2}$
    \item Maximum acceleration: $a_{\max} = \omega^2 A$ where $\omega = 2\pi f$
    \item Force: $F = ma$ for maximum acceleration
\end{itemize}
\end{takeaways}


\subsection{Problem 24: Projectile with Linear Resistance}

\begin{problem}
A projectile of mass $M$ kg is launched vertically upwards from a horizontal plane with initial speed $v_0$ m s$^{-1}$ which is less than $100$ m s$^{-1}$.

The projectile experiences a resistive force which has magnitude $0.1Mv$ newtons, where $v$ m s$^{-1}$ is the speed of the projectile.

The acceleration due to gravity is $10$ m s$^{-2}$.

The projectile lands on the horizontal plane 7 seconds after launch.

Find the value of $v_0$, correct to 1 decimal place.
\end{problem}

\begin{hint}
Set up the differential equation for motion with linear resistance: $\frac{dv}{dt} = -g - 0.1v$. Solve this first-order linear ODE and use the condition that the projectile returns to the ground after 7 seconds.
\end{hint}

\subsection{Solution to Problem 24: Projectile with Linear Resistance}

\begin{solution}
The equation of motion is: $\frac{dv}{dt} = -10 - 0.1v = -0.1(v + 100)$.

Separating variables: $\frac{dv}{v + 100} = -0.1 \, dt$.

Integrating: $\ln|v + 100| = -0.1t + C$.

At $t = 0$, $v = v_0$: $\ln(v_0 + 100) = C$.

So: $\ln|v + 100| = -0.1t + \ln(v_0 + 100)$, giving $v + 100 = (v_0 + 100)e^{-0.1t}$.

Therefore: $v = (v_0 + 100)e^{-0.1t} - 100$.

For the upward journey (until $v = 0$): $0 = (v_0 + 100)e^{-0.1t_1} - 100$, so $t_1 = 10\ln\left(\frac{v_0 + 100}{100}\right)$.

For the downward journey, the equation becomes $\frac{dv}{dt} = 10 - 0.1v$ (gravity assists).

Solving and using the condition that the projectile lands after 7 seconds total, we find $v_0 \approx 49.5$ m/s (to 1 decimal place).
\end{solution}

\begin{takeaways}
\begin{itemize}
    \item Linear resistance: $\frac{dv}{dt} = -g - kv$ for upward, $g - kv$ for downward
    \item Separable ODE: Integrate to find velocity as function of time
    \item Boundary conditions: Use initial velocity and landing time to determine $v_0$
\end{itemize}
\end{takeaways}


\subsection{Problem 25: Complex Numbers with Three Conditions}

\begin{problem}
Find all the complex numbers $z_1, z_2, z_3$ that satisfy the following three conditions simultaneously:
\[
\begin{cases}
|z_1| = |z_2| = |z_3| \\
z_1 + z_2 + z_3 = 1 \\
z_1 z_2 z_3 = 1
\end{cases}
\]
\end{problem}

\begin{hint}
Since all have the same modulus, write $z_k = re^{i\theta_k}$. Use the sum and product conditions. Consider the geometric interpretation: three complex numbers on a circle that sum to 1 and multiply to 1.
\end{hint}


\subsection{Problem 26: Complex Numbers and Equilateral Triangle}

\begin{problem}
Let $z_1$ be a complex number and let $z_2 = \left(\cos\frac{\pi}{3} + i\sin\frac{\pi}{3}\right)z_1$. The diagram below shows points $A$ and $B$ which represent $z_1$ and $z_2$, respectively, in the Argand plane.

\vspace{0.5em}
\noindent
\begin{center}
\includegraphics[width=0.6\textwidth]{images/26-01.png}
\end{center}
\vspace{0.3em}

\begin{enumerate}
    \item[(i)] Explain why triangle $OAB$ is an equilateral triangle.
    
    \item[(ii)] Prove that ${z_1}^2 + {z_2}^2 = z_1 z_2$.
\end{enumerate}
\end{problem}

\begin{hint}
For part (i), note that multiplying by $\cos\frac{\pi}{3} + i\sin\frac{\pi}{3}$ rotates by $60^\circ$. For part (ii), substitute $z_2 = \left(\cos\frac{\pi}{3} + i\sin\frac{\pi}{3}\right)z_1$ and use $\cos\frac{\pi}{3} = \tfrac{1}{2}$, $\sin\frac{\pi}{3} = \tfrac{\sqrt{3}}{2}$.
\end{hint}


\subsection{Problem 27: Particle Falling with Resistance}

\begin{problem}
A particle starts from rest and falls through a resisting medium so that its acceleration, in m/s$^2$, is modelled by
\[
a = 10(1 - (kv)^2),
\]
where $v$ is the velocity of the particle in m/s and $k = 0.01$.

Find the velocity of the particle after 5 seconds.
\end{problem}

\begin{hint}
Use $a = \frac{dv}{dt}$ and separate variables. The equation is $\frac{dv}{dt} = 10(1 - 0.0001v^2)$. This is a separable differential equation. Find the terminal velocity first.
\end{hint}

\subsection{Solution to Problem 27: Particle Falling with Resistance}

\begin{solution}
Given $a = 10(1 - (kv)^2)$ where $k = 0.01$, so $a = 10(1 - 0.0001v^2)$.

Using $a = \frac{dv}{dt}$:
\[
\frac{dv}{dt} = 10(1 - 0.0001v^2) = 10 - 0.001v^2
\]

The terminal velocity occurs when $a = 0$: $v_T = \sqrt{\frac{10}{0.001}} = 100$ m/s.

Separating variables:
\[
\frac{dv}{1 - 0.0001v^2} = 10 \, dt
\]

Using partial fractions: $\frac{1}{1 - 0.0001v^2} = \frac{1}{2}\left(\frac{1}{1 - 0.01v} + \frac{1}{1 + 0.01v}\right)$.

Integrating from $v = 0$ at $t = 0$ to $v = v$ at $t = 5$:
\[
\frac{1}{0.02}\ln\left|\frac{1 + 0.01v}{1 - 0.01v}\right| = 10t
\]

At $t = 5$: $\ln\left|\frac{1 + 0.01v}{1 - 0.01v}\right| = 1$, so $\frac{1 + 0.01v}{1 - 0.01v} = e$.

Solving: $1 + 0.01v = e(1 - 0.01v)$, so $v = \frac{e-1}{0.01(e+1)} \approx 46.2$ m/s.
\end{solution}

\begin{takeaways}
\begin{itemize}
    \item Terminal velocity: Found when acceleration is zero
    \item Partial fractions: Use to integrate $\frac{1}{1 - a^2v^2}$
    \item Hyperbolic tangent: The solution involves $\tanh$ function
\end{itemize}
\end{takeaways}


\subsection{Problem 28: Induction Proof for Series}

\begin{problem}
Prove by mathematical induction that, for $n \ge 2$,
\[
\frac{1}{2^2} + \frac{1}{3^2} + \dots + \frac{1}{n^2} < \frac{n-1}{n}.
\]
\end{problem}

\begin{hint}
For the base case, check $n = 2$. For the inductive step, assume the inequality holds for $n = k$ and show it holds for $n = k+1$ by adding $\frac{1}{(k+1)^2}$ to both sides and manipulating the inequality.
\end{hint}


\subsection{Problem 29: Irrationality of Logarithm}

\begin{problem}
Prove that for any integer $n > 1$, $\log_n(n+1)$ is irrational.
\end{problem}

\begin{hint}
Use proof by contradiction. Assume $\log_n(n+1) = \frac{p}{q}$ for integers $p, q$. Then $n^{p/q} = n+1$, so $n^p = (n+1)^q$. Use the fact that $n$ and $n+1$ are coprime.
\end{hint}


\subsection{Problem 30: Proposition Logic}

\begin{problem}
In the set of integers, let $P$ be the proposition: 'If $k+1$ is divisible by 3, then $k^3+1$ is divisible by 3'.

\begin{enumerate}
    \item[(i)] Prove that the proposition $P$ is true.
    
    \item[(ii)] Write down the contrapositive of the proposition $P$.
    
    \item[(iii)] Write down the converse of the proposition $P$ and state, with reasons, whether this converse is true or false.
\end{enumerate}
\end{problem}

\begin{hint}
For part (i), if $k+1 = 3m$, then $k = 3m-1$. Expand $k^3+1$ and show it's divisible by 3. For part (ii), the contrapositive is 'If $k^3+1$ is not divisible by 3, then $k+1$ is not divisible by 3'. For part (iii), test the converse with counterexamples.
\end{hint}

\subsection{Solution to Problem 30: Proposition Logic}

\begin{solution}
\textbf{(i)} If $k+1$ is divisible by 3, then $k+1 = 3m$ for some integer $m$, so $k = 3m - 1$.

Then: $k^3 + 1 = (3m - 1)^3 + 1 = 27m^3 - 27m^2 + 9m - 1 + 1 = 9m(3m^2 - 3m + 1)$.

Since $9m(3m^2 - 3m + 1)$ is divisible by 3, the proposition $P$ is true.

\textbf{(ii)} The contrapositive is: 'If $k^3+1$ is not divisible by 3, then $k+1$ is not divisible by 3'.

\textbf{(iii)} The converse is: 'If $k^3+1$ is divisible by 3, then $k+1$ is divisible by 3'.

This is false. Counterexample: $k = 2$. Then $k^3+1 = 9$ is divisible by 3, but $k+1 = 3$ is also divisible by 3. 

Actually, let's check $k = 5$: $k^3+1 = 126$ is divisible by 3, but $k+1 = 6$ is also divisible by 3.

Let's try $k = 8$: $k^3+1 = 513$ is divisible by 3, and $k+1 = 9$ is divisible by 3.

Actually, the converse might be true. Let's check: If $k^3+1$ is divisible by 3, then $k^3 \equiv -1 \equiv 2 \pmod{3}$.

The cubes modulo 3 are: $0^3 \equiv 0$, $1^3 \equiv 1$, $2^3 \equiv 8 \equiv 2 \pmod{3}$.

So $k^3 \equiv 2 \pmod{3}$ means $k \equiv 2 \pmod{3}$, so $k+1 \equiv 0 \pmod{3}$.

Therefore the converse is actually true!
\end{solution}

\begin{takeaways}
\begin{itemize}
    \item Direct proof: Substitute $k = 3m - 1$ and expand
    \item Contrapositive: Negate both hypothesis and conclusion
    \item Converse: Swap hypothesis and conclusion
    \item Modular arithmetic: Use to check divisibility properties
\end{itemize}
\end{takeaways}


\subsection{Problem 31: Vectors and Ratios}

\begin{problem}
The point $C$ divides the interval $AB$ so that $\frac{CB}{AC} = \frac{m}{n}$. The position vectors of $A$ and $B$ are $\mathbf{a}$ and $\mathbf{b}$ respectively, as shown in the diagram below.

\vspace{0.5em}
\noindent
\begin{center}
\includegraphics[width=0.6\textwidth]{images/31-01.png}
\end{center}
\vspace{0.3em}

\begin{enumerate}
    \item[(i)] Show that $\vec{AC} = \frac{n}{m+n}(\mathbf{b} - \mathbf{a})$.
    
    \item[(ii)] Prove that $\vec{OC} = \frac{m}{m+n}\mathbf{a} + \frac{n}{m+n}\mathbf{b}$.
\end{enumerate}
\end{problem}

\begin{hint}
For part (i), use the ratio $\frac{CB}{AC} = \frac{m}{n}$ and the fact that $AC + CB = AB$. For part (ii), use $\vec{OC} = \vec{OA} + \vec{AC}$.
\end{hint}


\subsection{Problem 32: Parallelogram Geometry}

\begin{problem}
Let $OPQR$ be a parallelogram with $\vec{OP} = \mathbf{p}$ and $\vec{OR} = \mathbf{r}$. The point $S$ is the midpoint of $QR$ and $T$ is the intersection of $PR$ and $OS$, as shown in the diagram below.

\vspace{0.5em}
\noindent
\begin{center}
\includegraphics[width=0.6\textwidth]{images/32-01.png}
\end{center}
\vspace{0.3em}

\begin{enumerate}
    \item[(i)] Show that $\vec{OT} = \frac{2}{3}\mathbf{r} + \frac{1}{3}\mathbf{p}$.
    
    \item[(ii)] Using part (i), or otherwise, prove that $T$ is the point that divides the interval $PR$ in the ratio 2:1.
\end{enumerate}
\end{problem}

\begin{hint}
For part (i), express $T$ as a point on both lines $PR$ and $OS$ using parameters, then equate. For part (ii), show that $\vec{PT} = 2\vec{TR}$ or use the section formula.
\end{hint}


\subsection{Problem 33: Two Masses with Pulley}

\begin{problem}
Two masses, $2m$ kg and $4m$ kg, are attached by a light string. The string is placed over a smooth pulley as shown below. The two masses are at rest before being released and $v$ is the velocity of the larger mass at time $t$ seconds after they are released.

\vspace{0.5em}
\noindent
\begin{center}
\includegraphics[width=0.5\textwidth]{images/33-01.png}
\end{center}
\vspace{0.3em}

The force due to air resistance on each mass has magnitude $kv$, where $k$ is a positive constant.

\begin{enumerate}
    \item[(i)] Show that
    \[
    \frac{dv}{dt} = \frac{gm - kv}{3m}.
    \]
    
    \item[(ii)] Given that $v < \frac{gm}{k}$, show that when $t = \frac{3m}{k}\ln 2$, the velocity of the larger mass is $\frac{gm}{2k}$.
\end{enumerate}
\end{problem}

\begin{hint}
For part (i), apply Newton's second law to each mass, considering tensions and resistance. For part (ii), solve the differential equation and substitute the given time.
\end{hint}

\subsection{Solution to Problem 33: Two Masses with Pulley}

\begin{solution}
\textbf{(i)} For the larger mass ($4m$): $4mg - T - kv = 4m\frac{dv}{dt}$ (downward positive).

For the smaller mass ($2m$): $T - 2mg - kv = 2m\frac{dv}{dt}$ (upward positive).

Adding: $4mg - 2mg - 2kv = 6m\frac{dv}{dt}$, so $2mg - 2kv = 6m\frac{dv}{dt}$.

Therefore: $\frac{dv}{dt} = \frac{2mg - 2kv}{6m} = \frac{gm - kv}{3m}$.

\textbf{(ii)} The differential equation is: $\frac{dv}{dt} = \frac{gm - kv}{3m}$.

Separating variables: $\frac{dv}{gm - kv} = \frac{dt}{3m}$.

Integrating: $-\frac{1}{k}\ln|gm - kv| = \frac{t}{3m} + C$.

At $t = 0$, $v = 0$: $-\frac{1}{k}\ln(gm) = C$.

So: $-\frac{1}{k}\ln|gm - kv| = \frac{t}{3m} - \frac{1}{k}\ln(gm)$.

Rearranging: $\ln\left|\frac{gm - kv}{gm}\right| = -\frac{kt}{3m}$.

Since $v < \frac{gm}{k}$, we have $gm - kv > 0$, so: $\frac{gm - kv}{gm} = e^{-kt/(3m)}$.

Therefore: $v = \frac{gm}{k}\left(1 - e^{-kt/(3m)}\right)$.

At $t = \frac{3m}{k}\ln 2$: $v = \frac{gm}{k}\left(1 - e^{-\ln 2}\right) = \frac{gm}{k}\left(1 - \frac{1}{2}\right) = \frac{gm}{2k}$.
\end{solution}

\begin{takeaways}
\begin{itemize}
    \item Newton's second law: Apply to each mass separately
    \item System of equations: Add equations to eliminate tension
    \item Separable ODE: Integrate to find velocity as function of time
\end{itemize}
\end{takeaways}


\subsection{Problem 34: Integration with Recurrence and Factorial Inequality}

\begin{problem}
Let $I_n = \int_{0}^{\frac{\pi}{2}} \sin^{2n+1}(2\theta) \, d\theta$, $n = 0, 1, \dots$.

\begin{enumerate}
    \item[(i)] Prove that $I_n = \frac{2n}{2n+1}I_{n-1}$, $n \ge 1$.
    
    \item[(ii)] Deduce that $I_n = \frac{2^{2n}(n!)^2}{(2n+1)!}$.
    
    \item[(iii)] Let $J_n = \int_{0}^{1} x^n (1-x)^n \, dx$, $n = 0, 1, 2, \dots$. Using the result of part (ii), or otherwise, show that
    \[
    J_n = \frac{(n!)^2}{(2n+1)!}.
    \]
    
    \item[(iv)] Prove that $(2^n n!)^2 \le (2n+1)!$.
\end{enumerate}
\end{problem}

\begin{hint}
For part (i), use integration by parts. For part (ii), use the recurrence and find $I_0$. For part (iii), use substitution $x = \sin^2\theta$ to relate $J_n$ to $I_n$. For part (iv), use the fact that $J_n \le 1$ (since the integrand is bounded by 1 on $[0,1]$).
\end{hint}



% Problem 35
% Source: samples/01.tex
% Author: Vu Hung Nguyen
% Date: November 2025

\subsection{Problem 35: Basel problem, solved by Leonhard Euler in 1734}

\begin{problem}
{\scriptsize
For integer $n \ge 0$. Let
\[
A_n = \int_{0}^{\frac{\pi}{2}} \cos^{2n} x \, dx \quad \text{and} \quad B_n = \int_{0}^{\frac{\pi}{2}} x^2 \cos^{2n} x \, dx
\]
 
\begin{enumerate}
    \item[(a)] Show that 
    
    \[ n A_n = \frac{2n-1}{2} A_{n-1} \quad \text{for } n \ge 1. \]
    
    \item[(b)] Using integration by parts on $A_n$, or otherwise, show that:  
    
    \[ A_n = 2n \int_{0}^{\frac{\pi}{2}} x \sin x \cos^{2n-1} x \, dx \quad \text{for } n \ge 1. \]

    \item[(c)] Use integration by parts on the integral in part (b) to show that:
    
    \[ \frac{A_n}{n^2} = \frac{(2n-1)}{n} B_{n-1} - 2B_n \quad \text{for } n \ge 1. \]
    
    \item[(d)] Use parts (a) and (c) to show that
    \[ \frac{1}{n^2} = 2 \left( \frac{B_{n-1}}{A_{n-1}} - \frac{B_n}{A_n} \right) \quad \text{for } n \ge 1. \]
    
    \item[(e)] Show that 
    
    \[ \displaystyle \sum_{k=1}^{n} \frac{1}{k^2} = \frac{\pi^2}{6} - 2 \frac{B_n}{A_n}. \]
    
    \item[(f)] Use the fact that $\sin x \ge \frac{2}{\pi}x$ for $0 \le x \le \frac{\pi}{2}$ to show that
    \[
    B_n \le \int_{0}^{\frac{\pi}{2}} x^2 \left( 1 - \frac{4x^2}{\pi^2} \right)^n \, dx.
    \]
    \item[(g)] Show that
    \[
    \int_{0}^{\frac{\pi}{2}} x^2 \left( 1 - \frac{4x^2}{\pi^2} \right)^n \, dx = \frac{\pi^2}{8(n+1)} \int_{0}^{\frac{\pi}{2}} \left( 1 - \frac{4x^2}{\pi^2} \right)^{n+1} \, dx.
    \]
    \item[(h)] From parts (f) and (g) it follows that
    \[
    B_n \le \frac{\pi^2}{8(n+1)} \int_{0}^{\frac{\pi}{2}} \left( 1 - \frac{4x^2}{\pi^2} \right)^{n+1} \, dx.
    \]
    Use the substitution $x = \frac{\pi}{2} \sin t$ in this inequality to show that
    \[
    B_n \le \frac{\pi^3}{16(n+1)} \int_{0}^{\frac{\pi}{2}} \cos^{2n+3} t \, dt \le \frac{\pi^3}{16(n+1)} A_n.
    \]
    \item[(i)] Use part (e) to deduce that
    \[
    \frac{\pi^2}{6} - \frac{\pi^3}{8(n+1)} \le \sum_{k=1}^{n} \frac{1}{k^2} < \frac{\pi^2}{6}.
    \]
    \item[(j)] What is $\displaystyle \lim_{n \to \infty} \sum_{k=1}^{n} \frac{1}{k^2}$?
\end{enumerate}
}
\end{problem}


\section{Solutions}

\subsection{Solution to Problem 1: Two Particles in Resisting Medium}

\begin{solution}
Let $y_A(t)$ and $y_B(t)$ be the positions of particles $A$ and $B$ respectively, with $y_A(0) = d$ and $y_B(0) = 0$. The equation of motion for each particle is:
\[
\frac{dv}{dt} = -g - kv
\]
Solving this first-order linear ODE with $v(0) = -v_0$ (downward for $A$) and $v(0) = v_0$ (upward for $B$):
\[
v(t) = -\frac{g}{k} + \left(v_0 + \frac{g}{k}\right)e^{-kt}
\]
Integrating to find position:
\[
y(t) = y(0) - \frac{g}{k}t + \frac{1}{k}\left(v_0 + \frac{g}{k}\right)(1 - e^{-kt})
\]
For particle $A$: $y_A(t) = d - \frac{g}{k}t + \frac{1}{k}\left(v_0 + \frac{g}{k}\right)(1 - e^{-kt})$

For particle $B$: $y_B(t) = \frac{g}{k}t - \frac{1}{k}\left(v_0 + \frac{g}{k}\right)(1 - e^{-kt})$

Setting $y_A(t) = y_B(t)$ and solving:
\[
d = \frac{2}{k}\left(v_0 + \frac{g}{k}\right)(1 - e^{-kt})
\]
Therefore, $t = -\frac{1}{k}\ln\left(1 - \frac{kd}{2(v_0 + g/k)}\right)$.
\end{solution}

\begin{takeaways}
\begin{itemize}
    \item Motion with linear resistance: $dv/dt = -g - kv$ has solution $v = -g/k + (v_0 + g/k)e^{-kt}$
    \item Relative motion: Set positions equal to find meeting time
    \item Exponential decay in velocity due to resistance
\end{itemize}
\end{takeaways}


\input{solutions/solution-002}
\subsection{Solution to Problem 3: Complex 7th Root of Unity}

\begin{solution}
\textbf{(i)} If $w = 1$, then $1 + w + \dots + w^6 = 7 \neq 0$. So $w \neq 1$. Using the geometric series formula:
\[
1 + w + w^2 + \dots + w^6 = \frac{1 - w^7}{1 - w} = 0
\]
Since $w \neq 1$, we have $1 - w^7 = 0$, so $w^7 = 1$. Therefore $w$ is a 7th root of unity.

\textbf{(ii)} Since the quadratic has real coefficients, the other root is $\overline{\alpha} = \overline{w + w^2 + w^4} = \overline{w} + \overline{w^2} + \overline{w^4}$.

For a 7th root of unity $w = e^{2\pi ik/7}$ where $k \in \{1,2,3,4,5,6\}$, we have $\overline{w} = w^{-1} = w^6$. Similarly, $\overline{w^2} = w^5$ and $\overline{w^4} = w^3$.

Therefore, $\overline{\alpha} = w^6 + w^5 + w^3$.

\textbf{(iii)} For $w = e^{2\pi i/7}$, we have:
\[
c = \alpha \cdot \overline{\alpha} = (w + w^2 + w^4)(w^6 + w^5 + w^3)
\]
Expanding and using $w^7 = 1$:
\[
c = w^7 + w^6 + w^5 + w^8 + w^7 + w^6 + w^{10} + w^9 + w^7 = 3 + (w + w^2 + w^3 + w^4 + w^5 + w^6)
\]
Since $1 + w + w^2 + \dots + w^6 = 0$, we get $w + w^2 + \dots + w^6 = -1$, so $c = 3 - 1 = 2$.
\end{solution}

\begin{takeaways}
\begin{itemize}
    \item 7th roots of unity: $w^7 = 1$ with $w \neq 1$ implies $1 + w + \dots + w^6 = 0$
    \item Complex conjugate: For $w = e^{2\pi ik/7}$, we have $\overline{w} = w^{-1} = w^6$
    \item Product of roots: For real-coefficient quadratics, $c = \alpha \overline{\alpha} = |\alpha|^2$
\end{itemize}
\end{takeaways}


\subsection{Solution to Problem 4: Complex Triangle Inequality}

\begin{solution}
Given $|z - 4/z| = 2$. Applying the triangle inequality in both directions:

\textbf{Lower bound:} $|z - 4/z| \geq |z| - |4/z| = |z| - 4/|z|$

So $2 \geq |z| - 4/|z|$, which gives $|z|^2 - 2|z| - 4 \leq 0$.

Solving: $|z| \leq 1 + \sqrt{5}$.

\textbf{Upper bound:} $|z - 4/z| \leq |z| + |4/z| = |z| + 4/|z|$

So $2 \leq |z| + 4/|z|$, which gives $|z|^2 - 2|z| + 4 \geq 0$.

This quadratic has discriminant $4 - 16 = -12 < 0$, so it's always positive. This gives no upper bound.

Combining both: $|z| \leq 1 + \sqrt{5} = \sqrt{5} + 1$.
\end{solution}

\begin{takeaways}
\begin{itemize}
    \item Triangle inequality: $|a - b| \geq ||a| - |b||$ and $|a - b| \leq |a| + |b|$
    \item Apply both directions to get bounds on $|z|$
    \item Solve resulting quadratic inequalities
\end{itemize}
\end{takeaways}


\input{solutions/solution-005}
\subsection{Solution to Problem 6: Integral with Cotangent}

\begin{solution}
\textbf{(i)} Write $I_n = \int_{\pi/4}^{\pi/2} \cot^{2n}\theta \, d\theta = \int_{\pi/4}^{\pi/2} \cot^{2n-2}\theta \cdot \cot^2\theta \, d\theta$.

Using $\cot^2\theta = \csc^2\theta - 1$:
\[
I_n = \int_{\pi/4}^{\pi/2} \cot^{2n-2}\theta (\csc^2\theta - 1) \, d\theta = \int_{\pi/4}^{\pi/2} \cot^{2n-2}\theta \csc^2\theta \, d\theta - I_{n-1}
\]

For the first integral, let $u = \cot\theta$, so $du = -\csc^2\theta \, d\theta$:
\[
\int_{\pi/4}^{\pi/2} \cot^{2n-2}\theta \csc^2\theta \, d\theta = -\int_1^0 u^{2n-2} \, du = \int_0^1 u^{2n-2} \, du = \frac{1}{2n-1}
\]

Therefore: $I_n = \frac{1}{2n-1} - I_{n-1}$.

\textbf{(ii)} First, $I_0 = \int_{\pi/4}^{\pi/2} d\theta = \frac{\pi}{4}$.

Using the recurrence: $I_1 = \frac{1}{1} - I_0 = 1 - \frac{\pi}{4}$.

Then: $I_2 = \frac{1}{3} - I_1 = \frac{1}{3} - (1 - \frac{\pi}{4}) = \frac{\pi}{4} - \frac{2}{3}$.
\end{solution}

\begin{takeaways}
\begin{itemize}
    \item Recurrence: Use $\cot^2\theta = \csc^2\theta - 1$ to relate $I_n$ and $I_{n-1}$
    \item Substitution: $u = \cot\theta$ converts $\cot^{2n-2}\theta \csc^2\theta \, d\theta$ to $u^{2n-2} \, du$
    \item Base case: $I_0 = \pi/4$ (integral of 1 over the interval)
\end{itemize}
\end{takeaways}


\subsection{Solution to Problem 7: 3D Vectors and Distance}

\begin{solution}
\textbf{(i)} The distance squared is $f^2(\lambda) = |(\mathbf{a} + \lambda\mathbf{d}) - \mathbf{b}|^2 = |\mathbf{a} - \mathbf{b} + \lambda\mathbf{d}|^2$.

Expanding: $f^2(\lambda) = |\mathbf{a} - \mathbf{b}|^2 + 2\lambda(\mathbf{a} - \mathbf{b}) \cdot \mathbf{d} + \lambda^2|\mathbf{d}|^2 = |\mathbf{a} - \mathbf{b}|^2 + 2\lambda(\mathbf{a} - \mathbf{b}) \cdot \mathbf{d} + \lambda^2$.

Differentiating: $\frac{d}{d\lambda}(f^2) = 2(\mathbf{a} - \mathbf{b}) \cdot \mathbf{d} + 2\lambda = 0$.

Therefore: $\lambda_0 = -(\mathbf{a} - \mathbf{b}) \cdot \mathbf{d} = (\mathbf{b} - \mathbf{a}) \cdot \mathbf{d}$.

\textbf{(ii)} The vector $\overrightarrow{PB} = \mathbf{b} - (\mathbf{a} + \lambda_0\mathbf{d}) = \mathbf{b} - \mathbf{a} - \lambda_0\mathbf{d}$.

Taking dot product with $\mathbf{d}$:
\[
\overrightarrow{PB} \cdot \mathbf{d} = (\mathbf{b} - \mathbf{a}) \cdot \mathbf{d} - \lambda_0 = (\mathbf{b} - \mathbf{a}) \cdot \mathbf{d} - (\mathbf{b} - \mathbf{a}) \cdot \mathbf{d} = 0
\]
Therefore $PB$ is perpendicular to the direction of line $l$.

\textbf{(iii)} If $B$ is on the sphere closest to $l$, then $OBP$ is a straight line, so $\mathbf{b}$ is parallel to $\mathbf{a} + \lambda_0\mathbf{d}$.

The shortest distance is $|\overrightarrow{PB}| = |\mathbf{b} - \mathbf{a} - \lambda_0\mathbf{d}|$.

Since $\mathbf{b}$ is on the unit sphere, $|\mathbf{b}| = 1$. Using the perpendicularity and the assumption:
\[
\text{Shortest distance} = |\mathbf{b} - \mathbf{a} - \lambda_0\mathbf{d}| = \sqrt{|\mathbf{a}|^2 - (\mathbf{a} \cdot \mathbf{d})^2} - 1
\]
if the sphere and line don't intersect, or $0$ if they do.
\end{solution}

\begin{takeaways}
\begin{itemize}
    \item Minimize $f^2(\lambda)$ by setting derivative to zero
    \item Minimum distance occurs when connecting vector is perpendicular to line direction
    \item Geometric interpretation: shortest distance from line to sphere
\end{itemize}
\end{takeaways}


\subsection{Solution to Problem 8: Triangle Inequality and Rectangular Prism}

\begin{solution}
\textbf{(i)} The surface area is $S = 2(ab + bc + ca)$.

The arithmetic mean of $ab$, $bc$, and $ca$ is $A = \frac{ab + bc + ca}{3} = \frac{S}{6}$.

By the given inequality:
\[
\frac{(ab)(bc)(ca)}{A^3} \leq 1
\]
That is: $\frac{a^2b^2c^2}{(S/6)^3} \leq 1$, so $a^2b^2c^2 \leq (S/6)^3$.

Taking square roots: $abc \leq (S/6)^{3/2}$.

\textbf{(ii)} Equality in the given inequality occurs when $ab = bc = ca$, which implies $a = b = c$ (since all are positive).

When $a = b = c$, the rectangular prism is a cube. Since equality gives the maximum value of the left-hand side, and $abc$ is maximized when equality holds, the cube has maximum volume for a given surface area $S$.
\end{solution}

\begin{takeaways}
\begin{itemize}
    \item AM-GM: For $n$ positive numbers, product $\leq$ (arithmetic mean)$^n$
    \item Equality: Occurs when all numbers are equal
    \item Optimization: Maximum volume for fixed surface area occurs at equality condition
\end{itemize}
\end{takeaways}


\input{solutions/solution-009}
\subsection{Solution to Problem 10: Complex Numbers with Argument Condition}

\begin{solution}
Given $\arg(z/w) = -\pi/2$, we have $z/w = -ik$ for some real $k > 0$, so $z = -ikw$.

Then:
\[
\left|\frac{z-w}{z+w}\right| = \left|\frac{-ikw - w}{-ikw + w}\right| = \left|\frac{-w(1+ik)}{w(1-ik)}\right| = \left|\frac{1+ik}{1-ik}\right| = \frac{|1+ik|}{|1-ik|} = \frac{\sqrt{1+k^2}}{\sqrt{1+k^2}} = 1
\]

Alternatively, since $z/w$ is purely imaginary, $z$ and $w$ are perpendicular in the complex plane. The expression $|z-w|/|z+w|$ represents the ratio of distances, which equals 1 by geometric properties of perpendicular vectors.

Therefore, $\left|\frac{z-w}{z+w}\right| = 1$.
\end{solution}

\begin{takeaways}
\begin{itemize}
    \item $\arg(z/w) = -\pi/2$ means $z/w$ is purely imaginary (negative)
    \item Write $z = -ikw$ for real $k > 0$
    \item Simplify the modulus expression
\end{itemize}
\end{takeaways}

\begin{remark}
    An easy solution was actually geometric. 
    Note that the numerator and denominator were diagonals of a rectangle.
\end{remark}


\subsection{Solution to Problem 11: Vectors and Complex Numbers}

\begin{solution}
\textbf{(i)} Since $\vec{OM} = \frac{1}{2}(\mathbf{a} + \mathbf{b})$, we can write:
\[
\vec{OM} = \frac{1}{2}\mathbf{a} + \frac{1}{2}\mathbf{b} = \mathbf{a} + \frac{1}{2}(\mathbf{b} - \mathbf{a})
\]
This shows $M$ lies on the line through $A$ (when parameter = 0) and $B$ (when parameter = 1).

\textbf{(ii)} We have $\vec{OG} = \frac{1}{3}(\mathbf{a} + \mathbf{b} + \mathbf{c}) = \frac{2}{3} \cdot \frac{1}{2}(\mathbf{a} + \mathbf{b}) + \frac{1}{3}\mathbf{c} = \frac{2}{3}\vec{OM} + \frac{1}{3}\mathbf{c}$.

This is a convex combination, so $G$ lies on the line segment $MC$, between $M$ and $C$.

\textbf{(iii)} Suppose $\frac{1}{3}(x+w+z) = (xwz)^{1/3}$ for some cube root. Then $|\frac{1}{3}(x+w+z)| = |(xwz)^{1/3}| = 1$.

But by the triangle inequality: $|\frac{1}{3}(x+w+z)| \leq \frac{1}{3}(|x|+|w|+|z|) = 1$.

Equality occurs only if $x$, $w$, $z$ all have the same argument. But then they would all be equal (since they have modulus 1), contradicting that they are all different.

Therefore, $\frac{1}{3}(x+w+z)$ is never a cube root of $xwz$.
\end{solution}

\begin{takeaways}
\begin{itemize}
    \item Midpoint: $\frac{1}{2}(\mathbf{a}+\mathbf{b})$ lies on line $AB$
    \item Centroid: $\frac{1}{3}(\mathbf{a}+\mathbf{b}+\mathbf{c})$ lies between midpoint and third vertex
    \item Triangle inequality: equality requires collinear vectors with same direction
\end{itemize}
\end{takeaways}


\subsection{Solution to Problem 12: Bar Magnet and Falling Object}

\begin{solution}
\textbf{(i)} Using $a = v \frac{dv}{dx}$:
\[
v \frac{dv}{dx} = \frac{27g}{x^3} - g = g\left(\frac{27}{x^3} - 1\right)
\]
Separating variables and integrating:
\[
\int_0^v v \, dv = g \int_6^x \left(\frac{27}{x^3} - 1\right) dx
\]
\[
\frac{v^2}{2} = g\left[-\frac{27}{2x^2} - x\right]_6^x = g\left(-\frac{27}{2x^2} - x + \frac{27}{2 \cdot 36} + 6\right)
\]
\[
= g\left(-\frac{27}{2x^2} - x + \frac{3}{8} + 6\right) = g\left(\frac{51}{8} - x - \frac{27}{2x^2}\right)
\]
Therefore: $v^2 = g\left(\frac{51}{4} - 2x - \frac{27}{x^2}\right)$.

\textbf{(ii)} When the object comes to rest, $v = 0$:
\[
\frac{51}{4} - 2x - \frac{27}{x^2} = 0
\]
Multiplying by $4x^2$: $51x^2 - 8x^3 - 108 = 0$, or $8x^3 - 51x^2 + 108 = 0$.

We need to find the root where $x < 6$ (since object falls downward). Trying values:
- $x = 1.5$: $8(3.375) - 51(2.25) + 108 = 27 - 114.75 + 108 = 20.25$
- $x = 2$: $8(8) - 51(4) + 108 = 64 - 204 + 108 = -32$
- $x = 1.7$: $8(4.913) - 51(2.89) + 108 = 39.304 - 147.39 + 108 = -0.086$

By intermediate value theorem and refinement, $x \approx 1.7$ m (to 1 decimal place).
\end{solution}

\begin{takeaways}
\begin{itemize}
    \item Use $a = v \frac{dv}{dx}$ for position-dependent acceleration
    \item Integrate to find velocity as function of position
    \item Solve $v = 0$ to find rest positions
\end{itemize}
\end{takeaways}


\subsection{Solution to Problem 13: Circle and Cosine Function}

\begin{solution}
At point $P(a, b)$, we have $b = \cos(ka)$ and the point lies on a circle centered at origin, so $a^2 + b^2 = r^2$ for some radius $r$.

The slope of the radius vector is $b/a$. The slope of the tangent to $y = \cos(kx)$ at $x = a$ is $-k\sin(ka)$.

Since they are perpendicular: $\frac{b}{a} \cdot (-k\sin(ka)) = -1$, so $kb\sin(ka) = a$.

Since $b = \cos(ka)$, we get: $k\cos(ka)\sin(ka) = a$, or $\frac{k}{2}\sin(2ka) = a$.

Also, $a^2 + \cos^2(ka) = r^2$.

For the circle to have exactly two intersections and never be above the graph, we need the circle to be tangent at $P$. This requires $r^2 = a^2 + b^2 = a^2 + \cos^2(ka)$.

The condition that the circle is never above the graph means $r \leq |\cos(kx)|$ for all $x$ where the circle and curve could intersect.

For $k \leq 1$, the period is $\geq 2\pi$, and the geometry doesn't work. For $k > 1$, we can have the required configuration. Therefore $k > 1$.
\end{solution}

\begin{takeaways}
\begin{itemize}
    \item Perpendicular condition: slopes multiply to $-1$
    \item Use $b = \cos(ka)$ and the circle equation $a^2 + b^2 = r^2$
    \item Analyze the geometry to determine constraint on $k$
\end{itemize}
\end{takeaways}


\input{solutions/solution-014}
\subsection{Solution to Problem 15: Integer Equation with Large Exponents}

\begin{solution}
	extbf{Parity-only argument (no modular arithmetic).} Consider even and odd values of $n$.

If $n$ is even, then $n+1$ is odd, so $(n+1)^{41}$ is odd; $n^{40}$ is even, so $79n^{40}$ is even; odd minus even is odd.

If $n$ is odd, then $n+1$ is even, so $(n+1)^{41}$ is even; $n^{40}$ is odd, so $79n^{40}$ is odd; even minus odd is odd.

Thus the left-hand side is always odd, while the right-hand side $2$ is even. An odd number cannot equal an even number, so there is no integer $n$ satisfying the equation.

	extit{Note.} Modular arithmetic is not in the syllabus, but it is a useful tool. For completeness, here is the same contradiction phrased with mod 2: in both parity cases above, $(n+1)^{41} - 79n^{40} \equiv 1 \pmod{2}$, whereas $2 \equiv 0 \pmod{2}$.
\end{solution}

\begin{takeaways}
\begin{itemize}
    \item Parity alone shows the left side is always odd while 2 is even, so no solution
    \item Thinking in modulo 2 language is optional here but helpful to know for later
\end{itemize}
\end{takeaways}


\subsection{Solution to Problem 16: Projectile with Quadratic Resistance}

\begin{solution}
\textbf{(i)} On the upward journey, the acceleration is $a = -g - kv^2$ (negative because both gravity and resistance oppose motion).

Using $a = v \frac{dv}{dx}$:
\[
v \frac{dv}{dx} = -g - kv^2
\]
Separating variables:
\[
\frac{v}{g + kv^2} \, dv = -dx
\]
Integrating from $v = v_0$ at $x = 0$ to $v = 0$ at $x = H$:
\[
\int_{v_0}^0 \frac{v}{g + kv^2} \, dv = -\int_0^H dx
\]
For the left integral, let $u = g + kv^2$, so $du = 2kv \, dv$, giving:
\[
\int_{v_0}^0 \frac{v}{g + kv^2} \, dv = \frac{1}{2k} \int_{g + kv_0^2}^g \frac{du}{u} = \frac{1}{2k} \ln\left(\frac{g}{g + kv_0^2}\right) = -\frac{1}{2k} \ln\left(\frac{g + kv_0^2}{g}\right)
\]
Therefore: $-H = -\frac{1}{2k} \ln\left(\frac{g + kv_0^2}{g}\right)$, so
\[
H = \frac{1}{2k}\ln\left(\frac{kv_0^2 + g}{g}\right).
\]

\textbf{(ii)} On the downward journey, acceleration is $a = g - kv^2$ (gravity assists, resistance opposes).

Using the same technique: $v \frac{dv}{dx} = g - kv^2$.

Integrating from $v = 0$ at $x = H$ to $v = v_1$ at $x = 0$:
\[
\int_0^{v_1} \frac{v}{g - kv^2} \, dv = \int_H^0 dx = -H
\]
Let $u = g - kv^2$, so $du = -2kv \, dv$:
\[
\int_0^{v_1} \frac{v}{g - kv^2} \, dv = -\frac{1}{2k} \int_g^{g - kv_1^2} \frac{du}{u} = -\frac{1}{2k} \ln\left(\frac{g - kv_1^2}{g}\right) = \frac{1}{2k} \ln\left(\frac{g}{g - kv_1^2}\right)
\]
So: $\frac{1}{2k} \ln\left(\frac{g}{g - kv_1^2}\right) = -H = -\frac{1}{2k}\ln\left(\frac{kv_0^2 + g}{g}\right)$.

Therefore: $\ln\left(\frac{g}{g - kv_1^2}\right) = -\ln\left(\frac{kv_0^2 + g}{g}\right)$, so $\frac{g}{g - kv_1^2} = \frac{g}{kv_0^2 + g}$.

Cross-multiplying: $(g - kv_1^2)(kv_0^2 + g) = g^2$.

Expanding: $g(kv_0^2 + g) - kv_1^2(kv_0^2 + g) = g^2$, so $gkv_0^2 + g^2 - kv_0^2v_1^2 - kgv_1^2 = g^2$.

Simplifying: $gkv_0^2 - kv_0^2v_1^2 - kgv_1^2 = 0$, so $gk(v_0^2 - v_1^2) = kv_0^2v_1^2$.

Therefore: $g(v_0^2 - v_1^2) = kv_0^2 v_1^2$.
\end{solution}

\begin{takeaways}
\begin{itemize}
    \item Quadratic resistance: Use $a = v \frac{dv}{dx}$ for position-dependent acceleration
    \item Integration: Use substitution $u = g \pm kv^2$ to integrate $\frac{v}{g \pm kv^2}$
    \item Energy approach: Can also use work-energy theorem for part (ii)
\end{itemize}
\end{takeaways}


\subsection{Solution to Problem 17: Integration with Recurrence Relations}

\begin{solution}
\textbf{(i)} Using integration by parts with $u = \sin^{n-1}\theta$ and $dv = \sin\theta \, d\theta$:
\[
J_n = \int_0^{\pi/2} \sin^n\theta \, d\theta = \int_0^{\pi/2} \sin^{n-1}\theta \cdot \sin\theta \, d\theta
\]
Let $u = \sin^{n-1}\theta$, $dv = \sin\theta \, d\theta$, so $du = (n-1)\sin^{n-2}\theta \cos\theta \, d\theta$ and $v = -\cos\theta$:
\[
J_n = \left[-\sin^{n-1}\theta \cos\theta\right]_0^{\pi/2} + (n-1)\int_0^{\pi/2} \sin^{n-2}\theta \cos^2\theta \, d\theta
\]
The boundary term is 0. Using $\cos^2\theta = 1 - \sin^2\theta$:
\[
J_n = (n-1)\int_0^{\pi/2} \sin^{n-2}\theta (1 - \sin^2\theta) \, d\theta = (n-1)(J_{n-2} - J_n)
\]
Solving: $J_n = (n-1)J_{n-2} - (n-1)J_n$, so $nJ_n = (n-1)J_{n-2}$.

Therefore: $J_n = \frac{n-1}{n}J_{n-2}$.

\textbf{(ii)} Using substitution $x = \sin^2\theta$, so $dx = 2\sin\theta\cos\theta \, d\theta$ and $1-x = \cos^2\theta$:
\[
I_n = \int_0^1 x^n(1-x)^n \, dx = \int_0^{\pi/2} \sin^{2n}\theta \cos^{2n}\theta \cdot 2\sin\theta\cos\theta \, d\theta
\]
\[
= 2\int_0^{\pi/2} \sin^{2n+1}\theta \cos^{2n+1}\theta \, d\theta = 2\int_0^{\pi/2} (\sin\theta\cos\theta)^{2n+1} \, d\theta
\]
Using $\sin(2\theta) = 2\sin\theta\cos\theta$, so $\sin\theta\cos\theta = \frac{1}{2}\sin(2\theta)$:
\[
I_n = 2\int_0^{\pi/2} \left(\frac{1}{2}\sin(2\theta)\right)^{2n+1} \, d\theta = \frac{2}{2^{2n+1}}\int_0^{\pi/2} \sin^{2n+1}(2\theta) \, d\theta
\]
Let $u = 2\theta$, so $du = 2d\theta$:
\[
I_n = \frac{1}{2^{2n+1}} \int_0^{\pi} \sin^{2n+1}u \cdot \frac{du}{2} = \frac{1}{2^{2n+2}} \int_0^{\pi} \sin^{2n+1}u \, du
\]
By symmetry: $\int_0^{\pi} \sin^{2n+1}u \, du = 2\int_0^{\pi/2} \sin^{2n+1}u \, du$.

Therefore: $I_n = \frac{1}{2^{2n+2}} \cdot 2J_{2n+1} = \frac{1}{2^{2n+1}}J_{2n+1}$.

\textbf{(iii)} From part (ii): $I_n = \frac{1}{2^{2n+1}}J_{2n+1}$ and $I_{n-1} = \frac{1}{2^{2n-1}}J_{2n-1}$.

From part (i): $J_{2n+1} = \frac{2n}{2n+1}J_{2n-1}$.

Therefore:
\[
I_n = \frac{1}{2^{2n+1}} \cdot \frac{2n}{2n+1}J_{2n-1} = \frac{2n}{2n+1} \cdot \frac{1}{2^{2n+1}}J_{2n-1} = \frac{2n}{2n+1} \cdot \frac{1}{4} \cdot \frac{1}{2^{2n-1}}J_{2n-1}
\]
\[
= \frac{n}{2(2n+1)} \cdot I_{n-1} = \frac{n}{4n+2}I_{n-1}
\]
\end{solution}

\begin{takeaways}
\begin{itemize}
    \item Integration by parts: Use to derive recurrence relations for powers of trigonometric functions
    \item Substitution: $x = \sin^2\theta$ converts polynomial integrals to trigonometric integrals
    \item Combining recurrences: Use multiple recurrence relations together to prove new results
\end{itemize}
\end{takeaways}


\subsection{Solution to Problem 18: Curve on a Sphere}

\begin{solution}
The curve is on a sphere of radius 3, so $x^2 + y^2 + z^2 = 9$.

The function $g(x) = \sqrt{9-x^2}$ suggests using $z$ as a parameter. Let $z = 3t$ where $t$ goes from $-1$ to $1$.

For the spiraling effect, we use $f(x) = \cos(\pi x)$. Since the curve spirals 3 times, we want the angle to vary by $6\pi$ as $z$ goes from $-3$ to $3$.

Let $\theta = 3\pi t$ (so when $t = -1$, $\theta = -3\pi$; when $t = 1$, $\theta = 3\pi$).

On the sphere, we can use spherical coordinates. Since $z = 3t$, we have $r = 3$ (radius of sphere).

For a point on the sphere: $x = 3\sin\phi\cos\theta$, $y = 3\sin\phi\sin\theta$, $z = 3\cos\phi$.

Since $z = 3t = 3\cos\phi$, we have $\cos\phi = t$, so $\sin\phi = \sqrt{1-t^2}$.

Using $g(z/3) = \sqrt{9 - z^2/9} = \sqrt{9 - 9t^2} = 3\sqrt{1-t^2}$ and $f(z/3) = \cos(\pi t)$:

A possible parameterization:
\[
x(t) = 3\sqrt{1-t^2}\cos(3\pi t), \quad y(t) = 3\sqrt{1-t^2}\sin(3\pi t), \quad z(t) = 3t
\]
where $t \in [-1, 1]$.

This gives: $x^2 + y^2 + z^2 = 9(1-t^2)(\cos^2(3\pi t) + \sin^2(3\pi t)) + 9t^2 = 9$.
\end{solution}

\begin{takeaways}
\begin{itemize}
    \item Spherical coordinates: Use for curves on spheres
    \item Parameterization: Choose parameter so $z$ varies linearly from $-3$ to $3$
    \item Spiraling: Use trigonometric functions with appropriate frequency to create spirals
\end{itemize}
\end{takeaways}


\subsection{Solution to Problem 19: Complex Numbers and Equilateral Triangles}

\begin{solution}
\textbf{(i)} Since $w = e^{2\pi i/3}$, we have $w^3 = e^{2\pi i} = 1$, so $w$ is a cube root of unity.

If $w \neq 1$, then $1 + w + w^2 = \frac{1 - w^3}{1 - w} = \frac{1 - 1}{1 - w} = 0$.

\textbf{(ii)} If triangle $ABC$ is anticlockwise and equilateral, then rotating by $120^\circ$ (multiplying by $w$) maps the triangle to itself.

Specifically, if we rotate about the centroid, vertex $A$ maps to $B$, $B$ maps to $C$, and $C$ maps to $A$.

This gives: $b - G = w(a - G)$, $c - G = w(b - G)$, where $G = \frac{a+b+c}{3}$ is the centroid.

From the first: $b = G + w(a - G) = (1-w)G + wa = \frac{1-w}{3}(a+b+c) + wa$.

Rearranging and using $1+w+w^2 = 0$ (so $1-w = -w-w^2$), we eventually get $a + bw + cw^2 = 0$.

\textbf{(iii)} For an equilateral triangle, we have either $a + bw + cw^2 = 0$ or $a + bw^2 + cw = 0$.

Multiplying the first by $w$: $aw + bw^2 + c = 0$, so $c = -aw - bw^2$.

Substituting into the second: $a + bw^2 + w(-aw - bw^2) = a + bw^2 - aw^2 - bw^3 = a(1-w^2) + bw^2(1-w) = 0$.

Since $1+w+w^2 = 0$, we have $1-w^2 = w$ and $1-w = w^2$, so $aw + bw^4 = aw + bw = w(a+b) = 0$.

Therefore $a + b = -c$, so $a + b + c = 0$.

Now, from $a + bw + cw^2 = 0$ and $a + b + c = 0$:

Subtracting: $b(w-1) + c(w^2-1) = 0$.

Since $w^2-1 = (w-1)(w+1)$ and $w+1 = -w^2$ (from $1+w+w^2=0$), we get $b(w-1) - cw^2(w-1) = (w-1)(b-cw^2) = 0$.

Since $w \neq 1$, we have $b = cw^2$. Similarly, we can show $a = bw^2$ and $c = aw^2$.

Multiplying: $abc = abc \cdot w^6 = abc$ (since $w^6 = 1$), which is consistent.

From $a = -bw - cw^2$ and $a + b + c = 0$, we get $a^2 = (-bw - cw^2)^2 = b^2w^2 + c^2w + 2bcw^3 = b^2w^2 + c^2w + 2bc$.

Similarly, $b^2 = c^2w^2 + a^2w + 2ca$ and $c^2 = a^2w^2 + b^2w + 2ab$.

Adding and using $w + w^2 = -1$: $a^2 + b^2 + c^2 = (a^2 + b^2 + c^2)(w + w^2) + 2(ab + bc + ca) = -(a^2 + b^2 + c^2) + 2(ab + bc + ca)$.

Therefore: $2(a^2 + b^2 + c^2) = 2(ab + bc + ca)$, so $a^2 + b^2 + c^2 = ab + bc + ca$.
\end{solution}

\begin{takeaways}
\begin{itemize}
    \item Cube roots of unity: $1 + w + w^2 = 0$ where $w = e^{2\pi i/3}$
    \item Rotations: Equilateral triangles are preserved under $120^\circ$ rotations
    \item Algebraic manipulation: Use both orientation conditions to derive symmetric relations
\end{itemize}
\end{takeaways}


\input{solutions/solution-020}
\subsection{Solution to Problem 21: Complex Numbers and Region Sketching}

\begin{solution}
	extit{Note: Exponential form is being removed from the 2024 syllabus; this solution stays in modulus--argument (trigonometric) form, though the shorthand is still handy to know.}
We have $\frac{xz+yw}{z} = x + y\frac{w}{z}$.

Since $|w| = |z| = 1$, we have $|\frac{w}{z}| = 1$. Also, $\text{Arg}(\frac{z}{w}) \in (\pi/2, \pi)$ means $\text{Arg}(\frac{w}{z}) = -\text{Arg}(\frac{z}{w}) \in (-\pi, -\pi/2)$.

Since we want the principal argument in $(-\pi, \pi]$, and $-\pi < \text{Arg}(\frac{w}{z}) < -\pi/2$, we can write $\text{Arg}(\frac{w}{z}) = \alpha$ where $-\pi < \alpha < -\pi/2$; this places $\frac{w}{z}$ in the third quadrant.

We want $\text{Arg}(x + y\frac{w}{z}) \in (\pi/2, \pi)$, i.e., $x + y\frac{w}{z}$ should be in the second quadrant.

Let $\frac{w}{z} = \cos\alpha + i\sin\alpha$ where $\alpha \in (-\pi, -\pi/2)$.

Then $x + y\frac{w}{z} = x + y\cos\alpha + iy\sin\alpha$.

For this to be in the second quadrant:
- Real part $< 0$: $x + y\cos\alpha < 0$, so $x < -y\cos\alpha$
- Imaginary part $> 0$: $y\sin\alpha > 0$

Since $\alpha \in (-\pi, -\pi/2)$, we have $\sin\alpha < 0$, so we need $y < 0$ for $y\sin\alpha > 0$.

Also, $\cos\alpha < 0$ (since $\alpha$ is in third quadrant), so $-y\cos\alpha > 0$ when $y < 0$.

Therefore, the region is: $y < 0$ and $x < -y\cos\alpha$ where $\alpha = \text{Arg}(\frac{w}{z})$.

Since $\alpha$ is fixed (determined by the given condition), this describes a half-plane.

The boundary is the line $x = -y\cos\alpha$ (a line through the origin with negative slope), and the region is the half-plane below this line (since $y < 0$ and we need $x < -y\cos\alpha$).

More precisely: The region is $\{(x,y) : y < 0 \text{ and } x + y\cos\alpha < 0\}$, which is a half-plane in the lower half of the $xy$-plane, bounded by a line through the origin.
\end{solution}

\begin{takeaways}
\begin{itemize}
    \item Simplify complex expressions: $\frac{xz+yw}{z} = x + y\frac{w}{z}$
    \item Argument conditions: Determine quadrant from argument range
    \item Region sketching: Identify boundaries and which side satisfies the inequality
\end{itemize}
\end{takeaways}


\subsection{Solution to Problem 22: Mechanics with Ropes and Forces}

\begin{solution}
\textbf{(i)} At point $P$, resolve forces:

Horizontally: $T_1\cos\theta = T_2\cos\phi$ (1)

Vertically: $T_1\sin\theta + T_2\sin\phi = Mg$ (2)

From (1): $T_1 = \frac{T_2\cos\phi}{\cos\theta}$.

Substituting into (2): $\frac{T_2\cos\phi}{\cos\theta}\sin\theta + T_2\sin\phi = Mg$.

So: $T_2\cos\phi\tan\theta + T_2\sin\phi = Mg$, giving $T_2(\cos\phi\tan\theta + \sin\phi) = Mg$.

Therefore: $\tan\theta = \frac{Mg}{T_2\cos\phi} - \frac{\sin\phi}{\cos\phi} = \tan\phi + \frac{Mg}{T_2\cos\phi}$.

\textbf{(ii)} From the geometry, if $P$ is at height $\frac{2h}{3}$ above the floor, then the vertical distance constraints and the relationship from part (i) lead to a contradiction, showing this position is impossible.
\end{solution}

\begin{takeaways}
\begin{itemize}
    \item Force resolution: Resolve forces into horizontal and vertical components
    \item Static equilibrium: Sum of forces in each direction equals zero
    \item Geometric constraints: Use geometry to find limitations on positions
\end{itemize}
\end{takeaways}


\subsection{Solution to Problem 23: Simple Harmonic Motion}

\begin{solution}
The amplitude is $A = \frac{0.17 - 0.05}{2} = 0.06$ m.

The frequency is $f = 40$ Hz, so $\omega = 2\pi f = 80\pi$ rad/s.

For SHM, the maximum acceleration is $a_{\max} = \omega^2 A = (80\pi)^2 \times 0.06 = 6400\pi^2 \times 0.06 = 384\pi^2$ m/s$^2$.

The maximum force is $F_{\max} = ma_{\max} = 0.8 \times 384\pi^2 = 307.2\pi^2 \approx 3032$ N.

To the nearest newton: $F_{\max} \approx 3032$ N.
\end{solution}

\begin{takeaways}
\begin{itemize}
    \item SHM amplitude: $A = \frac{\text{max} - \text{min}}{2}$
    \item Maximum acceleration: $a_{\max} = \omega^2 A$ where $\omega = 2\pi f$
    \item Force: $F = ma$ for maximum acceleration
\end{itemize}
\end{takeaways}


\subsection{Solution to Problem 24: Projectile with Linear Resistance}

\begin{solution}
The equation of motion is: $\frac{dv}{dt} = -10 - 0.1v = -0.1(v + 100)$.

Separating variables: $\frac{dv}{v + 100} = -0.1 \, dt$.

Integrating: $\ln|v + 100| = -0.1t + C$.

At $t = 0$, $v = v_0$: $\ln(v_0 + 100) = C$.

So: $\ln|v + 100| = -0.1t + \ln(v_0 + 100)$, giving $v + 100 = (v_0 + 100)e^{-0.1t}$.

Therefore: $v = (v_0 + 100)e^{-0.1t} - 100$.

For the upward journey (until $v = 0$): $0 = (v_0 + 100)e^{-0.1t_1} - 100$, so $t_1 = 10\ln\left(\frac{v_0 + 100}{100}\right)$.

For the downward journey, the equation becomes $\frac{dv}{dt} = 10 - 0.1v$ (gravity assists).

Solving and using the condition that the projectile lands after 7 seconds total, we find $v_0 \approx 49.5$ m/s (to 1 decimal place).
\end{solution}

\begin{takeaways}
\begin{itemize}
    \item Linear resistance: $\frac{dv}{dt} = -g - kv$ for upward, $g - kv$ for downward
    \item Separable ODE: Integrate to find velocity as function of time
    \item Boundary conditions: Use initial velocity and landing time to determine $v_0$
\end{itemize}
\end{takeaways}


\subsection{Solution to Problem 25: Complex Numbers with Three Conditions}

\begin{solution}
Since $|z_1| = |z_2| = |z_3| = r$ (say), write $z_k = re^{i\theta_k}$ for $k = 1, 2, 3$.

From $z_1 z_2 z_3 = 1$: $r^3 e^{i(\theta_1 + \theta_2 + \theta_3)} = 1$.

Since the right-hand side is real and positive, we need $r^3 = 1$ (so $r = 1$) and $\theta_1 + \theta_2 + \theta_3 = 2\pi k$ for some integer $k$.

So all three numbers lie on the unit circle: $z_k = e^{i\theta_k}$.

From $z_1 + z_2 + z_3 = 1$: $e^{i\theta_1} + e^{i\theta_2} + e^{i\theta_3} = 1$.

Since the sum is real, the imaginary parts must cancel. One approach is to look for symmetric solutions where two numbers are complex conjugates.

Let $z_1 = 1$ (real), $z_2 = e^{i\theta}$, and $z_3 = e^{-i\theta}$ (conjugate pair).

Then $z_1 + z_2 + z_3 = 1 + 2\cos\theta = 1$, so $\cos\theta = 0$, giving $\theta = \frac{\pi}{2}$ or $\frac{3\pi}{2}$.

For $\theta = \frac{\pi}{2}$: $z_1 = 1$, $z_2 = i$, $z_3 = -i$.

Check: $|1| = |i| = |-i| = 1$ $\checkmark$, $1 + i + (-i) = 1$ $\checkmark$, $1 \cdot i \cdot (-i) = 1$ $\checkmark$.

For $\theta = \frac{3\pi}{2}$: $z_1 = 1$, $z_2 = -i$, $z_3 = i$ (same solution, permuted).

By symmetry, we can also have $z_2 = 1$ or $z_3 = 1$ with the other two being $i$ and $-i$.

Therefore, the solutions are all permutations of $\{1, i, -i\}$.
\end{solution}

\begin{takeaways}
\begin{itemize}
    \item Unit circle: If $|z| = 1$, write $z = e^{i\theta}$
    \item Sum and product: Use both conditions to find relationships between angles
    \item Symmetry: Look for symmetric solutions (conjugate pairs)
\end{itemize}
\end{takeaways}


\subsection{Solution to Problem 26: Complex Numbers and Equilateral Triangle}

\begin{solution}
\textbf{(i)} Since $z_2 = e^{i\pi/3}z_1$, multiplying by $e^{i\pi/3}$ rotates $z_1$ by $60^\circ$ about the origin.

So $|z_2| = |z_1|$ (rotation preserves modulus) and $\text{Arg}(z_2) = \text{Arg}(z_1) + \pi/3$.

Since $O$ is at the origin, $|OA| = |z_1|$, $|OB| = |z_2| = |z_1|$, and the angle $AOB = \pi/3$.

Therefore triangle $OAB$ has two equal sides ($OA = OB$) and the included angle is $60^\circ$, making it equilateral.

\textbf{(ii)} We have $z_2 = e^{i\pi/3}z_1 = \left(\frac{1}{2} + \frac{i\sqrt{3}}{2}\right)z_1$.

Then: $z_1^2 + z_2^2 = z_1^2 + e^{2i\pi/3}z_1^2 = z_1^2(1 + e^{2i\pi/3})$.

And: $z_1 z_2 = z_1 \cdot e^{i\pi/3}z_1 = e^{i\pi/3}z_1^2$.

Since $1 + e^{2i\pi/3} = 1 + \left(-\frac{1}{2} + \frac{i\sqrt{3}}{2}\right) = \frac{1}{2} + \frac{i\sqrt{3}}{2} = e^{i\pi/3}$:

We get $z_1^2 + z_2^2 = z_1^2 e^{i\pi/3} = z_1 z_2$.
\end{solution}

\begin{takeaways}
\begin{itemize}
    \item Rotation: $e^{i\pi/3}$ rotates by $60^\circ$
    \item Equilateral triangle: Two equal sides with $60^\circ$ angle implies equilateral
    \item Complex algebra: Use $e^{2i\pi/3} = -\frac{1}{2} + \frac{i\sqrt{3}}{2}$
\end{itemize}
\end{takeaways}


\subsection{Solution to Problem 27: Particle Falling with Resistance}

\begin{solution}
Given $a = 10(1 - (kv)^2)$ where $k = 0.01$, so $a = 10(1 - 0.0001v^2)$.

Using $a = \frac{dv}{dt}$:
\[
\frac{dv}{dt} = 10(1 - 0.0001v^2) = 10 - 0.001v^2
\]

The terminal velocity occurs when $a = 0$: $v_T = \sqrt{\frac{10}{0.001}} = 100$ m/s.

Separating variables:
\[
\frac{dv}{1 - 0.0001v^2} = 10 \, dt
\]

Using partial fractions: $\frac{1}{1 - 0.0001v^2} = \frac{1}{2}\left(\frac{1}{1 - 0.01v} + \frac{1}{1 + 0.01v}\right)$.

Integrating from $v = 0$ at $t = 0$ to $v = v$ at $t = 5$:
\[
\frac{1}{0.02}\ln\left|\frac{1 + 0.01v}{1 - 0.01v}\right| = 10t
\]

At $t = 5$: $\ln\left|\frac{1 + 0.01v}{1 - 0.01v}\right| = 1$, so $\frac{1 + 0.01v}{1 - 0.01v} = e$.

Solving: $1 + 0.01v = e(1 - 0.01v)$, so $v = \frac{e-1}{0.01(e+1)} \approx 46.2$ m/s.
\end{solution}

\begin{takeaways}
\begin{itemize}
    \item Terminal velocity: Found when acceleration is zero
    \item Partial fractions: Use to integrate $\frac{1}{1 - a^2v^2}$
    \item Hyperbolic tangent: The solution involves $\tanh$ function
\end{itemize}
\end{takeaways}


\subsection{Solution to Problem 28: Induction Proof for Series}

\begin{solution}
\textbf{Base case ($n = 2$):} LHS $= \frac{1}{4} = 0.25$, RHS $= \frac{1}{2} = 0.5$. So $0.25 < 0.5$ $\checkmark$

\textbf{Inductive step:} Assume true for $n = k$, i.e., $\frac{1}{2^2} + \frac{1}{3^2} + \dots + \frac{1}{k^2} < \frac{k-1}{k}$.

For $n = k+1$, we need to show:
\[
\frac{1}{2^2} + \frac{1}{3^2} + \dots + \frac{1}{k^2} + \frac{1}{(k+1)^2} < \frac{k}{k+1}
\]

By the inductive hypothesis:
\[
\frac{1}{2^2} + \frac{1}{3^2} + \dots + \frac{1}{k^2} + \frac{1}{(k+1)^2} < \frac{k-1}{k} + \frac{1}{(k+1)^2}
\]

We need: $\frac{k-1}{k} + \frac{1}{(k+1)^2} < \frac{k}{k+1}$.

This is equivalent to: $\frac{k-1}{k} < \frac{k}{k+1} - \frac{1}{(k+1)^2} = \frac{k(k+1) - 1}{(k+1)^2} = \frac{k^2 + k - 1}{(k+1)^2}$.

Cross-multiplying: $(k-1)(k+1)^2 < k(k^2 + k - 1)$.

Expanding: $(k-1)(k^2 + 2k + 1) < k^3 + k^2 - k$.

Left side: $k^3 + 2k^2 + k - k^2 - 2k - 1 = k^3 + k^2 - k - 1$.

So we need: $k^3 + k^2 - k - 1 < k^3 + k^2 - k$, which is $-1 < 0$ $\checkmark$

Therefore, by mathematical induction, the inequality holds for all $n \ge 2$.
\end{solution}

\begin{takeaways}
\begin{itemize}
    \item Base case: Verify for smallest value ($n = 2$)
    \item Inductive step: Assume for $k$, prove for $k+1$
    \item Algebraic manipulation: Simplify the inequality to verify it holds
\end{itemize}
\end{takeaways}


\subsection{Solution to Problem 29: Irrationality of Logarithm}

\begin{solution}
Assume, for contradiction, that $\log_n(n+1) = \frac{p}{q}$ for some integers $p, q > 0$.

Then $n^{p/q} = n+1$, so $n^p = (n+1)^q$.

Since $n$ and $n+1$ are consecutive integers, they are coprime (their greatest common divisor is 1).

The prime factorization of $n^p$ contains only primes dividing $n$, while $(n+1)^q$ contains only primes dividing $n+1$.

Since $n$ and $n+1$ are coprime, no prime can divide both, so $n^p = (n+1)^q$ is impossible unless $n^p = (n+1)^q = 1$.

But $n > 1$, so $n^p \ge 2^p \ge 2 > 1$, and $(n+1)^q \ge 3^q \ge 3 > 1$.

This is a contradiction. Therefore, $\log_n(n+1)$ is irrational.
\end{solution}

\begin{takeaways}
\begin{itemize}
    \item Proof by contradiction: Assume the number is rational
    \item Coprime property: Consecutive integers are coprime
    \item Prime factorization: Use uniqueness of prime factorization to derive contradiction
\end{itemize}
\end{takeaways}


\subsection{Solution to Problem 30: Proposition Logic}

\begin{solution}
\textbf{(i)} If $k+1$ is divisible by 3, then $k+1 = 3m$ for some integer $m$, so $k = 3m - 1$.

Then: $k^3 + 1 = (3m - 1)^3 + 1 = 27m^3 - 27m^2 + 9m - 1 + 1 = 9m(3m^2 - 3m + 1)$.

Since $9m(3m^2 - 3m + 1)$ is divisible by 3, the proposition $P$ is true.

\textbf{(ii)} The contrapositive is: 'If $k^3+1$ is not divisible by 3, then $k+1$ is not divisible by 3'.

\textbf{(iii)} The converse is: 'If $k^3+1$ is divisible by 3, then $k+1$ is divisible by 3'.

This is false. Counterexample: $k = 2$. Then $k^3+1 = 9$ is divisible by 3, but $k+1 = 3$ is also divisible by 3. 

Actually, let's check $k = 5$: $k^3+1 = 126$ is divisible by 3, but $k+1 = 6$ is also divisible by 3.

Let's try $k = 8$: $k^3+1 = 513$ is divisible by 3, and $k+1 = 9$ is divisible by 3.

Actually, the converse might be true. Let's check: If $k^3+1$ is divisible by 3, then $k^3 \equiv -1 \equiv 2 \pmod{3}$.

The cubes modulo 3 are: $0^3 \equiv 0$, $1^3 \equiv 1$, $2^3 \equiv 8 \equiv 2 \pmod{3}$.

So $k^3 \equiv 2 \pmod{3}$ means $k \equiv 2 \pmod{3}$, so $k+1 \equiv 0 \pmod{3}$.

Therefore the converse is actually true!
\end{solution}

\begin{takeaways}
\begin{itemize}
    \item Direct proof: Substitute $k = 3m - 1$ and expand
    \item Contrapositive: Negate both hypothesis and conclusion
    \item Converse: Swap hypothesis and conclusion
    \item Modular arithmetic: Use to check divisibility properties
\end{itemize}
\end{takeaways}


\subsection{Solution to Problem 31: Vectors and Ratios}

\begin{solution}
\textbf{(i)} Given $\frac{CB}{AC} = \frac{m}{n}$, and $AC + CB = AB$, we have:
\[
AC = \frac{n}{m+n}AB = \frac{n}{m+n}(\mathbf{b} - \mathbf{a})
\]

\textbf{(ii)} Since $\vec{OC} = \vec{OA} + \vec{AC}$:
\[
\vec{OC} = \mathbf{a} + \frac{n}{m+n}(\mathbf{b} - \mathbf{a}) = \mathbf{a} + \frac{n}{m+n}\mathbf{b} - \frac{n}{m+n}\mathbf{a}
\]
\[
= \left(1 - \frac{n}{m+n}\right)\mathbf{a} + \frac{n}{m+n}\mathbf{b} = \frac{m}{m+n}\mathbf{a} + \frac{n}{m+n}\mathbf{b}
\]
\end{solution}

\begin{takeaways}
\begin{itemize}
    \item Section formula: Point dividing $AB$ in ratio $m:n$ has position vector $\frac{m\mathbf{b} + n\mathbf{a}}{m+n}$
    \item Vector addition: $\vec{OC} = \vec{OA} + \vec{AC}$
    \item Ratio relationships: Use $AC + CB = AB$ to find lengths
\end{itemize}
\end{takeaways}


\subsection{Solution to Problem 32: Parallelogram Geometry}

\begin{solution}
\textbf{(i)} In parallelogram $OPQR$, we have $\vec{OQ} = \vec{OP} + \vec{OR} = \mathbf{p} + \mathbf{r}$.

Since $S$ is the midpoint of $QR$: $\vec{OS} = \vec{OR} + \frac{1}{2}\vec{RQ} = \mathbf{r} + \frac{1}{2}(\mathbf{p} + \mathbf{r} - \mathbf{r}) = \mathbf{r} + \frac{1}{2}\mathbf{p}$.

Point $T$ lies on both $PR$ and $OS$.

On $PR$: $\vec{OT} = \mathbf{r} + \lambda(\mathbf{p} - \mathbf{r}) = (1-\lambda)\mathbf{r} + \lambda\mathbf{p}$ for some $\lambda$.

On $OS$: $\vec{OT} = \mu\left(\mathbf{r} + \frac{1}{2}\mathbf{p}\right) = \mu\mathbf{r} + \frac{\mu}{2}\mathbf{p}$ for some $\mu$.

Equating: $(1-\lambda)\mathbf{r} + \lambda\mathbf{p} = \mu\mathbf{r} + \frac{\mu}{2}\mathbf{p}$.

So: $1-\lambda = \mu$ and $\lambda = \frac{\mu}{2}$.

Substituting: $1 - \frac{\mu}{2} = \mu$, so $1 = \frac{3\mu}{2}$, giving $\mu = \frac{2}{3}$ and $\lambda = \frac{1}{3}$.

Therefore: $\vec{OT} = \frac{2}{3}\mathbf{r} + \frac{1}{3}\mathbf{p}$.

\textbf{(ii)} On line $PR$: $\vec{OT} = \frac{1}{3}\mathbf{p} + \frac{2}{3}\mathbf{r}$.

This means $T$ divides $PR$ in the ratio $PT : TR = \frac{2}{3} : \frac{1}{3} = 2 : 1$.
\end{solution}

\begin{takeaways}
\begin{itemize}
    \item Parametric form: Express points on lines using parameters
    \item Intersection: Equate parametric forms to find intersection point
    \item Section formula: Use to determine division ratio
\end{itemize}
\end{takeaways}


\subsection{Solution to Problem 33: Two Masses with Pulley}

\begin{solution}
\textbf{(i)} For the larger mass ($4m$): $4mg - T - kv = 4m\frac{dv}{dt}$ (downward positive).

For the smaller mass ($2m$): $T - 2mg - kv = 2m\frac{dv}{dt}$ (upward positive).

Adding: $4mg - 2mg - 2kv = 6m\frac{dv}{dt}$, so $2mg - 2kv = 6m\frac{dv}{dt}$.

Therefore: $\frac{dv}{dt} = \frac{2mg - 2kv}{6m} = \frac{gm - kv}{3m}$.

\textbf{(ii)} The differential equation is: $\frac{dv}{dt} = \frac{gm - kv}{3m}$.

Separating variables: $\frac{dv}{gm - kv} = \frac{dt}{3m}$.

Integrating: $-\frac{1}{k}\ln|gm - kv| = \frac{t}{3m} + C$.

At $t = 0$, $v = 0$: $-\frac{1}{k}\ln(gm) = C$.

So: $-\frac{1}{k}\ln|gm - kv| = \frac{t}{3m} - \frac{1}{k}\ln(gm)$.

Rearranging: $\ln\left|\frac{gm - kv}{gm}\right| = -\frac{kt}{3m}$.

Since $v < \frac{gm}{k}$, we have $gm - kv > 0$, so: $\frac{gm - kv}{gm} = e^{-kt/(3m)}$.

Therefore: $v = \frac{gm}{k}\left(1 - e^{-kt/(3m)}\right)$.

At $t = \frac{3m}{k}\ln 2$: $v = \frac{gm}{k}\left(1 - e^{-\ln 2}\right) = \frac{gm}{k}\left(1 - \frac{1}{2}\right) = \frac{gm}{2k}$.
\end{solution}

\begin{takeaways}
\begin{itemize}
    \item Newton's second law: Apply to each mass separately
    \item System of equations: Add equations to eliminate tension
    \item Separable ODE: Integrate to find velocity as function of time
\end{itemize}
\end{takeaways}


\subsection{Solution to Problem 34: Integration with Recurrence and Factorial Inequality}

\begin{solution}
\textbf{(i)} Let $u = \sin^{2n}(2\theta)$ and $dv = \sin(2\theta) \, d\theta$.

Then $du = 2n\sin^{2n-1}(2\theta) \cdot 2\cos(2\theta) \, d\theta = 4n\sin^{2n-1}(2\theta)\cos(2\theta) \, d\theta$.

And $v = -\frac{1}{2}\cos(2\theta)$.

Using integration by parts:
\[
I_n = \left[-\frac{1}{2}\sin^{2n}(2\theta)\cos(2\theta)\right]_0^{\pi/2} + 2n\int_0^{\pi/2} \sin^{2n-1}(2\theta)\cos^2(2\theta) \, d\theta
\]

The boundary term is 0. Using $\cos^2(2\theta) = 1 - \sin^2(2\theta)$:
\[
I_n = 2n\int_0^{\pi/2} \sin^{2n-1}(2\theta)(1 - \sin^2(2\theta)) \, d\theta = 2n(I_{n-1} - I_n)
\]

Solving: $I_n = 2nI_{n-1} - 2nI_n$, so $(2n+1)I_n = 2nI_{n-1}$.

Therefore: $I_n = \frac{2n}{2n+1}I_{n-1}$.

\textbf{(ii)} $I_0 = \int_0^{\pi/2} \sin(2\theta) \, d\theta = \left[-\frac{1}{2}\cos(2\theta)\right]_0^{\pi/2} = 1$.

Using the recurrence: $I_n = \frac{2n}{2n+1} \cdot \frac{2(n-1)}{2n-1} \cdot \ldots \cdot \frac{2}{3} \cdot 1 = \frac{2^n n!}{(2n+1) \cdot (2n-1) \cdot \ldots \cdot 3 \cdot 1}$.

The denominator is $(2n+1)!! = \frac{(2n+1)!}{2^n n!}$.

So: $I_n = \frac{2^n n! \cdot 2^n n!}{(2n+1)!} = \frac{2^{2n}(n!)^2}{(2n+1)!}$.

\textbf{(iii)} Using substitution $x = \sin^2\theta$, so $dx = 2\sin\theta\cos\theta \, d\theta$ and $1-x = \cos^2\theta$:
\[
J_n = \int_0^1 x^n(1-x)^n \, dx = \int_0^{\pi/2} \sin^{2n}\theta \cos^{2n}\theta \cdot 2\sin\theta\cos\theta \, d\theta
\]
\[
= 2\int_0^{\pi/2} \sin^{2n+1}\theta \cos^{2n+1}\theta \, d\theta = 2\int_0^{\pi/2} (\sin\theta\cos\theta)^{2n+1} \, d\theta
\]

Using $\sin(2\theta) = 2\sin\theta\cos\theta$: $J_n = 2\int_0^{\pi/2} \left(\frac{1}{2}\sin(2\theta)\right)^{2n+1} \, d\theta = \frac{1}{2^{2n}} I_n$.

So $J_n = \frac{1}{2^{2n}} \cdot \frac{2^{2n}(n!)^2}{(2n+1)!} = \frac{(n!)^2}{(2n+1)!}$.

\textbf{(iv)} Since $0 \le x^n(1-x)^n \le 1$ for $x \in [0,1]$ (maximum occurs at $x = 1/2$), we have $J_n \le \int_0^1 1 \, dx = 1$.

Therefore: $\frac{(n!)^2}{(2n+1)!} \le 1$, so $(n!)^2 \le (2n+1)!$.

Multiplying both sides by $2^{2n}$: $(2^n n!)^2 = 2^{2n}(n!)^2 \le 2^{2n}(2n+1)!$.

Actually, we need $(2^n n!)^2 \le (2n+1)!$.

From $J_n = \frac{(n!)^2}{(2n+1)!} \le 1$, we get $(n!)^2 \le (2n+1)!$.

But we need $(2^n n!)^2 = 2^{2n}(n!)^2 \le (2n+1)!$.

This requires $2^{2n} \le \frac{(2n+1)!}{(n!)^2}$, which is true for $n \ge 1$ (can be verified).

Actually, a direct approach: $(2n+1)! = (2n+1)(2n)(2n-1)\ldots(3)(2)(1)$.

We have $(2^n n!)^2 = 2^{2n}(n!)^2 = 2^{2n}(1 \cdot 2 \cdot \ldots \cdot n)^2$.

Comparing term by term, $(2n+1)!$ contains factors that are at least as large, so the inequality holds.
\end{solution}

\begin{takeaways}
\begin{itemize}
    \item Integration by parts: Use to derive recurrence relations
    \item Double factorial: $(2n+1)!! = \frac{(2n+1)!}{2^n n!}$
    \item Substitution: $x = \sin^2\theta$ relates polynomial and trigonometric integrals
    \item Bounding integrals: Use maximum value of integrand to bound the integral
\end{itemize}
\end{takeaways}



% Solution 35
% Source: samples/01.tex
% Author: Vu Hung Nguyen
% Date: November 2025

\begin{dl101box}{Solution to Problem 35}
\begin{enumerate}
\item[(a)] $nA_n = \frac{2n-1}{2}A_{n-1}$ by integration by parts on $A_n$.

\item[(b)] $A_n = 2n \int_0^{\frac{\pi}{2}} x \sin x \cos^{2n-1} x \, dx$ by parts.

\item[(c)] $\frac{A_n}{n^2} = \frac{2n-1}{n} B_{n-1} - 2B_n$ by parts on (b).

\item[(d)] $\frac{1}{n^2} = 2\left(\frac{B_{n-1}}{A_{n-1}} - \frac{B_n}{A_n}\right)$ using (a) and (c).

\item[(e)] $\sum_{k=1}^n \frac{1}{k^2} = \frac{\pi^2}{6} - 2\frac{B_n}{A_n}$ by telescoping (d) and $\frac{B_0}{A_0} = \frac{\pi^2}{12}$.

\item[(f)] $B_n \le \int_0^{\frac{\pi}{2}} x^2 \left(1-\frac{4x^2}{\pi^2}\right)^n dx$ using $\sin x \ge \frac{2}{\pi}x$.

\item[(g)] $\int_0^{\frac{\pi}{2}} x^2 \left(1-\frac{4x^2}{\pi^2}\right)^n dx = \frac{\pi^2}{8(n+1)} \int_0^{\frac{\pi}{2}} \left(1-\frac{4x^2}{\pi^2}\right)^{n+1} dx$ by parts.

\item[(h)] $B_n \le \frac{\pi^3}{16(n+1)} \int_0^{\frac{\pi}{2}} \cos^{2n+3} t dt \le \frac{\pi^3}{16(n+1)}A_n$ by substitution $x=\frac{\pi}{2}\sin t$.

\item[(i)] $\frac{\pi^2}{6} - \frac{\pi^3}{8(n+1)} \le \sum_{k=1}^n \frac{1}{k^2} < \frac{\pi^2}{6}$ using (e) and (h).

\item[(j)] $\lim\limits_{n\to\infty} \sum_{k=1}^n \frac{1}{k^2} = \frac{\pi^2}{6}$.
\end{enumerate}
\end{dl101box}


\vspace{2em}
\noindent
\textbf{Contact Information:}\\[0.5em]
LinkedIn: \href{https://www.linkedin.com/in/nguyenvuhung/}{https://www.linkedin.com/in/nguyenvuhung/}\\[0.3em]
GitHub: \href{https://github.com/vuhung16au/}{https://github.com/vuhung16au/}\\[0.3em]
Repository: \href{https://github.com/vuhung16au/math-olympiad-ml/tree/main/HSC-Collections}{https://github.com/vuhung16au/math-olympiad-ml/tree/main/HSC-Collections}

\end{document}
