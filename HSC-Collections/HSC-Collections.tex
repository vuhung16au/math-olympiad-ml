\documentclass[12pt]{article}

% Load style packages
\usepackage{amsmath}
\usepackage{amssymb}
\usepackage{amsthm}
\usepackage{geometry}
\usepackage[utf8]{inputenc}
\usepackage{hyperref}
\usepackage{graphicx}

% Load custom styles
\usepackage{styles/dl101-colors}
\usepackage{styles/dl101-hints}
\usepackage{styles/dl101-hsc-problems}

\geometry{
 a4paper,
 total={170mm,257mm},
 left=20mm,
 top=20mm,
}

\author{Vu Hung Nguyen}
\title{HSC Mathematics Extension 2: Collection of Hard Problems}
\date{}

\begin{document}

\maketitle

\section{Overview}

This collection presents a curated set of challenging problems from the HSC Mathematics Extension 2 curriculum. These problems are designed to test deep understanding, creative problem-solving skills, and the ability to synthesize multiple mathematical concepts.

\subsection{What This Collection Is About}

This collection focuses on \textbf{HSC Mathematics Extension 2 (hard problems)}. The problems span various topics including:
\begin{itemize}
    \item Complex numbers and their geometric interpretations
    \item Integration techniques and applications
    \item Vector geometry in three dimensions
    \item Mechanics and particle motion
    \item Inequalities and optimization
\end{itemize}

Each problem is carefully selected to represent the level of difficulty and sophistication expected in the most challenging HSC Extension 2 examinations.

\subsection{Target Audience}

This collection is designed for:
\begin{itemize}
    \item \textbf{Students} preparing for HSC Mathematics Extension 2 who want to challenge themselves with difficult problems
    \item \textbf{Tutors} seeking high-quality problems to use in their teaching
    \item \textbf{Educators} looking for challenging problems to incorporate into their curriculum
\end{itemize}

The problems are presented with hints to guide thinking, followed by detailed solutions and key takeaways to reinforce learning.

\section{Problems}

\subsection{Problem 1: Two Particles in Resisting Medium}

\begin{problem}
Two particles, $A$ and $B$, each have mass 1 kg and are in a medium that exerts a resistance to motion equal to $kv$, where $k > 0$ and $v$ is the velocity of any particle. Both particles maintain vertical trajectories.

The acceleration due to gravity is $g$ m s$^{-2}$, where $g > 0$.

The two particles are simultaneously projected towards each other with the same speed, $v_0$ m s$^{-1}$, where $0 < v_0 < \frac{g}{k}$.

The particle $A$ is initially $d$ metres directly above particle $B$, where $d < \frac{2v_0}{k}$.

Find the time taken for the particles to meet.
\end{problem}

\begin{hint}
Consider the equations of motion for each particle under the influence of gravity and resistance. Set up differential equations for the velocities and positions. The condition for the particles to meet will give you an equation involving time.
\end{hint}

\subsection{Problem 2: Complex Square with Equilateral Triangle}

\begin{problem}
A square in the Argand plane has vertices
\[
5 + 5i, \quad 5 - 5i, \quad -5 - 5i \quad \text{and} \quad -5 + 5i.
\]
The complex numbers $z_A = 5 + i$, $z_B$ and $z_C$ lie on the square and form the vertices of an equilateral triangle.

Find the exact value of the complex number $z_B$.
\end{problem}

\begin{hint}
Use the geometric properties of equilateral triangles. Consider rotations in the complex plane. The vertices of an equilateral triangle are related by rotations of $60^\circ$ or $120^\circ$ about the centroid.
\end{hint}

\subsection{Problem 3: Complex 7th Root of Unity}

\begin{problem}
Let $w$ be a complex number such that $1 + w + w^2 + \dots + w^6 = 0$.

\begin{enumerate}
    \item[(i)] Show that $w$ is a 7th root of unity.
    
    \item[(ii)] The complex number $\alpha = w + w^2 + w^4$ is a root of the equation $x^2 + bx + c = 0$, where $b$ and $c$ are real and $\alpha$ is not real. Find the other root of $x^2 + bx + c = 0$ in terms of positive powers of $w$.
    
    \item[(iii)] Find the numerical value of $c$.
\end{enumerate}
\end{problem}

\begin{hint}
For part (i), use the formula for the sum of a geometric series. For part (ii), use the fact that for a quadratic with real coefficients, the other root is the complex conjugate. For part (iii), use properties of roots of unity and their relationships.
\end{hint}

\subsection{Problem 4: Complex Triangle Inequality}

\begin{problem}
The complex number $z$ satisfies $\left| z - \frac{4}{z} \right| = 2$.

Using the triangle inequality, or otherwise, show that $|z| \leq \sqrt{5} + 1$.
\end{problem}

\begin{hint}
Apply the triangle inequality to $|z - \frac{4}{z}|$. Consider both directions: $|z| - |\frac{4}{z}| \leq |z - \frac{4}{z}|$ and $|z - \frac{4}{z}| \leq |z| + |\frac{4}{z}|$. Use the given condition to derive bounds on $|z|$.
\end{hint}

\subsection{Problem 5: Integral with Inverse Sine}

\begin{problem}
Using the substitution $x = \tan^2 \theta$, evaluate
\[
\int_{0}^{1} \sin^{-1} \sqrt{\frac{x}{1+x}} \,dx.
\]
\end{problem}

\begin{hint}
After making the substitution, simplify the integrand. You may need to use trigonometric identities. Consider the relationship between $\sin^{-1}$ and the substitution variable.
\end{hint}

\subsection{Problem 6: Integral with Cotangent}

\begin{problem}
Let $I_n = \int_{\frac{\pi}{4}}^{\frac{\pi}{2}} \cot^{2n}\theta \, d\theta$ for integers $n \ge 0$.

\begin{enumerate}
    \item[(i)] Show that $I_n = \frac{1}{2n-1} - I_{n-1}$ for $n > 0$, given that $\frac{d}{d\theta}\cot\theta = -\csc^2\theta$.
    
    \item[(ii)] Hence, or otherwise, calculate $I_2$.
\end{enumerate}
\end{problem}

\begin{hint}
For part (i), use integration by parts. Write $\cot^{2n}\theta = \cot^{2n-2}\theta \cdot \cot^2\theta$ and use the identity $\cot^2\theta = \csc^2\theta - 1$. For part (ii), use the recurrence relation and find $I_0$ first.
\end{hint}

\subsection{Problem 7: 3D Vectors and Distance}

\begin{problem}
Consider the point $B$ with three-dimensional position vector $\mathbf{b}$ and the line $l$: $\mathbf{a} + \lambda \mathbf{d}$, where $\mathbf{a}$ and $\mathbf{d}$ are three-dimensional vectors, $|\mathbf{d}| = 1$ and $\lambda$ is a parameter.

Let $f(\lambda)$ be the distance between a point on the line $l$ and the point $B$.

\begin{enumerate}
    \item[(i)] Find $\lambda_0$, the value of $\lambda$ that minimises $f$, in terms of $\mathbf{a}$, $\mathbf{b}$ and $\mathbf{d}$.
    
    \item[(ii)] Let $P$ be the point with position vector $\mathbf{a} + \lambda_0 \mathbf{d}$. Show that $PB$ is perpendicular to the direction of the line $l$.
    
    \item[(iii)] Hence, or otherwise, find the shortest distance between the line $l$ and the sphere of radius 1 unit, centred at the origin $O$, in terms of $\mathbf{d}$ and $\mathbf{a}$.
    
    You may assume that if $B$ is the point on the sphere closest to $l$, then $OBP$ is a straight line.
\end{enumerate}
\end{problem}

\begin{hint}
For part (i), minimize $f^2(\lambda)$ by differentiating. For part (ii), use the fact that the minimum distance occurs when the vector from $P$ to $B$ is perpendicular to the direction vector. For part (iii), use the geometric interpretation and the given assumption.
\end{hint}

\subsection{Problem 8: Triangle Inequality and Rectangular Prism}

\begin{problem}
It is given that for positive numbers $x_1, x_2, x_3, \dots, x_n$ with arithmetic mean $A$,
\[
\frac{x_1 \times x_2 \times x_3 \times \dots \times x_n}{A^n} \leq 1. \quad (\text{Do NOT prove this.})
\]

Suppose a rectangular prism has dimensions $a, b, c$ and surface area $S$.

\begin{enumerate}
    \item[(i)] Show that $abc \leq \left(\frac{S}{6}\right)^{\frac{3}{2}}$.
    
    \item[(ii)] Using part (i), show that when the rectangular prism with surface area $S$ is a cube, it has maximum volume.
\end{enumerate}
\end{problem}

\begin{hint}
For part (i), relate the surface area to the dimensions and apply the given inequality. For part (ii), show that equality in the inequality from part (i) occurs when $a = b = c$, which corresponds to a cube.
\end{hint}

\section{Solutions}

\subsection{Solution to Problem 1: Two Particles in Resisting Medium}

\begin{solution}
Let $y_A(t)$ and $y_B(t)$ be the positions of particles $A$ and $B$ respectively, with $y_A(0) = d$ and $y_B(0) = 0$. The equation of motion for each particle is:
\[
\frac{dv}{dt} = -g - kv
\]
Solving this first-order linear ODE with $v(0) = -v_0$ (downward for $A$) and $v(0) = v_0$ (upward for $B$):
\[
v(t) = -\frac{g}{k} + \left(v_0 + \frac{g}{k}\right)e^{-kt}
\]
Integrating to find position:
\[
y(t) = y(0) - \frac{g}{k}t + \frac{1}{k}\left(v_0 + \frac{g}{k}\right)(1 - e^{-kt})
\]
For particle $A$: $y_A(t) = d - \frac{g}{k}t + \frac{1}{k}\left(v_0 + \frac{g}{k}\right)(1 - e^{-kt})$

For particle $B$: $y_B(t) = \frac{g}{k}t - \frac{1}{k}\left(v_0 + \frac{g}{k}\right)(1 - e^{-kt})$

Setting $y_A(t) = y_B(t)$ and solving:
\[
d = \frac{2}{k}\left(v_0 + \frac{g}{k}\right)(1 - e^{-kt})
\]
Therefore, $t = -\frac{1}{k}\ln\left(1 - \frac{kd}{2(v_0 + g/k)}\right)$.
\end{solution}

\begin{takeaways}
\begin{itemize}
    \item Motion with linear resistance: $dv/dt = -g - kv$ has solution $v = -g/k + (v_0 + g/k)e^{-kt}$
    \item Relative motion: Set positions equal to find meeting time
    \item Exponential decay in velocity due to resistance
\end{itemize}
\end{takeaways}

\subsection{Solution to Problem 2: Complex Square with Equilateral Triangle}

\begin{solution}
The centroid of the equilateral triangle is at $z_G = \frac{z_A + z_B + z_C}{3}$. Since $z_A = 5 + i$ lies on the right side of the square, and the triangle is equilateral, the vertices are related by $120^\circ$ rotations about the centroid.

Alternatively, note that $z_B$ and $z_C$ must lie on the square's perimeter. Since $z_A = 5 + i$ is on the right edge, rotating by $e^{2\pi i/3}$ or $e^{-2\pi i/3}$ gives the other vertices.

Let $z_B = 5 + bi$ where $-5 < b < 5$ (on right edge) or $z_B = a + 5i$ where $-5 < a < 5$ (on top edge). Using the rotation property: $z_B - z_G = e^{2\pi i/3}(z_A - z_G)$.

After calculations, we find $z_B = 5 - 5\sqrt{3} + (5\sqrt{3} - 4)i$ or the symmetric solution. The exact value depends on the orientation, but one solution is:
\[
z_B = 5 - 5\sqrt{3} + (5\sqrt{3} - 4)i
\]
\end{solution}

\begin{takeaways}
\begin{itemize}
    \item Equilateral triangles: vertices are $120^\circ$ rotations of each other about the centroid
    \item Complex rotations: multiply by $e^{\pm 2\pi i/3}$ to rotate by $\pm 120^\circ$
    \item Constraint: vertices must lie on the square's perimeter
\end{itemize}
\end{takeaways}

\subsection{Solution to Problem 3: Complex 7th Root of Unity}

\begin{solution}
\textbf{(i)} If $w = 1$, then $1 + w + \dots + w^6 = 7 \neq 0$. So $w \neq 1$. Using the geometric series formula:
\[
1 + w + w^2 + \dots + w^6 = \frac{1 - w^7}{1 - w} = 0
\]
Since $w \neq 1$, we have $1 - w^7 = 0$, so $w^7 = 1$. Therefore $w$ is a 7th root of unity.

\textbf{(ii)} Since the quadratic has real coefficients, the other root is $\overline{\alpha} = \overline{w + w^2 + w^4} = \overline{w} + \overline{w^2} + \overline{w^4}$.

For a 7th root of unity $w = e^{2\pi ik/7}$ where $k \in \{1,2,3,4,5,6\}$, we have $\overline{w} = w^{-1} = w^6$. Similarly, $\overline{w^2} = w^5$ and $\overline{w^4} = w^3$.

Therefore, $\overline{\alpha} = w^6 + w^5 + w^3$.

\textbf{(iii)} For $w = e^{2\pi i/7}$, we have:
\[
c = \alpha \cdot \overline{\alpha} = (w + w^2 + w^4)(w^6 + w^5 + w^3)
\]
Expanding and using $w^7 = 1$:
\[
c = w^7 + w^6 + w^5 + w^8 + w^7 + w^6 + w^{10} + w^9 + w^7 = 3 + (w + w^2 + w^3 + w^4 + w^5 + w^6)
\]
Since $1 + w + w^2 + \dots + w^6 = 0$, we get $w + w^2 + \dots + w^6 = -1$, so $c = 3 - 1 = 2$.
\end{solution}

\begin{takeaways}
\begin{itemize}
    \item 7th roots of unity: $w^7 = 1$ with $w \neq 1$ implies $1 + w + \dots + w^6 = 0$
    \item Complex conjugate: For $w = e^{2\pi ik/7}$, we have $\overline{w} = w^{-1} = w^6$
    \item Product of roots: For real-coefficient quadratics, $c = \alpha \overline{\alpha} = |\alpha|^2$
\end{itemize}
\end{takeaways}

\subsection{Solution to Problem 4: Complex Triangle Inequality}

\begin{solution}
Given $|z - 4/z| = 2$. Applying the triangle inequality in both directions:

\textbf{Lower bound:} $|z - 4/z| \geq |z| - |4/z| = |z| - 4/|z|$

So $2 \geq |z| - 4/|z|$, which gives $|z|^2 - 2|z| - 4 \leq 0$.

Solving: $|z| \leq 1 + \sqrt{5}$.

\textbf{Upper bound:} $|z - 4/z| \leq |z| + |4/z| = |z| + 4/|z|$

So $2 \leq |z| + 4/|z|$, which gives $|z|^2 - 2|z| + 4 \geq 0$.

This quadratic has discriminant $4 - 16 = -12 < 0$, so it's always positive. This gives no upper bound.

Combining both: $|z| \leq 1 + \sqrt{5} = \sqrt{5} + 1$.
\end{solution}

\begin{takeaways}
\begin{itemize}
    \item Triangle inequality: $|a - b| \geq ||a| - |b||$ and $|a - b| \leq |a| + |b|$
    \item Apply both directions to get bounds on $|z|$
    \item Solve resulting quadratic inequalities
\end{itemize}
\end{takeaways}

\subsection{Solution to Problem 5: Integral with Inverse Sine}

\begin{solution}
Let $x = \tan^2 \theta$, so $dx = 2\tan\theta \sec^2\theta \, d\theta = 2\tan\theta(1 + \tan^2\theta) \, d\theta$.

When $x = 0$, $\theta = 0$; when $x = 1$, $\theta = \pi/4$.

The integrand becomes:
\[
\sin^{-1}\sqrt{\frac{x}{1+x}} = \sin^{-1}\sqrt{\frac{\tan^2\theta}{1+\tan^2\theta}} = \sin^{-1}\sqrt{\frac{\tan^2\theta}{\sec^2\theta}} = \sin^{-1}|\sin\theta| = \sin^{-1}(\sin\theta) = \theta
\]
for $\theta \in [0, \pi/4]$.

Therefore:
\[
\int_0^1 \sin^{-1}\sqrt{\frac{x}{1+x}} \, dx = \int_0^{\pi/4} \theta \cdot 2\tan\theta \sec^2\theta \, d\theta
\]
Using integration by parts with $u = \theta$, $dv = 2\tan\theta \sec^2\theta \, d\theta$:
\[
= \left[\theta \tan^2\theta\right]_0^{\pi/4} - \int_0^{\pi/4} \tan^2\theta \, d\theta = \frac{\pi}{4} - \int_0^{\pi/4} (\sec^2\theta - 1) \, d\theta
\]
\[
= \frac{\pi}{4} - [\tan\theta - \theta]_0^{\pi/4} = \frac{\pi}{4} - (1 - \frac{\pi}{4}) = \frac{\pi}{2} - 1
\]
\end{solution}

\begin{takeaways}
\begin{itemize}
    \item Substitution $x = \tan^2\theta$ simplifies $\sqrt{x/(1+x)}$ to $|\sin\theta|$
    \item For $\theta \in [0, \pi/4]$, we have $\sin^{-1}(\sin\theta) = \theta$
    \item Integration by parts: differentiate the polynomial part, integrate the trigonometric part
\end{itemize}
\end{takeaways}

\subsection{Solution to Problem 6: Integral with Cotangent}

\begin{solution}
\textbf{(i)} Write $I_n = \int_{\pi/4}^{\pi/2} \cot^{2n}\theta \, d\theta = \int_{\pi/4}^{\pi/2} \cot^{2n-2}\theta \cdot \cot^2\theta \, d\theta$.

Using $\cot^2\theta = \csc^2\theta - 1$:
\[
I_n = \int_{\pi/4}^{\pi/2} \cot^{2n-2}\theta (\csc^2\theta - 1) \, d\theta = \int_{\pi/4}^{\pi/2} \cot^{2n-2}\theta \csc^2\theta \, d\theta - I_{n-1}
\]

For the first integral, let $u = \cot\theta$, so $du = -\csc^2\theta \, d\theta$:
\[
\int_{\pi/4}^{\pi/2} \cot^{2n-2}\theta \csc^2\theta \, d\theta = -\int_1^0 u^{2n-2} \, du = \int_0^1 u^{2n-2} \, du = \frac{1}{2n-1}
\]

Therefore: $I_n = \frac{1}{2n-1} - I_{n-1}$.

\textbf{(ii)} First, $I_0 = \int_{\pi/4}^{\pi/2} d\theta = \frac{\pi}{4}$.

Using the recurrence: $I_1 = \frac{1}{1} - I_0 = 1 - \frac{\pi}{4}$.

Then: $I_2 = \frac{1}{3} - I_1 = \frac{1}{3} - (1 - \frac{\pi}{4}) = \frac{\pi}{4} - \frac{2}{3}$.
\end{solution}

\begin{takeaways}
\begin{itemize}
    \item Recurrence: Use $\cot^2\theta = \csc^2\theta - 1$ to relate $I_n$ and $I_{n-1}$
    \item Substitution: $u = \cot\theta$ converts $\cot^{2n-2}\theta \csc^2\theta \, d\theta$ to $u^{2n-2} \, du$
    \item Base case: $I_0 = \pi/4$ (integral of 1 over the interval)
\end{itemize}
\end{takeaways}

\subsection{Solution to Problem 7: 3D Vectors and Distance}

\begin{solution}
\textbf{(i)} The distance squared is $f^2(\lambda) = |(\mathbf{a} + \lambda\mathbf{d}) - \mathbf{b}|^2 = |\mathbf{a} - \mathbf{b} + \lambda\mathbf{d}|^2$.

Expanding: $f^2(\lambda) = |\mathbf{a} - \mathbf{b}|^2 + 2\lambda(\mathbf{a} - \mathbf{b}) \cdot \mathbf{d} + \lambda^2|\mathbf{d}|^2 = |\mathbf{a} - \mathbf{b}|^2 + 2\lambda(\mathbf{a} - \mathbf{b}) \cdot \mathbf{d} + \lambda^2$.

Differentiating: $\frac{d}{d\lambda}(f^2) = 2(\mathbf{a} - \mathbf{b}) \cdot \mathbf{d} + 2\lambda = 0$.

Therefore: $\lambda_0 = -(\mathbf{a} - \mathbf{b}) \cdot \mathbf{d} = (\mathbf{b} - \mathbf{a}) \cdot \mathbf{d}$.

\textbf{(ii)} The vector $\overrightarrow{PB} = \mathbf{b} - (\mathbf{a} + \lambda_0\mathbf{d}) = \mathbf{b} - \mathbf{a} - \lambda_0\mathbf{d}$.

Taking dot product with $\mathbf{d}$:
\[
\overrightarrow{PB} \cdot \mathbf{d} = (\mathbf{b} - \mathbf{a}) \cdot \mathbf{d} - \lambda_0 = (\mathbf{b} - \mathbf{a}) \cdot \mathbf{d} - (\mathbf{b} - \mathbf{a}) \cdot \mathbf{d} = 0
\]
Therefore $PB$ is perpendicular to the direction of line $l$.

\textbf{(iii)} If $B$ is on the sphere closest to $l$, then $OBP$ is a straight line, so $\mathbf{b}$ is parallel to $\mathbf{a} + \lambda_0\mathbf{d}$.

The shortest distance is $|\overrightarrow{PB}| = |\mathbf{b} - \mathbf{a} - \lambda_0\mathbf{d}|$.

Since $\mathbf{b}$ is on the unit sphere, $|\mathbf{b}| = 1$. Using the perpendicularity and the assumption:
\[
\text{Shortest distance} = |\mathbf{b} - \mathbf{a} - \lambda_0\mathbf{d}| = \sqrt{|\mathbf{a}|^2 - (\mathbf{a} \cdot \mathbf{d})^2} - 1
\]
if the sphere and line don't intersect, or $0$ if they do.
\end{solution}

\begin{takeaways}
\begin{itemize}
    \item Minimize $f^2(\lambda)$ by setting derivative to zero
    \item Minimum distance occurs when connecting vector is perpendicular to line direction
    \item Geometric interpretation: shortest distance from line to sphere
\end{itemize}
\end{takeaways}

\subsection{Solution to Problem 8: Triangle Inequality and Rectangular Prism}

\begin{solution}
\textbf{(i)} The surface area is $S = 2(ab + bc + ca)$.

The arithmetic mean of $ab$, $bc$, and $ca$ is $A = \frac{ab + bc + ca}{3} = \frac{S}{6}$.

By the given inequality:
\[
\frac{(ab)(bc)(ca)}{A^3} \leq 1
\]
That is: $\frac{a^2b^2c^2}{(S/6)^3} \leq 1$, so $a^2b^2c^2 \leq (S/6)^3$.

Taking square roots: $abc \leq (S/6)^{3/2}$.

\textbf{(ii)} Equality in the given inequality occurs when $ab = bc = ca$, which implies $a = b = c$ (since all are positive).

When $a = b = c$, the rectangular prism is a cube. Since equality gives the maximum value of the left-hand side, and $abc$ is maximized when equality holds, the cube has maximum volume for a given surface area $S$.
\end{solution}

\begin{takeaways}
\begin{itemize}
    \item AM-GM: For $n$ positive numbers, product $\leq$ (arithmetic mean)$^n$
    \item Equality: Occurs when all numbers are equal
    \item Optimization: Maximum volume for fixed surface area occurs at equality condition
\end{itemize}
\end{takeaways}

\vspace{2em}
\noindent
\textbf{Contact Information:}\\[0.5em]
LinkedIn: \href{https://www.linkedin.com/in/nguyenvuhung/}{https://www.linkedin.com/in/nguyenvuhung/}\\[0.3em]
GitHub: \href{https://github.com/vuhung16au/}{https://github.com/vuhung16au/}\\[0.3em]
Repository: \href{https://github.com/vuhung16au/math-olympiad-ml/tree/main/HSC-Collections}{https://github.com/vuhung16au/math-olympiad-ml/tree/main/HSC-Collections}

\end{document}

