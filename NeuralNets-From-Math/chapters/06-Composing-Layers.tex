\chapter{Composing Layers}
Stack linear steps with activations to form complex functions.

\begin{learningobjectives}
  \objective Define a layer as \( \vect h = g(W\vect x + \vect b) \).
  \objective Explain how two layers can form piecewise linear curves.
  \objective Understand hidden units as learned features.
\end{learningobjectives}

\section{Two-Layer Example}
Let \( \vect h = \mathrm{ReLU}(W_1\vect x + \vect b_1) \) and \( \hat y = W_2\vect h + b_2 \). Even with ReLU, the output is a flexible piecewise linear function of \( \vect x \).

\begin{example}
In one dimension, two ReLUs can create a ``bump'' by combining kinks at different locations. Consider a tiny network with one input \(x\), two hidden units, and one output:
\[
\vect h = \mathrm{ReLU}(W_1 x + \vect b_1), \qquad \hat y = W_2\,\vect h + b_2,
\]
where
\[
W_1 = \begin{bmatrix}1\\-1\end{bmatrix},\; \vect b_1 = \begin{bmatrix}-1\\-3\end{bmatrix},\; W_2=\begin{bmatrix}1 & -1\end{bmatrix},\; b_2=0.
\]
Thus the hidden units are
\[
h_1=\mathrm{ReLU}(x-1),\qquad h_2=\mathrm{ReLU}(-x-3)=\mathrm{ReLU}(-(x+3)).
\]
The output becomes \(\hat y = h_1 - h_2\). Evaluate at a few inputs:
\[
\begin{array}{c|c|c|c}
 x & h_1=\max(0,x-1) & h_2=\max(0,-x-3) & \hat y=h_1-h_2 \\
 \hline
 -5 & 0 & 2 & -2 \\
 -3 & 0 & 0 & 0 \\
 0 & 0 & 0 & 0 \\
 2 & 1 & 0 & 1 \\
 4 & 3 & 0 & 3 \\
\end{array}
\]
Interpretation:
\begin{itemize}
  \item For \(x\le -3\), only the second unit is active, so \(\hat y=-h_2\) decreases linearly.
  \item For \(-3<x<1\), both units are off, so \(\hat y=0\).
  \item For \(x\ge 1\), only the first unit is active, so \(\hat y=h_1\) increases linearly.
\end{itemize}
This piecewise linear shape forms a flat region with rising and falling flanks—one way to build a ``bump'' by composing ReLUs.
\end{example}

\section{Exercises}
\begin{exercisebox}[medium]
How many linear regions can a sum of two shifted ReLUs create on the line?
\end{exercisebox}

\begin{hint}
Up to three regions separated by the two kink locations.
\end{hint}
