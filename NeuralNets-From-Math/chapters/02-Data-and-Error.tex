\chapter{Data and Error}
Predictions meet reality through data. Error tells us how far off we are.

\begin{learningobjectives}
  \objective Define prediction, target, and error for a dataset.
  \objective Compute simple average absolute and squared error.
  \objective Explain why averaging error stabilises noisy measurements.
\end{learningobjectives}

\section{Measuring Error}
Given inputs \( x_i \) with observed outputs \( y_i \) and predictions \( \hat y_i \), an error for item \( i \) is \( e_i = \hat y_i - y_i \). Two popular summaries are the mean absolute error (MAE) and mean squared error (MSE):
\[ \text{MAE} = \frac{1}{n}\sum_i |e_i|, \quad \text{MSE} = \frac{1}{n}\sum_i e_i^2. \]

\begin{example}
Using the study–score data from the previous chapter
\[
\begin{array}{c|cccc}
 x & 1 & 2 & 3 & 4 \\
 \hline
 y & 2.1 & 4.0 & 6.2 & 7.9 \\
\end{array}
\]
consider the linear rule \( \hat y=2x \). The predictions and errors are:
\[
\begin{array}{c|cccc}
 x & 1 & 2 & 3 & 4 \\
 \hline
 \hat y=2x & 2.0 & 4.0 & 6.0 & 8.0 \\
 e=\hat y-y & 0.1 & 0.0 & -0.2 & 0.1 \\
 |e| & 0.1 & 0.0 & 0.2 & 0.1 \\
 e^2 & 0.01 & 0.00 & 0.04 & 0.01 \\
\end{array}
\]
Thus
\[
\text{MAE}=\tfrac{0.1+0.0+0.2+0.1}{4}=0.1,\qquad
\text{MSE}=\tfrac{0.01+0.00+0.04+0.01}{4}=0.015.
\]
The small MAE/MSE values indicate the line \( \hat y=2x \) matches the trend closely for these points.
\end{example}

\begin{remark}
Squaring emphasises large mistakes; absolute value treats all deviations proportionally. Choice depends on the application.
\end{remark}

\section{Exercises}
\begin{exercisebox}[medium]
For true values \( y=(1,2,3) \) and predictions \( \hat y=(1.2,1.9,3.4) \), compute MAE and MSE.
\end{exercisebox}

\begin{hint}
Errors: \( (0.2,-0.1,0.4) \). MAE = \( (0.2+0.1+0.4)/3=0.233\ldots \); MSE = \( (0.04+0.01+0.16)/3=0.07\ldots \).
\end{hint}
