\chapter{Vectors and Matrices}
Vectors collect features; matrices collect many vectors and define linear transformations.

\begin{learningobjectives}
  \objective Represent data points as vectors of features.
  \objective Use matrix–vector multiplication to combine features linearly.
  \objective Interpret a matrix as a transformation of space.
\end{learningobjectives}

\section{Notation}
Write a feature vector as \( \vect x = (x_1,\dots,x_d)^\top \). A weight vector \( \vect w \) and bias \( b \) define a linear score \( s = \vect w^\top \vect x + b \).

\begin{definition}
A \emph{matrix} \( W \in \mathbb{R}^{m\times d} \) applied to a vector \( \vect x\in\mathbb{R}^d \) yields \( W\vect x\in\mathbb{R}^m \), combining inputs into \( m \) outputs.
\end{definition}

\begin{example}
With two features (height, width) and two outputs (sum, difference),
\( W=\begin{bmatrix}1&1\\1&-1\end{bmatrix} \) maps a vector \(\vect x=(x_1,x_2)^\top\) to \( W\vect x=(x_1+x_2,\; x_1-x_2)^\top \).

For a concrete calculation, let \(\vect x=(3,2)^\top\). Then
\[
W\vect x=
\begin{bmatrix}1&1\\[2pt]1&-1\end{bmatrix}
\begin{bmatrix}3\\[2pt]2\end{bmatrix}
=
\begin{bmatrix}1\cdot3+1\cdot2\\[2pt]1\cdot3+(-1)\cdot2\end{bmatrix}
=
\begin{bmatrix}5\\[2pt]1\end{bmatrix}.
\]
So the first output (sum) is 5 and the second output (difference) is 1.
\end{example}

\section{Exercises}
\begin{exercisebox}[easy]
If \( W=\begin{bmatrix}2&0\\0&3\end{bmatrix} \) and \( \vect x=(1,2)^\top \), compute \( W\vect x \).
\end{exercisebox}

\begin{hint}
\( (2,6)^\top \).
\end{hint}
