\chapter{Overview}
This book takes you from familiar mathematics to a working understanding of simple neural networks. It is written in clear British English, and designed for undergraduates and motivated high‑school students.

\begin{learningobjectives}
  \objective Understand the book's roadmap and how chapters connect.
  \objective Recognise the key ingredients of a neural network.
  \objective Learn how examples, metaphors, and visuals support intuition.
\end{learningobjectives}

\section*{Roadmap}
We move from functions and graphs to data and error, then to vectors and matrices that compactly represent many inputs. Linear models give us a first predictive rule; nonlinear activations add the ``kinks'' that let us model more interesting patterns. Stacking layers composes these ideas. Finally, we discuss loss and optimisation, the training loop, a simple network, and practical limits and ethics.

\begin{example}
Consider predicting house price from size. A function maps size (input) to price (output). A linear model draws a straight line; a neural network allows bends via activations and layers, improving fit when reality is not a straight line.
\end{example}

\begin{remark}
Throughout, we prioritise intuition first, then formalism. Visuals accompany core definitions where helpful.
\end{remark}

\section*{How to Use This Book}
Skim the learning objectives at the start of each chapter. Work through examples; attempt exercises before revealing the upside‑down hints. Use the glossary to refresh key terms.
