\chapter{Linear Models}
A linear model scores inputs by a weighted sum plus a bias; decisions follow from thresholds or regression.

\begin{learningobjectives}
  \objective Write a linear model as \( \hat y = \vect w^\top \vect x + b \).
  \objective Explain the geometric view: lines and hyperplanes.
  \objective Recognise when linear models are insufficient.
\end{learningobjectives}

\section{Form and Geometry}
The set of points with constant score forms a hyperplane. When classes are linearly separable, a single hyperplane can divide them.

\begin{example}
Classify points above a line: take \( \hat y = 2x_1 - x_2 + 0.5 \) and predict positive when \( \hat y>0 \).

For \( (x_1,x_2)=(2,1) \),
\[
\hat y = 2\cdot 2 - 1 + 0.5 = 3.5 > 0 \;\Rightarrow\; \text{positive}.
\]
For \( (x_1,x_2)=(0,1) \),
\[
\hat y = 2\cdot 0 - 1 + 0.5 = -0.5 < 0 \;\Rightarrow\; \text{negative}.
\]
The decision boundary \( \hat y=0 \) is the straight line \( x_2 = 2x_1 + 0.5 \).
\end{example}

\section{Exercises}
\begin{exercisebox}[medium]
Find a line that separates points A: \( (0,2),(1,3) \) from B: \( (2,0),(3,1) \) if possible.
\end{exercisebox}

\begin{hint}
Try \( x_2 = x_1 + 1.5 \). Points A lie above, B below.
\end{hint}
