\chapter{Limits and Ethics}
Models have limits. Responsible practice matters.

\begin{learningobjectives}
  \objective Identify common failure modes: overfitting, data shift, bias.
  \objective Describe simple mitigations students can apply.
  \objective Reflect on responsible communication of model results.
\end{learningobjectives}

\section{Limits}
Overfitting memorises training noise; distribution shift breaks assumptions; biased data leads to unfair outcomes.

\begin{example}[Overfitting]
Fit a 9th‑degree polynomial to 10 noisy points sampled from the straight line \(y=2x\). On the training points, the curve can wiggle to achieve near‑zero error; on fresh test points from the same line, error is high because the wiggles chase noise rather than signal.
\end{example}

\begin{example}[Underfitting]
Fit a straight line to data generated by a clear U‑shaped quadratic like \(y=x^2\) with small noise. Both training and test errors remain large because a line cannot capture the curved relationship: the model is too simple for the pattern present.
\end{example}

\section{Ethics}
Use diverse, representative data when possible; report uncertainty; avoid over‑claiming; consider impacts on people.

\begin{exercisebox}[easy]
Give one reason a model trained on last year's data may underperform this year.
\end{exercisebox}

\begin{hint}
Data distribution can shift: the relationship between inputs and outputs changes.
\end{hint}

\section{Further Directions}
This book covers feed‑forward neural networks—the foundation for more advanced architectures. Transformers, for instance, extend these ideas to handle sequences (text, time series) using attention mechanisms that allow the model to focus on relevant parts of the input when making predictions. Other directions include convolutional networks for images, recurrent networks for sequences, and generative models that learn to create new data.

\begin{remark}
The core concepts you've learned—layers, activations, loss functions, optimisation—apply across these advanced architectures. Understanding the fundamentals prepares you to explore these exciting extensions.
\end{remark}
