\documentclass[12pt]{article}

% --- PACKAGES ---
\usepackage[utf8]{inputenc} % Handle input encoding
\usepackage[T1]{fontenc}    % Font encoding
\usepackage{amsmath}        % For advanced math environments
\usepackage{amssymb}        % For \mathbb
\usepackage{amsthm}         % For theorem environments
\usepackage[margin=1in]{geometry} % Set standard 1-inch margins
\usepackage{hyperref}       % For hyperlinks

% --- THEOREM STYLES ---
\theoremstyle{definition}
\newtheorem{problem}{Problem}
\newtheorem{example}{Example}
\theoremstyle{plain}
\newtheorem{theorem}{Theorem}

% --- TITLE ---
\title{Functional Equation: $f(f(x)) = x^2 - x + 1$}
\author{Vu Hung Nguyen}
\date{November 2025}

% --- DOCUMENT START ---
\begin{document}

\maketitle

\begin{abstract}
This document explores a functional equation of the form $f(f(x)) = x^2 - x + 1$ where $f: \mathbb{R} \to \mathbb{R}$. We analyze the properties of such functions, derive specific values, and explore generalizations and extensions of the problem. The polynomial $x^2 - x + 1$ has special properties related to the sixth roots of unity and shares the same discriminant as $x^2 + x + 1$, making this functional equation similarly challenging.
\end{abstract}

% --- OVERVIEW ---
\section{Overview}

Functional equations are equations in which the unknowns are functions rather than simple quantities. The functional equation $f(f(x)) = x^2 - x + 1$ is particularly interesting because:

\begin{itemize}
    \item The polynomial $x^2 - x + 1$ has no real roots (discriminant $\Delta = 1 - 4 = -3 < 0$)
    \item It has the same discriminant as $x^2 + x + 1$, showing a symmetric relationship
    \item The polynomial can be factored over the complex numbers as $(x - \zeta)(x - \overline{\zeta})$ where $\zeta = \frac{1 + i\sqrt{3}}{2}$
    \item Finding $f$ requires careful analysis of the iteration structure and domain-range relationships
\end{itemize}

The polynomial $x^2 - x + 1$ is related to sixth roots of unity, since if $\omega = e^{2\pi i/3}$, then $\omega^2 - \omega + 1 = 0$.

% --- PROBLEM STATEMENT ---
\section{Problem Statement}

\begin{problem}
Given a function $f: \mathbb{R} \to \mathbb{R}$ that satisfy the following functional equation for all $x \in \mathbb{R}$:
\[
f(f(x)) = x^2 - x + 1
\]
In particular, determine the value of $f(0)$.
\end{problem}

\subsection{Initial Observations}

Before diving into the solution, we make several key observations:

\begin{enumerate}
    \item The right-hand side $x^2 - x + 1$ is always positive for all real $x$, since its minimum value occurs at $x = \frac{1}{2}$ where it equals $\frac{3}{4}$
    \item This means $f(f(x)) \geq \frac{3}{4}$ for all $x$, implying that the range of $f$ must be such that composing $f$ with itself produces only values $\geq \frac{3}{4}$
    \item The polynomial $x^2 - x + 1$ has complex roots at $\frac{1 \pm i\sqrt{3}}{2}$, related to sixth roots of unity
\end{enumerate}

% --- SOLUTION ---
\section{Derivation Steps}

We will solve this problem systematically by substituting strategic values for $x$ and analyzing the resulting properties of the function.

\subsection{Step 1: Substitute $x=0$ and $x=1$}

We begin by plugging in the two simplest integer values for $x$.

\textbf{For $x = 0$:}
\[
f(f(0)) = (0)^2 + (0) + 1 \implies \mathbf{f(f(0)) = 1}
\]

\textbf{For $x = 1$:}
\[
f(f(1)) = (1)^2 - (1) + 1 \implies \mathbf{f(f(1)) = 1}
\]

\subsection{Step 2: Analyze $f(0)$ and $f(1)$}

Let $a = f(0)$ and $b = f(1)$. From Step 1, we have:
\begin{align}
f(a) &= 1 \label{eq:fa} \\
f(b) &= 1 \label{eq:fb}
\end{align}

\textbf{Key observation:} Both $f(a) = 1$ and $f(b) = 1$, meaning both $f(0)$ and $f(1)$ map to values that $f$ sends to 1.

Now, substitute $x = a$ into the original equation:
\[
f(f(a)) = a^2 - a + 1
\]
Since $f(a) = 1$ from equation~\eqref{eq:fa}, we get:
\[
f(1) = a^2 - a + 1
\]
But $f(1) = b$, so:
\begin{equation}
b = a^2 - a + 1 \label{eq:bina}
\end{equation}

Similarly, substitute $x = b$ into the original equation:
\[
f(f(b)) = b^2 - b + 1
\]
Since $f(b) = 1$ from equation~\eqref{eq:fb}, we get:
\[
f(1) = b^2 - b + 1 \label{eq:f3}
\]

\subsection{Step 3: Finding the Value of $a = f(0)$}

From equation~\eqref{eq:bina}, we have $b = a^2 - a + 1$. We also know from equation~\eqref{eq:f3} that:
\[
f(1) = b^2 - b + 1
\]

But we also know $f(1) = b$, so:
\[
b = b^2 - b + 1
\]
\[
b^2 - 2b + 1 = 0
\]
\[
(b - 1)^2 = 0
\]
\[
\boxed{b = 1}
\]

Now, using $b = a^2 - a + 1$ and $b = 1$:
\[
1 = a^2 - a + 1
\]
\[
a^2 - a = 0
\]
\[
a(a - 1) = 0
\]

So either $a = 0$ or $a = 1$.

\subsection{Step 4: Testing the Two Cases}

\textbf{Case 1: $a = 0$ (i.e., $f(0) = 0$)}

If $f(0) = 0$ and $f(1) = 1$, let's check consistency:
\begin{itemize}
    \item We need $f(f(0)) = f(0) = 0$
    \item But the equation requires $f(f(0)) = 0^2 - 0 + 1 = 1$
    \item Contradiction! \texttimes
\end{itemize}

\textbf{Case 2: $a = 1$ (i.e., $f(0) = 1$)}

If $f(0) = 1$ and $f(1) = 1$, let's check consistency:
\begin{itemize}
    \item We need $f(f(0)) = f(1) = 1$
    \item The equation requires $f(f(0)) = 0^2 - 0 + 1 = 1$ \checkmark
    \item We need $f(f(1)) = f(1) = 1$
    \item The equation requires $f(f(1)) = 1^2 - 1 + 1 = 1$ \checkmark
\end{itemize}

Both conditions are satisfied!

\subsection{Step 5: Verification and Fixed Point Analysis}

The value $f(0) = 1$ is consistent with the functional equation. Moreover, we observe that $x = 1$ is a \textbf{fixed point} of $f$:
\[
f(1) = 1
\]

This is consistent because if $f(1) = 1$, then:
\[
f(f(1)) = f(1) = 1
\]
and the functional equation requires:
\[
f(f(1)) = 1^2 - 1 + 1 = 1 \quad \checkmark
\]

The structure of this equation is fundamentally different from $f(f(x)) = x^2 + x + 1$ because:
\begin{itemize}
    \item Both $f(0)$ and $f(1)$ map to the same value under $f$ (namely 1)
    \item The value $x = 1$ is a fixed point: $f(1) = 1$
    \item This creates a consistent constraint system with a unique solution for $f(0)$
\end{itemize}

\subsection{Conclusion of Derivation}

Through systematic substitution and algebraic manipulation, we have definitively determined:
\[
\boxed{f(0) = 1}
\]

The key steps in our derivation were:
\begin{enumerate}
    \item From $f(f(0)) = 1$ and $f(f(1)) = 1$, we found that both $f(0)$ and $f(1)$ must map to values that $f$ sends to 1
    \item Setting $a = f(0)$ and $b = f(1)$, we derived $b = a^2 - a + 1$ and $b = b^2 - b + 1$
    \item Solving $b = b^2 - b + 1$ gave us $b = 1$, meaning $f(1) = 1$ (a fixed point)
    \item Substituting $b = 1$ into $b = a^2 - a + 1$ yielded $a = 0$ or $a = 1$
    \item Testing showed that only $a = 1$ is consistent with the functional equation
\end{enumerate}

This result is remarkably cleaner than the analogous problem with $f(f(x)) = x^2 + x + 1$, where the constraint system leads to contradictions.

% --- FURTHER WORK ---
\section{Further Work}

The functional equation $f(f(x)) = x^2 - x + 1$ opens several avenues for further investigation:

\subsection{Existence and Uniqueness}

\begin{itemize}
    \item \textbf{Does a solution exist?} Proving the existence of a function $f: \mathbb{R} \to \mathbb{R}$ satisfying this equation requires careful construction, potentially using techniques from dynamical systems or invoking the Axiom of Choice for pathological constructions.
    
    \item \textbf{Is the solution unique?} Given the constraints, is there a unique function satisfying the equation, or are there multiple solutions? Uniqueness typically requires additional constraints such as continuity, monotonicity, or analyticity.
    
    \item \textbf{Continuous solutions:} If we restrict to continuous functions, does a continuous solution exist? The composition of continuous functions is continuous, so $f \circ f$ would be continuous, matching the continuity of $x^2 + x + 1$. Likely impossible to find a continuous solution.
\end{itemize}

\subsection{Connection to Dynamical Systems}

The functional equation can be viewed through the lens of discrete dynamical systems:
\begin{itemize}
    \item The iteration $f^{(n)}(x) = \underbrace{f(f(\cdots f}_{n \text{ times}}(x)\cdots))$ generates an orbit
    \item Fixed points satisfy $f(x) = x$, which combined with $f(f(x)) = x^2 - x + 1$ would require $x = x^2 - x + 1$ or $x^2 - 2x + 1 = 0$, giving $x = 1$ as the unique fixed point
    \item The existence of the fixed point $x = 1$ is a key feature that makes this equation more tractable than $f(f(x)) = x^2 + x + 1$
\end{itemize}

\subsection{Complex Extension}

Extending the domain to $f: \mathbb{C} \to \mathbb{C}$ might provide more structure:
\begin{itemize}
    \item The polynomial $x^2 - x + 1$ factors as $(x - \zeta)(x - \overline{\zeta})$ where $\zeta = \frac{1 + i\sqrt{3}}{2}$
    \item Complex dynamics and Julia sets may provide insights
    \item The functional equation might have natural solutions in terms of complex analytic functions
\end{itemize}

% --- EXPAND THE PROBLEM ---
\section{Generalizations and Extensions}

\subsection{General Polynomial Case}

Consider the more general functional equation:
\[
f(f(x)) = P(x)
\]
where $P(x)$ is a polynomial of degree $n$. 

\begin{itemize}
    \item For $P(x) = x$: Solutions include $f(x) = x$ and any involution (functions satisfying $f(f(x)) = x$)
    \item For $P(x) = x + c$: No real solutions exist if $c \neq 0$ (can be shown by iteration)
    \item For $P(x) = -x$: Solutions are involutions
    \item For $P(x) = x^2$: Related to iterative square root functions
\end{itemize}

\subsection{Polynomial of Degree 2}

The general form $f(f(x)) = ax^2 + bx + c$ includes our equation as a special case with $a = 1, b = 1, c = 1$. Different values lead to different behaviors:

\begin{example}
For $f(f(x)) = x^2 - x + 1$ (the original problem had this, with discriminant $\Delta = 1 - 4 = -3 < 0$), the approach would be similar but yield different specific values.
\end{example}

\subsection{Higher Iterations}

Instead of $f \circ f$, consider:
\[
\underbrace{f \circ f \circ \cdots \circ f}_{n \text{ times}}(x) = x^2 - x + 1
\]

For $n = 3$: $f(f(f(x))) = x^2 - x + 1$

This creates a hierarchy of increasingly complex functional equations.

\subsection{System of Functional Equations}

Consider simultaneous equations:
\begin{align}
f(g(x)) &= x^2 + 1 \\
g(f(x)) &= x^2 - 1
\end{align}

This extends the problem to finding pairs of functions rather than a single function. Unlike the system with $x^2 - x + 1$ and $x^2 + x + 1$ (which has no real solutions due to the composition constraint $g(1)^2 = -1$), this modified system avoids such contradictions.

\subsubsection{Analysis of the System}

Applying $g$ to the first equation:
\[
g(f(g(x))) = g(x^2 + 1)
\]

Substituting $y = g(x)$ in the second equation:
\[
g(f(g(x))) = g(x)^2 - 1
\]

Therefore:
\begin{equation}
g(x^2 + 1) = g(x)^2 - 1 \label{eq:gsystem}
\end{equation}

Similarly, applying $f$ to the second equation and substituting $z = f(x)$ in the first:
\begin{equation}
f(x^2 - 1) = f(x)^2 + 1 \label{eq:fsystem}
\end{equation}

\textbf{Testing specific values:}

For $x = 0$ in equation~\eqref{eq:gsystem}:
\[
g(1) = g(0)^2 - 1
\]

For $x = 1$ in equation~\eqref{eq:gsystem}:
\[
g(2) = g(1)^2 - 1
\]

Unlike the impossible constraint $g(1)^2 = -1$ from the original system, these equations are solvable. For example, if $g(0) = 1$, then $g(1) = 0$ and $g(2) = -1$, which are all real values.

The existence of solutions to this system remains an open problem requiring deeper investigation into the consistency of these functional constraints.

% --- CONCLUSION ---
\section{Conclusion}

The functional equation $f(f(x)) = x^2 - x + 1$ represents a rich mathematical problem connecting several areas:

\begin{itemize}
    \item \textbf{Algebraic manipulation:} Systematic substitution and solving equations reveals constraints on $f$
    \item \textbf{Number theory:} The polynomial $x^2 - x + 1$ is related to sixth roots of unity
    \item \textbf{Dynamical systems:} Iteration and composition of functions create complex behavior, with $x = 1$ as a fixed point
    \item \textbf{Complex analysis:} Extension to complex domains may provide additional structure
\end{itemize}

While finding an explicit closed-form solution for $f$ over all real numbers remains challenging, the analysis demonstrates that such functions, if they exist, must satisfy specific constraints. The value $f(0) = 1$ emerged uniquely from our derivation, with the key insight being that $f(1) = 1$ forms a fixed point.

This problem illustrates that functional equations can be significantly more challenging than ordinary algebraic equations, often requiring creative approaches and deep insights into the structure of functions and their compositions.

\subsection{Open Questions}

Several questions remain open for further research:
\begin{enumerate}
    \item Can we construct an explicit formula for $f(x)$ for all real $x$?
    \item Does a continuous (or smooth) solution exist?
    \item What is the minimal regularity required for a solution to exist?
    \item How many distinct solutions exist, and what characterizes them?
    \item What happens when we extend the domain to complex numbers?
\end{enumerate}

\vspace{1em}
\noindent\textbf{Acknowledgments:} This exploration was motivated by the beautiful interplay between polynomial equations and function composition.

\end{document}
% --- DOCUMENT END ---