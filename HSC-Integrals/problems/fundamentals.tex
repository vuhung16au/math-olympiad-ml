\section{Standard Integrals \& The ``Reverse Chain Rule''}
Before applying complex techniques, always check if the integral fits a standard form or the reverse chain rule.

\begin{itemize}
    \item \textbf{Logarithmic Form:}
    \[
    \int \frac{f'(x)}{f(x)} \, dx = \ln|f(x)| + C
    \]
    \item \textbf{Power Rule (General):}
    \[
    \int [f(x)]^n f'(x) \, dx = \frac{[f(x)]^{n+1}}{n+1} + C \quad (\text{where } n \neq -1)
    \]
    \item \textbf{Inverse Trigonometric Forms:}
    \begin{align*}
        \int \frac{1}{\sqrt{a^2 - x^2}} \, dx &= \sin^{-1}\left(\frac{x}{a}\right) + C \\
        \int \frac{1}{a^2 + x^2} \, dx &= \frac{1}{a}\tan^{-1}\left(\frac{x}{a}\right) + C
    \end{align*}
\end{itemize}

\section{Integration by Parts}
Used for integrating products of functions (e.g., $x e^x$, $x \ln x$, $e^x \cos x$).

\textbf{The Formula:}
\[
\int u \, dv = uv - \int v \, du
\]

\textbf{Strategy (LIATE):}
Choose $u$ based on this priority list (top to bottom):
\begin{enumerate}
    \item \textbf{L} -- Logarithmic ($\ln x$)
    \item \textbf{I} -- Inverse Trigonometric ($\tan^{-1}x$)
    \item \textbf{A} -- Algebraic ($x^2, 3x$)
    \item \textbf{T} -- Trigonometric ($\sin x, \cos x$)
    \item \textbf{E} -- Exponential ($e^x$)
\end{enumerate}

\section{Integration by Substitution}

\subsection{General Substitution ($u$-sub)}
Used to simplify composite functions. Let $u = g(x)$, then find $du = g'(x)dx$.

\subsection{Trigonometric Substitution}
Used when the integrand contains quadratic roots:
\begin{itemize}
    \item $\sqrt{a^2 - x^2}$: Let $x = a \sin \theta$ \quad (uses $1 - \sin^2\theta = \cos^2\theta$)
    \item $\sqrt{a^2 + x^2}$: Let $x = a \tan \theta$ \quad (uses $1 + \tan^2\theta = \sec^2\theta$)
    \item $\sqrt{x^2 - a^2}$: Let $x = a \sec \theta$ \quad (uses $\sec^2\theta - 1 = \tan^2\theta$)
\end{itemize}

\subsection{The $t$-Formula Substitution}
Used for rational functions involving $\sin x$ and $\cos x$. Let $t = \tan\left(\frac{x}{2}\right)$.
\begin{align*}
    dx &= \frac{2}{1+t^2} \, dt \\
    \sin x &= \frac{2t}{1+t^2} \\
    \cos x &= \frac{1-t^2}{1+t^2}
\end{align*}

\section{Partial Fractions}
Used to integrate rational functions $\frac{P(x)}{Q(x)}$ where $\text{deg}(P) < \text{deg}(Q)$. If $\text{deg}(P) \geq \text{deg}(Q)$, perform \textbf{polynomial long division} first.

\begin{itemize}
    \item \textbf{Distinct Linear Factors:}
    \[
    \frac{1}{(x-a)(x-b)} = \frac{A}{x-a} + \frac{B}{x-b}
    \]
    \item \textbf{Repeated Linear Factors:}
    \[
    \frac{1}{(x-a)^2} = \frac{A}{x-a} + \frac{B}{(x-a)^2}
    \]
    \item \textbf{Irreducible Quadratic Factors:}
    \[
    \frac{1}{(x-a)(x^2+b)} = \frac{A}{x-a} + \frac{Bx+C}{x^2+b}
    \]
\end{itemize}

\section{Trigonometric Integrals}

\subsection{ Integrals of $\sin^m x \cos^n x$}
\begin{itemize}
    \item \textbf{One power is odd:} Save one factor of the odd power for $du$. Convert the rest using $\sin^2x + \cos^2x = 1$.
    \item \textbf{Both powers even:} Use double angle formulae:
    \[
    \sin^2x = \frac{1}{2}(1-\cos 2x) \quad \text{and} \quad \cos^2x = \frac{1}{2}(1+\cos 2x)
    \]
\end{itemize}

\subsection{ Integrals of $\tan^m x \sec^n x$}
\begin{itemize}
    \item \textbf{If $\sec$ power is even:} Save $\sec^2 x$ for $du$. Convert remaining $\sec$ to $\tan$.
    \item \textbf{If $\tan$ power is odd:} Save $\sec x \tan x$ for $du$. Convert remaining $\tan$ to $\sec$.
\end{itemize}

\section{Reduction Formulas ($I_n$)}
Involves finding a recurrence relation using \textbf{Integration by Parts}.

\textbf{Typical form:}
\[
I_n = \int x^n e^x \, dx \quad \text{or} \quad I_n = \int_0^{\frac{\pi}{2}} \sin^n x \, dx
\]
\textbf{Steps:} Apply parts, manipulate the integral to find $I_{n-1}$ or $I_{n-2}$, then rearrange for $I_n$.

\section{Definite Integral Properties}
\begin{itemize}
    \item \textbf{Odd Function:} If $f(-x) = -f(x)$, then $\int_{-a}^a f(x) \, dx = 0$.
    \item \textbf{Even Function:} If $f(-x) = f(x)$, then $\int_{-a}^a f(x) \, dx = 2\int_0^a f(x) \, dx$.
    \item \textbf{Reflection (King's) Property:}
    \[
    \int_a^b f(x) \, dx = \int_a^b f(a+b-x) \, dx
    \]
\end{itemize}
