% Part 2: Medium Problems (15 problems with hints only - upside-down format)
% Students attempt first, then rotate page 180° to read hint

% =============================================================================
% PROBLEM 1: Integration by parts twice
% =============================================================================
\begin{problem}
Evaluate $\displaystyle \int x^2 e^x \, dx$
\end{problem}

\vspace{0.5cm}
\rotatebox{180}{\parbox{\textwidth}{\small \textbf{Hint:} Apply by parts twice. First: $u_1 = x^2$, $dv_1 = e^x dx$. Second: $u_2 = 2x$, $dv_2 = e^x dx$. Answer: $e^x(x^2 - 2x + 2) + C$.}}

\vspace{1.5cm}

% =============================================================================
% PROBLEM 2: Partial fractions with repeated factor
% =============================================================================
\begin{problem}
Find $\displaystyle \int \frac{2x + 3}{(x-1)^2} \, dx$
\end{problem}

\vspace{0.5cm}
\rotatebox{180}{\parbox{\textwidth}{\small \textbf{Hint:} $\frac{2x+3}{(x-1)^2} = \frac{A}{x-1} + \frac{B}{(x-1)^2}$. Solving: $A=2$, $B=5$. Answer: $2\ln|x-1| - \frac{5}{x-1} + C$.}}

\vspace{1.5cm}

% =============================================================================
% PROBLEM 3: Substitution + completing square
% =============================================================================
\begin{problem}
Evaluate $\displaystyle \int \frac{1}{\sqrt{6x - x^2}} \, dx$
\end{problem}

\vspace{0.5cm}
\rotatebox{180}{\parbox{\textwidth}{\small \textbf{Hint:} Complete square: $6x - x^2 = 9 - (x-3)^2$. Let $u = x-3$. Answer: $\arcsin\left(\frac{x-3}{3}\right) + C$.}}

\vspace{1.5cm}

% =============================================================================
% PROBLEM 4: King's property with trig
% =============================================================================
\begin{problem}
Evaluate $\displaystyle \int_0^{\pi/2} \frac{x}{\sin x + \cos x} \, dx$
\end{problem}

\vspace{0.5cm}
\rotatebox{180}{\parbox{\textwidth}{\small \textbf{Hint:} Let $I = \int_0^{\pi/2} \frac{x}{\sin x + \cos x}dx$. King's: $I = \int_0^{\pi/2} \frac{\pi/2 - x}{\cos x + \sin x}dx$. Add: $2I = \frac{\pi}{2}\int_0^{\pi/2}\frac{1}{\sin x + \cos x}dx$. Use $t = \tan(x/2)$.}}

\vspace{1.5cm}

% =============================================================================
% PROBLEM 5: Trig substitution for √(a² - x²)
% =============================================================================
\begin{problem}
Find $\displaystyle \int \sqrt{16 - x^2} \, dx$
\end{problem}

\vspace{0.5cm}
\rotatebox{180}{\parbox{\textwidth}{\small \textbf{Hint:} Let $x = 4\sin\theta$, $dx = 4\cos\theta \, d\theta$. Then $\sqrt{16-x^2} = 4\cos\theta$. Answer: $8\arcsin(x/4) + \frac{x\sqrt{16-x^2}}{2} + C$.}}

\vspace{1.5cm}

% =============================================================================
% PROBLEM 6: Reduction formula - powers of sin
% =============================================================================
\begin{problem}
Derive and use: If $I_n = \displaystyle \int_0^{\pi/2} \sin^n x \, dx$, find $I_4$
\end{problem}

\vspace{0.5cm}
\rotatebox{180}{\parbox{\textwidth}{\small \textbf{Hint:} By parts: $I_n = \frac{n-1}{n}I_{n-2}$. Calculate: $I_0 = \frac{\pi}{2}$, $I_2 = \frac{\pi}{4}$, $I_4 = \frac{3}{4} \cdot \frac{\pi}{4} = \frac{3\pi}{16}$.}}

\vspace{1.5cm}

% =============================================================================
% PROBLEM 7: Partial fractions with quadratic
% =============================================================================
\begin{problem}
Evaluate $\displaystyle \int \frac{x^2 + 1}{(x-1)(x^2 + 1)} \, dx$
\end{problem}

\vspace{0.5cm}
\rotatebox{180}{\parbox{\textwidth}{\small \textbf{Hint:} Simplify first: $\frac{x^2+1}{(x-1)(x^2+1)} = \frac{1}{x-1}$. Answer: $\ln|x-1| + C$.}}

\vspace{1.5cm}

% =============================================================================
% PROBLEM 8: Integration by parts - ln
% =============================================================================
\begin{problem}
Find $\displaystyle \int x^2 \ln x \, dx$
\end{problem}

\vspace{0.5cm}
\rotatebox{180}{\parbox{\textwidth}{\small \textbf{Hint:} LIATE: $u = \ln x$, $dv = x^2 dx$. Then $v = \frac{x^3}{3}$. Answer: $\frac{x^3 \ln x}{3} - \frac{x^3}{9} + C = \frac{x^3}{9}(3\ln x - 1) + C$.}}

\vspace{1.5cm}

% =============================================================================
% PROBLEM 9: Definite integral with symmetry
% =============================================================================
\begin{problem}
Evaluate $\displaystyle \int_{-1}^{1} \frac{x^2}{1 + e^x} \, dx$
\end{problem}

\vspace{0.5cm}
\rotatebox{180}{\parbox{\textwidth}{\small \textbf{Hint:} Let $I = \int_{-1}^{1} \frac{x^2}{1+e^x}dx$. King's property: $I = \int_{-1}^{1} \frac{x^2 e^x}{1+e^x}dx$. Add: $2I = \int_{-1}^{1} x^2 dx = \frac{2}{3}$, so $I = \frac{1}{3}$.}}

\vspace{1.5cm}

% =============================================================================
% PROBLEM 10: Trig identity + substitution
% =============================================================================
\begin{problem}
Evaluate $\displaystyle \int \sin^4 x \, dx$
\end{problem}

\vspace{0.5cm}
\rotatebox{180}{\parbox{\textwidth}{\small \textbf{Hint:} Use $\sin^2 x = \frac{1-\cos 2x}{2}$. Then $\sin^4 x = \left(\frac{1-\cos 2x}{2}\right)^2 = \frac{1 - 2\cos 2x + \cos^2 2x}{4}$. Use $\cos^2 2x = \frac{1+\cos 4x}{2}$.}}

\vspace{1.5cm}

% =============================================================================
% PROBLEM 11: Volume of revolution
% =============================================================================
\begin{problem}
Find the volume when the region bounded by $y = \sqrt{x}$, $y = 0$, $x = 4$ is rotated about the $x$-axis.
\end{problem}

\vspace{0.5cm}
\rotatebox{180}{\parbox{\textwidth}{\small \textbf{Hint:} $V = \pi\int_0^4 (\sqrt{x})^2 dx = \pi\int_0^4 x \, dx = \pi \cdot \frac{16}{2} = 8\pi$ cubic units.}}

\vspace{1.5cm}

% =============================================================================
% PROBLEM 12: t-formula application
% =============================================================================
\begin{problem}
Evaluate $\displaystyle \int_0^{\pi/2} \frac{1}{3 + 5\cos x} \, dx$
\end{problem}

\vspace{0.5cm}
\rotatebox{180}{\parbox{\textwidth}{\small \textbf{Hint:} Let $t = \tan(x/2)$. Then $\cos x = \frac{1-t^2}{1+t^2}$, $dx = \frac{2}{1+t^2}dt$. Limits: $t=0$ to $t=1$. Answer: $\frac{\ln 3}{4}$.}}

\vspace{1.5cm}

% =============================================================================
% PROBLEM 13: Particle motion with integration
% =============================================================================
\begin{problem}
A particle moves with velocity $v = 2t - 3$ m/s. If $s(0) = 5$m, find $s(4)$.
\end{problem}

\vspace{0.5cm}
\rotatebox{180}{\parbox{\textwidth}{\small \textbf{Hint:} $s(t) = \int v \, dt = t^2 - 3t + C$. Use $s(0)=5$: $C=5$. Then $s(4) = 16 - 12 + 5 = 9$m.}}

\vspace{1.5cm}

% =============================================================================
% PROBLEM 14: Substitution creating ln + arctan
% =============================================================================
\begin{problem}
Find $\displaystyle \int \frac{3x + 5}{x^2 + 4} \, dx$
\end{problem}

\vspace{0.5cm}
\rotatebox{180}{\parbox{\textwidth}{\small \textbf{Hint:} Split: $\frac{3x}{x^2+4} + \frac{5}{x^2+4}$. First gives $\frac{3}{2}\ln(x^2+4)$, second gives $\frac{5}{2}\arctan(x/2)$. Answer: $\frac{3}{2}\ln(x^2+4) + \frac{5}{2}\arctan(x/2) + C$.}}

\vspace{1.5cm}

% =============================================================================
% PROBLEM 15: Complex numbers method
% =============================================================================
\begin{problem}
Use $e^{i\theta} = \cos\theta + i\sin\theta$ to find $\displaystyle \int e^x \cos x \, dx$
\end{problem}

\vspace{0.5cm}
\rotatebox{180}{\parbox{\textwidth}{\small \textbf{Hint:} Consider $\int e^x e^{ix}dx = \int e^{(1+i)x}dx = \frac{e^{(1+i)x}}{1+i}$. Multiply by $\frac{1-i}{1-i}$, take real part. Answer: $\frac{e^x(\cos x + \sin x)}{2} + C$.}}
