% Part 2: Medium Problems (15 problems with hints only - upside-down format)
% Students attempt first, then rotate page 180° to read hint

% =============================================================================
% PROBLEM 1: Integration by parts twice
% =============================================================================
\begin{problem}
Evaluate $\displaystyle \int x^2 e^x \, dx$
\end{problem}

\vspace{0.5cm}
\rotatebox{180}{\parbox{\textwidth}{\small \textbf{Hint:} Apply by parts twice. First: $u_1 = x^2$, $dv_1 = e^x dx$. Second: $u_2 = 2x$, $dv_2 = e^x dx$. Answer: $e^x(x^2 - 2x + 2) + C$.}}

\begin{solution}
First: $u=x^2$, $dv=e^xdx \implies I = x^2e^x - 2\int xe^xdx$

Second: $u=x$, $dv=e^xdx \implies \int xe^xdx = xe^x - e^x$

\[
I = x^2e^x - 2(xe^x - e^x) + C = e^x(x^2 - 2x + 2) + C
\]

\textbf{Answer:} $\boxed{e^x(x^2-2x+2) + C}$
\end{solution}

\begin{takeaways}
Apply parts repeatedly for polynomial × exponential. Factor result.
\end{takeaways}

\vspace{1cm}

% =============================================================================
% PROBLEM 2: Partial fractions with repeated factor
% =============================================================================
\begin{problem}
Find $\displaystyle \int \frac{2x + 3}{(x-1)^2} \, dx$
\end{problem}

\vspace{0.5cm}
\rotatebox{180}{\parbox{\textwidth}{\small \textbf{Hint:} $\frac{2x+3}{(x-1)^2} = \frac{A}{x-1} + \frac{B}{(x-1)^2}$. Solving: $A=2$, $B=5$. Answer: $2\ln|x-1| - \frac{5}{x-1} + C$.}}

\begin{solution}
Partial fractions: $2x+3 = A(x-1) + B$

Solve: $A=2$, $B=5$

\[
I = \int\left(\frac{2}{x-1} + \frac{5}{(x-1)^2}\right)dx = 2\ln|x-1| - \frac{5}{x-1} + C
\]

\textbf{Answer:} $\boxed{2\ln|x-1| - \frac{5}{x-1} + C}$
\end{solution}

\begin{takeaways}
Repeated factors need separate terms for each power: $\frac{A}{x-a} + \frac{B}{(x-a)^2}$.
\end{takeaways}

\vspace{1cm}

% =============================================================================
% PROBLEM 3: Substitution + completing square
% =============================================================================
\begin{problem}
Evaluate $\displaystyle \int \frac{1}{\sqrt{6x - x^2}} \, dx$
\end{problem}

\vspace{0.5cm}
\rotatebox{180}{\parbox{\textwidth}{\small \textbf{Hint:} Complete square: $6x - x^2 = 9 - (x-3)^2$. Let $u = x-3$. Answer: $\arcsin\left(\frac{x-3}{3}\right) + C$.}}

\begin{solution}
Complete square: $6x-x^2 = -(x^2-6x) = -(x^2-6x+9-9) = 9-(x-3)^2$

Standard form: $\displaystyle \int\frac{1}{\sqrt{9-(x-3)^2}}dx = \arcsin\left(\frac{x-3}{3}\right) + C$

\textbf{Answer:} $\boxed{\arcsin\left(\frac{x-3}{3}\right) + C}$
\end{solution}

\begin{takeaways}
Complete square for $\sqrt{ax-x^2}$ to get $\sqrt{a^2-(x-h)^2}$ form.
\end{takeaways}

\vspace{1cm}

% =============================================================================
% PROBLEM 4: King's property with trig
% =============================================================================
\begin{problem}
Evaluate $\displaystyle \int_0^{\pi/2} \frac{x}{\sin x + \cos x} \, dx$
\end{problem}

\vspace{0.5cm}
\rotatebox{180}{\parbox{\textwidth}{\small \textbf{Hint:} Let $I = \int_0^{\pi/2} \frac{x}{\sin x + \cos x}dx$. King's: $I = \int_0^{\pi/2} \frac{\pi/2 - x}{\cos x + \sin x}dx$. Add: $2I = \frac{\pi}{2}\int_0^{\pi/2}\frac{1}{\sin x + \cos x}dx$. Use $t = \tan(x/2)$.}}

\begin{solution}
King's property: $I = \int_0^{\pi/2}\frac{\pi/2-x}{\sin x+\cos x}dx$

Add: $2I = \frac{\pi}{2}\int_0^{\pi/2}\frac{1}{\sin x+\cos x}dx$

Using t-formula: $\int\frac{1}{\sin x+\cos x}dx = \ln|1+\tan(x/2)|$

Evaluate: $2I = \frac{\pi}{2}\ln 2 \implies I = \frac{\pi\ln 2}{4}$

\textbf{Answer:} $\boxed{\frac{\pi\ln 2}{4}}$
\end{solution}

\begin{takeaways}
King's property eliminates $x$ in numerator. t-formula handles rational trig expressions.
\end{takeaways}

\vspace{1cm}

% =============================================================================
% PROBLEM 5: Trig substitution for √(a² - x²)
% =============================================================================
\begin{problem}
Find $\displaystyle \int \sqrt{16 - x^2} \, dx$
\end{problem}

\vspace{0.5cm}
\rotatebox{180}{\parbox{\textwidth}{\small \textbf{Hint:} Let $x = 4\sin\theta$, $dx = 4\cos\theta \, d\theta$. Then $\sqrt{16-x^2} = 4\cos\theta$. Answer: $8\arcsin(x/4) + \frac{x\sqrt{16-x^2}}{2} + C$.}}

\begin{solution}
Let $x=4\sin\theta \implies dx=4\cos\theta\,d\theta$, $\sqrt{16-x^2}=4\cos\theta$

\[
I = 16\int\cos^2\theta\,d\theta = 8\int(1+\cos 2\theta)d\theta = 8\theta + 4\sin 2\theta + C
\]

Back-substitute: $\theta=\arcsin(x/4)$, $\sin 2\theta = \frac{x\sqrt{16-x^2}}{8}$

\textbf{Answer:} $\boxed{8\arcsin(x/4) + \frac{x\sqrt{16-x^2}}{2} + C}$
\end{solution}

\begin{takeaways}
For $\sqrt{a^2-x^2}$: use $x=a\sin\theta$. Double angle for $\cos^2\theta$ integration.
\end{takeaways}

\vspace{1cm}

% =============================================================================
% PROBLEM 6: Reduction formula - powers of sin
% =============================================================================
\begin{problem}
Derive and use: If $I_n = \displaystyle \int_0^{\pi/2} \sin^n x \, dx$, find $I_4$
\end{problem}

\vspace{0.5cm}
\rotatebox{180}{\parbox{\textwidth}{\small \textbf{Hint:} By parts: $I_n = \frac{n-1}{n}I_{n-2}$. Calculate: $I_0 = \frac{\pi}{2}$, $I_2 = \frac{\pi}{4}$, $I_4 = \frac{3}{4} \cdot \frac{\pi}{4} = \frac{3\pi}{16}$.}}

\begin{solution}
Reduction formula (derived using integration by parts): $I_n = \frac{n-1}{n}I_{n-2}$

Base: $I_0 = \frac{\pi}{2}$

\[
I_2 = \frac{1}{2}I_0 = \frac{\pi}{4}
\]

\[
I_4 = \frac{3}{4}I_2 = \frac{3}{4} \cdot \frac{\pi}{4} = \frac{3\pi}{16}
\]

\textbf{Answer:} $\boxed{\frac{3\pi}{16}}$
\end{solution}

\begin{takeaways}
Reduction formulae link $I_n$ to $I_{n-2}$. Work backward from base cases.
\end{takeaways}

\vspace{1cm}

% =============================================================================
% PROBLEM 7: Partial fractions with quadratic
% =============================================================================
\begin{problem}
Evaluate $\displaystyle \int \frac{x^2 + 11}{(x-1)(x^2 + 1)} \, dx$
\end{problem}

\vspace{0.5cm}
\rotatebox{180}{\parbox{\textwidth}{\small \textbf{Hint:} $\frac{x^2+11}{(x-1)(x^2+1)} = \frac{A}{x-1} + \frac{Bx+C}{x^2+1}$. Solve: $A=6$, $B=-6$, $C=5$. Answer: $6\ln|x-1| - 3\ln(x^2+1) + 5\arctan x + C$.}}

\begin{solution}
Partial fractions: $\frac{x^2+11}{(x-1)(x^2+1)} = \frac{A}{x-1} + \frac{Bx+C}{x^2+1}$

Multiply: $x^2+11 = A(x^2+1) + (Bx+C)(x-1)$

Solve: $x=1$: $12 = 2A \implies A=6$

Coefficients: $x^2$: $1 = A+B \implies B=-5$

Constant: $11 = A-C \implies C=A-11=-5$

Actually: $x^2+11 = A(x^2+1) + (Bx+C)(x-1) = (A+B)x^2 + (C-B)x + (A-C)$

So: $A+B=1$, $C-B=0$, $A-C=11$. This gives $A=6$, $B=-5$, $C=-5$.

\[
I = \int\left(\frac{6}{x-1} + \frac{-5x-5}{x^2+1}\right)dx = 6\ln|x-1| - \frac{5}{2}\ln(x^2+1) - 5\arctan x + C
\]

Wait, let me recalculate: For $\frac{-5x}{x^2+1}$: let $u=x^2+1$, $du=2xdx$, so $\int\frac{-5x}{x^2+1}dx = -\frac{5}{2}\ln(x^2+1)$

\[
I = 6\ln|x-1| - \frac{5}{2}\ln(x^2+1) - 5\arctan x + C
\]

\textbf{Answer:} $\boxed{6\ln|x-1| - \frac{5}{2}\ln(x^2+1) - 5\arctan x + C}$
\end{solution}

\begin{takeaways}
For partial fractions with irreducible quadratic: use $\frac{Bx+C}{x^2+1}$. Split into ln and arctan terms.
\end{takeaways}

\vspace{1cm}

% =============================================================================
% PROBLEM 8: Integration by parts - ln
% =============================================================================
\begin{problem}
Find $\displaystyle \int x^2 \ln x \, dx$
\end{problem}

\vspace{0.5cm}
\rotatebox{180}{\parbox{\textwidth}{\small \textbf{Hint:} LIATE: $u = \ln x$, $dv = x^2 dx$. Then $v = \frac{x^3}{3}$. Answer: $\frac{x^3 \ln x}{3} - \frac{x^3}{9} + C = \frac{x^3}{9}(3\ln x - 1) + C$.}}

\begin{solution}
By parts: $u=\ln x$, $dv=x^2dx \implies du=\frac{1}{x}dx$, $v=\frac{x^3}{3}$

\[
I = \frac{x^3\ln x}{3} - \int\frac{x^3}{3} \cdot \frac{1}{x}dx = \frac{x^3\ln x}{3} - \frac{1}{3}\int x^2dx = \frac{x^3\ln x}{3} - \frac{x^3}{9} + C
\]

\textbf{Answer:} $\boxed{\frac{x^3}{9}(3\ln x - 1) + C}$
\end{solution}

\begin{takeaways}
LIATE: logarithm before algebraic. Factor result for cleaner form.
\end{takeaways}

\vspace{1cm}

% =============================================================================
% PROBLEM 9: Definite integral with symmetry
% =============================================================================
\begin{problem}
Evaluate $\displaystyle \int_{-1}^{1} \frac{x^2}{1 + e^x} \, dx$
\end{problem}

\vspace{0.5cm}
\rotatebox{180}{\parbox{\textwidth}{\small \textbf{Hint:} Let $I = \int_{-1}^{1} \frac{x^2}{1+e^x}dx$. King's property: $I = \int_{-1}^{1} \frac{x^2 e^x}{1+e^x}dx$. Add: $2I = \int_{-1}^{1} x^2 dx = \frac{2}{3}$, so $I = \frac{1}{3}$.}}

\begin{solution}
Let $I = \int_{-1}^1 \frac{x^2}{1+e^x}dx$

King's: $I = \int_{-1}^1 \frac{x^2e^x}{e^x+1}dx$

Add: $2I = \int_{-1}^1 \frac{x^2(1+e^x)}{1+e^x}dx = \int_{-1}^1 x^2dx = \left[\frac{x^3}{3}\right]_{-1}^1 = \frac{2}{3}$

Therefore: $I = \frac{1}{3}$

\textbf{Answer:} $\boxed{\frac{1}{3}}$
\end{solution}

\begin{takeaways}
King's property: $\int_a^b f(x)dx = \int_a^b f(a+b-x)dx$. Add to simplify.
\end{takeaways}

\vspace{1cm}

% =============================================================================
% PROBLEM 10: Trig identity + substitution
% =============================================================================
\begin{problem}
Evaluate $\displaystyle \int \sin^4 x \, dx$
\end{problem}

\vspace{0.5cm}
\rotatebox{180}{\parbox{\textwidth}{\small \textbf{Hint:} Use $\sin^2 x = \frac{1-\cos 2x}{2}$. Then $\sin^4 x = \left(\frac{1-\cos 2x}{2}\right)^2 = \frac{1 - 2\cos 2x + \cos^2 2x}{4}$. Use $\cos^2 2x = \frac{1+\cos 4x}{2}$.}}

\begin{solution}
Apply double angle twice:

\[
\sin^4 x = \left(\frac{1-\cos 2x}{2}\right)^2 = \frac{1-2\cos 2x+\cos^2 2x}{4}
\]

\[
= \frac{1-2\cos 2x}{4} + \frac{1+\cos 4x}{8} = \frac{3-4\cos 2x+\cos 4x}{8}
\]

\[
I = \frac{1}{8}\left(3x - 2\sin 2x + \frac{\sin 4x}{4}\right) + C
\]

\textbf{Answer:} $\boxed{\frac{3x}{8} - \frac{\sin 2x}{4} + \frac{\sin 4x}{32} + C}$
\end{solution}

\begin{takeaways}
Repeatedly apply double angle formula for higher even powers.
\end{takeaways}

\vspace{1cm}

% =============================================================================
% PROBLEM 11: Volume of revolution
% =============================================================================
\begin{problem}
Find the volume when the region bounded by $y = \sqrt{x}$, $y = 0$, $x = 4$ is rotated about the $x$-axis.
\end{problem}

\vspace{0.5cm}
\rotatebox{180}{\parbox{\textwidth}{\small \textbf{Hint:} $V = \pi\int_0^4 (\sqrt{x})^2 dx = \pi\int_0^4 x \, dx = \pi \cdot \frac{16}{2} = 8\pi$ cubic units.}}

\begin{solution}
Disk method: $V = \pi\int_a^b [f(x)]^2dx$

\[
V = \pi\int_0^4 (\sqrt{x})^2dx = \pi\int_0^4 x\,dx = \pi\left[\frac{x^2}{2}\right]_0^4 = 8\pi
\]

\textbf{Answer:} $\boxed{8\pi \text{ cubic units}}$
\end{solution}

\begin{takeaways}
Disk method: $V=\pi\int[f(x)]^2dx$. Square the radius function.
\end{takeaways}

\vspace{1cm}

% =============================================================================
% PROBLEM 12: t-formula application
% =============================================================================
\begin{problem}
Evaluate $\displaystyle \int_0^{\pi/2} \frac{1}{3 + 5\cos x} \, dx$
\end{problem}

\vspace{0.5cm}
\rotatebox{180}{\parbox{\textwidth}{\small \textbf{Hint:} Let $t = \tan(x/2)$. Then $\cos x = \frac{1-t^2}{1+t^2}$, $dx = \frac{2}{1+t^2}dt$. Limits: $t=0$ to $t=1$. Answer: $\frac{\ln 3}{4}$.}}

\begin{solution}
t-formula: $t=\tan(x/2)$, $\cos x = \frac{1-t^2}{1+t^2}$, $dx=\frac{2}{1+t^2}dt$

Limits: $x=0 \implies t=0$; $x=\pi/2 \implies t=1$

\[
I = \int_0^1 \frac{2}{3(1+t^2)+5(1-t^2)}dt = \int_0^1 \frac{1}{4-t^2}dt = \frac{1}{4}[\ln|2+t|-\ln|2-t|]_0^1
\]

\[
= \frac{1}{4}\ln 3
\]

\textbf{Answer:} $\boxed{\frac{\ln 3}{4}}$
\end{solution}

\begin{takeaways}
t-formula transforms rational trig expressions to algebraic fractions.
\end{takeaways}

\vspace{1cm}

% =============================================================================
% PROBLEM 13: Particle motion with integration
% =============================================================================
\begin{problem}
A particle moves with velocity $v = 2t - 3$ m/s. If $s(0) = 5$m, find $s(4)$.
\end{problem}

\vspace{0.5cm}
\rotatebox{180}{\parbox{\textwidth}{\small \textbf{Hint:} $s(t) = \int v \, dt = t^2 - 3t + C$. Use $s(0)=5$: $C=5$. Then $s(4) = 16 - 12 + 5 = 9$m.}}

\begin{solution}
Integrate: $s(t) = \int(2t-3)dt = t^2 - 3t + C$

Initial condition: $s(0) = 5 \implies C = 5$

Therefore: $s(t) = t^2 - 3t + 5$

At $t=4$: $s(4) = 16 - 12 + 5 = 9$

\textbf{Answer:} $\boxed{9 \text{ m}}$
\end{solution}

\begin{takeaways}
Velocity integration gives displacement. Use initial conditions for constant.
\end{takeaways}

\vspace{1cm}

% =============================================================================
% PROBLEM 14: Substitution creating ln + arctan
% =============================================================================
\begin{problem}
Find $\displaystyle \int \frac{3x + 5}{x^2 + 4} \, dx$
\end{problem}

\vspace{0.5cm}
\rotatebox{180}{\parbox{\textwidth}{\small \textbf{Hint:} Split: $\frac{3x}{x^2+4} + \frac{5}{x^2+4}$. First gives $\frac{3}{2}\ln(x^2+4)$, second gives $\frac{5}{2}\arctan(x/2)$. Answer: $\frac{3}{2}\ln(x^2+4) + \frac{5}{2}\arctan(x/2) + C$.}}

\begin{solution}
Split numerator:

\[
I = \int\frac{3x}{x^2+4}dx + \int\frac{5}{x^2+4}dx
\]

First: $\frac{3}{2}\ln(x^2+4)$

Second: $\frac{5}{2}\arctan(x/2)$

\textbf{Answer:} $\boxed{\frac{3}{2}\ln(x^2+4) + \frac{5}{2}\arctan(x/2) + C}$
\end{solution}

\begin{takeaways}
Split numerator: match derivative for $\ln$, constant for $\arctan$.
\end{takeaways}

\vspace{1cm}

% =============================================================================
% PROBLEM 15: Complex numbers method
% =============================================================================
\begin{problem}
Use $e^{i\theta} = \cos\theta + i\sin\theta$ to find $\displaystyle \int e^x \cos x \, dx$
\end{problem}

\vspace{0.5cm}
\rotatebox{180}{\parbox{\textwidth}{\small \textbf{Hint:} Consider $\int e^x e^{ix}dx = \int e^{(1+i)x}dx = \frac{e^{(1+i)x}}{1+i}$. Multiply by $\frac{1-i}{1-i}$, take real part. Answer: $\frac{e^x(\cos x + \sin x)}{2} + C$.}}

\begin{solution}
Consider: $\int e^x(\cos x + i\sin x)dx = \int e^{(1+i)x}dx = \frac{e^{(1+i)x}}{1+i}$

Simplify: $\frac{e^x e^{ix}}{1+i} \cdot \frac{1-i}{1-i} = \frac{e^x(1-i)(\cos x+i\sin x)}{2}$

Real part: $\frac{e^x(\cos x + \sin x)}{2}$

\textbf{Answer:} $\boxed{\frac{e^x(\cos x + \sin x)}{2} + C}$
\end{solution}

\begin{takeaways}
Complex exponentials simplify products of exponential and trig. Take real part for result.
\end{takeaways}

\vspace{1cm}

% =============================================================================
% PROBLEM 16: The Beta Function and Fractional Recurrence
% =============================================================================
\begin{problem}[The Beta Function of the Third Degree]
	% \textbf{Historical Context}\\
The \textbf{Beta function}, also known as the Euler integral of the first kind, was first studied by Leonhard Euler and Adrien-Marie Legendre in the 18th century. It is a fundamental special function in mathematical analysis that generalizes the concept of factorials and binomial coefficients to non-integer values. In this problem, we explore its recursive properties and its application to evaluating complex integrals.

This problem will guide you through deriving a recurrence relation for the Beta function and using it to evaluate a specific integral.

\vspace{0.5em}
For $m > 0$ and $n > 0$, we define the function $B(m, n)$ by the definite integral:
$$ B(m, n) = \int_0^1 x^{m-1} (1-x)^{n-1} \, dx $$

\vspace{0.5em}

\begin{enumerate}
	\item[(i)] Use the substitution $x = \sin^2 \theta$ to show that $B\left(\frac{1}{2}, \frac{1}{2}\right) = \pi$.
    % \hfill \textbf{(2 marks)}
    
	\item[(ii)] Use integration by parts to show that for $m > 0$ and $n > 0$:
	$$ B(m, n+1) = \frac{n}{m+n} B(m, n) $$
	% \hfill \textbf{(3 marks)}
    
	\item[(iii)] Use the substitution $x = \frac{u}{1+u}$ to show that:
	$$ B(m, n) = \int_0^\infty \frac{u^{m-1}}{(1+u)^{m+n}} \, du $$
	% \hfill \textbf{(3 marks)}
    
	\item[(iv)] Hence, evaluate the exact value of:
	$$ \int_0^\infty \frac{\sqrt{x}}{(1+x)^3} \, dx $$
	% \hfill \textbf{(7 marks)}
\end{enumerate}
\end{problem}

\vspace{0.5cm}
\rotatebox{180}{\parbox{\textwidth}{\small \textbf{Hint:} \textbf{Part (i):} Recall that the domain of integration changes from $x \in [0,1]$ to $\theta \in [0, \frac{\pi}{2}]$. Don't forget to calculate $dx$ in terms of $d\theta$. The integrand should simplify to a constant.\\
	extbf{Part (ii):} Apply Integration by Parts with $u = (1-x)^n$ and $dv = x^{m-1} \, dx$. You will arrive at an expression involving $\int x^m (1-x)^{n-1} \, dx$. To relate this back to $B(m,n)$, try writing $x^m = x^{m-1} \cdot x$ or substitute $x = 1 - (1-x)$ to split the integral.\\
	extbf{Part (iii):} Be careful with the limits of integration. As $x \to 1$, notice that $u = \frac{x}{1-x} \to \infty$. Algebraic simplification of the powers is key.\\
	extbf{Part (iv):} First, match the given integral to the form in Part (iii) to determine the specific values of $m$ and $n$. You will need to use the recurrence relation from Part (ii). Note that the Beta function is symmetric, i.e., $B(m,n) = B(n,m)$, which allows you to switch indices if needed to apply the reduction formula.}}

\begin{solution}
\begin{enumerate}
	\item[(i)] Let $x = \sin^2\theta$, then $dx = 2\sin\theta\cos\theta \, d\theta$.
	\begin{align*}
		B(\tfrac{1}{2}, \tfrac{1}{2}) &= \int_0^{\pi/2} (\sin^2\theta)^{-1/2} (1-\sin^2\theta)^{-1/2} \cdot 2\sin\theta\cos\theta \, d\theta \\
		&= \int_0^{\pi/2} \frac{1}{\sin\theta \cos\theta} \cdot 2\sin\theta\cos\theta \, d\theta = \int_0^{\pi/2} 2 \, d\theta = [2\theta]_0^{\pi/2} = \pi
	\end{align*}

	\item[(ii)] Let $I = B(m, n+1) = \int_0^1 x^{m-1}(1-x)^n \, dx$.
	\begin{itemize}
		\item Let $u = (1-x)^n \implies du = -n(1-x)^{n-1} \, dx$.
		\item Let $dv = x^{m-1} \, dx \implies v = \frac{x^m}{m}$.
	\end{itemize}
	$$ I = \left[ \frac{x^m}{m}(1-x)^n \right]_0^1 + \frac{n}{m} \int_0^1 x^m (1-x)^{n-1} \, dx $$
	The boundary term vanishes. For the integral, use $x^m = x^{m-1} - x^{m-1}(1-x)$:

	\begin{align*}
	\int_0^1 x^m (1-x)^{n-1} \, dx &= \int_0^1 x^{m-1}(1-x)^{n-1} \, dx 
	\, - \int_0^1 x^{m-1}(1-x)^n \, dx \\
	&= B(m,n) - B(m, n+1)
	\end{align*}
    
	Substituting back:
	$$ B(m, n+1) = \frac{n}{m} \left( B(m,n) - B(m, n+1) \right) $$
	$$ B(m, n+1) \left( 1 + \frac{n}{m} \right) = \frac{n}{m} B(m,n) \implies B(m, n+1) = \frac{n}{m+n} B(m,n) $$

	\item[(iii)] Let $x = \frac{u}{1+u}$, then $dx = \frac{1}{(1+u)^2} \, du$. Also $1-x = \frac{1}{1+u}$.
	$$ B(m,n) = \int_0^\infty \left( \frac{u}{1+u} \right)^{m-1} \left( \frac{1}{1+u} \right)^{n-1} \frac{1}{(1+u)^2} \, du $$
	Combining the powers of $(1+u)$ in the denominator: $(m-1) + (n-1) + 2 = m+n$.
	$$ = \int_0^\infty \frac{u^{m-1}}{(1+u)^{m+n}} \, du $$

	\item[(iv)] Comparing $\int_0^\infty \frac{x^{1/2}}{(1+x)^3} dx$ with Part (iii):
	$m-1 = \frac{1}{2} \implies m = \frac{3}{2}$. $m+n = 3 \implies n = \frac{3}{2}$.
	Target: Evaluate $B(\frac{3}{2}, \frac{3}{2})$.
	\begin{itemize}
		\item Apply recurrence (Part ii): $B(\frac{3}{2}, \frac{3}{2}) = B(\frac{3}{2}, \frac{1}{2} + 1) = \frac{1/2}{3/2 + 1/2} B(\frac{3}{2}, \frac{1}{2}) = \frac{1}{4} B(\frac{3}{2}, \frac{1}{2})$.
		\item By symmetry, $B(\frac{3}{2}, \frac{1}{2}) = B(\frac{1}{2}, \frac{3}{2})$.
		\item Apply recurrence again: $B(\frac{1}{2}, \frac{3}{2}) = B(\frac{1}{2}, \frac{1}{2} + 1) = \frac{1/2}{1/2 + 1/2} B(\frac{1}{2}, \frac{1}{2}) = \frac{1}{2} \pi$.
		\item Final Result: $\frac{1}{4} \times \frac{\pi}{2} = \frac{\pi}{8}$.
	\end{itemize}
\end{enumerate}
\end{solution}

\begin{takeaways}
\begin{itemize}
	\item \textbf{Generalization:} The Beta function generalizes the concept of binomial coefficients to non-integer values.
	\item \textbf{Technique:} Transforming a finite integral on $[0,1]$ to an infinite integral on $[0, \infty)$ is a standard technique for evaluating complex rational integrals.
	\item \textbf{Structure:} Recursive relationships allow us to compute values for higher indices (like $3/2$) by reducing them to a known "base case" (like $1/2$).
	\item For integer values of $m$ and $n$, the Beta function relates to factorials via:
            $$ B(m, n) = \frac{(m-1)!(n-1)!}{(m+n-1)!} $$ 
\end{itemize}
\end{takeaways}
