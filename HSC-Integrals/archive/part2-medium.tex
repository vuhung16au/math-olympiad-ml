% Part 2: Medium Problems (~15 problems with hints and concise solutions)

% Problem 1: Applications - Particle Motion (from sample 02.tex)
\begin{problem}[Particle Motion with Resistance]
A particle of mass $m$ kg moves along a horizontal line with initial velocity $V_0$ m/s. The motion is resisted by a constant force of $mk$ newtons and a variable force of $mv^2$ newtons, where $k$ is a positive constant and $v$ m/s is the velocity at time $t$ seconds.

Show that the distance travelled when the particle comes to rest is $\displaystyle \frac{1}{2}\ln\left(\frac{k + V_0^2}{k}\right)$ metres.
\end{problem}

\begin{hint}
Apply Newton's Second Law, then use $a = v\frac{dv}{dx}$ to change variables. Separate and integrate.
\end{hint}

\begin{solution}
Total resistive force: $F = -(mk + mv^2) \implies ma = -m(k + v^2) \implies a = -(k + v^2)$

Using $a = v\frac{dv}{dx}$:
\[
v \frac{dv}{dx} = -(k + v^2) \implies dx = -\frac{v}{k + v^2} dv
\]

Integrate with limits: $x=0$ to $x=D$, $v=V_0$ to $v=0$:
\[
\int_{0}^{D} dx = \int_{V_0}^{0} -\frac{v}{k + v^2} dv
\]

RHS: $\int \frac{v}{k+v^2} dv = \frac{1}{2} \ln(k+v^2)$

\[
D = -\frac{1}{2}[\ln k - \ln(k + V_0^2)] = \frac{1}{2}\ln\left(\frac{k + V_0^2}{k}\right)
\]

\textbf{Answer:} $\boxed{\frac{1}{2}\ln\left(\frac{k + V_0^2}{k}\right) \text{ metres}}$
\end{solution}

\begin{takeaways}
For distance problems, use $a = v\frac{dv}{dx}$ instead of $a = \frac{dv}{dt}$. Reverse chain rule for logarithms.
\end{takeaways}

\vspace{1cm}

% Problem 2: Mechanics with differential equation (from sample 18.tex)
\begin{problem}[Resistive Force Problem]
A particle with mass $1$ kg is moving along the $x$-axis. Initially at origin with speed $u$ m/s to the right. It experiences resistive force of magnitude $v + 3v^2$ newtons.

\begin{enumerate}[label=(\roman*)]
\item Show that $\displaystyle \frac{dv}{dx} = -(1 + 3v)$.
\item Hence find $x$ as a function of $v$.
\end{enumerate}
\end{problem}

\begin{hint}
Part (i): Use $F = ma$ with $a = v\frac{dv}{dx}$, then simplify. Part (ii): Separate variables and integrate.
\end{hint}

\begin{solution}
\textbf{(i)} $F = -(v + 3v^2) = ma = 1 \cdot v\frac{dv}{dx}$

\[
v\frac{dv}{dx} = -v(1 + 3v) \implies \frac{dv}{dx} = -(1 + 3v) \quad \text{(since } v \neq 0\text{)}
\]

\textbf{(ii)} Invert: $\frac{dx}{dv} = -\frac{1}{1 + 3v}$

Integrate: $x = -\frac{1}{3}\ln(1 + 3v) + C$

Initial conditions: $x=0, v=u \implies 0 = -\frac{1}{3}\ln(1 + 3u) + C$

Therefore: $C = \frac{1}{3}\ln(1 + 3u)$

\textbf{Answer:} $\boxed{x = \frac{1}{3}\ln\left(\frac{1 + 3u}{1 + 3v}\right)}$
\end{solution}

\begin{takeaways}
Invert $\frac{dv}{dx}$ to $\frac{dx}{dv}$ when finding $x$ as function of $v$. Use initial conditions for constant.
\end{takeaways}

\vspace{1cm}

% Problem 3: t-formula with more complex form (from sample 17.tex)
\begin{problem}[Weierstrass Substitution - Complex Form]
Using the substitution $t = \tan\frac{x}{2}$, find $\displaystyle \int \frac{dx}{1 + \cos x - \sin x}$.
\end{problem}

\begin{hint}
Apply t-formulae: $\cos x = \frac{1-t^2}{1+t^2}$, $\sin x = \frac{2t}{1+t^2}$, $dx = \frac{2}{1+t^2}dt$. Simplify denominator carefully.
\end{hint}

\begin{solution}
Substitute t-formulae:
\[
I = \int \frac{1}{1 + \frac{1-t^2}{1+t^2} - \frac{2t}{1+t^2}} \cdot \frac{2}{1+t^2} dt
\]

Multiply denominator by $(1+t^2)$:
\[
I = \int \frac{2}{(1+t^2) + (1-t^2) - 2t} dt = \int \frac{2}{2 - 2t} dt = \int \frac{1}{1 - t} dt
\]

Integrate: $I = -\ln|1 - t| + C$

\textbf{Answer:} $\boxed{-\ln\left|1 - \tan\frac{x}{2}\right| + C}$
\end{solution}

\begin{takeaways}
t-formula simplifies rational trig expressions. Combine fractions carefully before integrating.
\end{takeaways}

\vspace{1cm}

% Problem 4: SHM integration (from sample 13.tex - simplified)
\begin{problem}[Simple Harmonic Motion]
An object moves in SHM with acceleration $\ddot{x} = -4(x-3)$. Initially at $x=5.5$ m moving toward origin with speed $8$ m/s at origin.

Find the range of oscillation (between which two values of $x$).
\end{problem}

\begin{hint}
Identify $n^2 = 4$ and center $c = 3$. Use $v^2 = n^2(A^2 - (x-c)^2)$ with $v=8$ when $x=0$ to find amplitude $A$.
\end{hint}

\begin{solution}
Standard SHM form: $\ddot{x} = -n^2(x - c)$

Comparing: $n^2 = 4 \implies n = 2$, $c = 3$

Velocity-displacement relation: $v^2 = n^2(A^2 - (x-c)^2)$

At $x=0$, $v=8$:
\[
64 = 4(A^2 - 9) \implies 16 = A^2 - 9 \implies A = 5
\]

Range: $[c-A, c+A] = [3-5, 3+5]$

\textbf{Answer:} $\boxed{x \in [-2, 8]}$
\end{solution}

\begin{takeaways}
SHM amplitude found from energy equation. Particle oscillates between $c \pm A$.
\end{takeaways}

\vspace{1cm}

% Problem 5: Reduction formula (from sample 05.tex)
\begin{problem}[Cotangent Reduction Formula]
Let $I_n = \int_{\frac{\pi}{4}}^{\frac{\pi}{2}} \cot^{2n}\theta \, d\theta$ for $n \geq 0$.

Show that $I_n = \frac{1}{2n-1} - I_{n-1}$ for $n > 0$.
\end{problem}

\begin{hint}
Add $I_n + I_{n-1}$ and use $\cot^2\theta + 1 = \text{cosec}^2\theta$. Substitute $u = \cot\theta$ with $du = -\text{cosec}^2\theta \, d\theta$.
\end{hint}

\begin{solution}
Add: $I_n + I_{n-1} = \int_{\frac{\pi}{4}}^{\frac{\pi}{2}} \cot^{2n-2}\theta(\cot^2\theta + 1) d\theta$

Use identity: $= \int_{\frac{\pi}{4}}^{\frac{\pi}{2}} \cot^{2n-2}\theta \cdot \text{cosec}^2\theta \, d\theta$

Substitute $u = \cot\theta$, $-du = \text{cosec}^2\theta \, d\theta$

Limits: $\theta=\frac{\pi}{4} \implies u=1$; $\theta=\frac{\pi}{2} \implies u=0$

\[
I_n + I_{n-1} = -\int_1^0 u^{2n-2} du = \int_0^1 u^{2n-2} du = \frac{1}{2n-1}
\]

\textbf{Result:} $\boxed{I_n = \frac{1}{2n-1} - I_{n-1}}$ ✓
\end{solution}

\begin{takeaways}
Reduction formulae created by strategic addition. Trig identities convert to substitution-friendly forms.
\end{takeaways}

\vspace{1cm}

% Problem 6: Logarithmic reduction (from sample 11.tex)
\begin{problem}[Logarithmic Power Reduction]
Let $I_n = \int_1^e (\ln x)^n \, dx$ for $n \ge 0$.

Show that $I_n = e - nI_{n-1}$ for $n \ge 1$.
\end{problem}

\begin{hint}
Integration by parts: $u = (\ln x)^n$, $dv = dx$. Boundary term at $x=1$ gives zero since $\ln 1 = 0$.
\end{hint}

\begin{solution}
Parts: $u = (\ln x)^n \implies du = n(\ln x)^{n-1} \cdot \frac{1}{x}dx$; $v = x$

\[
I_n = [x(\ln x)^n]_1^e - \int_1^e x \cdot n(\ln x)^{n-1} \cdot \frac{1}{x} dx
\]

Boundary: At $x=e$: $e \cdot 1^n = e$; at $x=1$: $1 \cdot 0^n = 0$

\[
I_n = e - n\int_1^e (\ln x)^{n-1} dx = e - nI_{n-1}
\]

\textbf{Result:} $\boxed{I_n = e - nI_{n-1}}$ ✓
\end{solution}

\begin{takeaways}
Parts on repeated functions creates recursion. Logarithm boundaries simplify at $e$ and $1$.
\end{takeaways}

\vspace{1cm}

% Problem 7: Complex substitution (from sample 04.tex)
\begin{problem}[King's Property with t-Formula]
Evaluate $\int_{0}^{\frac{\pi}{2}} \frac{u}{1 + \sin u + \cos u} \, du$ using substitution $u = \frac{\pi}{2} - x$.
\end{problem}

\begin{hint}
After substitution, add original and transformed. Numerators combine to $\frac{\pi}{2}$. Then use t-formula on resulting $\frac{1}{1+\sin u+\cos u}$.
\end{hint}

\begin{solution}
Let $u = \frac{\pi}{2} - x$:
\[
I = \int_0^{\frac{\pi}{2}} \frac{\frac{\pi}{2}-u}{1+\sin u+\cos u} du
\]

Add with original: $2I = \frac{\pi}{2}\int_0^{\frac{\pi}{2}} \frac{1}{1+\sin u+\cos u} du$

t-formula ($t = \tan\frac{u}{2}$): $\sin u = \frac{2t}{1+t^2}$, $\cos u = \frac{1-t^2}{1+t^2}$, $du = \frac{2}{1+t^2}dt$

Limits: $u=0 \implies t=0$; $u=\frac{\pi}{2} \implies t=1$

\[
2I = \frac{\pi}{2}\int_0^1 \frac{2}{(1+t^2) + 2t + (1-t^2)} dt = \frac{\pi}{2}\int_0^1 \frac{1}{1+t} dt
\]

\[
= \frac{\pi}{2}[\ln(1+t)]_0^1 = \frac{\pi\ln 2}{2} \implies I = \frac{\pi\ln 2}{4}
\]

\textbf{Answer:} $\boxed{\frac{\pi\ln 2}{4}}$
\end{solution}

\begin{takeaways}
King's property plus t-formula powerful combination. Self-similar integrals simplify via addition.
\end{takeaways}

\vspace{1cm}

% Problem 8: Power of cosine reduction (from sample 08.tex - part iii only)
\begin{problem}[Quartic Cosine via Complex Numbers]
Using the identity $\cos^4\theta = \frac{1}{8}(\cos 4\theta + 4\cos 2\theta + 3)$, evaluate:
\[
\int_0^{\frac{\pi}{2}} \cos^4\theta \, d\theta
\]
\end{problem}

\begin{hint}
Expand using given identity. Integrate term by term. Note $\sin(2\pi) = \sin(\pi) = 0$.
\end{hint}

\begin{solution}
Apply identity:
\[
I = \frac{1}{8}\int_0^{\frac{\pi}{2}} (\cos 4\theta + 4\cos 2\theta + 3) d\theta
\]

\[
= \frac{1}{8}\left[\frac{1}{4}\sin 4\theta + 2\sin 2\theta + 3\theta\right]_0^{\frac{\pi}{2}}
\]

At $\theta=\frac{\pi}{2}$: $\frac{1}{4}\sin(2\pi) + 2\sin(\pi) + \frac{3\pi}{2} = 0 + 0 + \frac{3\pi}{2}$

At $\theta=0$: All terms zero

\[
I = \frac{1}{8} \cdot \frac{3\pi}{2} = \frac{3\pi}{16}
\]

\textbf{Answer:} $\boxed{\frac{3\pi}{16}}$
\end{solution}

\begin{takeaways}
Power reduction formulae via complex exponentials. Check boundary values carefully for trig terms.
\end{takeaways}

\vspace{1cm}

% Problem 9: Partial fractions with quadratic (from sample 29.tex)
\begin{problem}[Partial Fractions - Mixed]
Find $\int \frac{3x^2 + 2x + 1}{(x-1)(x^2+1)} \, dx$.
\end{problem}

\begin{hint}
Decompose: $\frac{A}{x-1} + \frac{Bx+C}{x^2+1}$. Cover-up for $A$, compare coefficients for $B, C$.
\end{hint}

\begin{solution}
Partial fractions: $3x^2+2x+1 = A(x^2+1) + (Bx+C)(x-1)$

At $x=1$: $6 = 2A \implies A=3$

Expand: $3x^2+2x+1 = 3x^2+3 + Bx^2 + (C-B)x - C$

Coefficients: $x^2$: $3 = 3+B \implies B=0$; constant: $1 = 3-C \implies C=2$

\[
I = \int\left(\frac{3}{x-1} + \frac{2}{x^2+1}\right) dx = 3\ln|x-1| + 2\arctan x + C
\]

\textbf{Answer:} $\boxed{3\ln|x-1| + 2\arctan x + C}$
\end{solution}

\begin{takeaways}
Irreducible quadratics need $Bx+C$ form. Coefficient comparison after substitution.
\end{takeaways}

\vspace{1cm}

% Problem 10: Half-angle identity (from sample 38.tex)
\begin{problem}[Perfect Square Form]
Evaluate $\int \sqrt{1 + \sin x} \, dx$.
\end{problem}

\begin{hint}
Use identities: $1 = \sin^2\frac{x}{2} + \cos^2\frac{x}{2}$ and $\sin x = 2\sin\frac{x}{2}\cos\frac{x}{2}$ to form perfect square $(\sin\frac{x}{2} + \cos\frac{x}{2})^2$.
\end{hint}

\begin{solution}
Apply identities:
\[
1 + \sin x = \sin^2\frac{x}{2} + \cos^2\frac{x}{2} + 2\sin\frac{x}{2}\cos\frac{x}{2} = \left(\sin\frac{x}{2} + \cos\frac{x}{2}\right)^2
\]

Therefore:
\[
I = \int \left(\sin\frac{x}{2} + \cos\frac{x}{2}\right) dx = -2\cos\frac{x}{2} + 2\sin\frac{x}{2} + C
\]

\textbf{Answer:} $\boxed{2\left(\sin\frac{x}{2} - \cos\frac{x}{2}\right) + C}$
\end{solution}

\begin{takeaways}
Half-angle identities create perfect squares. Simplifies radicals with trig expressions.
\end{takeaways}

\vspace{1cm}

% Problem 11: Product-to-sum (from sample 39.tex)
\begin{problem}[Product-to-Sum Formula]
Prove $\int \sin x \sin 2x \, dx = \frac{2\sin^3 x}{3} + c$ using the product-to-sum formula.
\end{problem}

\begin{hint}
Use $\sin A \sin B = \frac{1}{2}[\cos(A-B) - \cos(A+B)]$. Then use triple angle $\sin 3x = 3\sin x - 4\sin^3 x$ to verify.
\end{hint}

\begin{solution}
Product-to-sum: $\sin x \sin 2x = \frac{1}{2}[\cos(-x) - \cos 3x] = \frac{1}{2}(\cos x - \cos 3x)$

\[
I = \frac{1}{2}\int(\cos x - \cos 3x)dx = \frac{1}{2}\left(\sin x - \frac{1}{3}\sin 3x\right) + c
\]

Use $\sin 3x = 3\sin x - 4\sin^3 x$:
\[
= \frac{1}{2}\left[\sin x - \frac{1}{3}(3\sin x - 4\sin^3 x)\right] + c = \frac{1}{2} \cdot \frac{4\sin^3 x}{3} + c
\]

\textbf{Result:} $\boxed{\frac{2\sin^3 x}{3} + c}$ ✓
\end{solution}

\begin{takeaways}
Product-to-sum converts trig products to sums. Triple angle back-substitution verifies equivalence.
\end{takeaways}

\vspace{1cm}

% Problem 12: Error in integration (from sample 30.tex)
\begin{problem}[Integration Paradox]
Explain the error in this argument:

Using parts on $\int \frac{1}{x} dx$ with $u=\frac{1}{x}$, $dv=dx$ gives:
\[
\int \frac{1}{x} dx = 1 + \int \frac{1}{x} dx \implies 0 = 1
\]
What is wrong?
\end{problem}

\begin{hint}
Indefinite integrals represent families of functions, each with different constant $C$. Cannot "cancel" integrals algebraically.
\end{hint}

\begin{solution}
The flaw: treating indefinite integral as single value.

Correctly: $\ln|x| + C_1 = 1 + \ln|x| + C_2$

Simplifying: $C_1 = 1 + C_2 \implies C_1 - C_2 = 1$ ✓

This says constants differ by 1, not that $0=1$.

\textbf{Lesson:} Cannot algebraically cancel indefinite integrals; they represent function families, not values.
\end{solution}

\begin{takeaways}
Indefinite integrals are families with arbitrary constants. Cannot treat as algebraic equations without accounting for constants.
\end{takeaways}

\vspace{1cm}

% Problem 13: Complex induction problem (from sample 20.tex - simplified)
\begin{problem}[Exponential Integral Bounds]
Let $J_n = \int_0^1 x^n e^{-x} dx$ for $n \ge 0$.

\begin{enumerate}[label=(\roman*)]
\item Show $J_0 = 1 - \frac{1}{e}$.
\item Show $J_n = nJ_{n-1} - \frac{1}{e}$ for $n \ge 1$.
\end{enumerate}
\end{problem}

\begin{hint}
Part (i): Direct integration. Part (ii): Parts with $u=x^n$, $dv=e^{-x}dx$. Boundary terms simplify at $x=0,1$.
\end{hint}

\begin{solution}
\textbf{(i)} $J_0 = \int_0^1 e^{-x} dx = [-e^{-x}]_0^1 = -\frac{1}{e} - (-1) = 1 - \frac{1}{e}$ ✓

\textbf{(ii)} Parts: $u=x^n$, $du=nx^{n-1}dx$; $v=-e^{-x}$

\[
J_n = [-x^n e^{-x}]_0^1 + n\int_0^1 x^{n-1}e^{-x} dx
\]

Boundary: $-1^n e^{-1} - 0 = -\frac{1}{e}$

\[
J_n = -\frac{1}{e} + nJ_{n-1} = nJ_{n-1} - \frac{1}{e}
\]

\textbf{Result:} $\boxed{J_n = nJ_{n-1} - \frac{1}{e}}$ ✓
\end{solution}

\begin{takeaways}
Reduction formulae via parts. Check boundaries carefully - exponentials decay, powers vanish at zero.
\end{takeaways}

\vspace{1cm}

% Problem 14: Reverse chain rule trickier
\begin{problem}[Chain Rule with Substitution]
Evaluate $\int \frac{x}{\sqrt{1-x^4}} \, dx$.
\end{problem}

\begin{hint}
Let $u = x^2$, then $du = 2x \, dx$, so $x \, dx = \frac{1}{2}du$. Results in $\arcsin u$ form.
\end{hint}

\begin{solution}
Let $u = x^2 \implies du = 2x \, dx$

\[
I = \frac{1}{2}\int \frac{1}{\sqrt{1-u^2}} du = \frac{1}{2}\arcsin u + C
\]

Back-substitute: $u = x^2$

\textbf{Answer:} $\boxed{\frac{1}{2}\arcsin(x^2) + C}$
\end{solution}

\begin{takeaways}
Recognize hidden derivatives after simple substitutions. $\sqrt{1-u^2}$ suggests inverse sine.
\end{takeaways}

\vspace{1cm}

% Problem 15: Trig substitution application
\begin{problem}[Trigonometric Substitution]
Evaluate $\int \frac{x^2}{\sqrt{4-x^2}} \, dx$.
\end{problem}

\begin{hint}
Let $x = 2\sin\theta$. Then $\sqrt{4-x^2} = 2\cos\theta$. Use $\sin^2\theta = \frac{1-\cos 2\theta}{2}$.
\end{hint}

\begin{solution}
Let $x = 2\sin\theta \implies dx = 2\cos\theta \, d\theta$

\[
I = \int \frac{4\sin^2\theta}{2\cos\theta} \cdot 2\cos\theta \, d\theta = 4\int \sin^2\theta \, d\theta
\]

\[
= 4\int \frac{1-\cos 2\theta}{2} d\theta = 2\left(\theta - \frac{1}{2}\sin 2\theta\right) + C
\]

Back-substitute: $\theta = \arcsin\frac{x}{2}$, $\sin 2\theta = 2\sin\theta\cos\theta = 2 \cdot \frac{x}{2} \cdot \frac{\sqrt{4-x^2}}{2} = \frac{x\sqrt{4-x^2}}{2}$

\textbf{Answer:} $\boxed{2\arcsin\frac{x}{2} - \frac{x\sqrt{4-x^2}}{2} + C}$
\end{solution}

\begin{takeaways}
Trig substitution for $\sqrt{a^2-x^2}$ forms. Double angle formula for $\sin^2\theta$ integration.
\end{takeaways}

