\section{Measure Zero (Nowhere)}

While rational numbers are dense and appear "everywhere," measure theory reveals that they actually occupy "no space" on the number line. This section introduces Lebesgue measure and explains why countable sets have measure zero.

\subsection{Introduction to Lebesgue Measure}

\begin{definition}[Lebesgue Measure]
The \textbf{Lebesgue measure} of an interval $[a, b]$ (where $a \leq b$) is defined as its length:
\[
\mu([a, b]) = b - a
\]

For more general sets, the Lebesgue measure extends this notion of "length" in a consistent way.
\end{definition}

The Lebesgue measure generalizes our intuitive notion of length, area, and volume to more complicated sets. For intervals, it simply gives the length, but it can also measure sets that are not intervals.

\begin{example}[Riemann Integral vs Lebesgue Measure]
To understand how Lebesgue measure relates to integration, consider calculating the integral of the constant function $f(x) = 1$ from 0 to 1 using the traditional Riemann integral.

The Riemann integral partitions the \textit{domain} (the $x$-axis) into subintervals and approximates the area under the curve using rectangles. For $f(x) = 1$ on $[0, 1]$, we have:
\[
\int_0^1 1 \, dx = [x]_0^1 = 1 - 0 = 1
\]

The Riemann approach works by:
\begin{enumerate}
    \item Dividing the interval $[0, 1]$ into smaller subintervals
    \item Approximating the function value on each subinterval
    \item Summing the areas of rectangles: $\sum f(x_i) \cdot \Delta x_i$
    \item Taking the limit as the partition becomes finer
\end{enumerate}

The \textbf{Lebesgue integral} (which uses Lebesgue measure) takes a different approach:
\begin{enumerate}
    \item It partitions the \textit{range} (the $y$-axis) instead of the domain
    \item For each value $y$ in the range, it considers the set $\{x : f(x) \geq y\}$
    \item It uses the \textit{measure} of these sets (their "length") rather than just their endpoints
    \item For $f(x) = 1$, the set $\{x : 1 \geq y\}$ equals $[0, 1]$ when $y \leq 1$ (with measure 1), and is empty when $y > 1$ (with measure 0)
    \item The Lebesgue integral becomes: $\int_0^1 1 \, d\mu = \mu([0, 1]) = 1$
\end{enumerate}

For this simple function, both methods give the same result ($1$), but the Lebesgue approach is more powerful because:
\begin{itemize}
    \item It can integrate functions that Riemann integration cannot handle (e.g., functions with many discontinuities)
    \item It provides better convergence theorems (e.g., the dominated convergence theorem)
    \item It naturally extends to higher dimensions and more abstract spaces
\end{itemize}

The key insight is that Lebesgue measure allows us to measure the "size" of sets (like $\{x : f(x) \geq y\}$) even when they are not simple intervals, which makes the Lebesgue integral more general and powerful than the Riemann integral.
\end{example}

\begin{theorem}[Measure of a Single Point]
A single point has Lebesgue measure zero. That is, for any $x \in \mathbb{R}$:
\[
\mu(\{x\}) = 0
\end{theorem}

\begin{proof}
For any $\epsilon > 0$, the point $\{x\}$ is contained in the interval $[x - \frac{\epsilon}{2}, x + \frac{\epsilon}{2}]$, which has measure $\epsilon$. Since we can make $\epsilon$ arbitrarily small, the measure of $\{x\}$ must be zero.
\end{proof}

\begin{figure}[h]
\centering
\begin{tikzpicture}[scale=1.2]
    % Number line
    \draw[thick, ->] (0,0) -- (8,0);
    
    % Point
    \draw[fill=bookred] (4, 0) circle (2pt);
    \node[above] at (4, 0.2) {$x$};
    
    % Interval around point
    \draw[thick, bookpurple, <->] (3.5, -0.4) -- (4.5, -0.4);
    \node[left] at (3.5, -0.4) {$\epsilon$};
    
    % Shrinking intervals
    \draw[thick, bookpurple!70, <->] (3.75, -0.8) -- (4.25, -0.8);
    \node[left] at (3.75, -0.8) {$\epsilon/2$};
    
    \draw[thick, bookpurple!50, <->] (3.875, -1.2) -- (4.125, -1.2);
    \node[left] at (3.875, -1.2) {$\epsilon/4$};
    
    \node[below] at (4, -1.7) {As $\epsilon \to 0$, measure $\to 0$};
\end{tikzpicture}
\caption{Visualization that a single point has measure zero}
\end{figure}

\subsection{Countable Sets Have Measure Zero}

The key result connecting countability to measure is the following theorem.

\begin{theorem}[Countable Sets Have Measure Zero]
If $A$ is a countable set, then $\mu(A) = 0$.
\end{theorem}

\begin{proof}
Since $A$ is countable, we can list its elements: $A = \{a_1, a_2, a_3, \ldots\}$.

For any $\epsilon > 0$, cover each point $a_i$ with an interval of length $\frac{\epsilon}{2^i}$:
\[
I_i = \left[a_i - \frac{\epsilon}{2^{i+1}}, a_i + \frac{\epsilon}{2^{i+1}}\right]
\]

The total measure of these intervals is:
\[
\sum_{i=1}^{\infty} \frac{\epsilon}{2^i} = \epsilon \sum_{i=1}^{\infty} \frac{1}{2^i} = \epsilon \cdot 1 = \epsilon
\]

Since $A \subseteq \bigcup_{i=1}^{\infty} I_i$ and we can make $\epsilon$ arbitrarily small, we conclude that $\mu(A) = 0$.
\end{proof}

\begin{figure}[h]
\centering
\begin{tikzpicture}[scale=1]
    % Number line
    \draw[thick, ->] (0,0) -- (10,0);
    
    % Rational points
    \foreach \x in {1, 2.5, 3.8, 5.2, 6.7, 8.3, 9.5} {
        \draw[fill=bookred] (\x, 0) circle (1.5pt);
    }
    
    % Intervals covering each point
    \foreach \x/\i in {1/1, 2.5/2, 3.8/3, 5.2/4, 6.7/5, 8.3/6, 9.5/7} {
        \draw[thick, bookpurple] (\x - 0.15/\i, -0.3) -- (\x + 0.15/\i, -0.3);
        \draw[thick, bookpurple] (\x - 0.15/\i, -0.3) -- (\x - 0.15/\i, -0.25);
        \draw[thick, bookpurple] (\x + 0.15/\i, -0.3) -- (\x + 0.15/\i, -0.25);
    }
    
    \node[below] at (5, -0.5) {Each rational covered by interval of length $\epsilon/2^i$};
    \node[below] at (5, -0.8) {Total length = $\epsilon \sum_{i=1}^{\infty} 1/2^i = \epsilon$};
\end{tikzpicture}
\caption{Visual representation of covering countable set with intervals of decreasing size}
\end{figure}

\subsection{The Rational Numbers Have Measure Zero}

\begin{corollary}[Rational Numbers Have Measure Zero]
The set of rational numbers $\mathbb{Q}$ has Lebesgue measure zero:
\[
\mu(\mathbb{Q}) = 0
\end{corollary}

\begin{proof}
Since $\mathbb{Q}$ is countable (as shown in Section 3), the theorem immediately implies that $\mu(\mathbb{Q}) = 0$.
\end{proof}

This is the "nowhere" part of our paradox! Even though rational numbers are dense (everywhere), they take up zero total "space" on the number line.

\begin{figure}[h]
\centering
\begin{tikzpicture}[scale=1]
    % Number line
    \draw[thick, ->] (0,0) -- (10,0);
    \node[below] at (0,0) {$0$};
    \node[below] at (10,0) {$1$};
    
    % Mark many rational points
    \foreach \x in {0.1, 0.2, 0.3, 0.4, 0.5, 0.6, 0.7, 0.8, 0.9, 1.0, 1.1, 1.2, 1.3, 1.4, 1.5, 1.6, 1.7, 1.8, 1.9, 2.0, 2.1, 2.2, 2.3, 2.4, 2.5, 2.6, 2.7, 2.8, 2.9, 3.0, 3.1, 3.2, 3.3, 3.4, 3.5, 3.6, 3.7, 3.8, 3.9, 4.0, 4.1, 4.2, 4.3, 4.4, 4.5, 4.6, 4.7, 4.8, 4.9, 5.0, 5.1, 5.2, 5.3, 5.4, 5.5, 5.6, 5.7, 5.8, 5.9, 6.0, 6.1, 6.2, 6.3, 6.4, 6.5, 6.6, 6.7, 6.8, 6.9, 7.0, 7.1, 7.2, 7.3, 7.4, 7.5, 7.6, 7.7, 7.8, 7.9, 8.0, 8.1, 8.2, 8.3, 8.4, 8.5, 8.6, 8.7, 8.8, 8.9, 9.0, 9.1, 9.2, 9.3, 9.4, 9.5, 9.6, 9.7, 9.8, 9.9} {
        \draw[fill=bookred] (\x, 0) circle (0.5pt);
    }
    
    % Label showing measure
    \node[above] at (5, 0.5) {Infinitely many rational points};
    \node[above] at (5, 0.8) {But total measure = 0};
    
    % Measure indicator
    \draw[thick, bookpurple, <->] (0, -0.5) -- (10, -0.5);
    \node[below] at (5, -0.5) {Measure of [0,10] = 10};
    \node[below] at (5, -0.8) {Measure of $\mathbb{Q} \cap [0,10]$ = 0};
\end{tikzpicture}
\caption{Number line showing rationals as "dust" with zero total length}
\end{figure}

\subsection{Properties of Measure Zero}

\begin{theorem}[Properties of Measure Zero]
\begin{enumerate}
    \item If $A \subseteq B$ and $\mu(B) = 0$, then $\mu(A) = 0$.
    \item If $A_1, A_2, A_3, \ldots$ are sets with measure zero, then $\bigcup_{i=1}^{\infty} A_i$ also has measure zero.
    \item Any finite set has measure zero.
\end{enumerate}
\end{theorem}

These properties help us understand why countable unions of measure zero sets (like individual rational numbers) still have measure zero.

\begin{remark}
The concept of measure zero is fundamental in analysis and probability theory. Sets of measure zero are often called "negligible" because they don't affect integrals or probabilities in most practical situations.
\end{remark}

