\section{Introduction to Rational Numbers}

\subsection{What Are Rational Numbers?}

\begin{definition}[Rational Numbers]
A \textbf{rational number} is any number that can be expressed as the ratio of two integers, where the denominator is not zero. Formally, the set of rational numbers is:
\[
\mathbb{Q} = \left\{ \frac{p}{q} : p, q \in \mathbb{Z}, q \neq 0 \right\}
\]
where $\mathbb{Z}$ denotes the set of integers.
\end{definition}

Rational numbers include familiar fractions like $\frac{1}{2}$, $\frac{3}{4}$, and $\frac{22}{7}$, as well as integers (which can be written as $\frac{n}{1}$) and terminating or repeating decimals (e.g., $0.5 = \frac{1}{2}$, $0.\overline{3} = \frac{1}{3}$).

\begin{example}
Examples of rational numbers:
\begin{itemize}
    \item $\frac{1}{2} = 0.5$ (terminating decimal)
    \item $\frac{1}{3} = 0.\overline{3}$ (repeating decimal)
    \item $\frac{22}{7} \approx 3.142857...$ (approximation of $\pi$)
    \item $-5 = \frac{-5}{1}$ (integer)
    \item $0 = \frac{0}{1}$ (zero)
\end{itemize}
\end{example}

\subsection{Historical Review: From Ancient Greek Mathematics to Modern Post-Quantum Computing}

The study of rational numbers has a rich history spanning millennia, from ancient mathematical discoveries to cutting-edge cryptographic applications.

\subsubsection{Ancient Greek Mathematics}

The ancient Greeks made fundamental contributions to number theory. Pythagoras and his followers discovered that not all numbers could be expressed as ratios of integers—the discovery of irrational numbers (like $\sqrt{2}$) was revolutionary and initially disturbing to their worldview. The Euclidean algorithm, developed around 300 BCE, provided a method to find the greatest common divisor of two integers, which is essential for working with rational numbers in their simplest form.

\subsubsection{Evolution of Number Systems}

The development of number systems followed a natural progression:

\begin{enumerate}
    \item \textbf{Natural Numbers ($\mathbb{N}$)}: The counting numbers $1, 2, 3, \ldots$ used by ancient civilizations for basic arithmetic and counting.
    
    \item \textbf{Integers ($\mathbb{Z}$)}: The extension to include zero and negative numbers, enabling subtraction without restrictions. This was formalized by mathematicians like Brahmagupta (7th century CE).
    
    \item \textbf{Rational Numbers ($\mathbb{Q}$)}: Fractions and ratios, allowing division to be meaningful for any pair of integers (except division by zero). The Greeks extensively studied these.
    
    \item \textbf{Real Numbers ($\mathbb{R}$)}: The completion of the rationals, including irrational numbers like $\sqrt{2}$, $\pi$, and $e$. This was rigorously developed in the 19th century by mathematicians like Dedekind and Cantor.
    
    \item \textbf{Complex Numbers ($\mathbb{C}$)}: Numbers of the form $a + bi$ where $i^2 = -1$, extending the reals to solve equations like $x^2 + 1 = 0$. First systematically studied by Cardano and Bombelli in the 16th century.
\end{enumerate}

\begin{figure}[h]
\centering
\begin{tikzpicture}[scale=0.8]
    % Timeline
    \draw[thick, ->] (0,0) -- (12,0);
    
    % Time periods
    \node[below] at (1,0) {Ancient};
    \node[below] at (4,0) {Medieval};
    \node[below] at (7,0) {Renaissance};
    \node[below] at (10,0) {Modern};
    
    % Number systems - boxes with text inside
    \node[draw, fill=bookpurple!30, minimum width=0.4cm, minimum height=0.6cm, rectangle] (N) at (0.5,0.5) {$\mathbb{N}$};
    \node[draw, fill=bookpurple!40, minimum width=0.4cm, minimum height=0.6cm, rectangle] (Z) at (2,0.5) {$\mathbb{Z}$};
    \node[draw, fill=bookpurple!50, minimum width=0.4cm, minimum height=0.6cm, rectangle] (Q) at (3.5,0.5) {$\mathbb{Q}$};
    \node[draw, fill=bookpurple!60, minimum width=0.4cm, minimum height=0.6cm, rectangle] (R) at (5.5,0.5) {$\mathbb{R}$};
    \node[draw, fill=bookpurple!70, minimum width=0.4cm, minimum height=0.6cm, rectangle] (C) at (7.5,0.5) {$\mathbb{C}$};
    
    % Title
    \node[below] at (6, -0.5) {Timeline of Number System Evolution};
    
    % Arrows showing progression
    \draw[->, thick, bookpurple] (N.east) -- (Z.west);
    \draw[->, thick, bookpurple] (Z.east) -- (Q.west);
    \draw[->, thick, bookpurple] (Q.east) -- (R.west);
    \draw[->, thick, bookpurple] (R.east) -- (C.west);
\end{tikzpicture}
\caption{Evolution of number systems from natural numbers to complex numbers}
\end{figure}

\subsubsection{Modern Applications: Post-Quantum Cryptography}

In the 21st century, rational numbers and their properties have found crucial applications in cryptography, particularly in the development of post-quantum cryptographic systems. Lattice-based cryptography, which relies on the hardness of problems involving rational approximations, is one of the leading candidates for secure communication in the quantum computing era.

The connection between rational numbers and cryptography stems from:
\begin{itemize}
    \item \textbf{Diophantine Approximation}: Finding rational numbers close to irrational numbers, which underlies lattice problems
    \item \textbf{Measure Zero Properties}: Understanding that "weak" cryptographic keys form a set of measure zero, ensuring security
    \item \textbf{Countability}: Distinguishing between countable and uncountable sets in security proofs
\end{itemize}

\subsection{Basic Properties of Rational Numbers}

\begin{theorem}[Closure Properties]
The set of rational numbers $\mathbb{Q}$ is closed under addition, subtraction, multiplication, and division (by non-zero rationals). That is, if $a, b \in \mathbb{Q}$ and $b \neq 0$, then:
\begin{align*}
a + b &\in \mathbb{Q} \\
a - b &\in \mathbb{Q} \\
a \cdot b &\in \mathbb{Q} \\
\frac{a}{b} &\in \mathbb{Q}
\end{align*}
\end{theorem}

\begin{proof}
If $a = \frac{p}{q}$ and $b = \frac{r}{s}$ where $p, q, r, s \in \mathbb{Z}$ and $q, s \neq 0$, then:
\begin{align*}
a + b &= \frac{p}{q} + \frac{r}{s} = \frac{ps + qr}{qs} \in \mathbb{Q} \\
a \cdot b &= \frac{p}{q} \cdot \frac{r}{s} = \frac{pr}{qs} \in \mathbb{Q}
\end{align*}
Similar arguments hold for subtraction and division.
\end{proof}

\begin{remark}
The closure properties ensure that rational numbers form a \textit{field}, making them suitable for algebraic operations without leaving the set. This property is fundamental to their use in cryptography, where operations must remain within well-defined mathematical structures.
\end{remark}

