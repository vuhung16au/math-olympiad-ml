\section{Countability}

While rational numbers are dense (appearing everywhere), they are fundamentally different from real numbers in terms of their "size" or cardinality. This section explores the concept of countability and shows that rational numbers, despite being infinite, are in a sense "smaller" than the real numbers.

\subsection{Countable Sets}

\begin{definition}[Countable Set]
A set $A$ is called \textbf{countable} if there exists a bijection (one-to-one correspondence) between $A$ and the set of natural numbers $\mathbb{N} = \{1, 2, 3, \ldots\}$. If a set is countable and infinite, we say it is \textbf{countably infinite}.
\end{definition}

In simpler terms, a countable set can be "listed" or enumerated—we can assign each element a natural number index, even if the set is infinite.

\begin{example}
The set of even positive integers $\{2, 4, 6, 8, \ldots\}$ is countable because we can list them as:
\begin{align*}
1 &\mapsto 2 \\
2 &\mapsto 4 \\
3 &\mapsto 6 \\
4 &\mapsto 8 \\
&\vdots
\end{align*}
The bijection is $f(n) = 2n$.
\end{example}

\subsection{Cantor's Enumeration of Rational Numbers}

The remarkable fact that rational numbers are countable was first proven by Georg Cantor in the 19th century. His proof uses a clever enumeration method.

\begin{theorem}[Countability of Rational Numbers]
The set of rational numbers $\mathbb{Q}$ is countably infinite.
\end{theorem}

\begin{proof}
We will show that the positive rationals are countable by arranging them in a grid and enumerating them systematically.

Consider the infinite grid where row $i$ contains all fractions with numerator $i$:
\[
\begin{array}{cccccc}
\frac{1}{1} & \frac{1}{2} & \frac{1}{3} & \frac{1}{4} & \frac{1}{5} & \cdots \\
\frac{2}{1} & \frac{2}{2} & \frac{2}{3} & \frac{2}{4} & \frac{2}{5} & \cdots \\
\frac{3}{1} & \frac{3}{2} & \frac{3}{3} & \frac{3}{4} & \frac{3}{5} & \cdots \\
\frac{4}{1} & \frac{4}{2} & \frac{4}{3} & \frac{4}{4} & \frac{4}{5} & \cdots \\
\vdots & \vdots & \vdots & \vdots & \vdots & \ddots
\end{array}
\]

We enumerate by following diagonal paths:
\begin{enumerate}
    \item Start with $\frac{1}{1}$ (position 1)
    \item Move diagonally: $\frac{1}{2}, \frac{2}{1}$ (positions 2, 3)
    \item Next diagonal: $\frac{1}{3}, \frac{2}{2}, \frac{3}{1}$ (positions 4, 5, 6)
    \item Continue this pattern...
\end{enumerate}

When we encounter a fraction that is not in lowest terms (like $\frac{2}{2} = 1$), we skip it since we've already counted its reduced form. This gives us an enumeration of all positive rationals.

To include negative rationals and zero, we can interleave: $0, \frac{1}{1}, -\frac{1}{1}, \frac{1}{2}, -\frac{1}{2}, \frac{2}{1}, -\frac{2}{1}, \ldots$

Therefore, $\mathbb{Q}$ is countable.
\end{proof}

\begin{figure}[h]
\centering
\begin{tikzpicture}[scale=0.7]
    % Grid
    \foreach \i in {1,...,5} {
        \foreach \j in {1,...,5} {
            \node[draw, rectangle, minimum size=0.8cm] (n\i-\j) at (\j, -\i+1) {$\frac{\i}{\j}$};
        }
    }
    
    % Diagonal enumeration path - following the correct sequence
    % Position 1: 1/1 at (1, 0)
    % Position 2: 1/2 at (2, 0)
    % Position 3: 2/1 at (1, -1)
    % Position 4: 1/3 at (3, 0)
    % Position 5: 2/2 at (2, -1)
    % Position 6: 3/1 at (1, -2)
    % Position 7: 1/4 at (4, 0)
    % Position 8: 2/3 at (3, -1)
    % Position 9: 3/2 at (2, -2)
    % Position 10: 4/1 at (1, -3)
    
    % Draw arrows following the diagonal pattern
    \draw[thick, bookred, ->] (1, 0) -- (2, 0);  % 1/1 -> 1/2
    \draw[thick, bookred, ->] (2, 0) -- (1, -1);  % 1/2 -> 2/1
    \draw[thick, bookred, ->] (1, -1) -- (3, 0);  % 2/1 -> 1/3
    \draw[thick, bookred, ->] (3, 0) -- (2, -1);  % 1/3 -> 2/2
    \draw[thick, bookred, ->] (2, -1) -- (1, -2); % 2/2 -> 3/1
    \draw[thick, bookred, ->] (1, -2) -- (4, 0);  % 3/1 -> 1/4
    \draw[thick, bookred, ->] (4, 0) -- (3, -1);  % 1/4 -> 2/3
    \draw[thick, bookred, ->] (3, -1) -- (2, -2); % 2/3 -> 3/2
    \draw[thick, bookred, ->] (2, -2) -- (1, -3); % 3/2 -> 4/1
    
    % Labels showing enumeration order
    \node[above] at (1, 0.5) {\small 1};
    \node[above] at (2, 0.5) {\small 2};
    \node[above] at (1, -0.5) {\small 3};
    \node[above] at (3, 0.5) {\small 4};
    \node[above] at (2, -0.5) {\small 5};
    \node[above] at (1, -1.5) {\small 6};
    \node[above] at (4, 0.5) {\small 7};
    \node[above] at (3, -0.5) {\small 8};
    \node[above] at (2, -1.5) {\small 9};
    \node[above] at (1, -2.5) {\small 10};
    
    \node[below] at (3, -4.5) {Cantor's diagonal enumeration of rational numbers};
\end{tikzpicture}
\caption{Grid showing Cantor's enumeration method for rational numbers}
\end{figure}

\subsection{Uncountability of Real Numbers}

In contrast to rational numbers, the real numbers are \textit{uncountable}—they cannot be put into a one-to-one correspondence with the natural numbers.

\begin{theorem}[Uncountability of Real Numbers]
The set of real numbers $\mathbb{R}$ is uncountable.
\end{theorem}

\begin{proof}[Cantor's Diagonal Argument]
Assume for contradiction that $\mathbb{R}$ is countable. Then we can list all real numbers between 0 and 1:
\begin{align*}
r_1 &= 0.d_{11}d_{12}d_{13}d_{14}\ldots \\
r_2 &= 0.d_{21}d_{22}d_{23}d_{24}\ldots \\
r_3 &= 0.d_{31}d_{32}d_{33}d_{34}\ldots \\
r_4 &= 0.d_{41}d_{42}d_{43}d_{44}\ldots \\
&\vdots
\end{align*}
where each $d_{ij}$ is a digit.

Now construct a number $s = 0.s_1s_2s_3s_4\ldots$ where $s_i \neq d_{ii}$ (and $s_i \neq 9$ to avoid the $0.999\ldots = 1$ issue). Then $s$ differs from every $r_i$ in at least one decimal place, so $s$ is not in our list—a contradiction.

Therefore, $\mathbb{R}$ is uncountable.
\end{proof}

\begin{figure}[h]
\centering
\begin{tikzpicture}[scale=1]
    % Countable set visualization
    \draw[fill=bookpurple!30, rounded corners] (0, 0) rectangle (3, 4);
    \node[above] at (1.5, 4) {Countable Sets};
    \node at (1.5, 3) {$\mathbb{N}$};
    \node at (1.5, 2) {$\mathbb{Z}$};
    \node at (1.5, 1) {$\mathbb{Q}$};
    \node at (1.5, 0.5) {Can be listed};
    
    % Uncountable set visualization
    \draw[fill=bookred!30, rounded corners] (5, 0) rectangle (8, 4);
    \node[above] at (6.5, 4) {Uncountable Sets};
    \node at (6.5, 3) {$\mathbb{R}$};
    \node at (6.5, 2) {$\mathbb{C}$};
    \node at (6.5, 1) {Power set of $\mathbb{N}$};
    \node at (6.5, 0.5) {Cannot be listed};
    
    % Arrow
    \draw[->, thick, bookpurple] (3.2, 2) -- (4.8, 2);
    \node[above] at (4, 2.3) {Much larger!};
\end{tikzpicture}
\caption{Comparison of countable and uncountable sets}
\end{figure}

\subsection{Paradoxes of Infinite Sets}

The concept of infinity leads to many counterintuitive results that challenge our everyday intuition. These paradoxes help illustrate the strange and fascinating properties of infinite sets, including the countably infinite set of rational numbers.

\begin{example}[Hilbert's Grand Hotel]
Imagine a hotel with infinitely many rooms, numbered $1, 2, 3, \ldots$, and suppose every room is occupied. One might think the hotel is full and cannot accommodate any new guests. However, this is not the case!

When a new guest arrives, the hotel manager can accommodate them by asking each current guest in room $n$ to move to room $2n$. This frees up all the odd-numbered rooms ($1, 3, 5, 7, \ldots$), allowing the new guest to take room 1. In fact, the hotel can accommodate infinitely many new guests by having guest in room $n$ move to room $2n$, freeing up all odd-numbered rooms.

This paradox illustrates a fundamental property of countably infinite sets: they can be "rearranged" to accommodate additional elements while maintaining the same cardinality. Just as the rational numbers can be enumerated (listed) in different ways, the hotel's guests can be reassigned to different rooms. This property is unique to infinite sets—a finite hotel with all rooms occupied truly cannot accommodate new guests, but an infinite hotel always can.
\end{example}

\begin{figure}[h]
\centering
\begin{tikzpicture}[scale=0.9]
    % Before: All rooms occupied
    \begin{scope}[xshift=0cm]
        \node[above] at (2, 2.5) {\textbf{Before: All rooms occupied}};
        \foreach \i in {1,...,8} {
            \draw[fill=bookpurple!50, draw=bookpurple, thick] (\i*0.6, 0) rectangle (\i*0.6+0.5, 1);
            \node at (\i*0.6+0.25, 0.5) {\small \i};
            \draw[fill=bookpurple] (\i*0.6+0.25, 1.3) circle (2pt);
        }
    \end{scope}
    
    % Arrow
    \draw[->, thick, bookred] (5.2, 0.5) -- (6.2, 0.5);
    \node[above] at (5.7, 0.8) {Move guest};
    \node[above] at (5.7, 0.5) {from room $n$};
    \node[below] at (5.7, 0.2) {to room $2n$};
    
    % After: Guests in even rooms, odd rooms free
    \begin{scope}[xshift=7.5cm]
        \node[above] at (2, 2.5) {\textbf{After: Odd rooms free}};
        \foreach \i in {1,3,5,7} {
            \draw[fill=bookred!30, draw=bookred, thick, dashed] (\i*0.6, 0) rectangle (\i*0.6+0.5, 1);
            \node at (\i*0.6+0.25, 0.5) {\small \i};
        }
        \foreach \i in {2,4,6,8} {
            \draw[fill=bookpurple!50, draw=bookpurple, thick] (\i*0.6, 0) rectangle (\i*0.6+0.5, 1);
            \node at (\i*0.6+0.25, 0.5) {\small \i};
            \draw[fill=bookpurple] (\i*0.6+0.25, 1.3) circle (2pt);
        }
        \node[above] at (0.3, 1.3) {\small New};
        \draw[fill=bookred] (0.3, 1.3) circle (2pt);
    \end{scope}
\end{tikzpicture}
\caption{Hilbert's Grand Hotel: Rearranging guests to free up infinitely many rooms}
\end{figure}

\begin{example}[Russell's Paradox]
Consider the set $R$ of all sets that do not contain themselves as an element. Does $R$ contain itself?

If $R$ contains itself, then by definition it should not contain itself (since $R$ only contains sets that don't contain themselves). But if $R$ does not contain itself, then it should contain itself (since it's a set that doesn't contain itself). This creates an inescapable contradiction.

This paradox, discovered by Bertrand Russell in 1901, revealed a fundamental flaw in naive set theory and led to the development of axiomatic set theory (such as Zermelo-Fraenkel set theory with the Axiom of Choice, or ZFC). The resolution required restricting what can be considered a "set" and introducing careful axioms to prevent such self-referential contradictions.

While Russell's paradox doesn't directly involve rational numbers, it highlights the importance of rigorous foundations in set theory. Our understanding of number systems, including the rational numbers $\mathbb{Q}$, rests on these carefully constructed foundations. The fact that we can rigorously define and work with infinite sets like $\mathbb{Q}$ is a testament to the success of modern set theory in resolving such paradoxes.
\end{example}

\begin{figure}[h]
\centering
\begin{tikzpicture}[scale=1]
    % Set R box
    \node[draw, rectangle, fill=bookpurple!30, minimum width=4cm, minimum height=1.5cm, align=center] (R) at (0, 0) {
        \textbf{Set $R$}\\
        All sets that do not\\
        contain themselves
    };
    
    % Self-referential arrow
    \draw[->, thick, bookred, out=45, in=135, looseness=1.5] (R.north east) to node[above, align=center] {\textbf{?}\\Does $R$ contain\\itself?} (R.north west);
    
    % Contradiction branches
    \node[draw, rectangle, fill=bookred!30, minimum width=2.5cm, minimum height=1cm, align=center] (yes) at (-3, -2.5) {
        If YES:\\
        Contradiction!
    };
    \node[draw, rectangle, fill=bookred!30, minimum width=2.5cm, minimum height=1cm, align=center] (no) at (3, -2.5) {
        If NO:\\
        Contradiction!
    };
    
    \draw[->, thick, bookred] (R.south west) -- (yes.north);
    \draw[->, thick, bookred] (R.south east) -- (no.north);
    
    \node[below] at (0, -3.8) {Russell's Paradox: An inescapable logical contradiction};
\end{tikzpicture}
\caption{Russell's Paradox: The self-referential set that leads to contradiction}
\end{figure}

\begin{example}[Tristram Shandy's Paradox]
In Laurence Sterne's novel \textit{The Life and Opinions of Tristram Shandy, Gentleman}, the protagonist takes two years to write the history of the first two days of his life. He laments that he will never finish his autobiography because he cannot keep up with the accumulating events—as he writes about day $n$, days $n+1, n+2, \ldots$ continue to pass.

At first glance, this seems like an impossible task. However, if we assume Tristram lives forever and continues writing at the same pace (one day's history per year), he will eventually complete his autobiography! After year 1, he has written about day 1. After year 2, he has written about days 1 and 2. After year $n$, he has written about days 1 through $n$. Since there are infinitely many natural numbers, he can eventually cover all days of his infinite life.

This paradox relates to the nature of countably infinite processes. Just as we can enumerate the rational numbers (assigning each rational a natural number index), Tristram can enumerate the days of his life. The key insight is that with infinite time, a countably infinite task can be completed, even if the task grows over time. This mirrors how we can "complete" the enumeration of all rational numbers, despite there being infinitely many of them.
\end{example}

\begin{figure}[h]
\centering
\begin{tikzpicture}[scale=0.85]
    % Timeline axis
    \draw[thick, ->] (0, 0) -- (10, 0);
    \node[below] at (0, 0) {Year 1};
    \node[below] at (2.5, 0) {Year 2};
    \node[below] at (5, 0) {Year 3};
    \node[below] at (7.5, 0) {Year $n$};
    \node[below] at (10, 0) {$\ldots$};
    
    % Progress bars showing days written
    \draw[fill=bookpurple!50, draw=bookpurple, thick] (0, 0.5) rectangle (1, 1.5);
    \node[above] at (0.5, 1.5) {Day 1};
    \node[above] at (0.5, 1.8) {written};
    
    \draw[fill=bookpurple!50, draw=bookpurple, thick] (2.5, 0.5) rectangle (3.5, 1.5);
    \draw[fill=bookpurple!50, draw=bookpurple, thick] (2.5, 1.5) rectangle (3.5, 2.5);
    \node[above] at (3, 2.5) {Days 1-2};
    \node[above] at (3, 2.8) {written};
    
    \draw[fill=bookpurple!50, draw=bookpurple, thick] (5, 0.5) rectangle (6, 1.5);
    \draw[fill=bookpurple!50, draw=bookpurple, thick] (5, 1.5) rectangle (6, 2.5);
    \draw[fill=bookpurple!50, draw=bookpurple, thick] (5, 2.5) rectangle (6, 3.5);
    \node[above] at (5.5, 3.5) {Days 1-3};
    \node[above] at (5.5, 3.8) {written};
    
    \draw[fill=bookpurple!50, draw=bookpurple, thick] (7.5, 0.5) rectangle (8.5, 1.5);
    \draw[fill=bookpurple!50, draw=bookpurple, thick] (7.5, 1.5) rectangle (8.5, 2.5);
    \draw[fill=bookpurple!50, draw=bookpurple, thick] (7.5, 2.5) rectangle (8.5, 3.5);
    \draw[fill=bookpurple!50, draw=bookpurple, thick] (7.5, 3.5) rectangle (8.5, 4.5);
    \node[above] at (8, 4.5) {Days 1-$n$};
    \node[above] at (8, 4.8) {written};
    
    % Arrow showing progression
    \draw[->, thick, bookred] (1.2, 1) -- (2.3, 1);
    \draw[->, thick, bookred] (3.7, 1.5) -- (4.8, 1.5);
    \draw[->, thick, bookred] (6.2, 2) -- (7.3, 2);
    
    \node[below] at (5, -0.5) {With infinite time, all days will eventually be written};
\end{tikzpicture}
\caption{Tristram Shandy's Paradox: The infinite autobiography can be completed}
\end{figure}

\begin{remark}
These paradoxes demonstrate that infinite sets behave in ways that defy our finite intuition. The counterintuitive nature of infinity helps explain why the properties of rational numbers—being both dense (everywhere) and having measure zero (nowhere)—can seem paradoxical at first. Understanding these paradoxes of infinity provides valuable intuition for appreciating the deeper mathematical structures we explore in this document.
\end{remark}

\subsection{Implications of Countability}

The fact that rational numbers are countable while real numbers are uncountable has profound implications:

\begin{enumerate}
    \item \textbf{Size Difference}: Even though both sets are infinite, there are "more" real numbers than rational numbers in a precise mathematical sense.
    
    \item \textbf{Probability}: As we'll see in Section 6, if you pick a real number at random, the probability it's rational is 0, even though rationals are dense.
    
    \item \textbf{Cryptography}: The distinction between countable and uncountable sets is crucial in security proofs, where we need to show that "weak" keys form a negligible (measure zero) set.
\end{enumerate}

\begin{remark}
The countability of rational numbers, combined with their density, creates the paradox we explore in later sections: they are "everywhere" (dense) but "nowhere" (measure zero) at the same time.
\end{remark}

