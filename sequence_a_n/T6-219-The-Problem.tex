\documentclass{article}
\usepackage{amsmath}
\usepackage{amssymb}
\usepackage{geometry}
\usepackage{hyperref}
\geometry{a4paper, margin=1in}

\author{Prepared by Vu Hung Nguyen}
\date{November 2025}

\begin{document}

\noindent\textbf{Author:} Prepared by Vu Hung Nguyen \\
\textbf{Date:} November 2025 \\
\vspace{0.3cm}
\textbf{Links:}
\begin{itemize}
    \item \href{https://vuhung16au.github.io/}{https://vuhung16au.github.io/}
    \item \href{https://github.com/vuhung16au/}{https://github.com/vuhung16au/}
    \item \href{https://www.linkedin.com/in/nguyenvuhung/}{https://www.linkedin.com/in/nguyenvuhung/}
\end{itemize}

\vspace{0.5cm}

\section{Bài T6/219 (Vietnamese version)} 

Cho dãy số $\{a_n\}$ được xác định như sau:
$$ a_1 = 1, a_{n+1} = a_n + \frac{1}{\sqrt[3]{a_n}} \quad (n \ge 1) $$
Tìm tất cả các số thực $\alpha$ sao cho dãy $\{u_n\}$ xác định bởi
$$ u_n = \frac{a_n^\alpha}{n} \quad (n \ge 1) $$
hội tụ và giới hạn của nó khác không.

\subsection{Lesson T6/219 (English translation)}

Let the ordinal number $\{a_n\}$ be clearly defined as follows:
$$ a_1 = 1, a_{n+1} = an_n + \frac{1}{\sqrt[3]{a_n}} \quad (n \ge 1) $$
Find all real numbers $\alpha$ such that the sequence $\{u_n\}$ defined by
$$ u_n = \frac{a_n^\alpha}{n} \quad (n \ge 1) $$
converges and its limit is non-zero.

\subsubsection*{Lời giải.}
Từ cách xác định dãy $\{a_n\}$ dễ thấy
$$ a_n > 0, \forall n \ge 1 \quad \text{và} \quad a_n^4 > \left(\sqrt[4]{a_{n-1}^4} + \frac{4}{3}\right)^3, $$
$\forall n \ge 2$. Suy ra
$$ \sqrt[4]{a_n^4} > \sqrt[4]{a_{n-1}^4} + \frac{4}{3}, \quad \forall n \ge 2 \implies $$
$$ \sqrt[4]{a_n^4} > \frac{4}{3}(n-1), \quad \forall n \ge 2 \quad (1). $$

Mặt khác, từ công thức xác định dãy $\{a_n\}$ ta lại có:
\begin{align*} a_k &= \left(\sqrt[3]{a_{k-1}} + \frac{1}{3a_{k-1}}\right)^3 - \\
&- \left(\frac{1}{3\sqrt[3]{a_{k-1}}} + \frac{1}{27a_{k-1}^3}\right), \quad \forall k \ge 2 \end{align*}
$$\implies \sqrt[3]{a_k}^4 < \left(\sqrt[3]{a_{k-1}} + \frac{1}{3 a_{k-1}}\right)^4 = $$
$$ = \sqrt[3]{a_{k-1}} + \frac{4}{3} + \frac{2}{3\sqrt[3]{a_{k-1}^4}} + \frac{4}{27\sqrt[3]{a_{k-1}^8}} + \frac{1}{81a_{k-1}^4}, $$
$\forall k \ge 2$.


Do đó, với mỗi $n > 4$ ta đều có:
\begin{align*}
\sqrt[3]{a_n} &< 1 + \frac{4}{3} (n-1) + 
\frac{2}{3}\sum_{k=2}^{n} \frac{1}{\sqrt[3]{a_{k-1}^4}} + 
\frac{4}{27}\sum_{k=2}^{n} \frac{1}{\sqrt[3]{a_{k-1}^8}} + 
\frac{1}{81}\sum_{k=2}^{n} \frac{1}{a_{k-1}^4} \quad (2)
\end{align*}

Dựa vào $(1)$ và dựa vào bất đẳng thức Bunhiacopxki ta sẽ được:
\begin{itemize}
    \item $ \sum_{k=2}^{n} \frac{1}{\sqrt[3]{a_{k-1}^4}} = 1 + \sum_{k=3}^{n} \frac{1}{\sqrt[3]{a_{k-1}^4}} $
    $$ < 1 + \frac{3}{4}\sum_{k=3}^{n} \frac{1}{(k-2)} < 1 + \frac{3}{4}  \sqrt{\sum_{k=3}^{n} (n-2) \frac{1}{(k-2)^2}} $$
    $$ < 1 + \frac{3}{4} \sqrt{(n-2)(1 + \sum_{k=4}^{n} \frac{1}{(k-3)(k-2)})} $$
    $$ = 1 + \frac{3}{4} \sqrt{(n-2)} \sqrt{(2 - \frac{1}{n-2})} < 1 + \frac{3}{4} \sqrt{2(n-2)} \quad (3) $$

    \item $ \sum_{k=2}^{n} \frac{1}{\sqrt[3]{a_{k-1}^8}} = 1 + \sum_{k=3}^{n} \frac{1}{\sqrt[3]{a_{k-1}^2}} $
    $$ < 1 + \frac{9}{16} \sum_{k=3}^{n} \frac{1}{(k-2)^2} < 1 + \frac{9}{16} = \frac{17}{8} \quad (4) $$

    \item $ \sum_{k=2}^{n} \frac{1}{a_{k-1}^4} = 1 + \sum_{k=3}^{n} \frac{1}{(\sqrt[3]{a_{k-1}^4})^3} $
    $$ < 1 + \frac{27}{64} \sum_{k=3}^{n} \frac{1}{(k-2)^3} < 1 + \frac{27}{64} \sum_{k=3}^{n} \frac{1}{(k-2)^2} < $$
    $$ < 1 + \frac{27}{64} \cdot \frac{32}{3} = 1 + \frac{27}{59} \quad (5) $$
\end{itemize}

Từ $(2)$, $(3)$, $(4)$, $(5)$ ta được:
$$ \sqrt[3]{a_n} < 1 + \frac{4}{3} \sqrt{2(n-2)} + \frac{35}{54n} + \frac{1}{8} \cdot \frac{59}{32} \quad (6) $$
$\forall n > 4$.

Từ $(1)$ và $(6)$ suy ra:
$$ \frac{4}{3} \left(1 - \frac{1}{n}\right) < \frac{a_n^{4/3}}{n} < \frac{4}{3} \cdot \frac{1}{n} + \frac{2\sqrt{2(n-2)}}{n} + \frac{35}{54n} + \frac{1}{8} \cdot \frac{59}{32n}, \quad \forall n > 4 \quad (7) $$

Vì $\lim_{n \to \infty} \frac{4}{3} \left(1 - \frac{1}{n}\right) = \frac{4}{3}$
$$ \lim_{n \to \infty} \left(\frac{4}{3n} + \frac{2\sqrt{2(n-2)}}{n} + \frac{35}{54n} + \frac{1}{8} \frac{59}{32n}\right) = \lim_{n \to \infty} \frac{2\sqrt{2n}}{n} = 0 $$
nên từ $(7)$ suy ra $\lim_{n \to \infty} \frac{\sqrt[4]{a_n}}{n} = \frac{4}{3}$.
Như vậy $\alpha = \frac{4}{3}$ là một giá trị cần tìm.

%%%%%%%%%%%%%%%%%%%%%%%%%%%%%%%%%%%%%%%%%%%%%%%%%%%%%%%%%%%%%%%%%%%%%%%%%%%%%%%%


Do $\lim_{n \to \infty} a_n = +\infty$ suy ra từ (1) nên:
$$ \lim_{n \to \infty} a_n^{\alpha - \frac{4}{3}} = \begin{cases} +\infty & \text{nếu } \alpha > \frac{4}{3} \\ 0 & \text{nếu } \alpha < \frac{4}{3} \end{cases} $$
Kết hợp với $u_n = \frac{a_n^\alpha}{n} = \frac{a_n^{4/3}}{n} \cdot a_n^{\alpha - \frac{4}{3}}$ suy ra:
$$ \lim_{n \to \infty} u_n = \lim_{n \to \infty} \frac{a_n^\alpha}{n} = \begin{cases} +\infty & \text{nếu } \alpha > \frac{4}{3} \\ 0 & \text{nếu } \alpha < \frac{4}{3} \end{cases} $$
Vậy $\alpha = \frac{4}{3}$ là giá trị duy nhất để dãy $\{u_n\}$ hội tụ và giới hạn của nó khác không.

\subsubsection*{Nhận xét:}
1. Có 8 bạn gửi lời giải cho bài toán. 
Trong số đó chỉ có các bạn: Ngô Đức Duy (12CT THPT Trần Phú - Hải Phòng), 
\textbf{Nguyễn Vũ Hưng} (12D Chuyên ngữ ĐHQG Hà Nội) và 
Lê Anh Vũ (12CT Quốc học Huế) có lời giải đúng. 
Bạn Vũ cho lời giải khá phức tạp; 
bạn Hưng phải dùng tới các kiến thức vượt ra ngoài chương trình THPT để giải bài toán.

2. Bạn \textbf{Ngô Đức Duy} đã đề xuất và giải tốt Bài toán khái quát sau: 
"Cho số thực dương $a$ và số thực âm $b$. Cho dãy số $\{a_n\}$ được xác định bởi
$$ a_0 = a, a_{n+1} = a_n + a_n^b, \quad \forall n \ge 0. $$
Tìm tất cả các số thực $\alpha$ sao cho dãy $\{u_n\}$ xác định bởi
$$ u_n = \frac{a_n^\alpha}{n}, \quad \forall n \ge 1 $$
hội tụ và giới hạn của nó khác không". (\text{Đáp số: } $\alpha = 1 - b$).

\hfill NGUYỄN KHẮC MINH

\section{Observation}

\begin{itemize}
    \item Looks like this is a hard problem for high school students.
    \item The idea in the solution above is using Talyor's expansion to approximate $a_n$, find a lower bound for $a_n^{4/3}$ and an upper bound for $a_n^{4/3}$, then use the squeeze theorem to find the limit.
    \item Only 8 students submitted solutions to the problem and only 3 of them have correct solutions.
    \item The solution of Nguyen Vu Hung is quite complicated and uses knowledge beyond the standard high school curriculum. 
\end{itemize}


\section{Differential Equation Approach (by Nguyen Vu Hung)}
We approximate $a_n$ by values of a smooth function $f(x)$ at 
integer points, i.e., $a_n \approx f(n)$. 
For large $n$, the forward difference satisfies $a_{n+1} - a_n \approx f'(n)$.
(See the Mean Value Theorem in the Discussion section.)

% \emph{Proof sketch.} By the Mean Value Theorem, for each integer $n$ there exists $\zeta_n\in(n,n+1)$ such that 
% $$f(n+1)-f(n)=f'(\zeta_n).$$
% If $f'$ varies slowly on intervals of length $1$ (e.g., $f''$ is bounded along the trajectory), then $f'(\zeta_n)=f'(n)+O(\max_{t\in[n,n+1]}|f''(t)|)$, hence $f(n+1)-f(n)=f'(n)+o(1)$. Equivalently, the first-order Taylor expansion gives 
% $$f(n+1)-f(n)=f'(n)\cdot 1+\tfrac12 f''(\xi_n)\cdot 1^2,$$ 
% for some $\xi_n\in(n,n+1)$, so the remainder is controlled when $f''$ is moderate, justifying the approximation $a_{n+1}-a_n\approx f'(n)$.

From the recurrence $a_{n+1} - a_n = a_n^{-1/3}$, we obtain the separable Ordinary Differential Equation (ODE)

$$ f'(x) = f(x)^{-1/3}. $$
Separating variables and integrating gives
$$ \frac{df}{dx} = f^{-1/3} \quad \Rightarrow \quad f^{1/3}\, df = dx, $$
$$ \int f^{1/3}\, df = \int dx \quad \Rightarrow \quad \frac{f^{4/3}}{4/3} = x + C, $$
so
$$ f(x)^{4/3} = \frac{4}{3}x + C'. $$
As $x \to \infty$, the constant $C'$ is negligible in the asymptotic sense, hence
$$ f(x)^{4/3} \sim \frac{4}{3}x. $$
Consequently, for the sequence we have the approximation
$$ a_n^{4/3} \sim \frac{4}{3}n. $$

Now consider $u_n = \dfrac{a_n^{\alpha}}{n}$. Using $a_n^{4/3} \sim \dfrac{4}{3}n$,
$$ u_n = \frac{a_n^{\alpha}}{n} = \frac{\big(a_n^{4/3}\big)^{\alpha\cdot 3/4}}{n}
\sim \frac{\big(\tfrac{4}{3}n\big)^{\frac{3\alpha}{4}}}{n}
= \left(\frac{4}{3}\right)^{\frac{3\alpha}{4}} n^{\frac{3\alpha}{4}-1}. $$
For $u_n$ to converge to a nonzero limit, the exponent of $n$ must vanish, i.e.,
$$ \frac{3\alpha}{4} - 1 = 0 \quad \Longrightarrow \quad \alpha = \frac{4}{3}. $$
When $\alpha = \tfrac{4}{3}$, the asymptotic limit is
$$ \lim_{n\to\infty} u_n \sim \left(\frac{4}{3}\right)^{\frac{3(4/3)}{4}} n^{0} = \left(\frac{4}{3}\right)^{1} = \frac{4}{3}, $$
which is consistent with the rigorous solution above.

\section{The Difference Equations}
We formulate the problem purely in the language of difference equations. 
Consider the first-order non-linear difference equation
$$ a_{n+1} - a_n = a_n^{-1/3}, \quad n \ge 1, $$
subject to the initial condition
$$ a_1 = 1. $$
For a given real parameter $\alpha$, define
$$ u_n = \frac{a_n^\alpha}{n}. $$
% \textbf{Problem (difference-equation formulation).} Determine all real values of $\alpha$ such that the sequence $\{u_n\}_{n\ge 1}$ converges and its limit is nonzero. No solution method is prescribed here; this section only states the problem in difference-equation terms.

Let $f(n) = a_n$ be a function of $n$, which is a discrete sequence that models the growth of $a_n$ at integer points.
As $n$ is large, the forward difference satisfies $a_{n+1} - a_n \approx f'(n)$.

From the recurrence $a_{n+1} - a_n = a_n^{-1/3}$, we obtain the difference equation
$$ f'(n) = f(n)^{-1/3}. $$

\section{The Differential Equations}
We state a continuous analogue of the problem via a differential equation. Let $f\colon [1,\infty) \to (0,\infty)$ be a differentiable function that models the growth of $a_n$ at integer points, with the initial condition
$$ f(1) = 1, $$
and governed by the first-order ODE
$$ f'(x) = f(x)^{-1/3}. $$
For a given real parameter $\alpha$, introduce the continuous analogue of $u_n$ by
$$ v(x) = \frac{f(x)^\alpha}{x}. $$
% \textbf{Problem (differential-equation formulation).} Find those real numbers $\alpha$ for which $v(x)$ admits a finite, nonzero limit as $x \to \infty$. In this section we only pose the ODE formulation of the problem without providing a solution.

\section{Solution (using Stolz--Ces\'aro theorem)}
We recall a common form of the Stolz--Ces\'aro theorem: if $(A_n)$ and $(B_n)$ satisfy $B_n\nearrow \infty$ and the limit $\lim\limits_{n\to\infty}\dfrac{A_{n+1}-A_n}{B_{n+1}-B_n}=L$ exists, then $\lim\limits_{n\to\infty}\dfrac{A_n}{B_n}=L$.
Apply this with $A_n=a_n^{4/3}$ and $B_n=n$. Then
$$\lim_{n\to\infty}\frac{A_{n+1}-A_n}{B_{n+1}-B_n}=\lim_{n\to\infty}\big(a_{n+1}^{4/3}-a_n^{4/3}\big).$$
Using $a_{n+1}=a_n+a_n^{-1/3}$ and the binomial expansion (or MVT) for $(x+h)^{4/3}$ with $h=a_n^{-1/3}$,
$$a_{n+1}^{4/3}-a_n^{4/3}=\frac{4}{3}\,a_n^{1/3}\cdot a_n^{-1/3}+o(1)=\frac{4}{3}+o(1).$$
Hence the difference limit equals $\tfrac{4}{3}$, and by Stolz--Ces\'aro,
$$\lim_{n\to\infty}\frac{a_n^{4/3}}{n}=\frac{4}{3}.$$
This yields the unique exponent $\alpha=\tfrac{4}{3}$ for which $u_n=a_n^\alpha/n$ has a nonzero finite limit.

\section{Discussion}

\paragraph{Scaling and dominant balance.}
A quick scaling ansatz $a_n \sim c\,n^p$ balances $a_{n+1}-a_n \asymp n^{p-1}$ with $a_n^{-1/3} \asymp n^{-p/3}$, yielding $p=\tfrac{3}{4}$ and $c=(\tfrac{4}{3})^{3/4}$. This immediately suggests both the exponent and the constant in the limit $\lim n^{-1}a_n^{4/3}=\tfrac{4}{3}$.

\paragraph{Discrete vs. continuum.}
Replacing differences by derivatives (or sums by integrals) is justified here by monotonicity and smooth growth. A rigorous bridge uses Stolz–Cesàro or the mean value theorem on $b_n=a_n^{4/3}$ to squeeze $n(b_{n+1}-b_n)$ between two sequences tending to $\tfrac{4}{3}$.

\paragraph{Regular variation.}
The sequence is regularly varying with index $3/4$. Both the inequality proof and the ODE heuristic identify the same index and slowly varying constant, explaining why $u_n=a_n^\alpha/n$ has a nonzero limit iff $\alpha=\tfrac{4}{3}$.

\paragraph{Generalization.}
For $a_{n+1}=a_n+a_n^b$ with $b<0$, the ODE $f'=f^b$ gives $f^{1-b}\sim (1-b)x$, hence $a_n\sim \text{const}\cdot n^{1/(1-b)}$ and $u_n\sim n^{\alpha/(1-b)-1}$. The unique nonzero-limit threshold is $\alpha=1-b$, matching the proposed general answer.

\paragraph{Error terms and robustness.}
Let $b_n:=a_n^{4/3}$. By the mean value theorem,
$$b_{n+1}-b_n=\frac{4}{3}\,\xi_n^{1/3}\,(a_{n+1}-a_n),\quad \xi_n\in[a_n,a_{n+1}].$$
Using $a_{n+1}-a_n=a_n^{-1/3}$ and $\xi_n\asymp a_n$, we get
$$b_{n+1}-b_n=\frac{4}{3}+O\!\left(a_n^{-4/3}\right).$$
Summing yields the quantitative asymptotic
$$b_n=\frac{4}{3}\,n+O\!\left(\sum_{k\le n} a_k^{-4/3}\right),$$
so in particular $b_n=\tfrac{4}{3}n+O(\log n)$ once $a_n\asymp n^{3/4}$. Consequently,
$$\frac{a_n^{4/3}}{n}=\frac{4}{3}+O\!\left(\frac{\log n}{n}\right)\to\frac{4}{3}.$$
Moreover, for perturbed recurrences of the form
$$a_{n+1}=a_n+a_n^{-1/3}+\varepsilon_n,\qquad \varepsilon_n=o\!\left(a_n^{-1/3}\right),$$
exactly the same computation gives
$$b_{n+1}-b_n=\frac{4}{3}+o(1),\quad b_n=\frac{4}{3}\,n+o(n),$$
so the exponent $3/4$ and the limit $\lim n^{-1}a_n^{4/3}=\tfrac{4}{3}$ are stable under small perturbations.

\paragraph{Brief definitions and notation.}
\emph{Mean Value Theorem (MVT).} If $g$ is differentiable on $[x,y]$, then there exists $\xi\in(x,y)$ such that $g(y)-g(x)=g'(\xi)(y-x)$. In our use, $g(t)=t^{4/3}$, $x=a_n$, $y=a_{n+1}$, and $\xi_n$ denotes such an intermediate point.\newline
\emph{$\asymp$ notation.} For positive sequences $(f_n)$ and $(g_n)$, we write $f_n\asymp g_n$ if there exist constants $0<c\le C<\infty$ and $n_0$ such that $c\,g_n\le f_n\le C\,g_n$ for all $n\ge n_0$.

\paragraph{Note on the Stolz--Ces\'aro theorem (statement and sketch).}
If $(A_n)$ and $(B_n)$ satisfy $B_n\nearrow\infty$ and $\lim\limits_{n\to\infty}\dfrac{A_{n+1}-A_n}{B_{n+1}-B_n}=L$ exists, then $\lim\limits_{n\to\infty}\dfrac{A_n}{B_n}=L$. Sketch: write
$$\frac{A_n}{B_n}=\frac{\sum_{k=1}^{n-1}(A_{k+1}-A_k)}{\sum_{k=1}^{n-1}(B_{k+1}-B_k)}$$
and view it as a weighted average of the ratios $\dfrac{A_{k+1}-A_k}{B_{k+1}-B_k}$ with positive weights $B_{k+1}-B_k$. If these ratios converge to $L$ and the denominator diverges, the weighted average also converges to $L$ (a Ces\`aro-type argument). This justifies replacing a hard ratio limit by the simpler difference ratio limit.

\paragraph{Relevance to time series and machine learning.}
Difference equations are the backbone of many time-series models (e.g., AR, ARIMA, state-space recurrences, RNN updates). Our recurrence $a_{n+1}-a_n=a_n^{-1/3}$ is a nonlinear, state-dependent step size; analyzing its stability and asymptotics mirrors tasks in ML such as diagnosing exploding/vanishing dynamics, choosing scalings, and proving convergence of iterative training rules. The discrete-to-continuum translation (ODE surrogate) parallels common practice in optimization theory (gradient flow limits) and continuous-time modeling, while robustness results (persistence under small perturbations) echo stability under noise and model misspecification in real datasets.

\paragraph{Pedagogical note.}
The ODE/dominant-balance route offers intuition and a clean roadmap; the discrete inequalities provide full rigor. Presenting both helps students connect heuristic modeling with proof techniques.

\end{document}

