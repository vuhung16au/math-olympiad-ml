\begin{problem}[Region with Real and Argument Constraints]
Let $R$ be the region in the complex plane defined by $1 < \text{Re}(z) \le 3$ and $\frac{\pi}{6} \le \text{Arg}(z) < \frac{\pi}{3}$. Sketch this region.
\end{problem}

\begin{hint}
The region is bounded by two vertical lines ($x=1$ and $x=3$) and two rays from the origin at angles $\pi/6$ and $\pi/3$.
\end{hint}

\begin{solution}[Sketch]
Draw vertical lines at $x=1$ (dashed, excluded) and $x=3$ (solid, included). Draw rays from origin at angles $30^\circ$ (solid) and $60^\circ$ (dashed). The region is the intersection in the first quadrant.

\vspace{0.5em}
\begin{center}
\begin{tikzpicture}[scale=2.2,>=Stealth]
  % Draw axes
  \draw[->,thick] (-0.3,0) -- (4,0) node[right,font=\small] {Re};
  \draw[->,thick] (0,-0.3) -- (0,3.5) node[above,font=\small] {Im};
  
  % Define angles
  \def\angleMin{30}  % π/6 = 30°
  \def\angleMax{60}  % π/3 = 60°
  
  % Calculate the corner points of the region
  % Bottom edge: x=1, angle=30° → point (1, 1·tan(30°))
  \coordinate (BL) at (1,{1*tan(\angleMin)});
  % Top-left edge: x=1, angle=60° → point (1, 1·tan(60°))
  \coordinate (TL) at (1,{1*tan(\angleMax)});
  % Bottom-right edge: x=3, angle=30° → point (3, 3·tan(30°))
  \coordinate (BR) at (3,{3*tan(\angleMin)});
  % Top-right edge: x=3, angle=60° → point (3, 3·tan(60°))
  \coordinate (TR) at (3,{3*tan(\angleMax)});
  
  % Fill the region (quadrilateral)
  \fill[purple!25,opacity=0.8] (BL) -- (BR) -- (TR) -- (TL) -- cycle;
  
  % Draw the boundaries of the region
  % Bottom boundary: along ray at angle 30° from x=1 to x=3 (solid, included)
  \draw[very thick,bookpurple!80!black,solid] (BL) -- (BR);
  
  % Top boundary: along ray at angle 60° from x=1 to x=3 (dashed, excluded)
  \draw[very thick,lawpurple,dashed] (TL) -- (TR);
  
  % Left boundary: vertical line at x=1 from angle 30° to 60° (dashed, excluded)
  \draw[very thick,bookred,dashed] (BL) -- (TL);
  
  % Right boundary: vertical line at x=3 from angle 30° to 60° (solid, included)
  \draw[very thick,bookpurple,solid] (BR) -- (TR);
  
  % Draw the complete rays from origin (extended for clarity)
  % Ray at angle π/6 = 30° (solid)
  \draw[thick,bookpurple!80!black,solid] (0,0) -- ({\angleMin}:3.8);
  \node[bookpurple!80!black,font=\scriptsize,right] at ({\angleMin}:3.8) {$\text{Arg}(z)=\frac{\pi}{6}$};
  
  % Ray at angle π/3 = 60° (dashed)
  \draw[thick,lawpurple,dashed] (0,0) -- ({\angleMax}:3.6);
  \node[lawpurple,font=\scriptsize,above left] at ({\angleMax}:3.6) {$\text{Arg}(z)=\frac{\pi}{3}$};
  
  % Draw the complete vertical lines (extended for clarity)
  % x = 1 (dashed)
  \draw[thick,bookred,dashed] (1,0) -- (1,3.2);
  \node[bookred,font=\scriptsize,below] at (1,-0.05) {$\text{Re}(z)=1$};
  
  % x = 3 (solid)
  \draw[thick,bookpurple,solid] (3,0) -- (3,3.2);
  \node[bookpurple,font=\scriptsize,below] at (3,-0.05) {$\text{Re}(z)=3$};
  
  % Mark the corner points with appropriate symbols
  % Bottom-left: x=1 (excluded), angle=30° (included)
  \draw[fill=white,draw=bookred,thick] (BL) circle (1.5pt);
  
  % Top-left: x=1 (excluded), angle=60° (excluded)
  \draw[fill=white,draw=bookred,thick] (TL) circle (1.5pt);
  
  % Bottom-right: x=3 (included), angle=30° (included)
  \fill[bookpurple] (BR) circle (1.5pt);
  
  % Top-right: x=3 (included), angle=60° (excluded)
  \draw[fill=white,draw=bookpurple,thick] (TR) circle (1.5pt);
  
  % Label the region
  \node[font=\normalsize,deepcharcoal,align=center] at (2,1.6) {\textbf{Region $R$}};
  
  % Origin
  \fill (0,0) circle (1pt);
  \node[below left,font=\scriptsize] at (0,0) {$O$};
  
  % Add angle markers at origin
  \draw[bookpurple!80!black,thick,->] (0.5,0) arc (0:\angleMin:0.5);
  \node[bookpurple!80!black,font=\scriptsize] at (15:0.7) {$\frac{\pi}{6}$};
  
  \draw[lawpurple,thick,->] (0.7,0) arc (0:\angleMax:0.7);
  \node[lawpurple,font=\scriptsize] at (30:0.95) {$\frac{\pi}{3}$};
  
  % Add a box with the conditions
  \node[font=\scriptsize,align=left,draw=purple!50,fill=white,rounded corners] at (3.2,2.8) {
    Conditions:\\
    $1 < \text{Re}(z) \le 3$\\
    $\frac{\pi}{6} \le \text{Arg}(z) < \frac{\pi}{3}$
  };
  
  % Legend for boundary symbols
  \node[font=\tiny,align=left,draw=warmstone!30,fill=white] at (0.5,3.1) {
    $\circ$ excluded\\
    $\bullet$ included
  };
\end{tikzpicture}
\end{center}
\vspace{0.5em}

\end{solution}

\begin{problem}[Statement Analysis]
Which statement about complex numbers is true?

\begin{enumerate}[label=(\Alph*),leftmargin=*]
  \item For $z = x + iy$, the argument is given by $\arg(z) = \arctan(y/x)$ for all $z \neq 0$.
  \item For any two complex numbers $z_1$ and $z_2$, we have $\text{Arg}(z_1 z_2) = \text{Arg}(z_1) + \text{Arg}(z_2)$.
  \item If two complex numbers have the same modulus and their arguments satisfy $\theta_1 = \theta_2 + 2\pi k$ for some integer $k$, then they are equal (modulo $2\pi$).
  \item For $z = x + iy$ with $x > 0$, the argument is $\arg(z) = \arctan(y/x)$.
\end{enumerate}
\end{problem}

\begin{hint}
Check each statement. Remember that $\text{Arg}(z_1z_2) = \text{Arg}(z_1) + \text{Arg}(z_2)$ only modulo $2\pi$.
\end{hint}

\begin{solution}[Sketch]
\begin{itemize}[leftmargin=*]
  \item \textbf{(A) False:} The formula $\arg(z) = \arctan(y/x)$ doesn't account for the quadrant. For example, if $z = -1 + i$, then $\arctan(-1) = -\pi/4$, but the actual argument is $3\pi/4$ (in the second quadrant).
  
  \item \textbf{(B) False:} Principal argument addition requires modulo $2\pi$ adjustment. For instance, if $\text{Arg}(z_1) = 3\pi/4$ and $\text{Arg}(z_2) = 3\pi/4$, then $\text{Arg}(z_1) + \text{Arg}(z_2) = 3\pi/2$, but we need to adjust to the principal range $(-\pi, \pi]$, giving $\text{Arg}(z_1 z_2) = -\pi/2$.
  
  \item \textbf{(C) True (with proper interpretation):} If $|z_1| = |z_2|$ and $\arg(z_1) = \arg(z_2) + 2\pi k$ for integer $k$, then $z_1 = z_2$ since arguments that differ by $2\pi k$ represent the same direction.
  
  \item \textbf{(D) True (but limited):} This is correct when $x > 0$ (first and fourth quadrants), as the arctan function gives the correct argument in these quadrants without adjustment.
\end{itemize}

\textbf{Answer:} The best answer is \textbf{(D)} for its specific domain, though \textbf{(C)} is also correct if ``equality'' means modulo $2\pi$. Statement (D) is the most straightforward true statement.
\end{solution}

\begin{problem}[Roots of Unity Application]
Suppose that $x + \frac{1}{x} = -1$. What is the value of $x^{2028} + \frac{1}{x^{2028}}$?
\end{problem}

\begin{hint}
Solve $x^2 + x + 1 = 0$ to find $x = e^{i2\pi/3}$ or $x = e^{-i2\pi/3}$. These are cube roots of unity.
\end{hint}

\begin{solution}[Sketch]
$x^2+x+1=0$ gives $x = e^{\pm i2\pi/3}$. Since $x^3=1$, we have $x^{2028} = x^{3 \cdot 676} = 1$. Thus $x^{2028} + \frac{1}{x^{2028}} = 1 + 1 = 2$.
\end{solution}

\begin{problem}[Shading a Region]
On an Argand diagram, shade the region where $0 \le \text{Re}(z) \le 2$ and $|z - (1-i)| \le 2$ both hold.
\end{problem}

\begin{hint}
The first inequality is a vertical strip. The second is a disk centered at $(1,-1)$ with radius 2.
\end{hint}

\begin{solution}[Sketch]
Draw vertical lines at $x=0$ and $x=2$. Draw circle centered at $(1,-1)$ with radius 2. Shade the intersection.

\vspace{0.5em}
\begin{center}
\begin{tikzpicture}[scale=1.5,>=Stealth]
  % Draw axes
  \draw[->,thick] (-1,0) -- (3.5,0) node[right,font=\small] {Re};
  \draw[->,thick] (0,-3.5) -- (0,2) node[above,font=\small] {Im};
  
  % Center of circle
  \coordinate (C) at (1,-1);
  \def\radius{2}
  
  % Draw the vertical strip boundaries
  % x = 0 (imaginary axis)
  \draw[very thick,bookpurple] (0,-3.5) -- (0,2);
  \node[bookpurple,font=\scriptsize,above] at (0,1.8) {$\text{Re}(z)=0$};
  
  % x = 2
  \draw[very thick,bookpurple] (2,-3.5) -- (2,2);
  \node[bookpurple,font=\scriptsize,above] at (2,1.8) {$\text{Re}(z)=2$};
  
  % Draw the circle
  \draw[very thick,bookred] (C) circle (\radius);
  
  % Mark the center
  \fill[bookred] (C) circle (1.5pt);
  \node[bookred,font=\scriptsize,below right] at (C) {$(1,-1)$};
  
  % Draw radius indicator
  \draw[bookred,dashed,thin] (C) -- ($(C)+(0:\radius)$);
  \node[bookred,font=\tiny] at ($(C)+(0:1)$) [above] {$r=2$};
  
  % Shade the intersection region
  % The intersection is the part of the circle between x=0 and x=2
  \begin{scope}
    \clip (0,-3.5) rectangle (2,2);
    \fill[purple!30,opacity=0.7] (C) circle (\radius);
  \end{scope}
  
  % Draw the clipped boundary more prominently
  \begin{scope}
    \clip (0,-3.5) rectangle (2,2);
    \draw[very thick,lawpurple] (C) circle (\radius);
  \end{scope}
  
  % Add label for the region
  \node[font=\normalsize,deepcharcoal,align=center] at (1,0) {\textbf{Region}};
  
  % Mark key intersection points
  % Circle intersects x=0 at (0, -1±√3)
  % Calculate: (x-1)² + (y+1)² = 4, when x=0: 1 + (y+1)² = 4, so (y+1)² = 3, y = -1±√3
  \coordinate (L1) at (0,{-1+sqrt(3)});
  \coordinate (L2) at (0,{-1-sqrt(3)});
  \fill[bookpurple!80!black] (L1) circle (1.5pt);
  \fill[bookpurple!80!black] (L2) circle (1.5pt);
  
  % Circle intersects x=2 at (2, -1±√3)
  \coordinate (R1) at (2,{-1+sqrt(3)});
  \coordinate (R2) at (2,{-1-sqrt(3)});
  \fill[bookpurple!80!black] (R1) circle (1.5pt);
  \fill[bookpurple!80!black] (R2) circle (1.5pt);
  
  % Add grid for reference
  \foreach \x in {-1,1,2,3} {
    \draw[warmstone!30,very thin] (\x,-3.5) -- (\x,2);
  }
  \foreach \y in {-3,-2,-1,1} {
    \draw[warmstone!30,very thin] (-1,\y) -- (3.5,\y);
  }
  
  % Origin
  \fill (0,0) circle (1pt);
  \node[below left,font=\scriptsize] at (0,0) {$O$};
  
  % Add condition box
  \node[font=\scriptsize,align=left,draw=purple!50,fill=white,rounded corners] at (3,1.2) {
    Conditions:\\
    $0 \le \text{Re}(z) \le 2$\\
    $|z-(1-i)| \le 2$
  };
  
  % Label the circle
  \node[bookred,font=\scriptsize] at (2.8,-1) {Circle: $|z-(1-i)|=2$};
\end{tikzpicture}
\end{center}
\vspace{0.5em}

\end{solution}

\begin{problem}[Shifted Reciprocal Locus]
If $z$ is a complex number, describe the locus defined by $\Re\!\left(\frac{1}{z-1}\right) = 2$.
\end{problem}

\begin{hint}
Write $z = x + iy$ and set $w = z - 1 = (x-1) + iy$. Compute $\Re\!\left(\frac{1}{w}\right)$ by multiplying numerator and denominator by $\bar{w}$, then complete the square to reveal the circle; note the locus must pass through $z = 1$.
\end{hint}

\begin{solution}[Sketch]
\[
\Re\!\left(\frac{1}{z-1}\right) = \frac{x-1}{(x-1)^2 + y^2} = 2 \;\Rightarrow\; (x-1)^2 - \tfrac{1}{2}(x-1) + y^2 = 0
\]
Completing the square gives $\left(x - \tfrac{5}{4}\right)^2 + y^2 = \left(\tfrac{1}{4}\right)^2$, a circle centered at $(\tfrac{5}{4}, 0)$ with radius $\tfrac{1}{4}$ that passes through $z = 1$ and is tangent to the line $x = 1$.

\vspace{0.4em}
\begin{center}
\begin{tikzpicture}[scale=5,>=Stealth]
  \def\cx{1.25}
  \def\r{0.25}
  % Axes
  \draw[->,gray] (-0.05,0) -- (1.6,0) node[right,font=\scriptsize] {Re};
  \draw[->,gray] (\cx,-0.35) -- (\cx,0.35) node[above,font=\scriptsize] {Im};
  \draw[gray,thin] (0,0.3) -- (0,-0.3);
  % Circle
  \draw[very thick,bookpurple] (\cx,0) circle (\r);
  % Center
  \fill[bookred] (\cx,0) circle (0.5pt) node[above right,font=\scriptsize] {Center $(\tfrac{5}{4}, 0)$};
  % Point z=1
  \fill[bookpurple!80!black] (1,0) circle (0.5pt) node[below left,font=\scriptsize] {$1$};
  % Tick marks
  \draw[gray] (0.5,0) -- (0.5,0.035) node[below,font=\tiny] {$\tfrac{1}{2}$};
  \draw[gray] (1,0) -- (1,0.035);
  \draw[gray] (1.5,0) -- (1.5,0.035) node[below,font=\tiny] {$\tfrac{3}{2}$};
  % Highlight tangency line
  \draw[dashed,warmstone!70] (1,-0.32) -- (1,0.32);
  \node[warmstone!80!black,font=\scriptsize,below] at (1,-0.32) {$x=1$};
\end{tikzpicture}
\end{center}
\vspace{0.4em}

\textbf{Takeaways:}
\begin{itemize}[leftmargin=*]
  \item Translating by $1$ then inverting sends vertical lines to circles that pass through the translation point.
  \item $\Re\!\left(\frac{1}{z-a}\right) = k \ne 0$ traces a circle through $z = a$ with center shifted by $\frac{1}{2k}$ along the real axis from $a$.
  \item Here the circle has center $(\tfrac{5}{4}, 0)$, radius $\tfrac{1}{4}$, and is tangent to the line $x = 1$.
\end{itemize}
\end{solution}

\begin{problem}[Modulus-Argument Division]
Given $\frac{1+\sqrt{3}i}{1+i} = \frac{1+\sqrt{3}}{2} + \frac{\sqrt{3}-1}{2}i$, express $\frac{1+i\sqrt{3}}{1+i}$ in modulus-argument form by converting both numerator and denominator.
\end{problem}

\begin{hint}
$|1+i\sqrt{3}| = 2$, $\arg(1+i\sqrt{3}) = \pi/3$; $|1+i| = \sqrt{2}$, $\arg(1+i) = \pi/4$.
\end{hint}

\begin{solution}[Sketch]
$\frac{1+i\sqrt{3}}{1+i} = \frac{2}{\sqrt{2}}\left(\cos\left(\frac{\pi}{3}-\frac{\pi}{4}\right) + i\sin\left(\frac{\pi}{3}-\frac{\pi}{4}\right)\right) = \sqrt{2}\left(\cos\frac{\pi}{12} + i\sin\frac{\pi}{12}\right)$.
\end{solution}

\begin{problem}[Power in Polar Form]
Let $\beta = 1 - i\sqrt{3}$. Express $\beta^5$ in modulus-argument form.
\end{problem}

\begin{hint}
$|\beta| = 2$, $\arg(\beta) = -\pi/3$.
\end{hint}

\begin{solution}[Sketch]
$\beta = 2(\cos(-\pi/3) + i\sin(-\pi/3))$. Then $\beta^5 = 32(\cos(-5\pi/3) + i\sin(-5\pi/3)) = 32(\cos(\pi/3) + i\sin(\pi/3))$.
\end{solution}

\begin{problem}[Deriving Trigonometric Value]
Find the exact value of $\sin \frac{\pi}{12}$ using the angle subtraction formula. Note that $\frac{\pi}{12} = \frac{\pi}{3} - \frac{\pi}{4} = 60° - 45° = 15°$.
\end{problem}

\begin{hint}
Use the angle subtraction formula: $\sin(A - B) = \sin A \cos B - \cos A \sin B$.

Recall the exact values:
\begin{itemize}
\item $\sin \frac{\pi}{3} = \frac{\sqrt{3}}{2}$, $\cos \frac{\pi}{3} = \frac{1}{2}$
\item $\sin \frac{\pi}{4} = \frac{\sqrt{2}}{2}$, $\cos \frac{\pi}{4} = \frac{\sqrt{2}}{2}$
\end{itemize}
\end{hint}

\begin{solution}[Sketch]
We use the fact that $\frac{\pi}{12} = \frac{\pi}{3} - \frac{\pi}{4}$.

Applying the angle subtraction formula:
\begin{align*}
\sin\frac{\pi}{12} &= \sin\left(\frac{\pi}{3} - \frac{\pi}{4}\right)\\
&= \sin\frac{\pi}{3}\cos\frac{\pi}{4} - \cos\frac{\pi}{3}\sin\frac{\pi}{4}\\
&= \frac{\sqrt{3}}{2} \cdot \frac{\sqrt{2}}{2} - \frac{1}{2} \cdot \frac{\sqrt{2}}{2}\\
&= \frac{\sqrt{6}}{4} - \frac{\sqrt{2}}{4}\\
&= \frac{\sqrt{6}-\sqrt{2}}{4}
\end{align*}

\textbf{Alternative approach using complex numbers:}

Consider the division $\frac{1+i\sqrt{3}}{1+i}$. In modulus-argument form:
\begin{itemize}
\item Numerator: $|1+i\sqrt{3}| = 2$, $\arg(1+i\sqrt{3}) = \frac{\pi}{3}$
\item Denominator: $|1+i| = \sqrt{2}$, $\arg(1+i) = \frac{\pi}{4}$
\end{itemize}

Therefore:
$$\frac{1+i\sqrt{3}}{1+i} = \frac{2}{\sqrt{2}}\left(\cos\left(\frac{\pi}{3}-\frac{\pi}{4}\right) + i\sin\left(\frac{\pi}{3}-\frac{\pi}{4}\right)\right) = \sqrt{2}\left(\cos\frac{\pi}{12} + i\sin\frac{\pi}{12}\right)$$

Computing the division algebraically:
$$\frac{1+i\sqrt{3}}{1+i} = \frac{(1+i\sqrt{3})(1-i)}{(1+i)(1-i)} = \frac{1-i+i\sqrt{3}+\sqrt{3}}{2} = \frac{1+\sqrt{3}}{2} + i\frac{\sqrt{3}-1}{2}$$

Comparing imaginary parts: $\sin\frac{\pi}{12} = \frac{\sqrt{3}-1}{2\sqrt{2}} = \frac{\sqrt{6}-\sqrt{2}}{4}$.

\textbf{Answer:} $\sin \frac{\pi}{12} = \frac{\sqrt{6}-\sqrt{2}}{4}$
\end{solution}

\begin{problem}[Marking Rotated Point]
A point $P$ represents the complex number $z = 3 + 2i$ on the Argand diagram. Mark the point $R$ representing $iz$.
\end{problem}

\begin{hint}
Multiply by $i$ rotates $90^\circ$ counterclockwise. If $z = a + bi$, then $iz = i(a+bi) = ai + bi^2 = -b + ai$.
\end{hint}

\begin{solution}[Sketch]
If $z$ is at $(a,b)$, then $iz$ is at $(-b,a)$. Draw accordingly.

For $z = 3 + 2i$ at point $(3,2)$:
\[
iz = i(3+2i) = 3i + 2i^2 = 3i - 2 = -2 + 3i
\]
So $R$ is at $(-2, 3)$.

\vspace{0.5em}
\begin{center}
\begin{tikzpicture}[scale=1.5,>=Stealth]
  % Draw axes
  \draw[->,thick] (-3,0) -- (4,0) node[right,font=\small] {Re};
  \draw[->,thick] (0,-1) -- (0,4) node[above,font=\small] {Im};
  
  % Original point z = 3 + 2i
  \coordinate (Z) at (3,2);
  \draw[->,very thick,bookpurple] (0,0) -- (Z);
  \fill[bookpurple] (Z) circle (2pt);
  \node[bookpurple,font=\small,right] at (Z) {$P: z = 3+2i$};
  
  % Rotated point iz = -2 + 3i
  \coordinate (IZ) at (-2,3);
  \draw[->,very thick,bookred] (0,0) -- (IZ);
  \fill[bookred] (IZ) circle (2pt);
  \node[bookred,font=\small,left] at (IZ) {$R: iz = -2+3i$};
  
  % Draw the rotation arc
  \draw[->,lawpurple,very thick] (Z) to[bend left=30] node[midway,above right,font=\scriptsize] {$\times i$} (IZ);
  
  % Show 90° rotation angle at origin
  \draw[bookpurple!80!black,thick] (0.8,0) arc (0:90:0.8);
  \node[bookpurple!80!black,font=\scriptsize] at (45:1.1) {$90°$};
  
  % Draw circles showing equal moduli
  \pgfmathsetmacro{\modulus}{sqrt(3*3 + 2*2)}
  \draw[dashed,warmstone!40,thin] (0,0) circle (\modulus);
  
  % Mark coordinates
  \node[bookpurple,font=\tiny,below] at (Z) {$(3,2)$};
  \node[bookred,font=\tiny,above left] at (IZ) {$(-2,3)$};
  
  % Draw projections (dashed lines)
  \draw[dashed,bookpurple!30,thin] (Z) -- (3,0) node[below,font=\tiny] {$3$};
  \draw[dashed,bookpurple!30,thin] (Z) -- (0,2) node[left,font=\tiny] {$2$};
  \draw[dashed,bookred!30,thin] (IZ) -- (-2,0) node[below,font=\tiny] {$-2$};
  \draw[dashed,bookred!30,thin] (IZ) -- (0,3) node[right,font=\tiny] {$3$};
  
  % Origin
  \fill (0,0) circle (1pt);
  \node[below left,font=\scriptsize] at (0,0) {$O$};
  
  % Add annotation box
  \node[font=\scriptsize,align=left,draw=purple!50,fill=white,rounded corners] at (3,3.5) {
    \textbf{Multiplication by $i$:}\\
    $z = a+bi \to iz = -b+ai$\\
    Rotates $90°$ CCW\\
    Preserves modulus
  };
  
  % Grid for reference
  \foreach \x in {-2,-1,1,2,3} {
    \draw[warmstone!15,very thin] (\x,-1) -- (\x,4);
  }
  \foreach \y in {1,2,3} {
    \draw[warmstone!15,very thin] (-3,\y) -- (4,\y);
  }
\end{tikzpicture}
\end{center}
\vspace{0.5em}

\textbf{Key observation:} Multiplying by $i$ rotates any complex number by $90°$ counterclockwise while keeping the same distance from the origin.

\end{solution}

\begin{problem}[Conjugate Root Property]
$2+i$ is a root of $P(z) = z^3 + rz^2 + sz + 20$ where $r,s$ are real. State why $2-i$ is also a root.
\end{problem}

\begin{hint}
Polynomials with real coefficients have complex roots in conjugate pairs.
\end{hint}

\begin{solution}[Sketch]
Since $P(z)$ has real coefficients, if $2+i$ is a root, then $\overline{2+i} = 2-i$ must also be a root.
\end{solution}

\begin{problem}[Factorizing Over Reals]
Factorise $P(z) = z^3 + rz^2 + sz + 20$ over the real numbers, given roots $2+i$ and $2-i$.
\end{problem}

\begin{hint}
$(z-(2+i))(z-(2-i)) = (z-2)^2 - (i)^2 = z^2 - 4z + 5$.
\end{hint}

\begin{solution}[Sketch]
$(z-2-i)(z-2+i) = z^2-4z+5$. Divide $P(z)$ by $z^2-4z+5$ to find the third factor. $P(z) = (z^2-4z+5)(z+4)$ after polynomial division.
\end{solution}

\begin{problem}[Vector Addition Point]
Points $P$ and $Q$ represent the complex numbers $z = 2 + i$ and $w = 1 + 3i$ respectively on the Argand diagram. Mark point $T$ representing $z+w$.
\end{problem}

\begin{hint}
Use parallelogram law: $T$ is the fourth vertex of parallelogram $OPTQ$ where $O$ is the origin.
\end{hint}

\begin{solution}[Sketch]
From $O$ to $P$ is vector $z$, from $O$ to $Q$ is vector $w$. The sum $z+w$ is at the opposite corner of the parallelogram.

For $z = 2 + i$ and $w = 1 + 3i$:
\[
z + w = (2+i) + (1+3i) = 3 + 4i
\]
So $T$ is at $(3, 4)$.

\vspace{0.5em}
\begin{center}
\begin{tikzpicture}[scale=1.4,>=Stealth]
  % Draw axes
  \draw[->,thick] (-0.5,0) -- (4,0) node[right,font=\small] {Re};
  \draw[->,thick] (0,-0.5) -- (0,4.8) node[above,font=\small] {Im};
  
  % Define points
  \coordinate (O) at (0,0);
  \coordinate (P) at (2,1);
  \coordinate (Q) at (1,3);
  \coordinate (T) at (3,4);
  
  % Draw the parallelogram
  \draw[thick,orange!50,dashed] (O) -- (P) -- (T) -- (Q) -- cycle;
  
  % Draw vectors from origin
  \draw[->,very thick,bookpurple] (O) -- (P) node[midway,below right,font=\scriptsize] {$z$};
  \draw[->,very thick,bookpurple!80!black] (O) -- (Q) node[midway,left,font=\scriptsize] {$w$};
  \draw[->,very thick,bookred] (O) -- (T) node[midway,above left,font=\scriptsize] {$z+w$};
  
  % Draw the parallelogram sides (showing vector addition)
  \draw[->,thick,bookpurple!80!black,dashed] (P) -- (T) node[midway,right,font=\tiny] {$w$};
  \draw[->,thick,bookpurple,dashed] (Q) -- (T) node[midway,above,font=\tiny] {$z$};
  
  % Mark the points
  \fill[bookpurple] (P) circle (2pt);
  \node[bookpurple,font=\small,below right] at (P) {$P: z=2+i$};
  
  \fill[bookpurple!80!black] (Q) circle (2pt);
  \node[bookpurple!80!black,font=\small,left] at (Q) {$Q: w=1+3i$};
  
  \fill[bookred] (T) circle (2.5pt);
  \node[bookred,font=\small,right] at (T) {$T: z+w=3+4i$};
  
  \fill (O) circle (1.5pt);
  \node[below left,font=\scriptsize] at (O) {$O$};
  
  % Mark coordinates
  \node[bookpurple,font=\tiny] at (2,-0.3) {$(2,1)$};
  \node[bookpurple!80!black,font=\tiny] at (0.5,3.3) {$(1,3)$};
  \node[bookred,font=\tiny] at (3.3,4.3) {$(3,4)$};
  
  % Draw coordinate projections (light dashed)
  \draw[dashed,warmstone!30,very thin] (P) -- (2,0);
  \draw[dashed,warmstone!30,very thin] (P) -- (0,1);
  \draw[dashed,warmstone!30,very thin] (Q) -- (1,0);
  \draw[dashed,warmstone!30,very thin] (Q) -- (0,3);
  \draw[dashed,warmstone!30,very thin] (T) -- (3,0);
  \draw[dashed,warmstone!30,very thin] (T) -- (0,4);
  
  % Add annotation box
  \node[font=\scriptsize,align=left,draw=purple!50,fill=white,rounded corners] at (1.5,0.5) {
    \textbf{Parallelogram Law:}\\
    $z + w$ forms the\\
    fourth vertex $T$
  };
  
  % Add label for parallelogram
  \node[orange!70,font=\scriptsize] at (1.5,2) {$OPTQ$};
  
  % Grid for reference
  \foreach \x in {1,2,3} {
    \draw[warmstone!15,very thin] (\x,-0.5) -- (\x,4.8);
  }
  \foreach \y in {1,2,3,4} {
    \draw[warmstone!15,very thin] (-0.5,\y) -- (4,\y);
  }
\end{tikzpicture}
\end{center}
\vspace{0.5em}

\textbf{Key observations:}
\begin{itemize}[leftmargin=*]
  \item The parallelogram $OPTQ$ has $OP$ parallel to $QT$ (both represent vector $z$)
  \item Similarly, $OQ$ is parallel to $PT$ (both represent vector $w$)
  \item Vector addition: starting from $O$, go to $P$ (add $z$), then to $T$ (add $w$)
  \item Alternatively: starting from $O$, go to $Q$ (add $w$), then to $T$ (add $z$)
  \item The diagonal $OT$ represents the sum $z+w$
\end{itemize}

\end{solution}

\begin{problem}[Sum in Polar Form]
Let $z = -2-2i$ and $w=3+i$. Express $z+w$ in modulus-argument form.
\end{problem}

\begin{hint}
First find $z+w = 1-i$, then convert to polar.
\end{hint}

\begin{solution}[Sketch]
$z+w = 1-i$. $|z+w| = \sqrt{2}$, $\arg(z+w) = -\pi/4$. So $z+w = \sqrt{2}(\cos(-\pi/4) + i\sin(-\pi/4))$.
\end{solution}

\begin{problem}[Division in Mixed Form]
Express $\frac{z}{w}$ in form $x+iy$ where $z=-2-2i, w=3+i$.
\end{problem}

\begin{hint}
$\frac{-2-2i}{3+i} = \frac{(-2-2i)(3-i)}{(3+i)(3-i)}$.
\end{hint}

\begin{solution}[Sketch]
$\frac{(-2-2i)(3-i)}{10} = \frac{-6+2i-6i+2i^2}{10} = \frac{-8-4i}{10} = -\frac{4}{5} - \frac{2}{5}i$.
\end{solution}

\begin{problem}[Sketching with Argument and Circle]
Sketch the region: $-\frac{\pi}{4} \le \arg z \le 0$ and $|z - (1-i)| \le 1$.
\end{problem}

\begin{hint}
The first is a wedge in the fourth quadrant. The second is a disk centered at $(1,-1)$.
\end{hint}

\begin{solution}[Sketch]
Draw rays at angles $-45^\circ$ and $0^\circ$ from origin. Draw circle centered at $(1,-1)$ radius 1. Shade intersection.
\end{solution}

\begin{problem}[High Power via De Moivre]
Express $z^9$ in form $x+iy$ for $z=\sqrt{3}-i$.
\end{problem}

\begin{hint}
$z = 2(\cos(-\pi/6) + i\sin(-\pi/6))$.
\end{hint}

\begin{solution}[Sketch]
$z^9 = 2^9(\cos(-3\pi/2) + i\sin(-3\pi/2)) = 512(0 + i) = 512i$.
\end{solution}

\begin{problem}[Polar Form of Specific Number]
Write $z=-1+i\sqrt{3}$ in modulus-argument form.
\end{problem}

\begin{hint}
$|z| = 2$, $z$ is in the second quadrant with reference angle $\pi/3$.
\end{hint}

\begin{solution}[Sketch]
$z = 2(\cos(2\pi/3) + i\sin(2\pi/3))$.
\end{solution}

\begin{problem}[Plotting a Squared Number]
Given $|z|=2, \arg(z)=\frac{\pi}{4}$, plot $u=z^2$.
\end{problem}

\begin{hint}
$|u| = |z|^2 = 4$, $\arg(u) = 2\arg(z) = \pi/2$.
\end{hint}

\begin{solution}[Sketch]
$u$ has modulus 4 and argument $\pi/2$, so $u = 4i$. Plot at $(0,4)$.

\vspace{0.5em}
\begin{center}
\begin{tikzpicture}[scale=1.3,>=Stealth]
  % Draw axes
  \draw[->,thick] (-0.5,0) -- (3,0) node[right,font=\small] {Re};
  \draw[->,thick] (0,-0.5) -- (0,4.8) node[above,font=\small] {Im};
  
  % Define z: |z| = 2, arg(z) = π/4 = 45°
  \def\angleZ{45}
  \def\modulusZ{2}
  \coordinate (Z) at ({\modulusZ*cos(\angleZ)},{\modulusZ*sin(\angleZ)});
  
  % Define u = z²: |u| = 4, arg(u) = π/2 = 90°
  \def\angleU{90}
  \def\modulusU{4}
  \coordinate (U) at ({\modulusU*cos(\angleU)},{\modulusU*sin(\angleU)});
  
  % Draw circles showing moduli
  \draw[dashed,bookpurple!30,thin] (0,0) circle (\modulusZ);
  \draw[dashed,bookred!30,thin] (0,0) circle (\modulusU);
  
  % Draw z
  \draw[->,very thick,bookpurple] (0,0) -- (Z);
  \fill[bookpurple] (Z) circle (2pt);
  \node[bookpurple,font=\small,right] at (Z) {$z$};
  
  % Label for z
  \node[bookpurple,font=\scriptsize,align=left] at ($(Z)+(0.5,-0.5)$) {
    $|z|=2$\\
    $\arg(z)=\frac{\pi}{4}$
  };
  
  % Draw u = z²
  \draw[->,very thick,bookred] (0,0) -- (U);
  \fill[bookred] (U) circle (2pt);
  \node[bookred,font=\small,left] at (U) {$u=z^2$};
  
  % Label for u
  \node[bookred,font=\scriptsize,align=left] at ($(U)+(-1,0)$) {
    $|u|=4$\\
    $\arg(u)=\frac{\pi}{2}$
  };
  
  % Draw angle arcs
  % Arc for z (π/4)
  \draw[bookpurple,thick,->] (0.6,0) arc (0:\angleZ:0.6);
  \node[bookpurple,font=\tiny] at (22.5:0.85) {$\frac{\pi}{4}$};
  
  % Arc for u (π/2)
  \draw[bookred,thick,->] (1,0) arc (0:\angleU:1);
  \node[bookred,font=\tiny] at (45:1.3) {$\frac{\pi}{2}$};
  
  % Show the relationship with an arrow
  \draw[->,thick,lawpurple,dashed] (Z) to[bend left=20] node[midway,right,font=\scriptsize] {square} (U);
  
  % Mark the coordinates
  \node[bookpurple,font=\tiny,below] at (Z) {$(\sqrt{2},\sqrt{2})$};
  \node[bookred,font=\tiny,right] at (U) {$(0,4)$};
  
  % Origin
  \fill (0,0) circle (1pt);
  \node[below left,font=\scriptsize] at (0,0) {$O$};
  
  % Add annotation box
  \node[font=\scriptsize,align=left,draw=purple!50,fill=white,rounded corners] at (2.5,2.5) {
    \textbf{Squaring Rule:}\\
    $|z^2| = |z|^2$\\
    $\arg(z^2) = 2\arg(z)$
  };
  
  % Draw the modulus labels on the circles
  \node[blue!50,font=\tiny] at (45:{\modulusZ-0.3}) {$r=2$};
  \node[red!50,font=\tiny] at (120:{\modulusU-0.3}) {$r=4$};
  
  % Grid for reference (light)
  \foreach \x in {1,2} {
    \draw[warmstone!15,very thin] (\x,-0.5) -- (\x,4.8);
  }
  \foreach \y in {1,2,3,4} {
    \draw[warmstone!15,very thin] (-0.5,\y) -- (3,\y);
  }
\end{tikzpicture}
\end{center}
\vspace{0.5em}

\end{solution}

\begin{problem}[Fifth Roots of Unity and Cosine Sum]
Let $\omega$ be a non-real root of $z^5 = 1$
\begin{enumerate}[label=\alph*),leftmargin=*]
    \item Show that: $1 + \omega + \omega^2 + \omega^3 + \omega^4 = 0$
    \item Hence, show that: $\displaystyle \cos\left(\frac{2\pi}{5}\right) + \cos\left(\frac{4\pi}{5}\right) = -\frac{1}{2}$
\end{enumerate}
\end{problem}

\begin{hint}
For part (a): The fifth roots of unity are roots of $z^5 - 1 = 0$. Factor as $(z-1)(z^4+z^3+z^2+z+1)=0$. For part (b): Use $\omega = e^{i2\pi/5}$ and $\omega^4 = e^{i8\pi/5} = e^{-i2\pi/5} = \bar{\omega}$. Add $\omega + \omega^4$ using Euler's formula.
\end{hint}

\begin{solution}[Sketch]
\textbf{(a)} Since $\omega^5 = 1$, factor $z^5-1 = (z-1)(z^4+z^3+z^2+z+1)=0$. As $\omega \neq 1$ (non-real), we have $z^4+z^3+z^2+z+1=0$, so $1+\omega+\omega^2+\omega^3+\omega^4=0$. \quad $\square$

\textbf{(b)} Let $\omega = e^{i2\pi/5}$. Then $\omega^4 = e^{-i2\pi/5} = \bar{\omega}$ and $\omega^3 = \bar{\omega^2}$. From (a):
\[
1 + (\omega + \omega^4) + (\omega^2 + \omega^3) = 0 \implies 1 + 2\cos\frac{2\pi}{5} + 2\cos\frac{4\pi}{5} = 0
\]
Therefore $\cos\frac{2\pi}{5} + \cos\frac{4\pi}{5} = -\frac{1}{2}$. \quad $\square$
\end{solution}

\begin{problem}[Cube Roots of Unity Properties]
If $\omega$ is a complex cube root of unity, show that:
\begin{enumerate}[label=\alph*),leftmargin=*]
    \item $1 + \omega + \omega^2 = 0$
    \item $1 + \omega + \bar{\omega} = 0$
    \item $(6\omega + 1)(6\omega^2 + 1) = 31$
    \item $(1 + \omega^2)^3 (2 + 3\omega + 3\omega^2) = 1$
\end{enumerate}
\end{problem}

\begin{hint}
A complex cube root of unity satisfies $\omega^3 = 1$ and $\omega \neq 1$. Use $\omega = e^{i2\pi/3}$ and the fundamental identity $1 + \omega + \omega^2 = 0$ to prove all parts. For (b), note that $\omega^2 = \bar{\omega}$. For (c) and (d), expand and repeatedly substitute $\omega^3 = 1$ and $1 + \omega + \omega^2 = 0$.
\end{hint}

\begin{solution}[Sketch]
\textbf{Key facts:} A complex cube root of unity $\omega$ satisfies:
\begin{itemize}[leftmargin=*]
  \item $\omega^3 = 1$
  \item $\omega = e^{i2\pi/3} = -\frac{1}{2} + i\frac{\sqrt{3}}{2}$
  \item $\omega^2 = e^{i4\pi/3} = -\frac{1}{2} - i\frac{\sqrt{3}}{2}$
  \item $\omega^2 = \bar{\omega}$ (conjugates)
\end{itemize}

\textbf{Part (a):} Show $1 + \omega + \omega^2 = 0$

Since $\omega$ is a cube root of unity: $\omega^3 = 1$, so $\omega^3 - 1 = 0$.

Factor: $(\omega - 1)(\omega^2 + \omega + 1) = 0$

Since $\omega \neq 1$ (it's a complex root), we must have:
\[
\omega^2 + \omega + 1 = 0
\]

Therefore: $1 + \omega + \omega^2 = 0$ \quad $\square$

\textbf{Part (b):} Show $1 + \omega + \bar{\omega} = 0$

Since $\omega = e^{i2\pi/3}$ and $\omega^2 = e^{i4\pi/3} = e^{-i2\pi/3}$, we have:
\[
\omega^2 = \overline{\omega}
\]

From part (a): $1 + \omega + \omega^2 = 0$

Substituting $\omega^2 = \bar{\omega}$:
\[
1 + \omega + \bar{\omega} = 0 \quad \square
\]

\textbf{Part (c):} Show $(6\omega + 1)(6\omega^2 + 1) = 31$

Expand the product:
\begin{align*}
(6\omega + 1)(6\omega^2 + 1) &= 36\omega \cdot \omega^2 + 6\omega + 6\omega^2 + 1\\
&= 36\omega^3 + 6\omega + 6\omega^2 + 1
\end{align*}

Use $\omega^3 = 1$:
\[
= 36(1) + 6\omega + 6\omega^2 + 1 = 36 + 1 + 6(\omega + \omega^2)
\]

From part (a): $\omega + \omega^2 = -1$

Therefore:
\[
= 37 + 6(-1) = 37 - 6 = 31 \quad \square
\]

\textbf{Part (d):} From (a), $1+\omega^2=-\omega$, so $(1+\omega^2)^3=(-\omega)^3=-\omega^3=-1$. Also, $2+3\omega+3\omega^2=2+3(-1)=-1$. Therefore $(1+\omega^2)^3(2+3\omega+3\omega^2)=(-1)(-1)=1$. \quad $\square$
\end{solution}

\begin{problem}[Circle Locus in Complex Plane]
Show that the following relationship geometrically describes a circle in the complex plane:
\[
\frac{1}{z} - \frac{1}{\bar{z}} = i
\]
\end{problem}

\begin{hint}
Let $z = x + iy$ where $x, y \in \mathbb{R}$. Express $\frac{1}{z}$ and $\frac{1}{\bar{z}}$ in terms of $x$ and $y$, then separate real and imaginary parts. The result should be an equation of the form $(x-a)^2 + (y-b)^2 = r^2$.
\end{hint}

\begin{solution}[Sketch]
Let $z = x + iy$. Then:
\[
\frac{1}{z} - \frac{1}{\bar{z}} = \frac{x-iy}{x^2+y^2} - \frac{x+iy}{x^2+y^2} = \frac{-2iy}{x^2+y^2}
\]
Given this equals $i$: $\frac{-2y}{x^2+y^2}=1 \implies x^2+y^2+2y=0 \implies x^2+(y+1)^2=1$. This is a circle centered at $-i$ with radius $1$.

\vspace{0.5em}
\begin{center}
\begin{tikzpicture}[scale=2.5,>=Stealth]
  % Draw axes
  \draw[->,thick] (-1.8,0) -- (1.8,0) node[right,font=\small] {Re};
  \draw[->,thick] (0,-2.5) -- (0,1) node[above,font=\small] {Im};
  
  % Draw the circle: x^2 + (y+1)^2 = 1
  % Center at (0, -1), radius 1
  \coordinate (Center) at (0,-1);
  \def\radius{1}
  
  \draw[very thick,bookpurple] (Center) circle (\radius);
  
  % Mark the center
  \fill[bookred] (Center) circle (1.5pt);
  \node[bookred,font=\small,right] at (Center) {Center: $-i$};
  
  % Draw radius indicator
  \draw[bookred,dashed,thin] (Center) -- ($(Center)+(90:\radius)$);
  \node[bookred,font=\scriptsize,right] at ($(Center)+(45:0.5)$) {$r=1$};
  
  % Mark some key points on the circle
  \fill[bookpurple] ($(Center)+(90:\radius)$) circle (1pt);
  \node[bookpurple,font=\scriptsize,above] at ($(Center)+(90:\radius)$) {$(0,0)$};
  
  \fill[bookpurple] ($(Center)+(-90:\radius)$) circle (1pt);
  \node[bookpurple,font=\scriptsize,below] at ($(Center)+(-90:\radius)$) {$(0,-2)$};
  
  \fill[bookpurple] ($(Center)+(0:\radius)$) circle (1pt);
  \node[bookpurple,font=\scriptsize,right] at ($(Center)+(0:\radius)$) {$(1,-1)$};
  
  \fill[bookpurple] ($(Center)+(180:\radius)$) circle (1pt);
  \node[bookpurple,font=\scriptsize,left] at ($(Center)+(180:\radius)$) {$(-1,-1)$};
  
  % Add grid for reference
  \foreach \x in {-1,1} {
    \draw[warmstone!30,very thin] (\x,-2.5) -- (\x,1);
  }
  \foreach \y in {-2,-1} {
    \draw[warmstone!30,very thin] (-1.8,\y) -- (1.8,\y);
  }
  
  % Origin
  \fill (0,0) circle (1pt);
  \node[below right,font=\scriptsize] at (0,0) {$O$};
  
  % Add equation label
  \node[font=\small,align=center,draw=purple!50,fill=white,rounded corners] at (1.3,0.5) {
    $x^2 + (y+1)^2 = 1$
  };
  
  % Note that origin is ON the circle
  \node[bookpurple!80,font=\tiny,align=left] at (-1.5,-2.2) {
    Note: Origin $(0,0)$\\is on the circle
  };
\end{tikzpicture}
\end{center}
\vspace{0.5em}

\textbf{Verification:} The origin $(0,0)$ satisfies the equation: $0^2 + (0+1)^2 = 1$ \checkmark

\textbf{Geometric interpretation:} For any point $z$ on this circle, the difference $\frac{1}{z} - \frac{1}{\bar{z}}$ equals $i$. This relates the reciprocals and their conjugates in a specific way that constrains $z$ to lie on a circle.
\end{solution}

\begin{problem}[Square OABC with Complex Numbers]
$OABC$ is a square with point $A$ represented by the complex number $z = 2 + i$.

\begin{enumerate}[label=\alph*),leftmargin=*]
    \item Let $w$ represent the complex number at point $C$. Prove that: $z^2 + w^2 = 0$
    \item Find the complex numbers represented by $B$ and $C$.
    \item Find the complex number represented by the vector $\overrightarrow{AC}$
\end{enumerate}
\end{problem}

\begin{hint}
For a square $OABC$ with $O$ at the origin, $C$ is obtained by rotating $A$ by $90°$ counterclockwise, so $w = iz$. For part (a), substitute and use $i^2 = -1$. For part (b), $B$ is the fourth vertex, obtained by $B = A + C$ (vector addition).
\end{hint}

\begin{solution}[Sketch]
\textbf{Given:} Square $OABC$ with $O$ at origin and $A$ at $z = 2+i$.

\textbf{Part (a):} Prove $z^2 + w^2 = 0$

For a square with one vertex at the origin $O$ and adjacent vertex at $A$ (represented by $z$), the next vertex $C$ (going counterclockwise) is obtained by rotating $OA$ by $90°$ counterclockwise.

Rotation by $90°$ is achieved by multiplying by $i$:
\[
w = iz
\]

Now calculate $z^2 + w^2$:
\begin{align*}
z^2 + w^2 &= z^2 + (iz)^2\\
&= z^2 + i^2z^2\\
&= z^2 + (-1)z^2\\
&= z^2 - z^2\\
&= 0 \quad \square
\end{align*}

\textbf{(b)} $C$: $w = iz = i(2+i) = 2i-1 = -1+2i$. $B$: $B = A+C = (2+i)+(-1+2i) = 1+3i$.

\textbf{(c)} $\overrightarrow{AC} = C-A = (-1+2i)-(2+i) = -3+i$.

\vspace{0.5em}
\begin{center}
\begin{tikzpicture}[scale=1.8,>=Stealth]
  % Define coordinates
  \coordinate (O) at (0,0);
  \coordinate (A) at (2,1);
  \coordinate (C) at (-1,2);
  \coordinate (B) at (1,3);
  
  % Draw axes
  \draw[->,thick] (-1.5,0) -- (3,0) node[right,font=\small] {Re};
  \draw[->,thick] (0,-0.5) -- (0,3.8) node[above,font=\small] {Im};
  
  % Draw the square
  \draw[very thick,purple!60] (O) -- (A) -- (B) -- (C) -- cycle;
  \fill[purple!10,opacity=0.5] (O) -- (A) -- (B) -- (C) -- cycle;
  
  % Draw vectors
  \draw[->,very thick,bookpurple] (O) -- (A) node[midway,below right,font=\scriptsize] {$z=2+i$};
  \draw[->,very thick,bookred] (O) -- (C) node[midway,left,font=\scriptsize] {$w=-1+2i$};
  \draw[->,very thick,bookpurple!80!black] (O) -- (B) node[midway,above left,font=\tiny] {$1+3i$};
  
  % Draw AC vector
  \draw[->,very thick,lawpurple,dashed] (A) -- (C) node[midway,above right,font=\tiny] {$\overrightarrow{AC}=-3+i$};
  
  % Mark right angle at O
  \draw[thick] (0.2,0) -- (0.2,0.2) -- (0,0.2);
  
  % Mark right angle at A
  \draw[thick] ($(A)+(-0.1,0.2)$) -- ($(A)+(0.1,0.2)$) -- ($(A)+(0.1,0)$);
  
  % Label vertices
  \fill[black] (O) circle (1.5pt);
  \node[black,font=\small,below left] at (O) {$O$ (origin)};
  
  \fill[bookpurple] (A) circle (1.5pt);
  \node[bookpurple,font=\small,right] at (A) {$A: 2+i$};
  
  \fill[bookred] (C) circle (1.5pt);
  \node[bookred,font=\small,left] at (C) {$C: -1+2i$};
  
  \fill[bookpurple!80!black] (B) circle (1.5pt);
  \node[bookpurple!80!black,font=\small,above right] at (B) {$B: 1+3i$};
  
  % Mark coordinates
  \node[bookpurple,font=\tiny,below] at (A) {$(2,1)$};
  \node[bookred,font=\tiny,above] at (C) {$(-1,2)$};
  \node[bookpurple!80!black,font=\tiny,left] at (B) {$(1,3)$};
  
  % Add annotation
  \node[font=\scriptsize,align=left,draw=purple!50,fill=white,rounded corners] at (2.3,2.5) {
    \textbf{Key relations:}\\
    $w = iz$\\
    $B = A + C$\\
    $\overrightarrow{AC} = C - A$
  };
  
  % Grid for reference
  \foreach \x in {-1,1,2} {
    \draw[warmstone!15,very thin] (\x,-0.5) -- (\x,3.8);
  }
  \foreach \y in {1,2,3} {
    \draw[warmstone!15,very thin] (-1.5,\y) -- (3,\y);
  }
\end{tikzpicture}
\end{center}
\end{solution}

\begin{problem}[Square ABCD Vertices with Complex Numbers]
$ABCD$ is a square in the complex plane. The vertices $A$ and $B$ represent the complex numbers $9 + i$ and $4 + 13i$ respectively. Find the complex numbers corresponding to vertices $C$ and $D$.
\end{problem}

\begin{hint}
The vector from $A$ to $B$ is $\overrightarrow{AB} = B - A = (4+13i) - (9+i) = -5 + 12i$. To find $C$, rotate $\overrightarrow{AB}$ by $90°$ counterclockwise (multiply by $i$) and add to $B$. Similarly for $D$.
\end{hint}

\begin{solution}[Sketch]
$\overrightarrow{AB} = B-A = (4+13i)-(9+i) = -5+12i$. Rotate by $90°$: $\overrightarrow{BC} = i(-5+12i) = -12-5i$. Thus $C = B+\overrightarrow{BC} = (4+13i)+(-12-5i) = -8+8i$. Similarly, $D = A+\overrightarrow{BC} = (9+i)+(-12-5i) = -3-4i$.

\textbf{Answers:} $C = -8+8i$, $D = -3-4i$.
\end{solution}

\begin{problem}[Square ABCD Vector Relationship]
The points $A, B, C$ and $D$ on the Argand diagram represent the complex numbers $a, b, c$ and $d$ respectively. The points form a square $ABCD$.

By using vectors, or otherwise, show that $c = (1 + i)d - ia$.
\end{problem}

\begin{hint}
Use the fact that in a square, consecutive sides are equal in length and perpendicular. The key relationship is $\overrightarrow{DC} = i \cdot \overrightarrow{AD}$ (rotation by $90°$).
\end{hint}

\begin{solution}[Proof]
In square $ABCD$, $\overrightarrow{DC} = i \cdot \overrightarrow{AD}$ (rotation by $90°$). Thus:
\[
c-d = i(d-a) = id-ia \implies c = d+id-ia = (1+i)d-ia \quad \square
\]
\end{solution}

