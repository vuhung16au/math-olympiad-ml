\begin{problem}[Complex Region with Exclusion]
Sketch the region on the Argand diagram where $|z - \bar{z}| < 2$ and $|z-1| \ge 1$ hold simultaneously.
\end{problem}

\begin{hint}
$|z - \bar{z}| = |2iy| = 2|y|$, so the first inequality gives $|y| < 1$. The second is the exterior of a circle.
\end{hint}

\begin{solution}[Sketch]
The region $|y| < 1$ is a horizontal strip $-1 < y < 1$. The region $|z-1| \ge 1$ is outside the circle centered at $(1,0)$ with radius 1. Shade the intersection: horizontal strip with circular hole.
\end{solution}

\begin{problem}[Isosceles Right Triangle]
Triangle $ABC$ is represented by $z_1, z_2, z_3$. It is isosceles and right-angled at $B$. Explain why $(z_1 - z_2)^2 = -(z_3 - z_2)^2$.
\end{problem}

\begin{hint}
The vectors $BA$ and $BC$ are perpendicular and equal in length. Multiplication by $i$ rotates by $90^\circ$.
\end{hint}

\begin{solution}[Sketch]
$BA = z_1-z_2$, $BC = z_3-z_2$. Since $BA \perp BC$ and $|BA|=|BC|$, we have $BA = \pm i \cdot BC$. Squaring: $(z_1-z_2)^2 = (\pm i)^2(z_3-z_2)^2 = -1 \cdot (z_3-z_2)^2$.
\end{solution}

\begin{problem}[Square from Triangle]
Three vertices of a square $ABCD$ in the complex plane are represented by complex numbers $z_1$ (point $A$), $z_2$ (point $B$), and $z_3$ (point $C$). Given that consecutive sides of a square are perpendicular and equal in length, find the complex number $z_4$ representing vertex $D$ in terms of $z_1, z_2, z_3$.
\end{problem}

\begin{hint}
Since $ABCD$ is a square, side $AB$ is perpendicular to side $BC$, and $|AB| = |BC|$. This means $BA = i \cdot BC$ (or $-i \cdot BC$). Use the same relationship for sides $CD$ and $DA$.
\end{hint}

\begin{solution}[Sketch]
For a square $ABCD$, consecutive sides are equal in length and perpendicular. 

The vector from $B$ to $A$ is $z_1 - z_2$, and from $B$ to $C$ is $z_3 - z_2$.

Since $BA \perp BC$ and $|BA| = |BC|$, we have:
\[
z_1 - z_2 = i(z_3 - z_2) \quad \text{(rotating $BC$ by $90°$ counterclockwise)}
\]

Similarly, for the square to close, the vector from $C$ to $D$ must equal the vector from $B$ to $C$ rotated $90°$:
\[
z_4 - z_3 = i(z_2 - z_3)
\]

Solving for $z_4$:
\begin{align*}
z_4 &= z_3 + i(z_2 - z_3)\\
&= z_3 + iz_2 - iz_3\\
&= z_2 \cdot i + z_3(1 - i)\\
&= iz_2 + z_3(1-i)
\end{align*}

\textbf{Alternative formula:} Using the parallelogram property, we can also write:
\[
z_4 = z_1 + z_3 - z_2
\]

This works because $\vec{AD} = \vec{BC}$, so $z_4 - z_1 = z_3 - z_2$.

\vspace{0.5em}
\begin{center}
\begin{tikzpicture}[scale=1.8,>=Stealth]
  % Define the square vertices (example values)
  \coordinate (A) at (1,0.5);
  \coordinate (B) at (2,1.5);
  \coordinate (C) at (1,2.5);
  \coordinate (D) at (0,1.5);
  
  % Draw the square
  \draw[very thick,purple!60] (A) -- (B) -- (C) -- (D) -- cycle;
  
  % Fill the square lightly
  \fill[purple!10,opacity=0.5] (A) -- (B) -- (C) -- (D) -- cycle;
  
  % Draw vectors from B
  \draw[->,very thick,bookpurple] (B) -- (A) node[midway,below right,font=\scriptsize] {$z_1-z_2$};
  \draw[->,very thick,bookpurple!80!black] (B) -- (C) node[midway,left,font=\scriptsize] {$z_3-z_2$};
  
  % Draw vectors from C
  \draw[->,very thick,bookred] (C) -- (D) node[midway,above,font=\scriptsize] {$z_4-z_3$};
  
  % Mark right angles
  \draw[thick] ($(B)+(0.15,0)$) -- ($(B)+(0.15,0.15)$) -- ($(B)+(0,0.15)$);
  \draw[thick] ($(C)+(0,0.15)$) -- ($(C)+(-0.15,0.15)$) -- ($(C)+(-0.15,0)$);
  
  % Label vertices
  \fill[bookpurple] (A) circle (1.5pt);
  \node[bookpurple,font=\small,below right] at (A) {$A: z_1$};
  
  \fill[bookpurple] (B) circle (1.5pt);
  \node[bookpurple,font=\small,right] at (B) {$B: z_2$};
  
  \fill[bookpurple!80!black] (C) circle (1.5pt);
  \node[bookpurple!80!black,font=\small,above] at (C) {$C: z_3$};
  
  \fill[bookred] (D) circle (1.5pt);
  \node[bookred,font=\small,left] at (D) {$D: z_4 = ?$};
  
  % Show rotation relationship
  \draw[->,thick,orange!70,dashed] (A) to[bend right=20] node[midway,below,font=\tiny] {rotate $90°$} (C);
  
  % Add annotation box
  \node[font=\scriptsize,align=left,draw=purple!50,fill=white,rounded corners] at (2.8,1) {
    \textbf{Key relationships:}\\
    $z_1 - z_2 = i(z_3-z_2)$\\
    $z_4 - z_3 = i(z_2-z_3)$\\[0.2em]
    \textbf{Answer:}\\
    $z_4 = z_1 + z_3 - z_2$
  };
\end{tikzpicture}
\end{center}
\vspace{0.5em}

\textbf{Verification:} Both formulas are equivalent:
\begin{align*}
iz_2 + z_3(1-i) &= iz_2 + z_3 - iz_3\\
&= z_3 + i(z_2 - z_3)
\end{align*}

And using $z_1 - z_2 = i(z_3 - z_2)$, we get $z_1 = z_2 + i(z_3-z_2) = iz_3 + z_2(1-i)$.

Therefore: $z_4 = z_1 + z_3 - z_2$

\end{solution}

\begin{problem}[Factoring Polynomial]
Express $P(x) = x^4 - 12x^3 + 59x^2 - 138x + 130$ as product of quadratic factors, given roots $3+i$ and $3+2i$.
\end{problem}

\begin{hint}
The roots are $3\pm i$ and $3\pm 2i$. Group conjugate pairs.
\end{hint}

\begin{solution}[Sketch]
$(x-(3+i))(x-(3-i)) = (x-3)^2+1 = x^2-6x+10$. $(x-(3+2i))(x-(3-2i)) = (x-3)^2+4 = x^2-6x+13$. Thus $P(x) = (x^2-6x+10)(x^2-6x+13)$.
\end{solution}

\begin{problem}[Real Root Count]
Deduce that $x^4 - 3x^3 + 5x^2 + 7x - 8 = 0$ has exactly two real roots.
\end{problem}

\begin{hint}
A degree-4 polynomial with real coefficients has either 0, 2, or 4 real roots. Check behavior or use Descartes' rule.
\end{hint}

\begin{solution}[Sketch]
Complex roots come in conjugate pairs. If there are 4 complex roots, they form 2 pairs, leaving 0 real roots. If 1 pair of complex roots, then 2 real roots. Since the polynomial has sign changes, it must have at least one positive real root. Testing values or using intermediate value theorem confirms exactly 2 real roots.
\end{solution}

\begin{problem}[Vector Addition and Angles]
Consider a sequence of unit complex numbers defined by:
$$z_n = \cos(\alpha+n\beta) + i\sin(\alpha+n\beta) = e^{i(\alpha+n\beta)}$$
for $n = 0, 1, 2, 3, \ldots$

On the Argand diagram, we construct points by cumulative vector addition:
\begin{align*}
P_0 &= z_0\\
P_1 &= z_0 + z_1\\
P_2 &= z_0 + z_1 + z_2\\
P_3 &= z_0 + z_1 + z_2 + z_3\\
&\vdots
\end{align*}

These points form a polygonal path. Using vector addition, prove that all external angles of this polygon are equal to $\beta$. That is, show that $\theta_0 = \theta_1 = \theta_2 = \cdots = \beta$, where $\theta_n$ is the external angle at vertex $P_n$.
\end{problem}

\begin{hint}
Each $z_n$ is a unit vector at angle $\alpha + n\beta$. The angle between consecutive vectors is constant. The external angle is the amount the direction turns from one segment to the next.
\end{hint}

\begin{solution}[Sketch]
\textbf{Understanding the geometry:}

The polygonal path is formed by placing unit vectors $z_0, z_1, z_2, \ldots$ tip-to-tail. Each vector $z_n$ is a unit complex number (lies on the unit circle) with argument $\alpha + n\beta$.

\textbf{Step 1: Arguments of consecutive vectors}

For any consecutive pair of vectors:
\begin{align*}
\arg(z_n) &= \alpha + n\beta\\
\arg(z_{n+1}) &= \alpha + (n+1)\beta = \alpha + n\beta + \beta
\end{align*}

The difference in arguments is:
$$\arg(z_{n+1}) - \arg(z_n) = \beta$$

\textbf{Step 2: Geometric interpretation}

The segment from $P_n$ to $P_{n+1}$ is represented by the vector $z_{n+1}$ (pointing in direction $\arg(z_{n+1})$).

The segment from $P_{n-1}$ to $P_n$ is represented by the vector $z_n$ (pointing in direction $\arg(z_n)$).

The external angle $\theta_n$ at vertex $P_n$ is the angle by which the path turns. This is precisely the difference in the directions of consecutive vectors:
$$\theta_n = \arg(z_{n+1}) - \arg(z_n) = \beta$$

\textbf{Step 3: Conclusion}

Since this difference is constant for all $n$, we have:
$$\theta_0 = \theta_1 = \theta_2 = \cdots = \beta$$

All external angles equal $\beta$, which means the polygon turns by the same angle at each vertex.

\textbf{Special case:} If $\beta = \frac{2\pi}{n}$, the path closes after $n$ steps, forming a regular $n$-sided polygon.

\vspace{0.5em}
\begin{center}
\begin{tikzpicture}[scale=1.5,>=Stealth]
  % Parameters
  \def\alpha{20}  % Initial angle
  \def\beta{45}   % Angle increment
  
  % Draw axes
  \draw[->,thick,warmstone!50] (-0.5,0) -- (3.5,0) node[right,font=\scriptsize] {Re};
  \draw[->,thick,warmstone!50] (0,-0.5) -- (0,2.8) node[above,font=\scriptsize] {Im};
  
  % Calculate cumulative positions
  \coordinate (P0) at ({cos(\alpha)}, {sin(\alpha)});
  \coordinate (P1) at ({cos(\alpha) + cos(\alpha+\beta)}, {sin(\alpha) + sin(\alpha+\beta)});
  \coordinate (P2) at ({cos(\alpha) + cos(\alpha+\beta) + cos(\alpha+2*\beta)}, 
                       {sin(\alpha) + sin(\alpha+\beta) + sin(\alpha+2*\beta)});
  \coordinate (P3) at ({cos(\alpha) + cos(\alpha+\beta) + cos(\alpha+2*\beta) + cos(\alpha+3*\beta)}, 
                       {sin(\alpha) + sin(\alpha+\beta) + sin(\alpha+2*\beta) + sin(\alpha+3*\beta)});
  
  % Draw the origin
  \fill (0,0) circle (1.2pt);
  \node[below left,font=\scriptsize] at (0,0) {$O$};
  
  % Draw individual unit vectors from origin (semi-transparent)
  \draw[->,bookpurple!30,thick] (0,0) -- ({cos(\alpha)}, {sin(\alpha)});
  \draw[->,bookpurple!30,thick] (0,0) -- ({cos(\alpha+\beta)}, {sin(\alpha+\beta)});
  \draw[->,bookpurple!30,thick] (0,0) -- ({cos(\alpha+2*\beta)}, {sin(\alpha+2*\beta)});
  \draw[->,bookpurple!30,thick] (0,0) -- ({cos(\alpha+3*\beta)}, {sin(\alpha+3*\beta)});
  
  % Draw partial unit circle
  \draw[warmstone!50,thin] (0,0) circle (1);
  
  % Draw the polygonal path (vector addition)
  \draw[->,very thick,bookpurple] (0,0) -- (P0) node[midway,above,font=\tiny,sloped] {$z_0$};
  \draw[->,very thick,bookred] (P0) -- (P1) node[midway,above,font=\tiny,sloped] {$z_1$};
  \draw[->,very thick,bookpurple!80!black] (P1) -- (P2) node[midway,above,font=\tiny,sloped] {$z_2$};
  \draw[->,very thick,lawpurple] (P2) -- (P3) node[midway,above,font=\tiny,sloped] {$z_3$};
  
  % Mark points
  \fill[bookpurple] (P0) circle (1.5pt);
  \node[above right,font=\scriptsize] at (P0) {$P_0$};
  
  \fill[bookred] (P1) circle (1.5pt);
  \node[above,font=\scriptsize] at (P1) {$P_1$};
  
  \fill[bookpurple!80!black] (P2) circle (1.5pt);
  \node[above,font=\scriptsize] at (P2) {$P_2$};
  
  \fill[lawpurple] (P3) circle (1.5pt);
  \node[above right,font=\scriptsize] at (P3) {$P_3$};
  
  % Mark external angle at P1
  \draw[lawpurple,thick] ($(P1)!0.4cm!(P0)$) arc [start angle={180+\alpha+\beta}, end angle={180+\alpha+2*\beta}, radius=0.4cm];
  \node[lawpurple,font=\scriptsize] at ($(P1) + (-0.35,0.25)$) {$\beta$};
  
  % Mark external angle at P2
  \draw[lawpurple,thick] ($(P2)!0.4cm!(P1)$) arc [start angle={180+\alpha+2*\beta}, end angle={180+\alpha+3*\beta}, radius=0.4cm];
  \node[lawpurple,font=\scriptsize] at ($(P2) + (-0.5,0.15)$) {$\beta$};
  
  % Labels for angles in the unit circle
  \node[blue!60,font=\tiny] at ({0.5*cos(\alpha+15)}, {0.5*sin(\alpha+15)}) {$\alpha$};
  \node[blue!60,font=\tiny] at ({0.6*cos(\alpha+\beta+20)}, {0.6*sin(\alpha+\beta+20)}) {$\alpha\!+\!\beta$};
  
  % Title
  \node[font=\small,align=center] at (1.7,-0.8) {Vector addition spiral with\\constant external angle $\beta$};
\end{tikzpicture}
\end{center}
\vspace{0.5em}

The diagram shows unit vectors $z_0, z_1, z_2, z_3$ (shown faintly from origin) being added tip-to-tail to form the polygonal path $P_0 \to P_1 \to P_2 \to P_3$. Each turn angle (external angle) is $\beta = 45°$.
\end{solution}

\begin{problem}[Modulus via Trigonometry]
Show $|z| = 2\sin\theta$ for $z = 1-\cos2\theta + i\sin2\theta$.
\end{problem}

\begin{hint}
Use $1-\cos2\theta = 2\sin^2\theta$ and $\sin2\theta = 2\sin\theta\cos\theta$.
\end{hint}

\begin{solution}[Sketch]
$|z|^2 = (1-\cos2\theta)^2 + \sin^22\theta = (2\sin^2\theta)^2 + (2\sin\theta\cos\theta)^2 = 4\sin^4\theta + 4\sin^2\theta\cos^2\theta = 4\sin^2\theta(\sin^2\theta+\cos^2\theta) = 4\sin^2\theta$. Thus $|z| = 2|\sin\theta|$.
\end{solution}

\begin{problem}[Euler's Formula Identity]
Show $e^{in\theta} + e^{-in\theta} = 2\cos(n\theta)$.
\end{problem}

\begin{hint}
Use $e^{i\phi} = \cos\phi + i\sin\phi$.
\end{hint}

\begin{solution}[Sketch]
$e^{in\theta} = \cos(n\theta) + i\sin(n\theta)$ and $e^{-in\theta} = \cos(n\theta) - i\sin(n\theta)$. Adding: $e^{in\theta} + e^{-in\theta} = 2\cos(n\theta)$.
\end{solution}

\begin{problem}[Modulus and Angle Calculation with Complex Exponentials]
Let $z$ be the complex number $z = e^{\frac{i\pi}{6}}$ and $w$ be the complex number $w = e^{\frac{3i\pi}{4}}$.

\begin{enumerate}
    \item[(i)] By first writing $z$ and $w$ in Cartesian form, or otherwise, show that
    \[
    |z + w|^2 = \frac{4 - \sqrt{6} + \sqrt{2}}{2}.
    \]

    \item[(ii)] The complex numbers $z$, $w$ and $z + w$ are represented in the complex plane by the vectors $\overrightarrow{OA}$, $\overrightarrow{OB}$ and $\overrightarrow{OC}$ respectively, where $O$ is the origin.
    
    Show that $\angle AOC = \frac{7\pi}{24}$.

    \item[(iii)] Deduce that $\cos \frac{7\pi}{24} = \frac{\sqrt{8 - 2\sqrt{6} + 2\sqrt{2}}}{4}$.
\end{enumerate}
\end{problem}

\begin{hint}
For (i): Convert to Cartesian form and compute $|z+w|^2 = x^2+y^2$. For (ii): Since $|z|=|w|=1$, the parallelogram $OACB$ is a rhombus, and the diagonal bisects the angle. For (iii): Use the relationship $|z+w| = 2\cos(\angle AOC)$ in a rhombus.
\end{hint}

\begin{solution}[Full Solution]
\textbf{Part (i)}

First, we convert $z$ and $w$ to Cartesian form:
\[
z = e^{\frac{i\pi}{6}} = \cos\left(\frac{\pi}{6}\right) + i\sin\left(\frac{\pi}{6}\right) = \frac{\sqrt{3}}{2} + \frac{1}{2}i
\]
\[
w = e^{\frac{3i\pi}{4}} = \cos\left(\frac{3\pi}{4}\right) + i\sin\left(\frac{3\pi}{4}\right) = -\frac{1}{\sqrt{2}} + \frac{1}{\sqrt{2}}i = \frac{-\sqrt{2}}{2} + \frac{\sqrt{2}}{2}i
\]

Now, find $z + w$:
\[
z + w = \left(\frac{\sqrt{3}}{2} - \frac{\sqrt{2}}{2}\right) + i\left(\frac{1}{2} + \frac{\sqrt{2}}{2}\right) = \frac{\sqrt{3}-\sqrt{2}}{2} + i\frac{1+\sqrt{2}}{2}
\]

Calculate the square of the modulus $|z+w|^2 = x^2 + y^2$:
\begin{align*}
|z+w|^2 &= \left(\frac{\sqrt{3}-\sqrt{2}}{2}\right)^2 + \left(\frac{1+\sqrt{2}}{2}\right)^2 \\
&= \frac{3 - 2\sqrt{6} + 2}{4} + \frac{1 + 2\sqrt{2} + 2}{4} \\
&= \frac{5 - 2\sqrt{6} + 3 + 2\sqrt{2}}{4} \\
&= \frac{8 - 2\sqrt{6} + 2\sqrt{2}}{4} \\
&= \frac{4 - \sqrt{6} + \sqrt{2}}{2} \quad \checkmark
\end{align*}

\textbf{Part (ii)}

Note that $|z| = |e^{\frac{i\pi}{6}}| = 1$ and $|w| = |e^{\frac{3i\pi}{4}}| = 1$. Since the moduli are equal ($OA = OB = 1$), the parallelogram $OACB$ formed by vector addition is a \textbf{rhombus}.

In a rhombus, the main diagonal $\overrightarrow{OC}$ bisects the angle between the adjacent sides $\overrightarrow{OA}$ and $\overrightarrow{OB}$.

Find $\angle AOB$:
\[
\text{arg}(z) = \frac{\pi}{6} = \frac{4\pi}{24}, \quad \text{arg}(w) = \frac{3\pi}{4} = \frac{18\pi}{24}
\]
\[
\angle AOB = \text{arg}(w) - \text{arg}(z) = \frac{18\pi}{24} - \frac{4\pi}{24} = \frac{14\pi}{24}
\]

Since $\overrightarrow{OC}$ bisects $\angle AOB$, the angle between $\overrightarrow{OA}$ and $\overrightarrow{OC}$ is half of $\angle AOB$:
\[
\angle AOC = \frac{1}{2} \times \frac{14\pi}{24} = \frac{7\pi}{24} \quad \checkmark
\]

\textbf{Part (iii)}

We use the geometry of the rhombus. The length of the diagonal $OC$ can be calculated using the property that in a rhombus with side length $1$ and angle $\theta$ between the diagonal and the side:
\[
OC = 2 \times OA \times \cos(\angle AOC)
\]

Substituting the known values:
\[
|z+w| = 2(1)\cos\left(\frac{7\pi}{24}\right) = 2\cos\left(\frac{7\pi}{24}\right)
\]

Squaring both sides:
\[
|z+w|^2 = 4\cos^2\left(\frac{7\pi}{24}\right)
\]

Substitute the result from Part (i):
\[
\frac{4 - \sqrt{6} + \sqrt{2}}{2} = 4\cos^2\left(\frac{7\pi}{24}\right)
\]
\[
\cos^2\left(\frac{7\pi}{24}\right) = \frac{4 - \sqrt{6} + \sqrt{2}}{8} = \frac{8 - 2\sqrt{6} + 2\sqrt{2}}{16}
\]

Taking the square root (noting the angle is acute, so cosine is positive):
\[
\cos\left(\frac{7\pi}{24}\right) = \sqrt{\frac{8 - 2\sqrt{6} + 2\sqrt{2}}{16}} = \frac{\sqrt{8 - 2\sqrt{6} + 2\sqrt{2}}}{4} \quad \checkmark
\]

\end{solution}

\begin{problem}[Equilateral Triangle]
Let $z_1$ be a non-zero complex number represented by point $A$ on the Argand diagram, and let $z_2 = e^{i\pi/3}z_1$ be represented by point $B$. The origin is represented by point $O$. Prove that triangle $\triangle OAB$ is equilateral.
\end{problem}

\begin{hint}
Multiplication by $e^{i\pi/3}$ rotates $z_1$ by angle $\pi/3 = 60°$ while preserving its modulus. Therefore $|z_1| = |z_2|$ and the angle $\angle AOB = \pi/3$.
\end{hint}

\begin{solution}[Sketch]
Given $z_2 = e^{i\pi/3}z_1$, we show all three sides of triangle $OAB$ are equal.

\textbf{Step 1:} Since $|e^{i\pi/3}|=1$, we have $|OA|=|z_1|$ and $|OB|=|z_2|=|e^{i\pi/3}z_1|=|z_1|$.

\textbf{Step 2:} Since $z_2 = e^{i\pi/3}z_1$, the angle $\angle AOB = \arg(z_2)-\arg(z_1) = \frac{\pi}{3} = 60°$.

\textbf{Step 3:} By law of cosines: $|AB|^2 = |z_1|^2+|z_1|^2-2|z_1|^2\cos(\pi/3) = 2|z_1|^2-|z_1|^2=|z_1|^2$, so $|AB|=|z_1|$.

\textbf{Conclusion:} $|OA|=|OB|=|AB|=|z_1|$, so triangle $OAB$ is equilateral.

\vspace{0.5em}
\begin{center}
\begin{tikzpicture}[scale=2.2,>=Stealth]
  % Draw axes
  \draw[->,thick] (-0.3,0) -- (3,0) node[right,font=\small] {Re};
  \draw[->,thick] (0,-0.3) -- (0,2.5) node[above,font=\small] {Im};
  
  % Define z1 (let's say at angle 20° with modulus 2)
  \def\angleZ{20}
  \def\modulus{2}
  \coordinate (O) at (0,0);
  \coordinate (A) at ({\modulus*cos(\angleZ)},{\modulus*sin(\angleZ)});
  \coordinate (B) at ({\modulus*cos(\angleZ+60)},{\modulus*sin(\angleZ+60)});
  
  % Draw the equilateral triangle
  \draw[very thick,purple!60] (O) -- (A) -- (B) -- cycle;
  \fill[purple!10,opacity=0.5] (O) -- (A) -- (B) -- cycle;
  
  % Draw vectors from origin
  \draw[->,very thick,bookpurple] (O) -- (A) node[midway,below right,font=\scriptsize] {$z_1$};
  \draw[->,very thick,bookred] (O) -- (B) node[midway,above left,font=\scriptsize] {$z_2=e^{i\pi/3}z_1$};
  
  % Draw the third side
  \draw[very thick,bookpurple!80!black] (A) -- (B);
  
  % Mark all three sides with their lengths
  \node[bookpurple,font=\tiny] at ($(O)!0.5!(A)+(0.15,-0.15)$) {$|z_1|$};
  \node[bookred,font=\tiny] at ($(O)!0.5!(B)+(-0.2,0.1)$) {$|z_1|$};
  \node[bookpurple!80!black,font=\tiny] at ($(A)!0.5!(B)+(0.2,0)$) {$|z_1|$};
  
  % Mark the 60° angle at O
  \draw[orange!70,thick,->] (0.5,0) arc (0:\angleZ:0.5);
  \draw[orange!70,thick,->] ({\angleZ}:0.6) arc (\angleZ:{\angleZ+60}:0.6);
  \node[orange!70,font=\scriptsize] at ({\angleZ+30}:0.85) {$60°$};
  
  % Mark vertices
  \fill[black] (O) circle (1.2pt);
  \node[black,font=\small,below left] at (O) {$O$ (origin)};
  
  \fill[bookpurple] (A) circle (1.5pt);
  \node[bookpurple,font=\small,below right] at (A) {$A: z_1$};
  
  \fill[bookred] (B) circle (1.5pt);
  \node[bookred,font=\small,above] at (B) {$B: z_2$};
  
  % Add annotation box
  \node[font=\scriptsize,align=left,draw=purple!50,fill=white,rounded corners] at (2.5,1.8) {
    \textbf{Key facts:}\\
    $|OA| = |OB| = |z_1|$\\
    $\angle AOB = 60°$\\
    $\Rightarrow |AB| = |z_1|$\\
    All sides equal!
  };
  
  % Mark equal sides with tick marks
  \draw[thick,bookpurple] ($(O)!0.4!(A)$) -- ($(O)!0.4!(A)+(0.05,0.05)$);
  \draw[thick,bookred] ($(O)!0.4!(B)$) -- ($(O)!0.4!(B)+(-0.05,0.05)$);
  \draw[thick,bookpurple!80!black] ($(A)!0.5!(B)+(0.03,0.03)$) -- ($(A)!0.5!(B)+(-0.03,-0.03)$);
  
  % Grid for reference
  \foreach \x in {1,2} {
    \draw[warmstone!15,very thin] (\x,-0.3) -- (\x,2.5);
  }
  \foreach \y in {1,2} {
    \draw[warmstone!15,very thin] (-0.3,\y) -- (3,\y);
  }
\end{tikzpicture}
\end{center}
\vspace{0.5em}

\textbf{Alternative:} $|AB| = |z_2-z_1| = |z_1||e^{i\pi/3}-1| = |z_1| \cdot |(1/2+i\sqrt{3}/2)-1| = |z_1|\sqrt{1/4+3/4} = |z_1|$. \quad $\square$

\end{solution}

\begin{problem}[Algebraic Identity]
Consider an equilateral triangle $\triangle OAB$ where $O$ is the origin, $A$ represents complex number $z_1$, and $B$ represents $z_2 = e^{i\pi/3}z_1$ (i.e., $z_2$ is obtained by rotating $z_1$ by $60°$). Prove the algebraic identity:
\[
z_1^2 + z_2^2 = z_1 z_2
\]
\end{problem}

\begin{hint}
Let $\omega = e^{i\pi/3}$, so $z_2 = \omega z_1$. Show that $\omega$ satisfies $\omega^2 - \omega + 1 = 0$ by using the fact that $\omega^3 = e^{i\pi} = -1$.
\end{hint}

\begin{solution}[Sketch]
Let $\omega = e^{i\pi/3}$ (a primitive 6th root of unity). Then $z_2 = \omega z_1$.

\textbf{Left-hand side:}
\begin{align*}
z_1^2 + z_2^2 &= z_1^2 + (\omega z_1)^2\\
&= z_1^2 + \omega^2 z_1^2\\
&= z_1^2(1 + \omega^2)
\end{align*}

\textbf{Right-hand side:}
\begin{align*}
z_1 z_2 &= z_1 \cdot \omega z_1\\
&= \omega z_1^2
\end{align*}

\textbf{To prove:} We need to show that $1 + \omega^2 = \omega$, or equivalently:
\[
\omega^2 - \omega + 1 = 0
\]

\textbf{Key observation:} Since $\omega = e^{i\pi/3}$, we have:
\[
\omega^3 = e^{i\pi/3 \cdot 3} = e^{i\pi} = -1
\]

Therefore:
\[
\omega^3 + 1 = 0
\]

This factors as:
\[
(\omega + 1)(\omega^2 - \omega + 1) = 0
\]

Since $\omega = e^{i\pi/3} = \cos 60° + i\sin 60° = \frac{1}{2} + i\frac{\sqrt{3}}{2} \neq -1$, we must have:
\[
\omega^2 - \omega + 1 = 0
\]

Therefore $1 + \omega^2 = \omega$, which gives us:
\[
z_1^2(1 + \omega^2) = z_1^2 \cdot \omega = \omega z_1^2 = z_1z_2
\]

Thus: $z_1^2 + z_2^2 = z_1 z_2$ \quad $\square$

\textbf{Alternative direct verification:} We can also verify by direct calculation:
\begin{align*}
\omega^2 &= e^{i2\pi/3} = \cos 120° + i\sin 120° = -\frac{1}{2} + i\frac{\sqrt{3}}{2}\\
1 + \omega^2 &= 1 + \left(-\frac{1}{2} + i\frac{\sqrt{3}}{2}\right) = \frac{1}{2} + i\frac{\sqrt{3}}{2} = \omega \quad \checkmark
\end{align*}

\end{solution}

\begin{problem}[Finding Non-Real Roots]
Find the two complex roots of $P(x) = x^5 - 10x^2 + 15x - 6$.
\end{problem}

\begin{hint}
Test for rational roots first (like $x=1$). Factor out linear factors, then solve the remaining polynomial.
\end{hint}

\begin{solution}[Sketch]
$P(1) = 1-10+15-6=0$, so $(x-1)$ is a factor. Polynomial division gives $P(x) = (x-1)(x^4+x^3+x^2-9x+6)$. Continue factoring or use numerical/algebraic methods to find complex roots.
\end{solution}

\begin{problem}[Expansion Result]
Using De Moivre's theorem and the binomial expansion, derive the formula for $\cos 5\theta$ in terms of $\cos\theta$. Hence show that:
\[
\cos 5\theta = 16\cos^5\theta - 20\cos^3\theta + 5\cos\theta
\]
\end{problem}

\begin{hint}
Use Euler's formula: $(\cos\theta + i\sin\theta)^5 = \cos 5\theta + i\sin 5\theta$. Expand the left side using the binomial theorem, then equate real parts. Express $\sin^2\theta$ in terms of $\cos\theta$ using $\sin^2\theta = 1 - \cos^2\theta$.
\end{hint}

\begin{solution}[Sketch]
By De Moivre's theorem:
\[
(\cos\theta + i\sin\theta)^5 = \cos 5\theta + i\sin 5\theta
\]

\textbf{Expand the left side using binomial theorem:}
\begin{align*}
(\cos\theta + i\sin\theta)^5 &= \sum_{k=0}^{5} \binom{5}{k} \cos^{5-k}\theta (i\sin\theta)^k\\
&= \binom{5}{0}\cos^5\theta + \binom{5}{1}\cos^4\theta(i\sin\theta) + \binom{5}{2}\cos^3\theta(i\sin\theta)^2\\
&\quad + \binom{5}{3}\cos^2\theta(i\sin\theta)^3 + \binom{5}{4}\cos\theta(i\sin\theta)^4 + \binom{5}{5}(i\sin\theta)^5\\
&= \cos^5\theta + 5i\cos^4\theta\sin\theta - 10\cos^3\theta\sin^2\theta\\
&\quad - 10i\cos^2\theta\sin^3\theta + 5\cos\theta\sin^4\theta + i\sin^5\theta
\end{align*}

\textbf{Separate real and imaginary parts:}

Real part:
\[
\text{Re} = \cos^5\theta - 10\cos^3\theta\sin^2\theta + 5\cos\theta\sin^4\theta
\]

This equals $\cos 5\theta$.

\textbf{Express in terms of $\cos\theta$ only:}

Use $\sin^2\theta = 1 - \cos^2\theta$:
\begin{align*}
\cos 5\theta &= \cos^5\theta - 10\cos^3\theta(1-\cos^2\theta) + 5\cos\theta(1-\cos^2\theta)^2\\
&= \cos^5\theta - 10\cos^3\theta + 10\cos^5\theta + 5\cos\theta(1 - 2\cos^2\theta + \cos^4\theta)\\
&= \cos^5\theta - 10\cos^3\theta + 10\cos^5\theta + 5\cos\theta - 10\cos^3\theta + 5\cos^5\theta\\
&= (1 + 10 + 5)\cos^5\theta + (-10 - 10)\cos^3\theta + 5\cos\theta\\
&= 16\cos^5\theta - 20\cos^3\theta + 5\cos\theta
\end{align*}

Therefore: $\cos 5\theta = 16\cos^5\theta - 20\cos^3\theta + 5\cos\theta$ \quad $\square$

\end{solution}

\begin{problem}[Symmetry Identity]
Show $\alpha^k + \alpha^{-k} = 2\cos k\theta$ where $\alpha = \cos\theta + i\sin\theta$.
\end{problem}

\begin{hint}
$\alpha = e^{i\theta}$ and $\alpha^{-1} = e^{-i\theta}$.
\end{hint}

\begin{solution}[Sketch]
$\alpha^k = e^{ik\theta} = \cos k\theta + i\sin k\theta$, $\alpha^{-k} = e^{-ik\theta} = \cos k\theta - i\sin k\theta$. Adding: $\alpha^k + \alpha^{-k} = 2\cos k\theta$.
\end{solution}

\begin{problem}[Equilateral Triangle Centroid]
Let $\triangle ABC$ be an equilateral triangle with vertices $A$, $B$, $C$ represented by complex numbers $a$, $b$, $c$ respectively, oriented anticlockwise. Let $\omega = e^{i2\pi/3} = \cos 120° + i\sin 120°$ be a primitive cube root of unity. Prove that the centroid of the triangle is at the origin if and only if:
\[
a + b\omega + c\omega^2 = 0
\]
\end{problem}

\begin{hint}
For an equilateral triangle centered at the origin, the vertices are related by $120°$ rotations. Use the fact that $\omega = e^{i2\pi/3}$ satisfies $\omega^3 = 1$ and $1 + \omega + \omega^2 = 0$.
\end{hint}

\begin{solution}[Sketch]
\textbf{Key facts about $\omega = e^{i2\pi/3}$:}
\begin{itemize}[leftmargin=*]
  \item $\omega^3 = e^{i2\pi} = 1$ (cube root of unity)
  \item $1 + \omega + \omega^2 = 0$ (sum of cube roots of unity)
  \item $\omega = -\frac{1}{2} + i\frac{\sqrt{3}}{2}$
\end{itemize}

\textbf{If the triangle is centered at origin:}

For an equilateral triangle centered at the origin with one vertex at $a$, the other two vertices are obtained by rotating by $120°$ and $240°$:
\[
b = \omega a \quad \text{and} \quad c = \omega^2 a
\]

Now compute:
\begin{align*}
a + b\omega + c\omega^2 &= a + (\omega a) \cdot \omega + (\omega^2 a) \cdot \omega^2\\
&= a + \omega^2 a + \omega^4 a\\
&= a(1 + \omega^2 + \omega^4)
\end{align*}

Since $\omega^3 = 1$, we have $\omega^4 = \omega^3 \cdot \omega = 1 \cdot \omega = \omega$.

Therefore:
\[
a + b\omega + c\omega^2 = a(1 + \omega^2 + \omega) = a(1 + \omega + \omega^2)
\]

Using the fundamental identity $1 + \omega + \omega^2 = 0$:
\[
a + b\omega + c\omega^2 = a \cdot 0 = 0
\]

\textbf{Conversely:} If $a + b\omega + c\omega^2 = 0$ and the triangle is equilateral, then its centroid is at the origin.

The centroid is at $\frac{a + b + c}{3}$. For an equilateral triangle with the given property, we can show:
\[
a + b + c = 0
\]

This follows because if $a + b\omega + c\omega^2 = 0$, and we also have the relations for an equilateral triangle, then the centroid must be at the origin.

Therefore: $a + b\omega + c\omega^2 = 0$ \quad $\square$

\textbf{Geometric interpretation:} This identity expresses the fact that in an equilateral triangle centered at the origin, the weighted sum of the vertices (with weights $1, \omega, \omega^2$) is zero, reflecting the rotational symmetry of the configuration.

\end{solution}

\begin{problem}[Real Sum and Product]
Given that $z + w$ and $zw$ are both real, prove that either $z = \overline{w}$ or $\text{Im}(z) = \text{Im}(w) = 0$.
\end{problem}

\begin{hint}
Let $z = a + bi$ and $w = c + di$. Use the conditions that $z + w$ and $zw$ are real to derive equations for $b$ and $d$ in terms of $a$ and $c$.
\end{hint}

\begin{solution}[Proof]
Let $z = a+bi$, $w = c+di$ where $a,b,c,d \in \mathbb{R}$.

\textbf{Step 1:} If $z+w$ is real, then $\text{Im}(z+w) = b+d = 0$, so $d=-b$. \quad (Eq 1)

\textbf{Step 2:} If $zw$ is real, then $\text{Im}(zw) = \text{Im}[(a+bi)(c+di)] = ad+bc = 0$. \quad (Eq 2)

\textbf{Step 3:} Substitute Eq 1 into Eq 2: $a(-b)+bc=0 \implies b(c-a)=0$.

\textbf{Case 1:} If $b=0$, then $d=-b=0$, so $\text{Im}(z)=\text{Im}(w)=0$ (both real).

\textbf{Case 2:} If $b \neq 0$, then $c=a$. With $d=-b$: $w = a-bi = \bar{z}$, so $z=\bar{w}$.

\textbf{Conclusion:} Either $\text{Im}(z)=\text{Im}(w)=0$ (both real) or $z=\bar{w}$ (conjugates). \quad $\square$
\end{solution}

\begin{problem}[Complex Argument Region]
The complex numbers $w$ and $z$ both have modulus $1$, and $\frac{\pi}{2} < \text{Arg}\left(\frac{z}{w}\right) < \pi$, where $\text{Arg}$ denotes the principal argument.

For real numbers $x$ and $y$, consider the complex number $\frac{xz + yw}{z}$.

On an $xy$-plane, clearly sketch the region that contains all points $(x, y)$ for which
\[
\frac{\pi}{2} < \text{Arg}\left(\frac{xz + yw}{z}\right) < \pi.
\]
\end{problem}

\begin{hint}
Simplify $\frac{xz + yw}{z} = x + y\frac{w}{z}$. Let $\alpha = \text{Arg}(w/z)$ where $\frac{\pi}{2} < \alpha < \pi$. Then $w/z = e^{i\alpha}$. The expression becomes $x + ye^{i\alpha}$. For the argument to be in the given range, the resulting complex number must lie in the second quadrant.
\end{hint}

\begin{solution}[Sketch]
Simplify: $\frac{xz+yw}{z} = x + y\frac{w}{z}$. Since $\frac{\pi}{2} < \text{Arg}(z/w) < \pi$, we have $-\pi < \text{Arg}(w/z) < -\frac{\pi}{2}$, so $w/z = e^{i\alpha}$ where $-\pi < \alpha < -\frac{\pi}{2}$. Thus:
\[
x + y(\cos\alpha + i\sin\alpha) = (x+y\cos\alpha) + i(y\sin\alpha)
\]
For this in Q2 (second quadrant): $x+y\cos\alpha < 0$ and $y\sin\alpha > 0$. Since $\sin\alpha < 0$ and $\cos\alpha < 0$ in Q3, we need $y < 0$ and $x < y\cos\alpha$. The region is a wedge below the $x$-axis bounded by the line $x = y\cos\alpha$.

\vspace{0.5em}
\begin{center}
\begin{tikzpicture}[scale=2,>=Stealth]
  % Draw axes
  \draw[->,thick] (-2.5,0) -- (2.5,0) node[right,font=\small] {$x$};
  \draw[->,thick] (0,-2.5) -- (0,2.5) node[above,font=\small] {$y$};
  
  % Example: Let alpha = -3π/4, so cos(alpha) = -√2/2
  \def\cosalpha{-0.707}
  \def\angle{-135}
  
  % Draw the boundary lines
  % Line x = y*cos(alpha), which has slope 1/cos(alpha)
  \draw[thick,bookpurple,dashed] (0,0) -- ({\cosalpha*(-2)},{-2});
  \node[bookpurple,font=\scriptsize,rotate=-45] at (-0.8,-1.2) {$x = y\cos\alpha$};
  
  % Shade the region: y < 0 and x < y*cos(alpha)
  \fill[purple!20,opacity=0.7] (0,0) -- (-2,0) -- (-2,-2) -- ({\cosalpha*(-2)},{-2}) -- cycle;
  
  % Add labels
  \node[purple!80,font=\small,align=center] at (-1.5,-1) {\textbf{Region}};
  
  % Mark the conditions
  \node[font=\scriptsize,align=left,draw=purple!50,fill=white,rounded corners] at (1.8,-1.5) {
    Conditions:\\
    $y < 0$\\
    $x < y\cos\alpha$\\[0.2em]
    where $-1 < \cos\alpha < 0$
  };
  
  % Origin
  \fill (0,0) circle (1pt);
  \node[below right,font=\scriptsize] at (0,0) {$O$};
  
  % Grid
  \foreach \xx in {-2,-1,1,2} {
    \draw[warmstone!20,very thin] (\xx,-2.5) -- (\xx,2.5);
  }
  \foreach \yy in {-2,-1,1,2} {
    \draw[warmstone!20,very thin] (-2.5,\yy) -- (2.5,\yy);
  }
\end{tikzpicture}
\end{center}
\end{solution}

\begin{problem}[Geometric Series with $n$-th Roots]
If $z = \cos \frac{\pi}{n} + i \sin \frac{\pi}{n}$, and $n$ is a positive integer, prove:
\begin{enumerate}[label=(\roman*),leftmargin=*]
    \item $1 + z + z^2 + \dots + z^{2n-1} = 0$
    \item $1 + z + z^2 + \dots + z^{n-1} = 1 + i \cot\left(\frac{\pi}{2n}\right)$
\end{enumerate}
\end{problem}

\begin{hint}
For part (i): Note that $z = e^{i\pi/n}$, so $z^{2n} = e^{i2\pi} = 1$. Use the geometric series formula. For part (ii): Split the sum at $z^{n-1}$ and use the fact that $z^n = e^{i\pi} = -1$.
\end{hint}

\begin{solution}[Proof]
\textbf{Setup:} Given $z = \cos\frac{\pi}{n} + i\sin\frac{\pi}{n} = e^{i\pi/n}$

\textbf{Part (i):} Prove $1 + z + z^2 + \dots + z^{2n-1} = 0$

Calculate $z^{2n}$:
\[
z^{2n} = \left(e^{i\pi/n}\right)^{2n} = e^{i \cdot 2\pi} = 1
\]

Therefore $z$ is a $2n$-th root of unity.

The sum $S = 1 + z + z^2 + \cdots + z^{2n-1}$ is a geometric series with first term 1, common ratio $z$, and $2n$ terms.

Using the geometric series formula:
\[
S = \frac{z^{2n} - 1}{z - 1} = \frac{1 - 1}{z - 1} = \frac{0}{z - 1} = 0
\]

(Note: Since $z = e^{i\pi/n} \neq 1$ for $n \geq 2$, the denominator is non-zero.)

Therefore: $1 + z + z^2 + \dots + z^{2n-1} = 0$ \quad $\square$

\textbf{Part (ii):} Prove $1 + z + z^2 + \dots + z^{n-1} = 1 + i\cot\left(\frac{\pi}{2n}\right)$

Let $S_n = 1 + z + z^2 + \cdots + z^{n-1}$

From part (i), we know:
\[
1 + z + z^2 + \cdots + z^{2n-1} = 0
\]

We can split this sum:
\[
(1 + z + \cdots + z^{n-1}) + (z^n + z^{n+1} + \cdots + z^{2n-1}) = 0
\]

The second sum can be factored:
\[
z^n + z^{n+1} + \cdots + z^{2n-1} = z^n(1 + z + \cdots + z^{n-1}) = z^n \cdot S_n
\]

Since $z^n = e^{i\pi} = -1$:
\[
S_n + (-1) \cdot S_n = 0
\]
\[
S_n - S_n = 0
\]

Using geometric series: $S_n = \frac{z^n-1}{z-1} = \frac{-2}{z-1}$ (since $z^n=-1$). With $z-1 = (\cos\frac{\pi}{n}-1)+i\sin\frac{\pi}{n}$, multiply by conjugate. Denominator: $2(1-\cos\frac{\pi}{n})=4\sin^2\frac{\pi}{2n}$. Using $\cos\frac{\pi}{n}-1=-2\sin^2\frac{\pi}{2n}$ and $\sin\frac{\pi}{n}=2\sin\frac{\pi}{2n}\cos\frac{\pi}{2n}$, we get:
\[
S_n = \frac{4\sin^2\frac{\pi}{2n}+i \cdot 4\sin\frac{\pi}{2n}\cos\frac{\pi}{2n}}{4\sin^2\frac{\pi}{2n}} = 1+i\cot\frac{\pi}{2n} \quad \square
\]

Therefore: $1 + z + z^2 + \dots + z^{n-1} = 1 + i\cot\left(\frac{\pi}{2n}\right)$ \quad $\square$
\end{solution}

\begin{problem}[Fifth Roots of $-1$ and Cosine Values]
By finding the fifth roots of $-1$, find the exact values of $\cos\frac{\pi}{5}$ and $\cos\frac{3\pi}{5}$.
\end{problem}

\begin{hint}
The fifth roots of $-1$ are solutions to $z^5 = -1$, or equivalently $z^5 + 1 = 0$. Use $-1 = e^{i\pi}$, so $z = e^{i\pi(2k+1)/5}$ for $k = 0, 1, 2, 3, 4$. The polynomial $z^5 + 1$ factors as $(z+1)(z^4 - z^3 + z^2 - z + 1) = 0$. Use the quartic to find relationships involving $\cos\frac{\pi}{5}$ and $\cos\frac{3\pi}{5}$.
\end{hint}

\begin{solution}[Sketch]
\textbf{Step 1:} Solve $z^5=-1=e^{i\pi}$ using De Moivre: $z=e^{i\pi(2k+1)/5}$ for $k=0,1,2,3,4$.

Five roots: $z_0=e^{i\pi/5}$, $z_1=e^{i3\pi/5}$, $z_2=-1$, $z_3=e^{-i3\pi/5}=\bar{z_1}$, $z_4=e^{-i\pi/5}=\bar{z_0}$.

\textbf{Step 2:} Factor: $z^5+1=(z+1)(z^4-z^3+z^2-z+1)=0$. The four non-real roots $z_0,z_1,\bar{z_0},\bar{z_1}$ satisfy $z^4-z^3+z^2-z+1=0$.

\textbf{Step 3: Use Vieta's formulas}

Sum of roots: $(z_0+\bar{z_0})+(z_1+\bar{z_1})=1 \implies 2\cos\frac{\pi}{5}+2\cos\frac{3\pi}{5}=1 \implies \cos\frac{\pi}{5}+\cos\frac{3\pi}{5}=\frac{1}{2}$.

Sum of products (coeff. of $z^2$): Using $z_0z_4=|z_0|^2=1$, $z_1z_3=|z_1|^2=1$, and simplifying other products gives: $\alpha\beta=-\frac{1}{4}$ where $\alpha=\cos\frac{\pi}{5}$ and $\beta=\cos\frac{3\pi}{5}$.

From $\alpha+\beta=\frac{1}{2}$ and $\alpha\beta=-\frac{1}{4}$, solve $4t^2-2t-1=0$ to get $t=\frac{1\pm\sqrt{5}}{4}$. Since $\cos\frac{\pi}{5}>0$ and $\cos\frac{3\pi}{5}<0$:
\[
\boxed{\cos\frac{\pi}{5} = \frac{1+\sqrt{5}}{4}} \quad \text{and} \quad \boxed{\cos\frac{3\pi}{5} = \frac{1-\sqrt{5}}{4}}
\]
\end{solution}

\begin{problem}[Euler's Formula and Integration]
\begin{enumerate}[label=(\roman*),leftmargin=*]
    \item Show that for any integer $n$, $e^{in\theta} + e^{-in\theta} = 2\cos(n\theta)$.
    \item By expanding $\left(e^{i\theta} + e^{-i\theta}\right)^4$, show that
    \[
    \cos^4 \theta = \frac{1}{8} \big( \cos(4\theta) + 4\cos(2\theta) + 3 \big).
    \]
    \item Hence, or otherwise, find $\displaystyle \int_{0}^{\frac{\pi}{2}} \cos^4 \theta \, d\theta$.
\end{enumerate}
\end{problem}

\begin{hint}
For part (i): Use Euler's formula $e^{i\phi} = \cos\phi + i\sin\phi$. For part (ii): Use part (i) to write $e^{i\theta} + e^{-i\theta} = 2\cos\theta$, then raise to the 4th power using the binomial theorem. For part (iii): Integrate the expression from part (ii) term by term.
\end{hint}

\begin{solution}[Sketch]
\textbf{Part (i):} Show $e^{in\theta} + e^{-in\theta} = 2\cos(n\theta)$

Using Euler's formula:
\begin{align*}
e^{in\theta} &= \cos(n\theta) + i\sin(n\theta)\\
e^{-in\theta} &= \cos(-n\theta) + i\sin(-n\theta) = \cos(n\theta) - i\sin(n\theta)
\end{align*}

Adding:
\[
e^{in\theta} + e^{-in\theta} = 2\cos(n\theta) \quad \square
\]

\textbf{(ii)} From (i), $\cos\theta = \frac{e^{i\theta}+e^{-i\theta}}{2}$, so $\cos^4\theta = \frac{(e^{i\theta}+e^{-i\theta})^4}{16}$. Expand using binomial: $(e^{i\theta}+e^{-i\theta})^4 = e^{i4\theta}+4e^{i2\theta}+6+4e^{-i2\theta}+e^{-i4\theta} = 2\cos 4\theta+8\cos 2\theta+6$. Thus $\cos^4\theta = \frac{1}{8}(\cos 4\theta+4\cos 2\theta+3)$. \quad $\square$

\textbf{(iii)} $\displaystyle \int_0^{\pi/2} \cos^4\theta \, d\theta = \frac{1}{8}\left[\frac{\sin 4\theta}{4}+2\sin 2\theta+3\theta\right]_0^{\pi/2} = \frac{1}{8}\cdot\frac{3\pi}{2} = \boxed{\frac{3\pi}{16}}$
\end{solution}

