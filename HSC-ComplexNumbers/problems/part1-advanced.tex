\begin{problem}[Geometric Proof with Complex Numbers]
Points $A$ and $B$ represent complex numbers $z$ and $w$ on a circle centered at $O$. Point $C$ represents $z+w$ and also lies on the circle. Show geometrically that $\angle AOB = \frac{2\pi}{3}$.
\end{problem}

\begin{solution}
Given that $|z| = |w| = |z+w| = r$ (the radius of the circle), we need to find $\angle AOB$.

Since $OACB$ forms a parallelogram (by vector addition), and we know:
\begin{itemize}[leftmargin=*]
  \item $OA = |z| = r$
  \item $OB = |w| = r$
  \item $OC = |z+w| = r$
  \item $AC = |w| = r$ (opposite side of parallelogram)
  \item $BC = |z| = r$ (opposite side of parallelogram)
\end{itemize}

All sides of the rhombus $OACB$ have length $r$. Moreover, diagonal $OC$ also has length $r$.

Consider triangle $OAC$: all three sides have length $r$ (since $OA = AC = OC = r$), so it is equilateral. Therefore, $\angle AOC = \frac{\pi}{3}$.

Similarly, triangle $OBC$ has $OB = BC = OC = r$, so it is also equilateral. Therefore, $\angle BOC = \frac{\pi}{3}$.

Thus:
\[
\angle AOB = \angle AOC + \angle BOC = \frac{\pi}{3} + \frac{\pi}{3} = \frac{2\pi}{3}.
\]
\end{solution}

\begin{takeaways}
When complex numbers are represented geometrically, vector addition corresponds to the parallelogram law. A rhombus with all sides equal to its diagonal consists of two equilateral triangles.
\end{takeaways}

\begin{problem}[Complex Division and Real Parts]
Let $z = 2(\cos \theta + i \sin \theta)$. Show that the real part of $\frac{1}{1-z}$ is $\frac{1-2\cos\theta}{5-4\cos\theta}$.
\end{problem}

\begin{solution}
First, compute $1 - z$:
\[
1 - z = 1 - 2(\cos\theta + i\sin\theta) = (1-2\cos\theta) - 2i\sin\theta.
\]

To find $\frac{1}{1-z}$, multiply numerator and denominator by the conjugate of $1-z$:
\[
\frac{1}{1-z} = \frac{1}{(1-2\cos\theta) - 2i\sin\theta} \cdot \frac{(1-2\cos\theta) + 2i\sin\theta}{(1-2\cos\theta) + 2i\sin\theta}
\]
\[
= \frac{(1-2\cos\theta) + 2i\sin\theta}{(1-2\cos\theta)^2 + (2\sin\theta)^2}.
\]

Compute the denominator:
\begin{align*}
(1-2\cos\theta)^2 + 4\sin^2\theta &= 1 - 4\cos\theta + 4\cos^2\theta + 4\sin^2\theta \\
&= 1 - 4\cos\theta + 4(\cos^2\theta + \sin^2\theta) \\
&= 1 - 4\cos\theta + 4 \\
&= 5 - 4\cos\theta.
\end{align*}

Therefore:
\[
\frac{1}{1-z} = \frac{(1-2\cos\theta) + 2i\sin\theta}{5-4\cos\theta}.
\]

The real part is:
\[
\text{Re}\left(\frac{1}{1-z}\right) = \frac{1-2\cos\theta}{5-4\cos\theta}.
\]
\end{solution}

\begin{takeaways}
To find the real part of a complex fraction, rationalize by multiplying by the conjugate of the denominator. Use the identity $\cos^2\theta + \sin^2\theta = 1$ to simplify.
\end{takeaways}

\begin{problem}[Finding Complex Roots of Polynomials]
Two zeros of $P(x) = x^4 - 12x^3 + 59x^2 - 138x + 130$ are $a+ib$ and $a+2ib$ where $a, b$ are real and $b>0$. Find $a$ and $b$.
\end{problem}

\begin{solution}
Since $P(x)$ has real coefficients, complex roots occur in conjugate pairs. If $a+ib$ is a root, then $a-ib$ is also a root. Similarly, if $a+2ib$ is a root, then $a-2ib$ is also a root.

Therefore, the four roots are: $a+ib$, $a-ib$, $a+2ib$, $a-2ib$.

By Vieta's formulas, the sum of roots equals the negative of the coefficient of $x^3$ divided by the leading coefficient:
\[
(a+ib) + (a-ib) + (a+2ib) + (a-2ib) = 4a = 12,
\]
so $a = 3$.

The product of roots equals the constant term:
\[
[(a+ib)(a-ib)][(a+2ib)(a-2ib)] = (a^2+b^2)(a^2+4b^2) = 130.
\]

Substituting $a = 3$:
\[
(9+b^2)(9+4b^2) = 130.
\]

Expand:
\[
81 + 36b^2 + 9b^2 + 4b^4 = 130,
\]
\[
4b^4 + 45b^2 + 81 = 130,
\]
\[
4b^4 + 45b^2 - 49 = 0.
\]

Let $u = b^2$:
\[
4u^2 + 45u - 49 = 0.
\]

Using the quadratic formula:
\[
u = \frac{-45 \pm \sqrt{45^2 + 4 \cdot 4 \cdot 49}}{8} = \frac{-45 \pm \sqrt{2025 + 784}}{8} = \frac{-45 \pm \sqrt{2809}}{8} = \frac{-45 \pm 53}{8}.
\]

Since $u = b^2 \ge 0$: $u = \frac{-45+53}{8} = \frac{8}{8} = 1$.

Therefore, $b^2 = 1$, so $b = 1$ (since $b > 0$).

Thus, $a = 3$ and $b = 1$.
\end{solution}

\begin{takeaways}
For polynomials with real coefficients, complex roots come in conjugate pairs. Use Vieta's formulas (sum and product of roots) along with conjugate pairing to set up equations for unknown parameters.
\end{takeaways}

\begin{problem}[Trigonometric Identity via Complex Numbers]
Show that $\cos 4\theta = \cos^4\theta - 6\cos^2\theta\sin^2\theta + \sin^4\theta$.
\end{problem}

\begin{solution}
By De Moivre's theorem:
\[
(\cos\theta + i\sin\theta)^4 = \cos 4\theta + i\sin 4\theta.
\]

Expand the left side using the binomial theorem:
\begin{align*}
(\cos\theta + i\sin\theta)^4 &= \sum_{k=0}^{4} \binom{4}{k} \cos^{4-k}\theta (i\sin\theta)^k \\
&= \cos^4\theta + 4i\cos^3\theta\sin\theta + 6i^2\cos^2\theta\sin^2\theta \\
&\quad + 4i^3\cos\theta\sin^3\theta + i^4\sin^4\theta \\
&= \cos^4\theta + 4i\cos^3\theta\sin\theta - 6\cos^2\theta\sin^2\theta \\
&\quad - 4i\cos\theta\sin^3\theta + \sin^4\theta \\
&= (\cos^4\theta - 6\cos^2\theta\sin^2\theta + \sin^4\theta) \\
&\quad + i(4\cos^3\theta\sin\theta - 4\cos\theta\sin^3\theta).
\end{align*}

Equating real parts:
\[
\cos 4\theta = \cos^4\theta - 6\cos^2\theta\sin^2\theta + \sin^4\theta.
\]
\end{solution}

\begin{takeaways}
De Moivre's theorem combined with binomial expansion provides a systematic way to derive multiple-angle formulas. Equate real and imaginary parts to obtain separate identities for $\cos(n\theta)$ and $\sin(n\theta)$.
\end{takeaways}

\begin{problem}[Conjugate Pairs and De Moivre]
Show that $(1+i)^n + (1-i)^n = 2(\sqrt{2})^n \cos \frac{n\pi}{4}$.
\end{problem}

\begin{solution}
First, convert $1+i$ and $1-i$ to polar form.

For $1+i$:
\[
|1+i| = \sqrt{2}, \quad \arg(1+i) = \frac{\pi}{4},
\]
so $1+i = \sqrt{2}\left(\cos\frac{\pi}{4} + i\sin\frac{\pi}{4}\right) = \sqrt{2}e^{i\pi/4}$.

For $1-i$:
\[
|1-i| = \sqrt{2}, \quad \arg(1-i) = -\frac{\pi}{4},
\]
so $1-i = \sqrt{2}\left(\cos\left(-\frac{\pi}{4}\right) + i\sin\left(-\frac{\pi}{4}\right)\right) = \sqrt{2}e^{-i\pi/4}$.

By De Moivre's theorem:
\[
(1+i)^n = (\sqrt{2})^n e^{in\pi/4} = (\sqrt{2})^n\left(\cos\frac{n\pi}{4} + i\sin\frac{n\pi}{4}\right),
\]
\[
(1-i)^n = (\sqrt{2})^n e^{-in\pi/4} = (\sqrt{2})^n\left(\cos\frac{n\pi}{4} - i\sin\frac{n\pi}{4}\right).
\]

Adding:
\begin{align*}
(1+i)^n + (1-i)^n &= (\sqrt{2})^n\left[\left(\cos\frac{n\pi}{4} + i\sin\frac{n\pi}{4}\right) + \left(\cos\frac{n\pi}{4} - i\sin\frac{n\pi}{4}\right)\right] \\
&= (\sqrt{2})^n \cdot 2\cos\frac{n\pi}{4} \\
&= 2(\sqrt{2})^n\cos\frac{n\pi}{4}.
\end{align*}
\end{solution}

\begin{takeaways}
For conjugate pairs $z$ and $\bar{z}$ in polar form, $z^n + \bar{z}^n = 2|z|^n\cos(n\arg(z))$ because the imaginary parts cancel. This is useful for simplifying sums involving conjugate powers.
\end{takeaways}

