\begin{problem}[Multiplication by $i$ as Rotation]
Consider the complex number $z = 3 + 2i$ on the Argand diagram. Find the complex number $iz$ and describe its geometric relationship to $z$.
\end{problem}

\begin{hint}
Multiplication by $i$ rotates a complex number by $90°$ counterclockwise about the origin while preserving its modulus. If $z = a + bi$, calculate $iz$ algebraically.
\end{hint}

\begin{solution}[Sketch]
\textbf{Algebraic calculation:}

Given $z = 3 + 2i$, we compute:
\begin{align*}
iz &= i(3 + 2i)\\
&= 3i + 2i^2\\
&= 3i + 2(-1)\\
&= -2 + 3i
\end{align*}

Therefore: $iz = -2 + 3i$

\textbf{General rule:} If $z = a + bi$, then:
\[
iz = i(a+bi) = ai + bi^2 = -b + ai
\]

This shows that $(a,b) \to (-b,a)$, which is a $90°$ counterclockwise rotation.

\textbf{Verification for our example:}
- Original point: $z = 3 + 2i$ at position $(3, 2)$
- Rotated point: $iz = -2 + 3i$ at position $(-2, 3)$
- Check: $(3,2) \to (-2,3)$ is correct

\vspace{0.5em}
\begin{center}
\begin{tikzpicture}[scale=1.5,>=Stealth]
  % Draw axes
  \draw[->,thick] (-3,0) -- (4,0) node[right,font=\small] {Re};
  \draw[->,thick] (0,-1) -- (0,4) node[above,font=\small] {Im};
  
  % Original complex number z = 3 + 2i
  \coordinate (Z) at (3,2);
  \draw[->,very thick,bookpurple] (0,0) -- (Z);
  \fill[bookpurple] (Z) circle (2pt);
  \node[bookpurple,font=\small,right] at (Z) {$z = 3+2i$};
  
  % Rotated complex number iz = -2 + 3i
  \coordinate (IZ) at (-2,3);
  \draw[->,very thick,bookred] (0,0) -- (IZ);
  \fill[bookred] (IZ) circle (2pt);
  \node[bookred,font=\small,left] at (IZ) {$iz = -2+3i$};
  
  % Show the rotation with a curved arrow
  \draw[->,lawpurple,very thick] (Z) to[bend left=30] node[midway,above right,font=\scriptsize] {$\times i$} (IZ);
  
  % Show 90° rotation angle at origin
  \draw[bookred!80,thick] (0.8,0) arc (0:90:0.8);
  \node[bookred!80,font=\scriptsize] at (45:1.1) {$90°$};
  
  % Draw circles showing equal moduli
  \pgfmathsetmacro{\modulus}{sqrt(3*3 + 2*2)}
  \draw[dashed,warmstone!50,thin] (0,0) circle (\modulus);
  
  % Mark coordinates
  \node[bookpurple,font=\tiny,below] at (Z) {$(3,2)$};
  \node[bookred,font=\tiny,above left] at (IZ) {$(-2,3)$};
  
  % Draw projections (dashed lines)
  \draw[dashed,bookpurple!30,thin] (Z) -- (3,0) node[below,font=\tiny] {$3$};
  \draw[dashed,bookpurple!30,thin] (Z) -- (0,2) node[left,font=\tiny] {$2$};
  \draw[dashed,bookred!30,thin] (IZ) -- (-2,0) node[below,font=\tiny] {$-2$};
  \draw[dashed,bookred!30,thin] (IZ) -- (0,3) node[right,font=\tiny] {$3$};
  
  % Origin
  \fill (0,0) circle (1pt);
  \node[below left,font=\scriptsize] at (0,0) {$O$};
  
  % Add annotation box
  \node[font=\scriptsize,align=left,draw=lawpurple!50,fill=white,rounded corners] at (3,3.5) {
    \textbf{Multiplication by $i$:}\\
    $z = a+bi \to iz = -b+ai$\\
    Rotates $90°$ CCW\\
    Preserves modulus
  };
  
  % Grid for reference
  \foreach \x in {-2,-1,1,2,3} {
    \draw[warmstone!20,very thin] (\x,-1) -- (\x,4);
  }
  \foreach \y in {1,2,3} {
    \draw[warmstone!20,very thin] (-3,\y) -- (4,\y);
  }
\end{tikzpicture}
\end{center}
\vspace{0.5em}

\textbf{Key observations:}
\begin{itemize}[leftmargin=*]
  \item The modulus is preserved: $|z| = |iz| = \sqrt{13}$
  \item The transformation $(a,b) \to (-b,a)$ is a $90°$ counterclockwise rotation
  \item Both points lie on the same circle centered at the origin
  \item This geometric property makes $i$ rotation in the complex plane
\end{itemize}

\end{solution}

\begin{problem}[Quadrant Determination]
The Argand diagram shows $z$ in the first quadrant and $w$ in the second quadrant. Which complex number could lie in the 3rd quadrant: $-w$, $2iz$, $\bar{z}$, or $w - z$?
\end{problem}

\begin{hint}
Check the signs of the real and imaginary parts of each option. Third quadrant means both parts are negative.
\end{hint}

\begin{solution}[Sketch]
If $z$ is in Q1: $\text{Re}(z)>0, \text{Im}(z)>0$. If $w$ is in Q2: $\text{Re}(w)<0, \text{Im}(w)>0$. Then $w-z$ has $\text{Re}(w-z) = \text{Re}(w)-\text{Re}(z) < 0$ and $\text{Im}(w-z) = \text{Im}(w)-\text{Im}(z)$ which could be negative if $\text{Im}(z) > \text{Im}(w)$. Answer: (D) $w-z$.
\end{solution}

\begin{problem}[Doubling the Argument]
A complex number $z$ is on the unit circle at angle $\theta = 40°$. Which diagram best shows the location of $\frac{z^2}{|z|}$?

\vspace{0.5em}
\begin{center}
\begin{tikzpicture}[scale=0.8,>=Stealth]
  % OPTION A (CORRECT): z² at angle 2θ
  \begin{scope}
    % Draw axes
    \draw[->,thick] (-1.2,0) -- (1.3,0) node[right,font=\tiny] {Re};
    \draw[->,thick] (0,-1.2) -- (0,1.3) node[above,font=\tiny] {Im};
    % Draw unit circle
    \draw[lawpurple,thick] (0,0) circle (1);
    % z at 40°
    \coordinate (Z1) at ({cos(40)},{sin(40)});
    \draw[->,thick,bookpurple] (0,0) -- (Z1);
    \fill[blue] (Z1) circle (1.5pt);
    \node[above right,font=\scriptsize] at (Z1) {$z$};
    % z² at 80° (2θ)
    \coordinate (Z2) at ({cos(80)},{sin(80)});
    \draw[->,thick,bookred] (0,0) -- (Z2);
    \fill[red] (Z2) circle (1.5pt);
    \node[above left,font=\scriptsize] at (Z2) {$z^2$};
    % Angles
    \draw[bookpurple!70,thin] (0.3,0) arc (0:40:0.3);
    \node[bookpurple!70,font=\tiny] at (0.45,0.12) {$40°$};
    \draw[bookred!70,thin] (0.45,0) arc (0:80:0.45);
    \node[bookred!70,font=\tiny] at (0.52,0.32) {$80°$};
    % Label
    \node[font=\small\bfseries] at (0,-1.5) {(A)};
  \end{scope}
  
  % OPTION B: z² at angle θ/2 (halving)
  \begin{scope}[xshift=3.2cm]
    % Draw axes
    \draw[->,thick] (-1.2,0) -- (1.3,0) node[right,font=\tiny] {Re};
    \draw[->,thick] (0,-1.2) -- (0,1.3) node[above,font=\tiny] {Im};
    % Draw unit circle
    \draw[lawpurple,thick] (0,0) circle (1);
    % z at 40°
    \coordinate (Z1) at ({cos(40)},{sin(40)});
    \draw[->,thick,bookpurple] (0,0) -- (Z1);
    \fill[blue] (Z1) circle (1.5pt);
    \node[above right,font=\scriptsize] at (Z1) {$z$};
    % z² at 20° (θ/2)
    \coordinate (Z2) at ({cos(20)},{sin(20)});
    \draw[->,thick,bookred] (0,0) -- (Z2);
    \fill[red] (Z2) circle (1.5pt);
    \node[above right,font=\scriptsize] at (Z2) {$z^2$};
    % Angles
    \draw[bookpurple!70,thin] (0.3,0) arc (0:40:0.3);
    \node[bookpurple!70,font=\tiny] at (0.45,0.12) {$40°$};
    \draw[bookred!70,thin] (0.55,0) arc (0:20:0.55);
    \node[bookred!70,font=\tiny] at (0.7,0.1) {$20°$};
    % Label
    \node[font=\small\bfseries] at (0,-1.5) {(B)};
  \end{scope}
  
  % OPTION C: z² at angle θ + 90°
  \begin{scope}[xshift=6.4cm]
    % Draw axes
    \draw[->,thick] (-1.2,0) -- (1.3,0) node[right,font=\tiny] {Re};
    \draw[->,thick] (0,-1.2) -- (0,1.3) node[above,font=\tiny] {Im};
    % Draw unit circle
    \draw[lawpurple,thick] (0,0) circle (1);
    % z at 40°
    \coordinate (Z1) at ({cos(40)},{sin(40)});
    \draw[->,thick,bookpurple] (0,0) -- (Z1);
    \fill[blue] (Z1) circle (1.5pt);
    \node[above right,font=\scriptsize] at (Z1) {$z$};
    % z² at 130° (40° + 90°)
    \coordinate (Z2) at ({cos(130)},{sin(130)});
    \draw[->,thick,bookred] (0,0) -- (Z2);
    \fill[red] (Z2) circle (1.5pt);
    \node[above left,font=\scriptsize] at (Z2) {$z^2$};
    % Angles
    \draw[bookpurple!70,thin] (0.3,0) arc (0:40:0.3);
    \node[bookpurple!70,font=\tiny] at (0.45,0.12) {$40°$};
    \draw[bookred!70,thin] (0.4,0) arc (0:130:0.4);
    \node[bookred!70,font=\tiny] at (-0.1,0.5) {$130°$};
    % Label
    \node[font=\small\bfseries] at (0,-1.5) {(C)};
  \end{scope}
  
  % OPTION D: z² at angle -θ (reflection)
  \begin{scope}[xshift=9.6cm]
    % Draw axes
    \draw[->,thick] (-1.2,0) -- (1.3,0) node[right,font=\tiny] {Re};
    \draw[->,thick] (0,-1.2) -- (0,1.3) node[above,font=\tiny] {Im};
    % Draw unit circle
    \draw[lawpurple,thick] (0,0) circle (1);
    % z at 40°
    \coordinate (Z1) at ({cos(40)},{sin(40)});
    \draw[->,thick,bookpurple] (0,0) -- (Z1);
    \fill[blue] (Z1) circle (1.5pt);
    \node[above right,font=\scriptsize] at (Z1) {$z$};
    % z² at -40° (reflection)
    \coordinate (Z2) at ({cos(-40)},{sin(-40)});
    \draw[->,thick,bookred] (0,0) -- (Z2);
    \fill[red] (Z2) circle (1.5pt);
    \node[below right,font=\scriptsize] at (Z2) {$z^2$};
    % Angles
    \draw[bookpurple!70,thin] (0.3,0) arc (0:40:0.3);
    \node[bookpurple!70,font=\tiny] at (0.45,0.12) {$40°$};
    \draw[bookred!70,thin] (0.45,0) arc (0:-40:0.45);
    \node[bookred!70,font=\tiny] at (0.58,-0.15) {$-40°$};
    % Label
    \node[font=\small\bfseries] at (0,-1.5) {(D)};
  \end{scope}
\end{tikzpicture}
\end{center}
\vspace{0.5em}
\end{problem}

\begin{hint}
If $z = e^{i\theta}$, then $z^2 = e^{i2\theta}$ and $|z| = 1$.
\end{hint}

\begin{solution}[Sketch]
Since $|z| = 1$, we have $\frac{z^2}{|z|} = z^2$. By De Moivre's theorem, $z^2$ has the same modulus but double the argument. If $z$ is at angle $\theta = 40°$, then $z^2$ is at angle $2\theta = 80°$.

\textbf{Analysis of options:}
\begin{itemize}
\item[(A)] \textbf{CORRECT}: Shows $z^2$ at angle $80° = 2 \times 40°$. This correctly applies De Moivre's theorem.
\item[(B)] Incorrect: Shows $z^2$ at angle $20° = 40°/2$. This halves the angle instead of doubling it.
\item[(C)] Incorrect: Shows $z^2$ at angle $130° = 40° + 90°$. This adds a fixed angle, which would represent multiplication by $i$.
\item[(D)] Incorrect: Shows $z^2$ at angle $-40°$. This reflects across the real axis, which would give the conjugate $\bar{z}$.
\end{itemize}

\textbf{Answer: (A)}
\end{solution}

\begin{problem}[Polynomial with Complex Zero]
Which polynomial could have $2+i$ as a zero, given that $k$ is real: (A) $x^3 - 4x^2 + kx$, (B) $x^3 - 4x^2 + kx + 5$, (C) $x^3 - 5x^2 + kx$, (D) $x^3 - 5x^2 + kx + 5$?
\end{problem}

\begin{hint}
If the polynomial has real coefficients and $2+i$ is a root, then $2-i$ must also be a root. The sum of these roots is $4$.
\end{hint}

\begin{solution}[Sketch]
Roots $2+i$ and $2-i$ sum to $4$. Let the third root be $r$. Then sum of all roots $= 4+r$. From the coefficient of $x^2$: $-(4+r) = -4$ or $-5$. If $-4$, then $r=0$, giving option (A) or (B). Check: $(x)(x^2-4x+5) = x^3-4x^2+5x$, but we need $+5$ constant. Answer: (B).
\end{solution}

\begin{problem}[Squaring Complex Numbers]
Let $z = 3+i$ and $w = 2-5i$. Find $z^2$ in the form $x+iy$.
\end{problem}

\begin{hint}
Use $(a+bi)^2 = a^2 - b^2 + 2abi$.
\end{hint}

\begin{solution}[Sketch]
$z^2 = (3+i)^2 = 9 + 6i + i^2 = 9 + 6i - 1 = 8 + 6i$.
\end{solution}

\begin{problem}[Conjugate Subtraction]
Let $z = 4+i$ and $w = \bar{z}$. Find $w-z$ in the form $x+iy$.
\end{problem}

\begin{hint}
If $z = 4+i$, then $\bar{z} = 4-i$.
\end{hint}

\begin{solution}[Sketch]
$w = \bar{z} = 4-i$. Then $w-z = (4-i) - (4+i) = -2i$.
\end{solution}

\begin{problem}[Division by Complex Number]
Let $z = 2+i$. Find $\frac{4}{z}$ in the form $x+iy$.
\end{problem}

\begin{hint}
Multiply numerator and denominator by $\bar{z} = 2-i$.
\end{hint}

\begin{solution}[Sketch]
$\frac{4}{2+i} = \frac{4(2-i)}{(2+i)(2-i)} = \frac{8-4i}{4-i^2} = \frac{8-4i}{5} = \frac{8}{5} - \frac{4}{5}i$.
\end{solution}

\begin{problem}[Complex Multiplication]
Let $z = 3+i$ and $w = 1-i$. Find $zw$ in the form $x+iy$.
\end{problem}

\begin{hint}
Use FOIL: $(a+bi)(c+di) = ac - bd + (ad+bc)i$.
\end{hint}

\begin{solution}[Sketch]
$zw = (3+i)(1-i) = 3 - 3i + i - i^2 = 3 - 2i + 1 = 4 - 2i$.
\end{solution}

\begin{problem}[Another Division]
Let $w = 1-i$. Find $\frac{6}{w}$ in the form $x+iy$.
\end{problem}

\begin{hint}
Multiply by conjugate: $\bar{w} = 1+i$.
\end{hint}

\begin{solution}[Sketch]
$\frac{6}{1-i} = \frac{6(1+i)}{(1-i)(1+i)} = \frac{6+6i}{1-i^2} = \frac{6+6i}{2} = 3+3i$.
\end{solution}

\begin{problem}[Dividing by Conjugate]
Let $z = 4+i$ and $w = \bar{z}$. Find $\frac{z}{w}$ in the form $x+iy$.
\end{problem}

\begin{hint}
$w = 4-i$, so $\frac{z}{w} = \frac{4+i}{4-i}$.
\end{hint}

\begin{solution}[Sketch]
$\frac{4+i}{4-i} = \frac{(4+i)(4+i)}{(4-i)(4+i)} = \frac{16+8i+i^2}{16-i^2} = \frac{15+8i}{17} = \frac{15}{17} + \frac{8}{17}i$.
\end{solution}

\begin{problem}[Modulus from Polar Form]
Let $z = \frac{1}{2}(\cos \theta + i \sin \theta)$. Find $|z|$.
\end{problem}

\begin{hint}
For $z = r(\cos\theta + i\sin\theta)$, we have $|z| = r$.
\end{hint}

\begin{solution}[Sketch]
$|z| = \frac{1}{2}$.
\end{solution}

\begin{problem}[Division in Cartesian Form]
Express $\frac{3-i}{2+i}$ in the form $x+iy$.
\end{problem}

\begin{hint}
Multiply by $\frac{2-i}{2-i}$.
\end{hint}

\begin{solution}[Sketch]
$\frac{3-i}{2+i} = \frac{(3-i)(2-i)}{(2+i)(2-i)} = \frac{6-3i-2i+i^2}{4-i^2} = \frac{5-5i}{5} = 1-i$.
\end{solution}

\begin{problem}[Addition with Conjugate]
Let $z=1+3i$ and $w=2-i$. Find $z+\bar{w}$.
\end{problem}

\begin{hint}
$\bar{w} = 2+i$.
\end{hint}

\begin{solution}[Sketch]
$z+\bar{w} = (1+3i) + (2+i) = 3+4i$.
\end{solution}

\begin{problem}[Product with Conjugate]
Evaluate $w\bar{z}$ given $w=-1+4i$ and $z=2-i$.
\end{problem}

\begin{hint}
$\bar{z} = 2+i$.
\end{hint}

\begin{solution}[Sketch]
$w\bar{z} = (-1+4i)(2+i) = -2-i+8i+4i^2 = -2+7i-4 = -6+7i$.
\end{solution}

\begin{problem}[Polar Form of Given Number]
Let $w=1+i\sqrt{3}$. Express $w$ in modulus-argument form.
\end{problem}

\begin{hint}
$|w| = \sqrt{1+3} = 2$, and $\arg(w) = \tan^{-1}(\sqrt{3}/1) = \pi/3$.
\end{hint}

\begin{solution}[Sketch]
$w = 2\left(\cos\frac{\pi}{3} + i\sin\frac{\pi}{3}\right)$.
\end{solution}

\begin{problem}[Showing a Power is Real]
Let $z = \sqrt{3}-i$. Show that $z^6$ is real.
\end{problem}

\begin{hint}
Convert to polar form: $z = 2(\cos(-\pi/6) + i\sin(-\pi/6))$.
\end{hint}

\begin{solution}[Sketch]
$z^6 = 2^6(\cos(-\pi) + i\sin(-\pi)) = 64(-1 + 0i) = -64$, which is real.
\end{solution}

\begin{problem}[Principal Argument]
Evaluate $\text{Arg}(z)$ for $z=-\sqrt{3}+i$.
\end{problem}

\begin{hint}
$z$ is in the second quadrant. Use $\tan\theta = \frac{1}{-\sqrt{3}} = -\frac{1}{\sqrt{3}}$.
\end{hint}

\begin{solution}[Sketch]
Reference angle is $\pi/6$. Since $z$ is in Q2, $\text{Arg}(z) = \pi - \pi/6 = \frac{5\pi}{6}$.
\end{solution}

\begin{problem}[Polynomial with Complex Root]
It is given that $z = 2 + i$ is a root of $z^3 + az^2 - 7z + 15 = 0$, where $a$ is a real number.

What is the value of $a$?
\begin{enumerate}[label=(\Alph*),leftmargin=*]
  \item $-1$
  \item $1$
  \item $7$
  \item $-7$
\end{enumerate}
\end{problem}

\begin{hint}
Since the polynomial has real coefficients and $2+i$ is a root, then $2-i$ must also be a root. Use Vieta's formulas or substitute the root directly.
\end{hint}

\begin{solution}[Sketch]
\textbf{Method 1: Direct substitution}

Since $z = 2+i$ is a root, substitute into the equation:
\begin{align*}
(2+i)^3 + a(2+i)^2 - 7(2+i) + 15 &= 0
\end{align*}

Calculate $(2+i)^2$:
\[
(2+i)^2 = 4 + 4i + i^2 = 4 + 4i - 1 = 3 + 4i
\]

Calculate $(2+i)^3 = (2+i)(3+4i)$:
\[
(2+i)(3+4i) = 6 + 8i + 3i + 4i^2 = 6 + 11i - 4 = 2 + 11i
\]

Substitute into the equation:
\begin{align*}
(2+11i) + a(3+4i) - 7(2+i) + 15 &= 0\\
(2+11i) + (3a + 4ai) + (-14-7i) + 15 &= 0\\
(2 + 3a - 14 + 15) + (11 + 4a - 7)i &= 0\\
(3 + 3a) + (4 + 4a)i &= 0
\end{align*}

For this to equal zero, both real and imaginary parts must be zero:
\begin{align*}
\text{Real part: } &3 + 3a = 0 \implies a = -1\\
\text{Imaginary part: } &4 + 4a = 0 \implies a = -1
\end{align*}

\textbf{Method 2: Using conjugate roots}

Since the polynomial has real coefficients, if $2+i$ is a root, then $2-i$ is also a root.

Let the third root be $r$. By Vieta's formulas, the sum of roots equals $-a$:
\[
(2+i) + (2-i) + r = -a
\]
\[
4 + r = -a
\]

The product of roots equals $-15$:
\[
(2+i)(2-i) \cdot r = -15
\]
\[
(4 - i^2) \cdot r = -15
\]
\[
5r = -15
\]
\[
r = -3
\]

Therefore: $4 + (-3) = -a \implies a = -1$

\textbf{Answer: (A) $a = -1$}
\end{solution}

\begin{problem}[Power Using De Moivre's Theorem]
\begin{enumerate}[label=(\roman*),leftmargin=*]
  \item Write the number $\sqrt{3} + i$ in modulus-argument form.
  \item Hence, or otherwise, write $(\sqrt{3} + i)^7$ in exact Cartesian form.
\end{enumerate}
\end{problem}

\begin{hint}
For part (i): Find $|z| = \sqrt{3+1} = 2$ and $\arg(z) = \tan^{-1}(1/\sqrt{3}) = \pi/6$. For part (ii): Use De Moivre's theorem to find $(re^{i\theta})^7 = r^7e^{i7\theta}$.
\end{hint}

\begin{solution}[Sketch]
\textbf{Part (i): Convert to modulus-argument form}

Let $z = \sqrt{3} + i$.

Calculate the modulus:
\[
|z| = \sqrt{(\sqrt{3})^2 + 1^2} = \sqrt{3 + 1} = \sqrt{4} = 2
\]

Calculate the argument. Since $z$ is in the first quadrant:
\[
\arg(z) = \tan^{-1}\left(\frac{1}{\sqrt{3}}\right) = \tan^{-1}\left(\frac{1}{\sqrt{3}}\right) = \frac{\pi}{6}
\]

Therefore, in modulus-argument form:
\[
z = 2\left(\cos\frac{\pi}{6} + i\sin\frac{\pi}{6}\right) = 2e^{i\pi/6}
\]

\textbf{Part (ii): Calculate $z^7$ using De Moivre's theorem}

By De Moivre's theorem:
\begin{align*}
z^7 &= \left[2\left(\cos\frac{\pi}{6} + i\sin\frac{\pi}{6}\right)\right]^7\\
&= 2^7\left(\cos\frac{7\pi}{6} + i\sin\frac{7\pi}{6}\right)\\
&= 128\left(\cos\frac{7\pi}{6} + i\sin\frac{7\pi}{6}\right)
\end{align*}

Calculate the trigonometric values. The angle $\frac{7\pi}{6}$ is in the third quadrant (reference angle $\frac{\pi}{6}$):
\[
\cos\frac{7\pi}{6} = -\cos\frac{\pi}{6} = -\frac{\sqrt{3}}{2}
\]
\[
\sin\frac{7\pi}{6} = -\sin\frac{\pi}{6} = -\frac{1}{2}
\]

Therefore:
\begin{align*}
z^7 &= 128\left(-\frac{\sqrt{3}}{2} + i\left(-\frac{1}{2}\right)\right)\\
&= 128\left(-\frac{\sqrt{3}}{2} - \frac{i}{2}\right)\\
&= -64\sqrt{3} - 64i
\end{align*}

\textbf{Answer:} $(\sqrt{3} + i)^7 = -64\sqrt{3} - 64i$

\textbf{Verification (optional):} We can verify that $|z^7| = |z|^7 = 2^7 = 128$ and $\arg(z^7) = 7\arg(z) = 7 \cdot \frac{\pi}{6} = \frac{7\pi}{6}$ (in the range $(-\pi, \pi]$).
\end{solution}

