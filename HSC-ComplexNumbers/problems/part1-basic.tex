\begin{problem}[Basic Complex Arithmetic]
Let $z = 5 - i$ and $w = 2 + 3i$. What is the value of $2z + \overline{w}$?
\end{problem}

\begin{solution}
First, compute $2z$:
\[
2z = 2(5 - i) = 10 - 2i.
\]
Next, find the conjugate of $w$:
\[
\overline{w} = \overline{2 + 3i} = 2 - 3i.
\]
Now add these results:
\[
2z + \overline{w} = (10 - 2i) + (2 - 3i) = 12 - 5i.
\]
Therefore, $2z + \overline{w} = 12 - 5i$.
\end{solution}

\begin{takeaways}
The conjugate of $a+bi$ is $a-bi$. When adding complex numbers, combine real parts with real parts and imaginary parts with imaginary parts.
\end{takeaways}

\begin{problem}[Finding Square Roots]
What value of $z$ satisfies $z^2 = 7 - 24i$?
\end{problem}

\begin{solution}
Let $z = a + bi$ where $a, b$ are real. Then:
\[
z^2 = (a+bi)^2 = a^2 - b^2 + 2abi = 7 - 24i.
\]
Equating real and imaginary parts:
\begin{align*}
a^2 - b^2 &= 7 \\
2ab &= -24.
\end{align*}
From the second equation: $ab = -12$, so $b = -\frac{12}{a}$.

Substitute into the first equation:
\[
a^2 - \left(-\frac{12}{a}\right)^2 = 7 \implies a^2 - \frac{144}{a^2} = 7.
\]
Multiply by $a^2$:
\[
a^4 - 7a^2 - 144 = 0.
\]
Let $u = a^2$:
\[
u^2 - 7u - 144 = 0 \implies (u-16)(u+9) = 0.
\]
Since $u = a^2 \ge 0$, we have $u = 16$, so $a^2 = 16$ giving $a = \pm 4$.

If $a = 4$: $b = -\frac{12}{4} = -3$, so $z = 4 - 3i$.

If $a = -4$: $b = -\frac{12}{-4} = 3$, so $z = -4 + 3i$.

Verify: $(4-3i)^2 = 16 - 24i + 9i^2 = 16 - 24i - 9 = 7 - 24i$. \checkmark

Therefore, $z = 4 - 3i$ or $z = -4 + 3i$.
\end{solution}

\begin{takeaways}
To find square roots of complex numbers in Cartesian form, let $z = a+bi$ and equate real and imaginary parts after expanding $z^2$. This gives a system of two equations in two unknowns.
\end{takeaways}

\begin{problem}[Complex Roots of Quadratics]
Given that $z = 3 + i$ is a root of $z^2 + pz + q = 0$, where $p$ and $q$ are real, what are the values of $p$ and $q$?
\end{problem}

\begin{solution}
Since the polynomial has real coefficients and $z = 3 + i$ is a root, its complex conjugate $\overline{z} = 3 - i$ must also be a root.

By Vieta's formulas:
\[
p = -(z_1 + z_2) = -\bigl((3+i) + (3-i)\bigr) = -6,
\]
\[
q = z_1 \cdot z_2 = (3+i)(3-i) = 9 - i^2 = 9 - (-1) = 10.
\]
Therefore, $p = -6$ and $q = 10$.
\end{solution}

\begin{takeaways}
For polynomials with real coefficients, complex roots occur in conjugate pairs. Use Vieta's formulas: sum of roots $= -p$ and product of roots $= q$ for $z^2 + pz + q = 0$.
\end{takeaways}

\begin{problem}[Powers of $i$]
Write $i^9$ in the form $a+ib$ where $a$ and $b$ are real.
\end{problem}

\begin{solution}
Note the pattern of powers of $i$:
\begin{align*}
i^1 &= i \\
i^2 &= -1 \\
i^3 &= i^2 \cdot i = -i \\
i^4 &= (i^2)^2 = 1 \\
i^5 &= i^4 \cdot i = i
\end{align*}
The pattern repeats every 4 powers. To find $i^9$:
\[
9 = 4 \cdot 2 + 1,
\]
so $i^9 = i^{4 \cdot 2 + 1} = (i^4)^2 \cdot i = 1^2 \cdot i = i$.

Therefore, $i^9 = 0 + 1i$, giving $a = 0$ and $b = 1$.
\end{solution}

\begin{takeaways}
Powers of $i$ repeat with period 4: $i^1 = i$, $i^2 = -1$, $i^3 = -i$, $i^4 = 1$. Use division by 4 to find $i^n = i^{n \bmod 4}$.
\end{takeaways}

\begin{problem}[Polar Form Conversion]
Write $1+i$ in the form $r(\cos \theta + i \sin \theta)$.
\end{problem}

\begin{solution}
For $z = 1 + i$, first find the modulus:
\[
r = |z| = \sqrt{1^2 + 1^2} = \sqrt{2}.
\]
Next, find the argument. Since $z$ is in the first quadrant:
\[
\theta = \arg(z) = \tan^{-1}\left(\frac{1}{1}\right) = \tan^{-1}(1) = \frac{\pi}{4}.
\]
Therefore:
\[
1 + i = \sqrt{2}\left(\cos\frac{\pi}{4} + i\sin\frac{\pi}{4}\right).
\]
\end{solution}

\begin{takeaways}
To convert $z = x+iy$ to polar form: (1) Find modulus $r = \sqrt{x^2+y^2}$. (2) Find argument $\theta = \tan^{-1}(y/x)$ (adjust for quadrant). (3) Write $z = r(\cos\theta + i\sin\theta)$.
\end{takeaways}

