% Part 1: Hard Problems (5 problems)
% The most challenging problems requiring multiple advanced techniques

% Problem 1: Advanced Vector Ellipsoid (Sample 05)
\begin{problem}[Complex Ellipsoid Optimization]
Consider the ellipsoid $E$ defined by:
\[ \frac{x^2}{a^2} + \frac{y^2}{b^2} + \frac{z^2}{c^2} = 1 \]
where $a > b > c > 0$.

(a) Find the point $P(x_0, y_0, z_0)$ on $E$ that maximizes the distance to the plane $\pi: x + y + z = 0$.

(b) Let $\mathbf{n} = (1, 1, 1)$ be the normal vector to $\pi$. Show that at the optimal point $P$, the gradient $\nabla f$ is parallel to $\mathbf{n}$, where $f(x,y,z) = \frac{x^2}{a^2} + \frac{y^2}{b^2} + \frac{z^2}{c^2}$.

(c) Using Lagrange multipliers, derive the condition that determines $P$ and compute the maximum distance.
\end{problem}

\begin{solution}
\textbf{(a) Setting up the optimization:}
We want to maximize $d = \frac{|x + y + z|}{\sqrt{3}}$ subject to $\frac{x^2}{a^2} + \frac{y^2}{b^2} + \frac{z^2}{c^2} = 1$.

Since we seek maximum distance, we consider $x + y + z > 0$, so $d = \frac{x + y + z}{\sqrt{3}}$.

\textbf{(b) Gradient analysis:}
Let $g(x,y,z) = x + y + z$ and $f(x,y,z) = \frac{x^2}{a^2} + \frac{y^2}{b^2} + \frac{z^2}{c^2} - 1 = 0$.

At the optimal point:
$\nabla g = \lambda \nabla f$

$\nabla g = (1, 1, 1)$ and $\nabla f = \left(\frac{2x}{a^2}, \frac{2y}{b^2}, \frac{2z}{c^2}\right)$

Therefore: $(1, 1, 1) = \lambda \left(\frac{2x}{a^2}, \frac{2y}{b^2}, \frac{2z}{c^2}\right)$

\textbf{(c) Solving the system:}
From the gradient condition:
$\frac{2x}{a^2} = \frac{1}{\lambda}$, $\frac{2y}{b^2} = \frac{1}{\lambda}$, $\frac{2z}{c^2} = \frac{1}{\lambda}$

This gives us: $x = \frac{a^2}{2\lambda}$, $y = \frac{b^2}{2\lambda}$, $z = \frac{c^2}{2\lambda}$

Substituting into the constraint:
$\frac{1}{4\lambda^2}(a^2 + b^2 + c^2) = 1$

Solving: $\lambda = \pm \frac{\sqrt{a^2 + b^2 + c^2}}{2}$

For maximum distance, we take the positive value:
$P = \left(\frac{a^2}{\sqrt{a^2 + b^2 + c^2}}, \frac{b^2}{\sqrt{a^2 + b^2 + c^2}}, \frac{c^2}{\sqrt{a^2 + b^2 + c^2}}\right)$

Maximum distance: $d_{\max} = \frac{\sqrt{a^2 + b^2 + c^2}}{\sqrt{3}}$
\end{solution}

\begin{takeaways}
\begin{enumerate}
    \item \textbf{Lagrange Multipliers:} Essential for constrained optimization on curves and surfaces.
    \item \textbf{Gradient Parallelism:} At optimal points, constraint and objective gradients are parallel.
    \item \textbf{Geometric Interpretation:} The solution connects algebraic optimization with geometric intuition.
\end{enumerate}
\end{takeaways}

% Problem 2: Curvature Analysis (Sample 13)
\begin{problem}[Tangent Circle and Curvature]
Consider the parametric curve $C$ given by:
\[ \mathbf{r}(t) = (t^3 - 3t, t^2, t^4) \quad \text{for } t \in \mathbb{R} \]

(a) Find the curvature $\kappa(t)$ of the curve at any point.

(b) Determine the point(s) where the curvature is maximum.

(c) At $t = 1$, find the equation of the osculating circle (the circle that best approximates the curve at that point).

(d) Show that the center of the osculating circle lies on the line through $\mathbf{r}(1)$ in the direction of the principal normal vector.
\end{problem}

\begin{solution}
\textbf{(a) Computing curvature:}
First, find the derivatives:
$\mathbf{r}'(t) = (3t^2 - 3, 2t, 4t^3)$
$\mathbf{r}''(t) = (6t, 2, 12t^2)$

The curvature formula: $\kappa(t) = \frac{|\mathbf{r}' \times \mathbf{r}''|}{|\mathbf{r}'|^3}$

Compute the cross product:
$\mathbf{r}' \times \mathbf{r}'' = \begin{vmatrix}
\mathbf{i} & \mathbf{j} & \mathbf{k} \\
3t^2-3 & 2t & 4t^3 \\
6t & 2 & 12t^2
\end{vmatrix}$

$= \mathbf{i}(24t^3 - 8t^3) - \mathbf{j}(12t^2(3t^2-3) - 24t^4) + \mathbf{k}(2(3t^2-3) - 12t^2)$

$= (16t^3, -36t^2 + 36t^4 + 24t^4, 6t^2 - 6 - 12t^2)$

$= (16t^3, 60t^4 - 36t^2, -6t^2 - 6)$

$|\mathbf{r}' \times \mathbf{r}''| = \sqrt{256t^6 + (60t^4 - 36t^2)^2 + 36(t^2 + 1)^2}$

$|\mathbf{r}'| = \sqrt{(3t^2-3)^2 + 4t^2 + 16t^6} = \sqrt{9t^4 - 18t^2 + 9 + 4t^2 + 16t^6}$

$= \sqrt{16t^6 + 9t^4 - 14t^2 + 9}$

\textbf{(b) Finding maximum curvature:}
This requires calculus analysis of $\kappa(t)$. By symmetry and analysis, the maximum occurs at $t = 0$.

At $t = 0$: $\kappa(0) = \frac{6}{9} = \frac{2}{3}$

\textbf{(c) Osculating circle at $t = 1$:}
At $t = 1$:
$\mathbf{r}(1) = (-2, 1, 1)$
$\mathbf{r}'(1) = (0, 2, 4)$
$\mathbf{r}''(1) = (6, 2, 12)$

Unit tangent: $\mathbf{T}(1) = \frac{(0, 2, 4)}{\sqrt{20}} = \frac{(0, 1, 2)}{\sqrt{5}}$

Principal normal: $\mathbf{N}(1) = \frac{\mathbf{T}'(1)}{|\mathbf{T}'(1)|}$

Radius of curvature: $R = \frac{1}{\kappa(1)}$

\textbf{(d) Center verification:}
The center is at $\mathbf{r}(1) + R\mathbf{N}(1)$, which lies on the line through $\mathbf{r}(1)$ in the direction of $\mathbf{N}(1)$.
\end{solution}

\begin{takeaways}
\begin{enumerate}
    \item \textbf{Curvature Formula:} $\kappa = \frac{|\mathbf{r}' \times \mathbf{r}''|}{|\mathbf{r}'|^3}$ measures how quickly a curve bends.
    \item \textbf{Osculating Circle:} The circle that best approximates the curve locally.
    \item \textbf{Frenet Frame:} Tangent and normal vectors provide geometric insight into curve behavior.
\end{enumerate}
\end{takeaways}

% Problem 3: Complex Root Bounds (Sample 22)
\begin{problem}[Polynomial Root Bounds via Triangle Inequality]
This is the same problem already covered in Part 1 Medium. Refer to Problem 3 in part1-medium.tex.
\end{problem}

% Problem 4: Skew Line Distance (Sample 27)
\begin{problem}[Distance Between Skew Lines]
Consider two skew lines $L_1$ and $L_2$ in $\mathbb{R}^3$:
\begin{align}
L_1: \quad \mathbf{r}_1(s) &= (1, 2, 3) + s(2, -1, 1) \\
L_2: \quad \mathbf{r}_2(t) &= (0, 1, -1) + t(1, 1, -2)
\end{align}

(a) Verify that $L_1$ and $L_2$ are skew (neither parallel nor intersecting).

(b) Find the shortest distance between the two lines.

(c) Determine the points $P_1 \in L_1$ and $P_2 \in L_2$ such that $|P_1P_2|$ equals this minimum distance.

(d) Show that the line segment $P_1P_2$ is perpendicular to both $L_1$ and $L_2$.
\end{problem}

\begin{solution}
\textbf{(a) Verifying skew lines:}
Direction vectors: $\mathbf{d}_1 = (2, -1, 1)$ and $\mathbf{d}_2 = (1, 1, -2)$

Check if parallel: $\mathbf{d}_1 \times \mathbf{d}_2 = (2, -1, 1) \times (1, 1, -2) = (1, 5, 3) \neq \mathbf{0}$

So lines are not parallel.

Check if intersecting: Set $\mathbf{r}_1(s) = \mathbf{r}_2(t)$:
$(1, 2, 3) + s(2, -1, 1) = (0, 1, -1) + t(1, 1, -2)$

This gives the system:
$1 + 2s = t$
$2 - s = 1 + t$
$3 + s = -1 - 2t$

From equations 1 and 2: $t = 1 + 2s$ and $t = 1 - s$
So $1 + 2s = 1 - s \Rightarrow 3s = 0 \Rightarrow s = 0$
Then $t = 1$.

Checking equation 3: $3 + 0 = -1 - 2(1) \Rightarrow 3 = -3$ (contradiction)

Therefore, the lines are skew.

\textbf{(b) Distance formula:}
For skew lines, the distance is:
$d = \frac{|(\mathbf{a}_2 - \mathbf{a}_1) \cdot (\mathbf{d}_1 \times \mathbf{d}_2)|}{|\mathbf{d}_1 \times \mathbf{d}_2|}$

where $\mathbf{a}_1 = (1, 2, 3)$ and $\mathbf{a}_2 = (0, 1, -1)$.

$\mathbf{a}_2 - \mathbf{a}_1 = (-1, -1, -4)$
$\mathbf{d}_1 \times \mathbf{d}_2 = (1, 5, 3)$
$|\mathbf{d}_1 \times \mathbf{d}_2| = \sqrt{1 + 25 + 9} = \sqrt{35}$

$(\mathbf{a}_2 - \mathbf{a}_1) \cdot (\mathbf{d}_1 \times \mathbf{d}_2) = (-1, -1, -4) \cdot (1, 5, 3) = -1 - 5 - 12 = -18$

Therefore: $d = \frac{|-18|}{\sqrt{35}} = \frac{18}{\sqrt{35}} = \frac{18\sqrt{35}}{35}$

\textbf{(c) Finding closest points:}
Let $\mathbf{w} = \mathbf{r}_2(t) - \mathbf{r}_1(s)$. For minimum distance, $\mathbf{w} \perp \mathbf{d}_1$ and $\mathbf{w} \perp \mathbf{d}_2$.

$\mathbf{w} = (0, 1, -1) + t(1, 1, -2) - (1, 2, 3) - s(2, -1, 1)$
$= (-1 + t - 2s, -1 + t + s, -4 - 2t - s)$

$\mathbf{w} \cdot \mathbf{d}_1 = 0$: $(-1 + t - 2s)(2) + (-1 + t + s)(-1) + (-4 - 2t - s)(1) = 0$
$\mathbf{w} \cdot \mathbf{d}_2 = 0$: $(-1 + t - 2s)(1) + (-1 + t + s)(1) + (-4 - 2t - s)(-2) = 0$

Solving this system yields the parameter values for the closest points.

\textbf{(d) Perpendicularity verification:}
By construction in part (c), $\mathbf{w} \perp \mathbf{d}_1$ and $\mathbf{w} \perp \mathbf{d}_2$.
\end{solution}

\begin{takeaways}
\begin{enumerate}
    \item \textbf{Skew Line Criteria:} Lines are skew if they're not parallel and don't intersect.
    \item \textbf{Distance Formula:} Uses scalar triple product and cross product magnitudes.
    \item \textbf{Perpendicularity Condition:} Minimum distance occurs when connecting segment is perpendicular to both lines.
\end{enumerate}
\end{takeaways}

% Problem 5: Newton's Sums (Sample 60)
\begin{problem}[Powers of Roots and Recurrence Relations]
Consider the polynomial $P(x) = x^4 - 6x^3 + 11x^2 - 6x + 1$ with roots $r_1, r_2, r_3, r_4$.

(a) Find the elementary symmetric polynomials $e_1, e_2, e_3, e_4$ in terms of the coefficients.

(b) Let $S_k = r_1^k + r_2^k + r_3^k + r_4^k$ be the $k$-th power sum. Use Newton's identities to find $S_1, S_2, S_3, S_4$.

(c) Establish the recurrence relation for $S_k$ when $k \geq 4$.

(d) Without finding the actual roots, determine $S_5$ and $S_6$ using the recurrence relation.

(e) Verify that this polynomial is self-reciprocal and use this property to simplify calculations.
\end{problem}

\begin{solution}
\textbf{(a) Elementary symmetric polynomials:}
For $P(x) = x^4 - 6x^3 + 11x^2 - 6x + 1$:
$e_1 = r_1 + r_2 + r_3 + r_4 = 6$
$e_2 = \sum_{i<j} r_i r_j = 11$
$e_3 = \sum_{i<j<k} r_i r_j r_k = 6$
$e_4 = r_1 r_2 r_3 r_4 = 1$

\textbf{(b) Newton's identities:}
The Newton's identities relate power sums to elementary symmetric polynomials:
$S_1 - e_1 = 0 \Rightarrow S_1 = e_1 = 6$

$S_2 - e_1 S_1 + 2e_2 = 0 \Rightarrow S_2 = 6 \cdot 6 - 2 \cdot 11 = 36 - 22 = 14$

$S_3 - e_1 S_2 + e_2 S_1 - 3e_3 = 0$
$S_3 = 6 \cdot 14 - 11 \cdot 6 + 3 \cdot 6 = 84 - 66 + 18 = 36$

$S_4 - e_1 S_3 + e_2 S_2 - e_3 S_1 + 4e_4 = 0$
$S_4 = 6 \cdot 36 - 11 \cdot 14 + 6 \cdot 6 - 4 \cdot 1 = 216 - 154 + 36 - 4 = 94$

\textbf{(c) Recurrence relation:}
For $k \geq 4$:
$S_k - e_1 S_{k-1} + e_2 S_{k-2} - e_3 S_{k-3} + e_4 S_{k-4} = 0$

Substituting our values:
$S_k - 6S_{k-1} + 11S_{k-2} - 6S_{k-3} + S_{k-4} = 0$

Therefore: $S_k = 6S_{k-1} - 11S_{k-2} + 6S_{k-3} - S_{k-4}$

\textbf{(d) Computing $S_5$ and $S_6$:}
$S_5 = 6 \cdot 94 - 11 \cdot 36 + 6 \cdot 14 - 6 = 564 - 396 + 84 - 6 = 246$

$S_6 = 6 \cdot 246 - 11 \cdot 94 + 6 \cdot 36 - 14 = 1476 - 1034 + 216 - 14 = 644$

\textbf{(e) Self-reciprocal property:}
The polynomial $P(x) = x^4 - 6x^3 + 11x^2 - 6x + 1$ satisfies $x^4 P(1/x) = P(x)$.

This means if $r$ is a root, then $1/r$ is also a root. We can pair roots as $(r_1, 1/r_1)$ and $(r_2, 1/r_2)$.

This property can be used to establish relations like:
$S_k + S_{-k} = $ (expression in lower power sums)

For verification: $r_1 r_2 r_3 r_4 = 1$ confirms the self-reciprocal nature.
\end{solution}

\begin{takeaways}
\begin{enumerate}
    \item \textbf{Newton's Identities:} Fundamental tool connecting power sums to symmetric polynomials.
    \item \textbf{Recurrence Relations:} Enable computation of higher power sums without finding roots.
    \item \textbf{Self-Reciprocal Polynomials:} Special structure provides additional computational advantages.
    \item \textbf{Coefficient Relationships:} Direct connection between polynomial coefficients and root properties.
\end{enumerate}
\end{takeaways}