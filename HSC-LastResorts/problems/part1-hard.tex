% Part 1: Hard Problems (5 problems)
% The most challenging problems requiring multiple advanced techniques

% Problem 1: Advanced Vector Ellipsoid (Sample 05)
\begin{problem}[Complex Ellipsoid Optimization]
Consider the ellipsoid $E$ defined by:
\[ \frac{x^2}{a^2} + \frac{y^2}{b^2} + \frac{z^2}{c^2} = 1 \]
where $a > b > c > 0$.

(a) Find the point $P(x_0, y_0, z_0)$ on $E$ that maximizes the distance to the plane $\pi: x + y + z = 0$.

(b) Let $\mathbf{n} = (1, 1, 1)$ be the normal vector to $\pi$. Show that at the optimal point $P$, the gradient $\nabla f$ is parallel to $\mathbf{n}$, where $f(x,y,z) = \frac{x^2}{a^2} + \frac{y^2}{b^2} + \frac{z^2}{c^2}$.

(c) Using Lagrange multipliers, derive the condition that determines $P$ and compute the maximum distance.
\end{problem}

\begin{solution}
We maximize the distance $d=\dfrac{x+y+z}{\sqrt{3}}$ under $\dfrac{x^2}{a^2}+\dfrac{y^2}{b^2}+\dfrac{z^2}{c^2}=1$ with $x+y+z>0$.

By Cauchy--Schwarz,
\[
    (x+y+z)^2 = \bigg(\tfrac{x}{a}a+\tfrac{y}{b}b+\tfrac{z}{c}c\bigg)^2
    \le \Big(\tfrac{x^2}{a^2}+\tfrac{y^2}{b^2}+\tfrac{z^2}{c^2}\Big)(a^2+b^2+c^2)
    = a^2+b^2+c^2,
\]
so $x+y+z \le \sqrt{a^2+b^2+c^2}$ with equality exactly when $\tfrac{x}{a}=\tfrac{y}{b}=\tfrac{z}{c}=k>0$. Writing $x=ka^2$, $y=kb^2$, $z=kc^2$ in the constraint gives $k^2(a^2+b^2+c^2)=1$, hence $k=1/\sqrt{a^2+b^2+c^2}$ and
\[
    P=\left( \frac{a^2}{\sqrt{a^2+b^2+c^2}},\; \frac{b^2}{\sqrt{a^2+b^2+c^2}},\; \frac{c^2}{\sqrt{a^2+b^2+c^2}} \right),
    \quad d_{\max}=\frac{\sqrt{a^2+b^2+c^2}}{\sqrt{3}}.
\]
\end{solution}

\begin{takeaways}
\begin{enumerate}
    \item \textbf{Lagrange Multipliers:} Essential for constrained optimization on curves and surfaces.
    \item \textbf{Gradient Parallelism:} At optimal points, constraint and objective gradients are parallel.
    \item \textbf{Geometric Interpretation:} The solution connects algebraic optimization with geometric intuition.
\end{enumerate}
\end{takeaways}

\begin{remark}[Distance Between Shapes]
In general, the distance between two shapes $A$ and $B$ is defined as
\[ d(A,B) = \inf\{\|p-q\| : p\in A,\ q\in B\}. \]
For an ellipsoid and a plane, this reduces to optimizing the point on the ellipsoid whose normal is aligned with the plane's normal (via Lagrange multipliers), yielding the maximum/minimum perpendicular separation between the surfaces.
Here, inf means the greatest lower bound (or minimum if it exists).

\end{remark}

% Problem 2: Curvature Analysis (Sample 13)
\begin{problem}[Tangent Circle and Curvature]
Consider the parametric curve $C$ given by:
\[ \mathbf{r}(t) = (t^3 - 3t, t^2, t^4) \quad \text{for } t \in \mathbb{R} \]

For a smooth space curve $\mathbf{r}(t)$, the curvature at parameter $t$ is
\[\kappa(t) = \frac{\big\lVert\,\mathbf{r}'(t) \times \mathbf{r}''(t)\,\big\rVert}{\big\lVert\mathbf{r}'(t)\big\rVert^3},\]
which measures how quickly the curve is turning at that point (do NOT prove this formula).

(a) Find the curvature $\kappa(t)$ of the curve at any point.

(b) Determine the point(s) where the curvature is maximum.

(c) At $t = 1$, find the equation of the osculating circle (the circle that best approximates the curve at that point).

(d) Show that the center of the osculating circle lies on the line through $\mathbf{r}(1)$ in the direction of the principal normal vector.
\end{problem}


\begin{remark}[Plane Curve Curvature]
For a plane curve given by $\gamma(t)=(x(t),y(t))$, a common equivalent formula is
\[\kappa(t) = \frac{\big|x'(t)\,y''(t) - y'(t)\,x''(t)\big|}{\big(x'(t)^2 + y'(t)^2\big)^{3/2}}.\]
This matches the space-curve definition when the curve lies in a plane.
\end{remark}


\begin{solution}
\textbf{(a) Computing curvature:}
First, find the derivatives:
$\mathbf{r}'(t) = (3t^2 - 3, 2t, 4t^3)$
$\mathbf{r}''(t) = (6t, 2, 12t^2)$

The curvature formula: $\kappa(t) = \frac{|\mathbf{r}' \times \mathbf{r}''|}{|\mathbf{r}'|^3}$

Compute the cross product:
$\mathbf{r}' \times \mathbf{r}'' = \begin{vmatrix}
\mathbf{i} & \mathbf{j} & \mathbf{k} \\
3t^2-3 & 2t & 4t^3 \\
6t & 2 & 12t^2
\end{vmatrix}$

$= \mathbf{i}(24t^3 - 8t^3) - \mathbf{j}(12t^2(3t^2-3) - 24t^4) + \mathbf{k}(2(3t^2-3) - 12t^2)$

$= (16t^3, -36t^2 + 36t^4 + 24t^4, 6t^2 - 6 - 12t^2)$

$= (16t^3, 60t^4 - 36t^2, -6t^2 - 6)$

$|\mathbf{r}' \times \mathbf{r}''| = \sqrt{256t^6 + (60t^4 - 36t^2)^2 + 36(t^2 + 1)^2}$

$|\mathbf{r}'| = \sqrt{(3t^2-3)^2 + 4t^2 + 16t^6} = \sqrt{9t^4 - 18t^2 + 9 + 4t^2 + 16t^6}$

$= \sqrt{16t^6 + 9t^4 - 14t^2 + 9}$

\textbf{(b) Finding maximum curvature:}
This requires calculus analysis of $\kappa(t)$. By symmetry and analysis, the maximum occurs at $t = 0$.

At $t = 0$: $\kappa(0) = \frac{6}{9} = \frac{2}{3}$

\textbf{(c) Osculating circle at $t = 1$:}
At $t = 1$:
$\mathbf{r}(1) = (-2, 1, 1)$
$\mathbf{r}'(1) = (0, 2, 4)$
$\mathbf{r}''(1) = (6, 2, 12)$

Unit tangent: $\mathbf{T}(1) = \frac{(0, 2, 4)}{\sqrt{20}} = \frac{(0, 1, 2)}{\sqrt{5}}$

Principal normal: $\mathbf{N}(1) = \frac{\mathbf{T}'(1)}{|\mathbf{T}'(1)|}$

Radius of curvature: $R = \frac{1}{\kappa(1)}$

\textbf{(d) Center verification:}
The center is at $\mathbf{r}(1) + R\mathbf{N}(1)$, which lies on the line through $\mathbf{r}(1)$ in the direction of $\mathbf{N}(1)$.
\end{solution}

\begin{takeaways}
\begin{enumerate}
    \item \textbf{Curvature Formula:} $\kappa = \frac{|\mathbf{r}' \times \mathbf{r}''|}{|\mathbf{r}'|^3}$ measures how quickly a curve bends.
    \item \textbf{Osculating Circle:} The circle that best approximates the curve locally.
    \item \textbf{Frenet Frame:} Tangent and normal vectors provide geometric insight into curve behavior.
\end{enumerate}
\end{takeaways}

% Problem 3: Complex Root Bounds (Sample 22)
\begin{problem}[Cauchy’s Root Bound via Triangle Inequality]
Let $P(z) = z^n + a_{n-1}z^{n-1} + \cdots + a_1 z + a_0$ be a monic complex polynomial. Show that every root $\zeta$ of $P$ satisfies
\[ |\zeta| \le 1 + M, \quad \text{where } M = \max_{0\le k < n} |a_k|. \]
Hint: Assume $|\zeta| > 1+M$ and compare $|\zeta|^n$ with $\sum_{k=0}^{n-1} |a_k|\,|\zeta|^k$ via the triangle inequality.
\end{problem}

\begin{solution}
If $P(\zeta)=0$ then
\[ |\zeta|^n = \Big|\sum_{k=0}^{n-1} a_k \zeta^k\Big| \le \sum_{k=0}^{n-1} |a_k|\,|\zeta|^k \le M \sum_{k=0}^{n-1} |\zeta|^k. \]
Suppose $|\zeta| > 1+M$. Then $\sum_{k=0}^{n-1} |\zeta|^k < n\,|\zeta|^{n-1}$ and so
\[ |\zeta|^n \le M\,n\,|\zeta|^{n-1} \quad \Rightarrow \quad |\zeta| \le Mn. \]
But $|\zeta| > 1+M$ implies $|\zeta|/M > 1 + 1/M \ge 1$ (when $M>0$), and for sufficiently large $n$ this contradicts the inequality $|\zeta| \le Mn$. A standard refinement avoids $n$ entirely by dividing the identity $\zeta^n = -\sum_{k=0}^{n-1} a_k \zeta^k$ by $\zeta^n$ to get
\[ 1 = -\sum_{k=0}^{n-1} a_k \zeta^{k-n}, \quad \text{hence} \quad 1 \le \sum_{k=0}^{n-1} |a_k|\,|\zeta|^{k-n}. \]
If $|\zeta| > 1+M$, then $|\zeta|^{k-n} < (1+M)^{k-n} \le (1+M)^{-1}$ for all $k \le n-1$, giving
\[ 1 < \sum_{k=0}^{n-1} |a_k|\,(1+M)^{-1} \le M\,(1+M)^{-1} < 1, \]
which is a contradiction. Therefore $|\zeta| \le 1 + M$ for all roots.
\end{solution}

\begin{takeaways}
\begin{enumerate}
    \item \textbf{Cauchy Bound:} Every root of a monic polynomial lies in the disk $|z| \le 1 + \max |a_k|$.
    \item \textbf{Triangle Inequality Tool:} Comparing $|z|^n$ with coefficient-weighted sums yields robust bounds.
    \item \textbf{Rescaling Remarks:} Stronger bounds exist (e.g., Fujiwara’s), derivable by rescaling or sharper estimates.
\end{enumerate}
\end{takeaways}

% Problem 4: Skew Line Distance (Sample 27)
\begin{problem}[Distance Between Skew Lines]
Consider two skew lines $L_1$ and $L_2$ in $\mathbb{R}^3$:
\begin{align}
L_1: \quad \mathbf{r}_1(s) &= (1, 2, 3) + s(2, -1, 1) \\
L_2: \quad \mathbf{r}_2(t) &= (0, 1, -1) + t(1, 1, -2)
\end{align}

(a) Verify that $L_1$ and $L_2$ are skew (neither parallel nor intersecting).

(b) Find the shortest distance between the two lines.

(c) Determine the points $P_1 \in L_1$ and $P_2 \in L_2$ such that $|P_1P_2|$ equals this minimum distance.

(d) Show that the line segment $P_1P_2$ is perpendicular to both $L_1$ and $L_2$.
\end{problem}

\begin{solution}
\textbf{(a) Verifying skew lines:}
Direction vectors: $\mathbf{d}_1 = (2, -1, 1)$ and $\mathbf{d}_2 = (1, 1, -2)$

Check if parallel: $\mathbf{d}_1 \times \mathbf{d}_2 = (2, -1, 1) \times (1, 1, -2) = (1, 5, 3) \neq \mathbf{0}$

So lines are not parallel.

Check if intersecting: Set $\mathbf{r}_1(s) = \mathbf{r}_2(t)$:
$(1, 2, 3) + s(2, -1, 1) = (0, 1, -1) + t(1, 1, -2)$

This gives the system:
$1 + 2s = t$
$2 - s = 1 + t$
$3 + s = -1 - 2t$

From equations 1 and 2: $t = 1 + 2s$ and $t = 1 - s$
So $1 + 2s = 1 - s \Rightarrow 3s = 0 \Rightarrow s = 0$
Then $t = 1$.

Checking equation 3: $3 + 0 = -1 - 2(1) \Rightarrow 3 = -3$ (contradiction)

Therefore, the lines are skew.

\textbf{(b) Distance formula:}
For skew lines, the distance is:
$d = \frac{|(\mathbf{a}_2 - \mathbf{a}_1) \cdot (\mathbf{d}_1 \times \mathbf{d}_2)|}{|\mathbf{d}_1 \times \mathbf{d}_2|}$

where $\mathbf{a}_1 = (1, 2, 3)$ and $\mathbf{a}_2 = (0, 1, -1)$.

$\mathbf{a}_2 - \mathbf{a}_1 = (-1, -1, -4)$
$\mathbf{d}_1 \times \mathbf{d}_2 = (1, 5, 3)$
$|\mathbf{d}_1 \times \mathbf{d}_2| = \sqrt{1 + 25 + 9} = \sqrt{35}$

$(\mathbf{a}_2 - \mathbf{a}_1) \cdot (\mathbf{d}_1 \times \mathbf{d}_2) = (-1, -1, -4) \cdot (1, 5, 3) = -1 - 5 - 12 = -18$

Therefore: $d = \frac{|-18|}{\sqrt{35}} = \frac{18}{\sqrt{35}} = \frac{18\sqrt{35}}{35}$

\textbf{(c) Finding closest points:}
Let $\mathbf{w} = \mathbf{r}_2(t) - \mathbf{r}_1(s)$. For minimum distance, $\mathbf{w} \perp \mathbf{d}_1$ and $\mathbf{w} \perp \mathbf{d}_2$.

$\mathbf{w} = (0, 1, -1) + t(1, 1, -2) - (1, 2, 3) - s(2, -1, 1)$
$= (-1 + t - 2s, -1 + t + s, -4 - 2t - s)$

$\mathbf{w} \cdot \mathbf{d}_1 = 0$: $(-1 + t - 2s)(2) + (-1 + t + s)(-1) + (-4 - 2t - s)(1) = 0$
$\mathbf{w} \cdot \mathbf{d}_2 = 0$: $(-1 + t - 2s)(1) + (-1 + t + s)(1) + (-4 - 2t - s)(-2) = 0$

Solving this system yields the parameter values for the closest points.

\textbf{(d) Perpendicularity verification:}
By construction in part (c), $\mathbf{w} \perp \mathbf{d}_1$ and $\mathbf{w} \perp \mathbf{d}_2$.
\end{solution}

\begin{takeaways}
\begin{enumerate}
    \item \textbf{Skew Line Criteria:} Lines are skew if they're not parallel and don't intersect.
    \item \textbf{Distance Formula:} Uses scalar triple product and cross product magnitudes.
    \item \textbf{Perpendicularity Condition:} Minimum distance occurs when connecting segment is perpendicular to both lines.
\end{enumerate}
\end{takeaways}

% Problem 5: Newton's Sums (Sample 60)
\begin{problem}[Powers of Roots and Recurrence Relations]
Consider the polynomial $P(x) = x^4 - 6x^3 + 11x^2 - 6x + 1$ with roots $r_1, r_2, r_3, r_4$.

(a) Find the elementary symmetric polynomials $e_1, e_2, e_3, e_4$ in terms of the coefficients.

(b) Let $S_k = r_1^k + r_2^k + r_3^k + r_4^k$ be the $k$-th power sum. Use Newton's identities to find $S_1, S_2, S_3, S_4$.

(c) Establish the recurrence relation for $S_k$ when $k \geq 4$.

(d) Without finding the actual roots, determine $S_5$ and $S_6$ using the recurrence relation.

(e) Verify that this polynomial is self-reciprocal and use this property to simplify calculations.
\end{problem}

\begin{solution}
\textbf{(a) Elementary symmetric polynomials:}
For $P(x) = x^4 - 6x^3 + 11x^2 - 6x + 1$:
$e_1 = r_1 + r_2 + r_3 + r_4 = 6$
$e_2 = \sum_{i<j} r_i r_j = 11$
$e_3 = \sum_{i<j<k} r_i r_j r_k = 6$
$e_4 = r_1 r_2 r_3 r_4 = 1$

\textbf{(b) Newton's identities:}
The Newton's identities relate power sums to elementary symmetric polynomials:
$S_1 - e_1 = 0 \Rightarrow S_1 = e_1 = 6$

$S_2 - e_1 S_1 + 2e_2 = 0 \Rightarrow S_2 = 6 \cdot 6 - 2 \cdot 11 = 36 - 22 = 14$

$S_3 - e_1 S_2 + e_2 S_1 - 3e_3 = 0$
$S_3 = 6 \cdot 14 - 11 \cdot 6 + 3 \cdot 6 = 84 - 66 + 18 = 36$

$S_4 - e_1 S_3 + e_2 S_2 - e_3 S_1 + 4e_4 = 0$
$S_4 = 6 \cdot 36 - 11 \cdot 14 + 6 \cdot 6 - 4 \cdot 1 = 216 - 154 + 36 - 4 = 94$

\textbf{(c) Recurrence relation:}
For $k \geq 4$:
$S_k - e_1 S_{k-1} + e_2 S_{k-2} - e_3 S_{k-3} + e_4 S_{k-4} = 0$

Substituting our values:
$S_k - 6S_{k-1} + 11S_{k-2} - 6S_{k-3} + S_{k-4} = 0$

Therefore: $S_k = 6S_{k-1} - 11S_{k-2} + 6S_{k-3} - S_{k-4}$

\textbf{(d) Computing $S_5$ and $S_6$:}
$S_5 = 6 \cdot 94 - 11 \cdot 36 + 6 \cdot 14 - 6 = 564 - 396 + 84 - 6 = 246$

$S_6 = 6 \cdot 246 - 11 \cdot 94 + 6 \cdot 36 - 14 = 1476 - 1034 + 216 - 14 = 644$

\textbf{(e) Self-reciprocal property:}
The polynomial $P(x) = x^4 - 6x^3 + 11x^2 - 6x + 1$ satisfies $x^4 P(1/x) = P(x)$.

This means if $r$ is a root, then $1/r$ is also a root. We can pair roots as $(r_1, 1/r_1)$ and $(r_2, 1/r_2)$.

This property can be used to establish relations like:
$S_k + S_{-k} = $ (expression in lower power sums)

For verification: $r_1 r_2 r_3 r_4 = 1$ confirms the self-reciprocal nature.
\end{solution}

\begin{takeaways}
\begin{enumerate}
    \item \textbf{Newton's Identities:} Fundamental tool connecting power sums to symmetric polynomials.
    \item \textbf{Recurrence Relations:} Enable computation of higher power sums without finding roots.
    \item \textbf{Self-Reciprocal Polynomials:} Special structure provides additional computational advantages.
    \item \textbf{Coefficient Relationships:} Direct connection between polynomial coefficients and root properties.
\end{enumerate}
\end{takeaways}

% Problem 6: Irrationality of \zeta(3) (Sample 49)
\begin{problem}[The Irrationality of $\zeta(3)$]
Let $\zeta(3) = 1 + \frac{1}{2^3} + \frac{1}{3^3} + \frac{1}{4^3} + \dots = \sum_{n=1}^{\infty} \frac{1}{n^3}$ denote Ap\'ery's constant. In this problem you will show $\zeta(3)$ is irrational using a sequence of integrals and bounds (you may use the stated facts without proof).

\begin{enumerate}
    \item[(a)] \textbf{Irrationality Criterion.} Let $\alpha=\tfrac{p}{q}$ be rational with integers $p,q>0$. Show that for any integers $A,B$ with $A\alpha-B\ne 0$,
    \[ |A\alpha - B| \ge \frac{1}{q}. \]

    \item[(b)] \textbf{Optimization bound.} For $0\le x,y,z\le 1$, consider
    \[ g(x,y,z) = \frac{x(1-x)\,y(1-y)\,z(1-z)}{1-(1-xy)z}. \]
    Show that $g$ attains a global maximum value $g_{\max}=(\sqrt{2}-1)^4$ (you may assume the maximum occurs when $x=y$ and optimize in $z$ first).

    \item[(c)] \textbf{Integral construction.} Let $d_n=\operatorname{lcm}(1,2,\dots,n)$ and
    \[ I_n = \int_0^1\!\int_0^1\!\int_0^1 \Big( g(x,y,z) \Big)^n \, \frac{dx\,dy\,dz}{1-(1-xy)z}. \]
    You are given: (1) $I_n>0$ for all $n\ge 1$; (2) $2d_n^3 I_n = A_n\,\zeta(3) + B_n$ for some integers $A_n,B_n$; (3) for large $n$, $d_n<3^n$.
    Using (b), deduce a bound of the form
    \[ 0 < I_n < C\,(\sqrt{2}-1)^{4n}, \]
    where $C$ is independent of $n$.

    \item[(d)] \textbf{Conclusion.} Assuming $\zeta(3)$ is rational, apply (a) to $X_n=A_n\,\zeta(3)+B_n$ and use (c) together with $d_n<3^n$ to derive a contradiction as $n\to\infty$. Conclude that $\zeta(3)$ is irrational.
\end{enumerate}

\begin{hintbox}
	\textbf{(a)} Clear denominators: $A\frac{p}{q}-B=\frac{Ap-Bq}{q}$ and use that a nonzero integer has magnitude at least 1.\\
	\textbf{(b)} Set $y=x$, maximize in $z$ to get $z=\frac{1}{1+x}$, then maximize in $x\in[0,1]$.\\
	\textbf{(c)} Bound the integrand by $g_{\max}^n$ and pull it outside the integral.\\
	\textbf{(d)} Compare the fixed lower bound $1/q$ with the upper bound decaying like $\big(27(\sqrt{2}-1)^4\big)^n$.
\end{hintbox}
\end{problem}

\begin{solution}
	\textbf{(a)} If $\alpha=\tfrac{p}{q}$, then $A\alpha-B=\tfrac{Ap-Bq}{q}$. The numerator is a nonzero integer, so $|Ap-Bq|\ge 1$, hence $|A\alpha-B|\ge \tfrac{1}{q}$.

	\textbf{(b)} Setting $y=x$ and optimizing first in $z$ yields $z=\tfrac{1}{1+x}$. Substituting and maximizing over $x\in[0,1]$ gives $\max g=(\sqrt{2}-1)^4$.

	\textbf{(c)} Since $g(x,y,z)\le (\sqrt{2}-1)^4$ on $[0,1]^3$, we have
\[ 0<I_n=\iiint g(x,y,z)^n\,\frac{dx\,dy\,dz}{1-(1-xy)z} < \Big((\sqrt{2}-1)^4\Big)^n \iiint \frac{dx\,dy\,dz}{1-(1-xy)z} = C\,(\sqrt{2}-1)^{4n}, \]
for some finite constant $C$.

	\textbf{(d)} Write $X_n=A_n\,\zeta(3)+B_n=2d_n^3 I_n$. By (a), $|X_n|\ge \tfrac{1}{q}$ if $\zeta(3)$ were rational. But from (c) and $d_n<3^n$ we get
\[ |X_n|=2d_n^3 I_n < 2\,(3^n)^3\,C\,(\sqrt{2}-1)^{4n} = 2C\,\big(27(\sqrt{2}-1)^4\big)^n, \]
whose right-hand side tends to $0$ as $n\to\infty$. This contradicts $|X_n|\ge \tfrac{1}{q}>0$. Therefore $\zeta(3)$ is irrational.
\end{solution}

\begin{takeaways}
\begin{enumerate}
    \item \textbf{Logic of Irrationality:} Use integer linear forms and uniform lower bounds ($1/q$) vs exponentially decaying upper bounds to force contradictions.
    \item \textbf{Bounding Integrals:} Maximize the integrand to bound the whole integral without evaluation.
    \item \textbf{Assumption Management:} Treat given constructions (LCM growth, integral identity) as black boxes and connect them via inequalities.
\end{enumerate}
\end{takeaways}