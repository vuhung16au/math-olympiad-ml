% Hard Problems (Part 1)
% Problem 11: Advanced Vector Inequalities
\section{Problem 11}
From samples/05.tex

(i) State the Cauchy-Schwarz inequality relating the dot product $\mathbf{u} \cdot \mathbf{v}$ and the magnitudes $|\mathbf{u}|$ and $|\mathbf{v}|$.

(ii) The point $P(x, y, z)$ lies on the surface of the ellipsoid $\frac{x^2}{4} + \frac{y^2}{9} + \frac{z^2}{25} = 1$. By choosing appropriate vectors and using part (i), show that $|x + y + z| \le \sqrt{38}$.

(iii) Find the coordinates of the point $P(x,y,z)$ with $x,y,z > 0$ for which $x+y+z$ is a maximum.

% Problem 12: 3D Geometry and Projections
\section*{Problem 12}
From samples/05.tex

(i) The points $A, B,$ and $C$ lie on the $x, y,$ and $z$ axes respectively, such that their position vectors are $\vec{OA} = a\mathbf{i}$, $\vec{OB} = b\mathbf{j}$, and $\vec{OC} = c\mathbf{k}$ ($a,b,c > 0$).
By considering the cross product $\vec{AB} \times \vec{AC}$, show that the area of triangle $ABC$ is:
\[ \text{Area}_{ABC} = \frac{1}{2} \sqrt{b^2 c^2 + a^2 c^2 + a^2 b^2} \]

% Problem 13: Zero Sum of Squares
\section*{Problem 13}
From samples/12.tex

Find all complex numbers $z_1, z_2, z_3$ satisfying:
\[
\begin{cases}
    |z_1| = |z_2| = |z_3| = 1 \\
    z_1 + z_2 + z_3 = 1 \\
    z_1^2 + z_2^2 + z_3^2 = 1
\end{cases}
\]

% Problem 14: The Tangent Circle
\section*{Problem 14}
From samples/13.tex

Consider the curve $y = \sin x$ for $x \in [0, \pi]$. A circle $\mathcal{C}$ passes through the origin $O(0,0)$ and has its centre $K$ on the positive $y$-axis at $(0, R)$.
(i) Write down the equation of the circle $\mathcal{C}$.
(ii) The circle is tangent to the curve $y = \sin x$ at the origin. Show that if the circle lies entirely "above" the sine curve for $x > 0$ (locally near the origin), the radius must satisfy $R \ge 1$.

% Problem 15: Circle Inside an Ellipse
\section*{Problem 15}
From samples/15.tex

Consider the ellipse $\frac{x^2}{16} + \frac{y^2}{4} = 1$ (major axis $a=4$, minor axis $b=2$).
A circle $x^2 + y^2 = r^2$ is centered at the origin.
(i) Determine the range of values for $r$ such that the circle intersects the ellipse at exactly four distinct points.
(ii) Visualize the scenario where $r$ creates exactly 2 intersection points.

% Problem 16: The Irrationality of ζ(3)
\section*{Problem 16}
% From samples/49.tex

Let $\zeta(3) = 1 + \frac{1}{2^3} + \frac{1}{3^3} + \frac{1}{4^3} + \dots = \sum_{n=1}^\infty \frac{1}{n^3}$.

This sequence converges to a real number known as Apéry's constant (DO NOT prove this)

\begin{enumerate}
    \item[(a)] \textbf{The Irrationality Criterion}
    
    Let $\alpha$ be a real number. Suppose $\alpha$ is rational, meaning $\alpha = \frac{p}{q}$ where $p, q$ are integers and $q > 0$.
    
    Show that for any integers $A$ and $B$, if $A\alpha - B \neq 0$, then:
    \[ |A\alpha - B| \ge \frac{1}{q} \]

    \item[(b)] \textbf{Optimization}
    
    Consider the function $f(x) = x(1-x)$ for $0 \le x \le 1$.
    Find the maximum value of the function:
    \[ g(x, y, z) = \frac{x(1-x)y(1-y)z(1-z)}{1 - (1-xy)z} \]
    for $0 \le x, y, z \le 1$.
    
    \textit{(Hint: You may assume the maximum occurs when $x=y$. First substitute $y=x$, then maximize with respect to $z$, then with respect to $x$.)}

    \item[(c)] \textbf{The Integral Construction}
    
    Let $d_n = \text{lcm}(1, 2, \dots, n)$.
    Consider the sequence of integrals defined for $n \ge 1$:
    \[ I_n = \int_0^1 \int_0^1 \int_0^1 \left( \frac{x(1-x)y(1-y)z(1-z)}{1 - (1-xy)z} \right)^n \frac{dx \, dy \, dz}{1 - (1-xy)z} \]
    
    You are given the following properties of $I_n$ (Do \textbf{NOT} prove these):
    \begin{itemize}
        \item[(1)] $I_n > 0$ for all $n \ge 1$.
        \item[(2)] $2 d_n^3 I_n = A_n \zeta(3) + B_n$, where $A_n$ and $B_n$ are integers.
        \item[(3)] For sufficiently large $n$, $d_n < 3^n$.
    \end{itemize}
    
    Using part (b), show that:
    \[ 0 < I_n < (\sqrt{2}-1)^{4n} \cdot C \]
    where $C$ is a constant independent of $n$.

    \item[(d)] \textbf{Conclusion}
    
    Using parts (a) and (c), prove that $\zeta(3)$ is irrational.

\end{enumerate}
