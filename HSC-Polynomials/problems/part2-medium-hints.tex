% Part 2: Medium Problems (13 problems with hints - upside-down format)
% Students attempt first, then rotate page 180° to read hint

% =============================================================================
% PROBLEM 13: Polynomial remainder with complex numbers
% =============================================================================
\begin{problem}
Suppose that $P(x) = x^3 - x^2 + mx + n$, where $m$ and $n$ are integers.
\begin{enumerate}
    \item Show that $P(-i) = (1+n) + i(1-m)$.
    \item When $P(x)$ is divided by $x^2 + 1$ the remainder is $6x - 3$. Find the values of $m$ and $n$.
\end{enumerate}
\end{problem}

\vspace{0.5cm}
\rotatebox{180}{\parbox{\textwidth}{\small \textbf{Hint:} (a) Compute $(-i)^2=-1, (-i)^3=i$. (b) $P(-i)$ equals remainder evaluated at $-i$: $6(-i)-3 = -3-6i$. Equate real/imaginary parts.}}

\begin{solution}
\textbf{(a)} $P(-i) = (-i)^3 - (-i)^2 + m(-i) + n = i - (-1) - mi + n = (1+n) + i(1-m)$

\textbf{(b)} Since $x^2+1$ has root $-i$, by remainder theorem:
$$
P(-i) = 6(-i) - 3 = -3 - 6i
$$

Equating with part (a): $(1+n) + i(1-m) = -3 - 6i$

Real parts: $1+n = -3 \implies n = -4$

Imaginary parts: $1-m = -6 \implies m = 7$

\textbf{Answer:} $m = 7, n = -4$.
\end{solution}

\begin{takeaways}
Remainder theorem applies to complex divisors; equate real and imaginary parts separately.
\end{takeaways}

\vspace{1cm}

% =============================================================================
% PROBLEM 19: De Moivre and complex equations
% =============================================================================
\begin{problem}
\begin{enumerate}
    \item Show that $(1 + i \tan \theta)^n + (1 - i \tan \theta)^n = \frac{2 \cos n\theta}{\cos^n \theta}$ where $\cos \theta \neq 0$ and $n$ is a positive integer.
    \item Hence show that if $z$ is purely imaginary, the roots of $(1 + z)^4 + (1 - z)^4 = 0$ are $z = \pm i \tan \frac{\pi}{8}, \pm i \tan \frac{3\pi}{8}$.
\end{enumerate}
\end{problem}

\vspace{0.5cm}
\rotatebox{180}{\parbox{\textwidth}{\small \textbf{Hint:} (a) Rewrite $1 \pm i\tan\theta = \frac{\cos\theta \pm i\sin\theta}{\cos\theta}$, apply De Moivre. (b) Set $z=i\tan\theta$, equation becomes $\frac{2\cos 4\theta}{\cos^4\theta}=0$, so $\cos 4\theta=0$.}}

\begin{solution}
\textbf{(a)} $1 + i\tan\theta = \frac{\cos\theta + i\sin\theta}{\cos\theta} = \frac{e^{i\theta}}{\cos\theta}$

By De Moivre:
$$
(1 + i\tan\theta)^n = \frac{e^{in\theta}}{\cos^n\theta}, \quad (1 - i\tan\theta)^n = \frac{e^{-in\theta}}{\cos^n\theta}
$$

Sum: $\frac{e^{in\theta} + e^{-in\theta}}{\cos^n\theta} = \frac{2\cos n\theta}{\cos^n\theta}$

\textbf{(b)} Let $z = i\tan\theta$. Then $(1+z)^4 + (1-z)^4 = \frac{2\cos 4\theta}{\cos^4\theta} = 0$

Since $\cos\theta \neq 0$: $\cos 4\theta = 0 \implies 4\theta = \frac{\pi}{2} + k\pi \implies \theta = \frac{\pi}{8} + \frac{k\pi}{4}$

For $k=0,1,2,3$: $\theta = \frac{\pi}{8}, \frac{3\pi}{8}, \frac{5\pi}{8}, \frac{7\pi}{8}$

Using $\tan(\pi-x) = -\tan x$: $z = \pm i\tan\frac{\pi}{8}, \pm i\tan\frac{3\pi}{8}$

\textbf{Answer:} As shown.
\end{solution}

\begin{takeaways}
De Moivre's theorem applies to expressions of form $\frac{e^{i\theta}}{\cos^n\theta}$; tangent values related by symmetry.
\end{takeaways}

\vspace{1cm}

% =============================================================================
% PROBLEM 22: Complex cotangent identity
% =============================================================================
\begin{problem}
\begin{enumerate}
    \item Show that $\dfrac{1+\cos \theta+i \sin \theta}{1-\cos \theta-i \sin \theta} = i \cot \dfrac{\theta}{2}$.
    \item Hence solve $\left(\dfrac{z-1}{z+1}\right)^8 = -1$.
\end{enumerate}
\end{problem}

\vspace{0.5cm}
\rotatebox{180}{\parbox{\textwidth}{\small \textbf{Hint:} (a) Use half-angle formulas: $1+\cos\theta=2\cos^2(\theta/2)$, $\sin\theta=2\sin(\theta/2)\cos(\theta/2)$. (b) Set $w=(z-1)/(z+1)$, find $w^8=-1$, then $z=\frac{1+w}{1-w}$.}}

\begin{solution}
\textbf{(a)} Using half-angle identities:
$$
\frac{1+\cos\theta+i\sin\theta}{1-\cos\theta-i\sin\theta} = \frac{2\cos^2(\theta/2) + 2i\sin(\theta/2)\cos(\theta/2)}{2\sin^2(\theta/2) - 2i\sin(\theta/2)\cos(\theta/2)}
$$
$$
= \frac{\cos(\theta/2)[\cos(\theta/2) + i\sin(\theta/2)]}{\sin(\theta/2)[\sin(\theta/2) - i\cos(\theta/2)]} = \cot(\theta/2) \cdot \frac{\cos(\theta/2) + i\sin(\theta/2)}{-i[\cos(\theta/2) + i\sin(\theta/2)]} = i\cot(\theta/2)
$$

\textbf{(b)} Let $w = \frac{z-1}{z+1}$. Then $w^8 = -1 = e^{i\pi(2k+1)}$, so $w = e^{i\pi(2k+1)/8}$, $k=0,\ldots,7$.

From $w = \frac{z-1}{z+1}$: $z = \frac{1+w}{1-w} = \frac{1+\cos\alpha_k+i\sin\alpha_k}{1-\cos\alpha_k-i\sin\alpha_k} = i\cot(\alpha_k/2)$

where $\alpha_k = \frac{(2k+1)\pi}{8}$. Thus $z = i\cot(\frac{(2k+1)\pi}{16})$ for $k=0,\ldots,7$.

\textbf{Answer:} $z = \pm i\cot\frac{\pi}{16}, \pm i\cot\frac{3\pi}{16}, \pm i\cot\frac{5\pi}{16}, \pm i\cot\frac{7\pi}{16}$.
\end{solution}

\begin{takeaways}
Half-angle formulas simplify complex fractions; eighth roots of -1 give eight solutions.
\end{takeaways}

\vspace{1cm}

% =============================================================================
% PROBLEM 24: Inequality proof by polynomial factorization (a=1 case)
% =============================================================================
\begin{problem}
Prove that for $0 \le b < 1$:
$$
\frac{1 - b^{n+1}}{1-b} < n+1
$$
where $n \in \mathbb{Z}^+$.
\end{problem}

\vspace{0.5cm}
\rotatebox{180}{\parbox{\textwidth}{\small \textbf{Hint:} Factor LHS as geometric series: $1 + b + b^2 + \dots + b^n$. Compare term-by-term with $\underbrace{1+1+\dots+1}_{n+1}$. Each $b^k < 1$ for $k \ge 1$.}}

\begin{solution}
LHS: $\frac{1-b^{n+1}}{1-b} = 1 + b + b^2 + \dots + b^n$ (geometric series)

RHS: $n+1 = \underbrace{1 + 1 + \dots + 1}_{n+1 \text{ terms}}$

Term-by-term comparison:
- First term: $b^0 = 1 = 1$ (equal)
- Terms $k=1,\ldots,n$: $b^k < 1$ since $0 \le b < 1$

Therefore: $1 + b + b^2 + \dots + b^n < 1 + 1 + \dots + 1 = n+1$

\textbf{Answer:} Inequality proven.
\end{solution}

\begin{takeaways}
Geometric series allow term-by-term comparison; strict inequality holds when $b<1$.
\end{takeaways}

\vspace{1cm}

% =============================================================================
% PROBLEM 26: Induction with symmetric polynomial
% =============================================================================
\begin{problem}
Prove by induction:
$$x^n + x^{n-2} + x^{n-4} + \dots + \frac{1}{x^{n-4}} + \frac{1}{x^{n-2}} + \frac{1}{x^n} \geq n+1$$
for $x > 0$ and $n \in \mathbb{Z}^+$. [Hint: separate base cases for $n$ even or odd.]
\end{problem}

\vspace{0.5cm}
\rotatebox{180}{\parbox{\textwidth}{\small \textbf{Hint:} Base cases: $P(1)$: $x+1/x \ge 2$ (AM-GM); $P(2)$: $x^2+1+1/x^2 \ge 3$. Inductive step: $P(k) \implies P(k+2)$. Add $(x^{k+2}+1/x^{k+2}) \ge 2$.}}

\begin{solution}
\textbf{Base cases:}
- $n=1$: $x + \frac{1}{x} \ge 2$ by AM-GM: $\frac{x+1/x}{2} \ge \sqrt{x \cdot 1/x} = 1$. ✓
- $n=2$: $x^2 + 1 + \frac{1}{x^2} \ge 3$. Since $x^2 + \frac{1}{x^2} \ge 2$ by AM-GM, LHS $\ge 2+1=3$. ✓

\textbf{Inductive step:} Assume $P(k)$ holds. Prove $P(k+2)$:
$$
\text{LHS}_{k+2} = (x^{k+2} + \frac{1}{x^{k+2}}) + \sum_{i=0}^k x^{k-2i}
$$

By AM-GM: $x^{k+2} + \frac{1}{x^{k+2}} \ge 2$

By induction: $\sum_{i=0}^k x^{k-2i} \ge k+1$

Therefore: LHS$_{k+2} \ge 2 + (k+1) = k+3$, which equals RHS$_{k+2}$.

\textbf{Answer:} Proven by induction.
\end{solution}

\begin{takeaways}
Two base cases handle parity; inductive step jumps by 2 to preserve parity structure.
\end{takeaways}

\vspace{1cm}

% =============================================================================
% PROBLEM 28: Divisibility proof
% =============================================================================
\begin{problem}
Prove that $\frac{n^5}{5} + \frac{n^4}{2} + \frac{n^3}{3} - \frac{n}{30}$ is an integer for all integers $n \geq 1$.
\end{problem}

\vspace{0.5cm}
\rotatebox{180}{\parbox{\textwidth}{\small \textbf{Hint:} Combine over common denominator 30: $\frac{6n^5+15n^4+10n^3-n}{30} = \frac{n(6n^4+15n^3+10n^2-1)}{30}$. Show numerator divisible by 30 using Fermat's Little Theorem.}}

\begin{solution}
Combine: $P(n) = \frac{6n^5+15n^4+10n^3-n}{30} = \frac{n(6n^4+15n^3+10n^2-1)}{30}$

Note: $P(n) = \sum_{k=1}^n k^4$ (sum of fourth powers formula).

Alternatively, show $n^5-n$ divisible by 30:
- Divisible by 2: $(n-1)n(n+1)$ contains even number
- Divisible by 3: $(n-1)n(n+1)$ contains multiple of 3
- Divisible by 5: Fermat's Little Theorem gives $n^5 \equiv n \pmod{5}$

Since $n^5-n \equiv 0 \pmod{30}$ and remaining terms also yield multiples of 30, $P(n)$ is an integer.

\textbf{Answer:} Proven.
\end{solution}

\begin{takeaways}
Fermat's Little Theorem handles prime divisibility; consecutive integers ensure small prime factors.
\end{takeaways}

\vspace{1cm}

% =============================================================================
% PROBLEM 29: Cubic with constrained roots
% =============================================================================
\begin{problem}
Let $\alpha, \beta, \gamma$ be roots of $x^3 + Ax^2 + Bx + 8 = 0$ (real $A, B$).
Given $\alpha^2 + \beta^2 = 0$ and $\beta^2 + \gamma^2 = 0$:
\begin{enumerate}
    \item Explain why $\beta$ is real and $\alpha, \gamma$ are not real.
    \item Show $\alpha, \gamma$ are purely imaginary.
    \item Find $A$ and $B$.
\end{enumerate}
\end{problem}

\vspace{0.5cm}
\rotatebox{180}{\parbox{\textwidth}{\small \textbf{Hint:} (a) If $\beta$ non-real, conjugate must be root, but constraints force contradictions. (b) From $\alpha^2=-\beta^2<0$. (c) Let $\alpha=bi, \gamma=-bi, \beta=b$. Check product=$-8$ gives $b^2=-8$ (contradiction; typo in problem?).}}

\begin{solution}
\textbf{(a)} From conditions: $\alpha^2 = -\beta^2$ and $\gamma^2 = -\beta^2$, so $\alpha^2 = \gamma^2$.

If $\beta$ non-real, its conjugate $\bar{\beta}$ is also a root. But then $\alpha$ or $\gamma$ must be real. If $\gamma$ real, $\gamma^2 \ge 0$, but $\beta^2=-\gamma^2 \le 0$ implies $\beta$ purely imaginary. Testing: if $\beta=bi$, then $\alpha^2=-\beta^2=-(-b^2)=b^2>0$, making $\alpha$ real. But three roots must include conjugate pair. Contradiction forces $\beta$ real.

\textbf{(b)} Since $\beta$ real and $\beta \neq 0$, $\alpha^2 = -\beta^2 < 0$, so $\alpha = \pm i|\beta|$ (purely imaginary). Similarly for $\gamma$.

\textbf{(c)} Let $\alpha=bi, \beta=b, \gamma=-bi$. By Vieta: $\alpha\beta\gamma = -8 \implies bi \cdot b \cdot (-bi) = b^2 = -8$.

This is impossible for real $b$. [Note: Problem likely has typo; constant should be $-8$ not $+8$.]

If equation is $x^3+Ax^2+Bx-8=0$: $b^2=8 \implies b=\pm 2\sqrt{2}$, giving $A=\mp 2\sqrt{2}, B=8$.

\textbf{Answer:} (Assuming corrected problem) $A=\pm 2\sqrt{2}, B=8$.
\end{solution}

\begin{takeaways}
Conjugate root theorem constrains complex roots; Vieta's formulas connect roots to coefficients.
\end{takeaways}

\vspace{1cm}

% =============================================================================
% PROBLEM 30: Conjugate root theorem
% =============================================================================
\begin{problem}
\begin{enumerate}
    \item Given $z$ is a root of $az^3 + bz^2 + cz + d = 0$ (real $a,b,c,d$), prove $\bar{z}$ is also a root.
    \item Find all roots of $z^3 - 6z^2 + 13z - 20 = 0$ given $1 + 2i$ is one root.
\end{enumerate}
\end{problem}

\vspace{0.5cm}
\rotatebox{180}{\parbox{\textwidth}{\small \textbf{Hint:} (a) Take conjugate of $P(z)=0$: $\overline{P(z)}=\bar{0}=0$. Since coefficients real, $\overline{P(z)}=P(\bar{z})$. (b) If $z_1=1+2i$, then $z_2=1-2i$. Use sum of roots $=-(-6)/1=6$.}}

\begin{solution}
\textbf{(a)} If $P(z) = az^3+bz^2+cz+d=0$, take conjugate:
$$
\overline{P(z)} = \bar{a}\overline{z^3} + \bar{b}\overline{z^2} + \bar{c}\bar{z} + \bar{d} = 0
$$
Since $a,b,c,d$ real: $\bar{a}=a$, etc. Thus $P(\bar{z})=0$, so $\bar{z}$ is a root.

\textbf{(b)} Given $z_1 = 1+2i$, by part (a): $z_2 = 1-2i$.

Sum of roots: $z_1+z_2+z_3 = 6$
$$
(1+2i)+(1-2i)+z_3 = 6 \implies 2+z_3=6 \implies z_3=4
$$

\textbf{Answer:} Roots are $1+2i, 1-2i, 4$.
\end{solution}

\begin{takeaways}
Conjugate root theorem: complex roots of real polynomials come in conjugate pairs.
\end{takeaways}

\vspace{1cm}

% =============================================================================
% PROBLEM 31: Primitive roots of unity
% =============================================================================
\begin{problem}
The roots of $z^n = 1$ are $z_k = e^{2\pi ik/n}$, $k = 1, \dots, n$. If $z_k^m$ generates all roots for $m=1,\ldots,n$, then $z_k$ is a primitive root.
\begin{enumerate}
    \item Show $z_1$ is a primitive root of $z^n = 1$.
    \item Show $z_5$ is a primitive root of $z^6 = 1$.
    \item If $\gcd(n,k)=h$, show $z_k$ primitive implies $h=1$.
\end{enumerate}
\end{problem}

\vspace{0.5cm}
\rotatebox{180}{\parbox{\textwidth}{\small \textbf{Hint:} (a) $z_1^m = e^{2\pi im/n}$ gives all $n$ roots. (b) For $z^6=1$: $5m \bmod 6$ gives $5,4,3,2,1,0$. (c) $z_k^m=z_1^{km}$; generates all roots iff $\gcd(k,n)=1$.}}

\begin{solution}
\textbf{(a)} $z_1^m = e^{2\pi im/n}$ for $m=1,\ldots,n$ generates all $n$-th roots.

\textbf{(b)} $z_5 = e^{10\pi i/6}$. Powers: $z_5^m = e^{10\pi im/6}$.
Values $5m \bmod 6$: $5,10\equiv 4, 15\equiv 3, 20\equiv 2, 25\equiv 1, 30\equiv 0$.
These are $\{5,4,3,2,1,0\}$—all residues, so all roots generated.

\textbf{(c)} $z_k$ generates all roots iff powers $z_k^m = z_1^{km}$ hit all roots. This requires $km \bmod n$ to cover all residues $0,\ldots,n-1$, which happens iff $\gcd(k,n)=1$. Thus $h=1$.

\textbf{Answer:} All parts proven.
\end{solution}

\begin{takeaways}
Primitive roots generate all roots of unity; requires coprimality of index and order.
\end{takeaways}

\vspace{1cm}

% =============================================================================
% PROBLEM 33: Infinite geometric series with complex terms
% =============================================================================
\begin{problem}
Given $z = \cos \alpha + i \sin \alpha$ with $\sin \alpha \neq 0$:
\begin{enumerate}
    \item Prove that $\frac{1}{1-z \cos \alpha} = 1 + i \cot \alpha$
    \item Hence, by considering $\sum_{k=0}^{\infty}(z \cos \alpha)^k$, deduce:
    $$ \sin \alpha \cos \alpha + \sin 2\alpha \cos^2 \alpha + \dots = \cot \alpha $$
\end{enumerate}
\end{problem}

\vspace{0.5cm}
\rotatebox{180}{\parbox{\textwidth}{\small \textbf{Hint:} (a) $1-z\cos\alpha = \sin^2\alpha - i\sin\alpha\cos\alpha = \sin\alpha(\sin\alpha-i\cos\alpha)$. Rationalize. (b) Geometric series $=1+i\cot\alpha$; extract imaginary part.}}

\begin{solution}
\textbf{(a)} $1-z\cos\alpha = 1-\cos^2\alpha - i\sin\alpha\cos\alpha = \sin^2\alpha - i\sin\alpha\cos\alpha = \sin\alpha(\sin\alpha - i\cos\alpha)$

Rationalize:
$$
\frac{1}{\sin\alpha(\sin\alpha-i\cos\alpha)} \cdot \frac{\sin\alpha+i\cos\alpha}{\sin\alpha+i\cos\alpha} = \frac{\sin\alpha+i\cos\alpha}{\sin\alpha(\sin^2\alpha+\cos^2\alpha)} = 1+i\cot\alpha
$$

\textbf{(b)} Geometric series: $\sum_{k=0}^\infty (z\cos\alpha)^k = \frac{1}{1-z\cos\alpha} = 1+i\cot\alpha$

Expand LHS using $z^k = \cos k\alpha + i\sin k\alpha$:
$$
1 + \sum_{k=1}^\infty (\cos k\alpha + i\sin k\alpha)\cos^k\alpha = 1 + \text{(real part)} + i\sum_{k=1}^\infty \sin k\alpha \cos^k\alpha
$$

Equating imaginary parts: $\sum_{k=1}^\infty \sin k\alpha \cos^k\alpha = \cot\alpha$

\textbf{Answer:} As required.
\end{solution}

\begin{takeaways}
Geometric series with complex terms; separate real/imaginary parts to extract trigonometric identities.
\end{takeaways}

\vspace{1cm}

% =============================================================================
% PROBLEM 34: Bound on polynomial roots
% =============================================================================
\begin{problem}
Let $\beta$ be a root of $P(z) = z^n + a_{n-1}z^{n-1} + \cdots + a_0$ with $M = \max(|a_{n-1}|, \ldots, |a_0|)$.
\begin{enumerate}
    \item Show that $|\beta|^n \le M(|\beta|^{n-1} + \cdots + |\beta| + 1)$.
    \item Hence show for any root $\beta$: $|\beta| < 1 + M$.
\end{enumerate}
\end{problem}

\vspace{0.5cm}
\rotatebox{180}{\parbox{\textwidth}{\small \textbf{Hint:} (a) $P(\beta)=0 \implies \beta^n = -(a_{n-1}\beta^{n-1}+\cdots+a_0)$. Triangle inequality: $|\beta|^n \le |a_{n-1}||\beta|^{n-1}+\cdots \le M(\cdots)$. (b) Assume $|\beta| \ge 1+M$, get contradiction.}}

\begin{solution}
\textbf{(a)} From $P(\beta)=0$: $\beta^n = -(a_{n-1}\beta^{n-1}+\cdots+a_0)$

Triangle inequality: $|\beta|^n \le |a_{n-1}||\beta|^{n-1}+\cdots+|a_0| \le M(|\beta|^{n-1}+\cdots+1)$

\textbf{(b)} Assume $|\beta| \ge 1+M$. Then $|\beta|-1 \ge M$.

From (a) with geometric series (for $|\beta|>1$):
$$
|\beta|^n \le M\frac{|\beta|^n-1}{|\beta|-1} \le M\frac{|\beta|^n-1}{M} = |\beta|^n-1
$$

This gives $|\beta|^n \le |\beta|^n - 1$, i.e., $0 \le -1$. Contradiction.

\textbf{Answer:} $|\beta| < 1+M$.
\end{solution}

\begin{takeaways}
Triangle inequality bounds root moduli; contradiction argument establishes strict inequality.
\end{takeaways}

\vspace{1cm}

% =============================================================================
% PROBLEM 35: Weighted sum of roots of unity
% =============================================================================
\begin{problem}
Consider $\omega$ an $n$-th root of unity with $\omega \neq 1$. Given $1 + \omega + \cdots + \omega^{n-1} = 0$:
\begin{enumerate}
    \item Show that $1 + 2\omega + 3\omega^2 + \cdots + n\omega^{n-1} = \frac{n}{\omega-1}$
    \item By factoring $z^n - 1$, deduce $(1 - \omega)(1 - \omega^2)\cdots (1 - \omega^{n-1}) = n$
\end{enumerate}
\end{problem}

\vspace{0.5cm}
\rotatebox{180}{\parbox{\textwidth}{\small \textbf{Hint:} (a) Let $S=1+2\omega+\cdots+n\omega^{n-1}$. Compute $S-\omega S$, use $\omega^n=1$ and given sum. (b) $z^n-1=(z-1)(z-\omega)\cdots(z-\omega^{n-1})$; evaluate at $z=1$.}}

\begin{solution}
\textbf{(a)} Let $S = 1+2\omega+\cdots+n\omega^{n-1}$.

$\omega S = \omega+2\omega^2+\cdots+(n-1)\omega^{n-1}+n\omega^n = \omega+2\omega^2+\cdots+n$ (using $\omega^n=1$)

$S-\omega S = 1+\omega+\cdots+\omega^{n-1}-n = 0-n = -n$

Thus $S(1-\omega)=-n \implies S = \frac{n}{\omega-1}$

\textbf{(b)} $z^n-1 = (z-1)(z-\omega)(z-\omega^2)\cdots(z-\omega^{n-1})$

Divide by $z-1$: $\frac{z^n-1}{z-1} = 1+z+\cdots+z^{n-1} = (z-\omega)(z-\omega^2)\cdots(z-\omega^{n-1})$

At $z=1$: $n = (1-\omega)(1-\omega^2)\cdots(1-\omega^{n-1})$

\textbf{Answer:} Both identities proven.
\end{solution}

\begin{takeaways}
Arithmetic-geometric series via shift-and-subtract; factorization of $z^n-1$ gives product formula.
\end{takeaways}

\vspace{1cm}

% =============================================================================
% PROBLEM 36: Factorization and quadratic formation
% =============================================================================
\begin{problem}
\begin{enumerate}
    \item Show that $z^5 - 1 = (z-1)(z^2 - 2z \cos \frac{2\pi}{5} + 1)(z^2 - 2z \cos \frac{4\pi}{5} + 1)$.
    \item Form a quadratic with roots $\cos \frac{2\pi}{5}$ and $\cos \frac{4\pi}{5}$.
    \item Find exact values of these roots.
\end{enumerate}
\end{problem}

\vspace{0.5cm}
\rotatebox{180}{\parbox{\textwidth}{\small \textbf{Hint:} (a) Fifth roots: $1, e^{2\pi i/5}, e^{4\pi i/5}, e^{-4\pi i/5}, e^{-2\pi i/5}$. Pair conjugates. (b) Divide by $z-1$, substitute $z=1$; or use $u=z+1/z$. (c) Solve $4x^2+2x-1=0$.}}

\begin{solution}
\textbf{(a)} Fifth roots: $1, e^{2\pi ik/5}$ for $k=1,2,3,4$. Conjugate pairs: $(e^{2\pi i/5}, e^{-2\pi i/5})$ and $(e^{4\pi i/5}, e^{-4\pi i/5})$.

Factor: $(z-e^{2\pi i/5})(z-e^{-2\pi i/5}) = z^2 - 2\cos(2\pi/5)z + 1$

Similarly for second pair. Result follows.

\textbf{(b)} $\frac{z^5-1}{z-1} = z^4+z^3+z^2+z+1$. Substitute $z=1$: $5 = (2-2\cos\frac{2\pi}{5})(2-2\cos\frac{4\pi}{5})$

Let $x = \cos\frac{2\pi}{5}$ or $\cos\frac{4\pi}{5}$. From algebra: sum $=-1/2$, product $=-1/4$.

Quadratic: $4x^2+2x-1=0$

\textbf{(c)} $x = \frac{-2\pm\sqrt{4+16}}{8} = \frac{-2\pm 2\sqrt{5}}{8} = \frac{-1\pm\sqrt{5}}{4}$

Since $\cos\frac{2\pi}{5}>0$ and $\cos\frac{4\pi}{5}<0$:
$$
\cos\frac{2\pi}{5} = \frac{-1+\sqrt{5}}{4}, \quad \cos\frac{4\pi}{5} = \frac{-1-\sqrt{5}}{4}
$$

\textbf{Answer:} As shown.
\end{solution}

\begin{takeaways}
Conjugate pairs give real quadratic factors; Vieta's formulas from factorization substitution.
\end{takeaways}

\vspace{1cm}
