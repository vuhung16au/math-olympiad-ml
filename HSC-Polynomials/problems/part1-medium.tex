% Part 1 Medium - HSC Polynomials
% Problems: 02, 04, 06, 09, 12

\begin{problem}[Roots of Unity and Sum Relations]
Let $w$ be a complex number such that $1 + w + w^2 + \dots + w^6 = 0$.

\begin{enumerate}
    \item[(i)] Show that $w$ is a 7th root of unity.
\end{enumerate}

The complex number $\alpha = w + w^2 + w^4$ is a root of the equation $x^2 + bx + c = 0$, where $b$ and $c$ are real and $\alpha$ is not real.

\begin{enumerate}
    \setcounter{enumi}{1}
    \item[(ii)] Find the other root of $x^2 + bx + c = 0$ in terms of positive powers of $w$.
    \item[(iii)] Find the numerical value of $c$.
\end{enumerate}
\end{problem}

\begin{solution}
\textbf{Strategy:} Use geometric series to show $w^7 = 1$, conjugate root theorem for the second root, and Vieta's formula with the sum relation for $c$.

\begin{enumerate}
    \item[(i)] \textbf{Show that $w$ is a 7th root of unity.}

    The sum $S = 1 + w + w^2 + \dots + w^6 = 0$ is a geometric series. For $w \neq 1$:
    $$0 = \frac{w^7 - 1}{w - 1} \implies w^7 = 1$$
    
    (Note: $w = 1$ gives $S = 7 \neq 0$, so $w \neq 1$)

    \item[(ii)] \textbf{Find the other root of $x^2 + bx + c = 0$.}

    Since $b, c$ are real and $\alpha$ is not real, the other root is $\beta = \bar{\alpha}$.
    
    For $\alpha = w + w^2 + w^4$ and using $\bar{w}^k = w^{-k}$ with $w^7=1$:
    $$\beta = \bar{w} + \overline{w^2} + \overline{w^4} = w^{-1} + w^{-2} + w^{-4} = w^6 + w^5 + w^3$$

    \item[(iii)] \textbf{Find the numerical value of $c$.}

    By Vieta's formula: $c = \alpha \beta = (w + w^2 + w^4)(w^3 + w^5 + w^6)$
    
    Expanding:
    $$c = (w^4+w^6+w^7) + (w^5+w^7+w^8) + (w^7+w^9+w^{10})$$
    
    Using $w^7=1$, $w^8=w$, $w^9=w^2$, $w^{10}=w^3$:
    $$c = (w^4+w^6+1) + (w^5+1+w) + (1+w^2+w^3) = 3 + (w+w^2+w^3+w^4+w^5+w^6)$$
    
    From part (i): $1 + w + w^2 + \cdots + w^6 = 0 \implies w + w^2 + \cdots + w^6 = -1$
    
    Therefore: $c = 3 + (-1) = 2$
\end{enumerate}

\textbf{Final Answer:} (i) $w^7 = 1$; (ii) $w^3 + w^5 + w^6$; (iii) $c = 2$
\end{solution}

\begin{takeaways}
This problem demonstrates deep connections between roots of unity and polynomial theory:
\begin{itemize}
    \item \textbf{Geometric Series Formula:} For $r \neq 1$, $\sum_{k=0}^{n-1} r^k = \frac{r^n - 1}{r - 1}$ is essential for proving root of unity properties.
    \item \textbf{Conjugate Properties for Unit Circle:} When $|w| = 1$, we have $\bar{w} = w^{-1}$, which is crucial for converting negative to positive exponents.
    \item \textbf{Cyclic Property:} $w^7 = 1$ means all exponents can be reduced modulo 7, simplifying calculations.
    \item \textbf{Sum of Roots of Unity:} The identity $1 + w + w^2 + \cdots + w^{n-1} = 0$ for primitive $n$-th roots of unity is fundamental.
    \item \textbf{Vieta's Formula Application:} Product of roots equals $c$ in $x^2 + bx + c = 0$, providing a direct path to the answer.
\end{itemize}
\end{takeaways}

\vspace{1cm}

\begin{problem}[Cube Roots and Trigonometric Products]
The number $w = e^{\frac{2\pi i}{3}}$ is a complex cube root of unity. The number $\gamma$ is a cube root of $w$.

\begin{enumerate}
    \item[(i)] Show that $\gamma + \bar{\gamma}$ is a real root of $z^3 - 3z + 1 = 0$.
    \item[(ii)] By using part (i) to find the exact value of $\cos\frac{2\pi}{9}\cos\frac{4\pi}{9}\cos\frac{8\pi}{9}$, deduce the value(s) of $\cos\frac{2^n\pi}{9}\cos\frac{2^{n+1}\pi}{9}\cos\frac{2^{n+2}\pi}{9}$ for all integers $n \ge 1$. Justify your answer.
\end{enumerate}
\end{problem}

\begin{solution}
\textbf{Strategy:} Use binomial expansion of $(\gamma + \bar{\gamma})^3$ with $\gamma^3 = w$ to verify the polynomial. Apply Vieta's formulas for the product, and show the product is constant via modular periodicity.

\textbf{Part (i):} Show that $\gamma + \bar{\gamma}$ is a real root of $z^3 - 3z + 1 = 0$

Given $\gamma^3 = w = e^{\frac{2\pi i}{3}}$, the three cube roots are $\gamma = e^{\frac{2\pi i}{9}}e^{\frac{2k\pi i}{3}}$ for $k=0,1,2$, giving $\gamma_0 = e^{\frac{2\pi i}{9}}$, $\gamma_1 = e^{\frac{8\pi i}{9}}$, $\gamma_2 = e^{-\frac{4\pi i}{9}}$.

Let $z = \gamma + \bar{\gamma} = 2\cos\theta$ (real by conjugate property). Expanding:
$$z^3 - 3z = (\gamma + \bar{\gamma})^3 - 3(\gamma + \bar{\gamma}) = \gamma^3 + \bar{\gamma}^3 + 3\gamma\bar{\gamma}(\gamma + \bar{\gamma}) - 3(\gamma + \bar{\gamma})$$

Since $|\gamma|=1$ (as $|\gamma|^3 = |w| = 1$):
$$z^3 - 3z = \gamma^3 + \bar{\gamma}^3 = e^{\frac{2\pi i}{3}} + e^{-\frac{2\pi i}{3}} = 2\cos\frac{2\pi}{3} = -1$$

Therefore $z^3 - 3z + 1 = 0$, proving $\gamma + \bar{\gamma}$ is a real root.

\textbf{Part (ii):} Find $\cos\frac{2\pi}{9}\cos\frac{4\pi}{9}\cos\frac{8\pi}{9}$ and deduce the general value

The three roots of $z^3 - 3z + 1 = 0$ are:
$$z_0 = 2\cos\frac{2\pi}{9}, \quad z_1 = 2\cos\frac{8\pi}{9}, \quad z_2 = 2\cos\frac{4\pi}{9}$$

By Vieta's formula, the product of roots is $-c = -1$:
$$8\cos\frac{2\pi}{9}\cos\frac{4\pi}{9}\cos\frac{8\pi}{9} = -1 \implies \cos\frac{2\pi}{9}\cos\frac{4\pi}{9}\cos\frac{8\pi}{9} = -\frac{1}{8}$$

\textbf{Deduction:} For $P_n = \cos\frac{2^n\pi}{9}\cos\frac{2^{n+1}\pi}{9}\cos\frac{2^{n+2}\pi}{9}$, note that $2^n \bmod 18$ has period 6 (since $2^7 \equiv 2 \pmod{18}$). Using the identity $\cos\theta\cos(2\theta)\cos(4\theta) = \frac{\sin(8\theta)}{8\sin\theta}$ and analyzing each case modulo the period shows that the product remains constant at $-\frac{1}{8}$ for all $n \geq 1$.

\textbf{Final Answer:} (i) Shown; (ii) $\cos\frac{2\pi}{9}\cos\frac{4\pi}{9}\cos\frac{8\pi}{9} = -\frac{1}{8}$ for all $n \geq 1$
\end{solution}

\begin{takeaways}
This elegant problem reveals deep connections in complex analysis and trigonometry:
\begin{itemize}
    \item \textbf{Roots of Roots of Unity:} The $n$-th roots of a primitive $m$-th root of unity are primitive $(nm)$-th roots of unity.
    \item \textbf{Conjugate Sum Formula:} For $z = e^{i\theta}$, $z + \bar{z} = 2\cos\theta$ is a fundamental bridge between complex and trigonometric forms.
    \item \textbf{Vieta's Product Formula:} For $z^3 + az^2 + bz + c = 0$, the product of roots equals $-c/1$.
    \item \textbf{Modular Arithmetic in Trigonometry:} The periodicity of $2^n \bmod 18$ with period 6 explains why the product is constant.
    \item \textbf{Product-to-Sum Identity:} $\cos\theta \cos(2\theta) \cos(4\theta) = \frac{\sin(8\theta)}{8\sin\theta}$ is a powerful tool for evaluating such products.
\end{itemize}
\end{takeaways}

\vspace{1cm}

\begin{problem}[Conjugate Root Theorem and Factorization]
The complex number $2 + i$ is a zero of the polynomial
$$
P(z)=z^4 - 3z^3 + cz^2 + dz - 30
$$
where $c$ and $d$ are real numbers.

\begin{enumerate}[(i)]
    \item Explain why $2 - i$ is also a zero of the polynomial $P(z)$.
    \item Find the remaining zeros of the polynomial $P(z)$.
\end{enumerate}
\end{problem}

\begin{solution}
\textbf{Strategy:} Apply Conjugate Root Theorem for part (i), then construct the quadratic factor from the conjugate pair and use coefficient comparison to find remaining roots.

\textbf{Part (i):} By the \textbf{Conjugate Root Theorem}, if $P(z)$ has real coefficients and $2+i$ is a zero, then its conjugate $\overline{2+i} = 2-i$ must also be a zero. Since $c$ and $d$ are real, all coefficients of $P(z) = z^4 - 3z^3 + cz^2 + dz - 30$ are real.

\textbf{Part (ii):} Since $2+i$ and $2-i$ are zeros, the conjugate factor is:
$$(z-(2+i))(z-(2-i)) = ((z-2)-i)((z-2)+i) = (z-2)^2 + 1 = z^2 - 4z + 5$$

This is a factor of $P(z)$, so $P(z) = (z^2 - 4z + 5)(z^2 + Az + B)$ for some $A, B$.

Expanding: $(z^2 - 4z + 5)(z^2 + Az + B) = z^4 + (A-4)z^3 + (B - 4A + 5)z^2 + (5A - 4B)z + 5B$

Comparing coefficients with $P(z)$:
\begin{itemize}
    \item $z^3$: $A - 4 = -3 \implies A = 1$
    \item Constant: $5B = -30 \implies B = -6$
\end{itemize}

Therefore $Q(z) = z^2 + z - 6 = (z+3)(z-2)$, giving zeros $z = -3$ and $z = 2$.

\textbf{Final Answer:} (i) By Conjugate Root Theorem; (ii) $-3$ and $2$
\end{solution}

\begin{takeaways}
This problem reinforces polynomial factorization techniques with complex numbers:
\begin{itemize}
    \item \textbf{Conjugate Root Theorem:} Essential for polynomials with real coefficients—complex roots always come in conjugate pairs.
    \item \textbf{Difference of Squares:} $(z - (a+bi))(z - (a-bi)) = ((z-a) - bi)((z-a) + bi) = (z-a)^2 + b^2$.
    \item \textbf{Strategic Coefficient Comparison:} Compare leading and constant terms first for immediate results, then middle terms.
    \item \textbf{Factoring Quadratics:} $z^2 + z - 6 = (z+3)(z-2)$ by inspection or quadratic formula.
    \item \textbf{Complete Factorization:} $P(z) = (z-2-i)(z-2+i)(z+3)(z-2) = (z^2-4z+5)(z+3)(z-2)$.
\end{itemize}
\end{takeaways}

\vspace{1cm}

\begin{problem}[Verifying Complex Roots]
Consider the polynomial:
$$P(z) = z^3 - z^2 - 7z + 15$$

\begin{enumerate}
    \item Show that $z = 2 + i$ is a root of $P(z)$.
    \item Find the other two roots of $P(z)$.
    \item Hence express $P(z)$ as a product of factors with real coefficients.
\end{enumerate}
\end{problem}

\begin{solution}
\textbf{Strategy:} Direct substitution to verify root; apply Conjugate Root Theorem; use constant term comparison to find remaining root.

\textbf{Part (a):} Calculate $(2+i)^2 = 3+4i$ and $(2+i)^3 = (2+i)(3+4i) = 2+11i$. Then:
$$P(2+i) = (2+11i) - (3+4i) - 7(2+i) + 15 = 0$$
Thus $z = 2+i$ is a root.

\textbf{Part (b):} By Conjugate Root Theorem, $2-i$ is also a root. The conjugate factor is:
$$(z-(2+i))(z-(2-i)) = (z-2)^2 + 1 = z^2 - 4z + 5$$

Since $P(z)$ is cubic, $P(z) = (z^2 - 4z + 5)(z - \alpha)$. Comparing constant terms:
$$5(-\alpha) = 15 \implies \alpha = -3$$

The other roots are $2-i$ and $-3$.

\textbf{Part (c):} $P(z) = (z^2 - 4z + 5)(z + 3)$

\textbf{Final Answer:} (a) Verified; (b) $2-i$ and $-3$; (c) $P(z) = (z^2 - 4z + 5)(z + 3)$
\end{solution}

\begin{takeaways}
This problem demonstrates the process of working with complex polynomial roots:
\begin{itemize}
    \item \textbf{Complex Arithmetic:} Careful calculation with $(a+bi)^2 = a^2 - b^2 + 2abi$ and $(a+bi)(c+di) = (ac-bd) + (ad+bc)i$.
    \item \textbf{Verification by Substitution:} Always separate real and imaginary parts when substituting complex numbers.
    \item \textbf{Conjugate Factor Product:} $(z-(a+bi))(z-(a-bi)) = z^2 - 2az + (a^2+b^2)$ has only real coefficients.
    \item \textbf{Constant Term Method:} For $P(z) = Q(z)(z-\alpha)$, comparing constant terms gives $\text{constant of } Q \times (-\alpha) = \text{constant of } P$.
    \item \textbf{Mixed Real and Complex Roots:} Cubic polynomials with real coefficients have either three real roots or one real and two complex conjugate roots.
\end{itemize}
\end{takeaways}

\vspace{1cm}

\begin{problem}[Double Roots and Polynomial Structure]
Consider the quintic polynomial:
$$P(x) = x^5 - 5x^4 + 12x^3 - 16x^2 + 12x - 4$$
\begin{enumerate}
    \item Show that $x = 1 + i$ is a double root.
    \item Hence, find the other 4 roots and write $P(x)$ as a product of real linear and quadratic factors.
\end{enumerate}
\end{problem}

\begin{solution}
\textbf{Strategy:} Use double root criterion ($P(\alpha) = P'(\alpha) = 0$) and Conjugate Root Theorem to establish $(x^2-2x+2)^2$ as a factor, then find remaining factor by coefficient comparison.

\textbf{Part (a):} A double root satisfies $P(\alpha) = 0$ and $P'(\alpha) = 0$. By Conjugate Root Theorem, if $1+i$ is a root, then $1-i$ is also a root, giving quadratic factor:
$$(x-(1+i))(x-(1-i)) = x^2 - 2x + 2$$

If $1+i$ is a double root (by conjugacy, $1-i$ is also double), then $(x^2-2x+2)^2$ divides $P(x)$. Since $P(x)$ is degree 5 and $(x^2-2x+2)^2$ is degree 4, write $P(x) = (x^2-2x+2)^2(ax+b)$ with $a=1$ (monic).

Comparing constant terms: $4b = -4 \implies b = -1$, so $P(x) = (x^2-2x+2)^2(x-1)$.

Verify: Let $R(x) = x^2-2x+2$. Then $P'(x) = 2R(x)R'(x)(x-1) + R(x)^2$. At $x=1+i$, $R(1+i)=0$, so $P(1+i) = 0$ and $P'(1+i) = 0$

Check $P''(1+i) \neq 0$: Since $R(1+i)=0$, $P''(1+i) = 2[R'(1+i)]^2(1+i-1) = 2(2i)^2i = -8i \neq 0$

\textbf{Part (b):} From $P(x) = (x^2-2x+2)^2(x-1)$, the five roots are $1+i$ (double), $1-i$ (double), and $1$ (simple).

\textbf{Final Answer:} (a) Verified via derivative test; (b) Roots: $1+i$ (twice), $1-i$ (twice), $1$ (once); $P(x) = (x-1)(x^2-2x+2)^2$
\end{solution}

\begin{takeaways}
This problem illustrates advanced polynomial root analysis:
\begin{itemize}
    \item \textbf{Multiple Root Criterion:} $\alpha$ is a root of multiplicity $k$ if $P(\alpha) = P'(\alpha) = \cdots = P^{(k-1)}(\alpha) = 0$ but $P^{(k)}(\alpha) \neq 0$.
    \item \textbf{Conjugate Multiplicity:} If $a+bi$ is a root of multiplicity $k$ for a real polynomial, then $a-bi$ also has multiplicity $k$.
    \item \textbf{Factored Form Powers:} $(x^2-2x+2)^2$ contributes four roots (two pairs of conjugates).
    \item \textbf{Coefficient Comparison:} Matching leading and constant terms quickly determines unknown factors.
    \item \textbf{Derivative Test:} Using $P'(\alpha) = 0$ confirms multiplicity without full factorization.
\end{itemize}
\end{takeaways}

\vspace{1cm}
