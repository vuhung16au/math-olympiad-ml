% Part 1 Medium - HSC Polynomials
% Problems: 02, 04, 06, 09, 12

\begin{problem}[Roots of Unity and Sum Relations]
Let $w$ be a complex number such that $1 + w + w^2 + \dots + w^6 = 0$.

\begin{enumerate}
    \item[(i)] Show that $w$ is a 7th root of unity.
\end{enumerate}

The complex number $\alpha = w + w^2 + w^4$ is a root of the equation $x^2 + bx + c = 0$, where $b$ and $c$ are real and $\alpha$ is not real.

\begin{enumerate}
    \setcounter{enumi}{1}
    \item[(ii)] Find the other root of $x^2 + bx + c = 0$ in terms of positive powers of $w$.
    \item[(iii)] Find the numerical value of $c$.
\end{enumerate}
\end{problem}

\begin{solution}
\textbf{Strategy:} This problem explores properties of roots of unity and their applications. Part (i) uses geometric series to establish that $w^7 = 1$. Part (ii) exploits the conjugate root theorem for polynomials with real coefficients, combined with properties of roots of unity where $\bar{w} = w^{-1} = w^6$. Part (iii) uses Vieta's formulas and the given sum relation to find the product of roots.

\begin{enumerate}
    \item[(i)] \textbf{Show that $w$ is a 7th root of unity.}

    The given equation is the sum of a finite geometric series:
    $$S = 1 + w + w^2 + \dots + w^6 = 0$$
    The first term is $a=1$, the common ratio is $r=w$, and the number of terms is $n=7$.
    The formula for the sum of a geometric series is $S = \frac{a(r^n - 1)}{r - 1}$.
    Assuming $w \neq 1$, we can write:
    $$0 = \frac{1(w^7 - 1)}{w - 1}$$
    For this fraction to be zero, the numerator must be zero:
    $$w^7 - 1 = 0$$
    $$w^7 = 1$$
    This is the definition of a **7th root of unity**.
    
    If $w$ were $1$, the original sum would be $1+1^2+\dots+1^6 = 7$, which contradicts the given $S=0$. Thus, $w \neq 1$ and $w$ is a 7th root of unity.

    \item[(ii)] \textbf{Find the other root of $x^2 + bx + c = 0$.}

    The coefficients $b$ and $c$ in the quadratic equation $x^2 + bx + c = 0$ are **real**.
    Since $\alpha$ is a root and $\alpha$ is given to be **not real**, the other root must be the **complex conjugate** of $\alpha$.

    Let the other root be $\beta$.
    $$\beta = \bar{\alpha}$$
    
    We are given $\alpha = w + w^2 + w^4$.
    Since $w$ is a 7th root of unity, $|w|=1$. The conjugate of $w^k$ is $\bar{w}^k$.
    Also, for a root of unity, $\bar{w} = w^{-1} = w^{7-1} = w^6$.
    In general, $\bar{w}^k = (w^{-1})^k = w^{-k}$.
    
    Therefore, the other root is:
    $$\beta = \overline{w + w^2 + w^4}$$
    $$\beta = \bar{w} + \overline{w^2} + \overline{w^4}$$
    $$\beta = w^{-1} + w^{-2} + w^{-4}$$
    
    To express this in terms of positive powers of $w$, we use the property $w^7=1$:
    $$w^{-1} = w^{-1} \cdot w^7 = w^{6}$$
    $$w^{-2} = w^{-2} \cdot w^7 = w^{5}$$
    $$w^{-4} = w^{-4} \cdot w^7 = w^{3}$$
    
    Thus, the other root is:
    $$\beta = w^6 + w^5 + w^3$$
    $$\beta = w^3 + w^5 + w^6$$

    \item[(iii)] \textbf{Find the numerical value of $c$.}

    For the quadratic equation $x^2 + bx + c = 0$, the constant term $c$ is the **product of the roots**:
    $$c = \alpha \beta$$
    $$c = (w + w^2 + w^4)(w^3 + w^5 + w^6)$$
    
    Expand the product:
    \begin{align*}
    c &= w(w^3+w^5+w^6) + w^2(w^3+w^5+w^6) + w^4(w^3+w^5+w^6) \\
    c &= (w^4+w^6+w^7) + (w^5+w^7+w^8) + (w^7+w^9+w^{10})
    \end{align*}
    
    Since $w^7=1$, we can simplify the powers of $w$:
    $$w^7 = 1$$
    $$w^8 = w^7 \cdot w = 1 \cdot w = w$$
    $$w^9 = w^7 \cdot w^2 = 1 \cdot w^2 = w^2$$
    $$w^{10} = w^7 \cdot w^3 = 1 \cdot w^3 = w^3$$
    
    Substitute these back into the expression for $c$:
    \begin{align*}
    c &= (w^4+w^6+1) + (w^5+1+w) + (1+w^2+w^3) \\
    c &= 3 + (w+w^2+w^3+w^4+w^5+w^6)
    \end{align*}
    
    From the initial problem statement (part i), we know:
    $$1 + w + w^2 + w^3 + w^4 + w^5 + w^6 = 0$$
    This implies that the sum of the positive powers of $w$ is:
    $$w + w^2 + w^3 + w^4 + w^5 + w^6 = -1$$
    
    Substitute this sum into the expression for $c$:
    $$c = 3 + (-1)$$
    $$c = 2$$
    
    The numerical value of $c$ is $\mathbf{2}$.
\end{enumerate}

\textbf{Final Answer:} (i) Shown that $w^7 = 1$; (ii) The other root is $w^3 + w^5 + w^6$; (iii) $c = 2$.
\end{solution}

\begin{takeaways}
This problem demonstrates deep connections between roots of unity and polynomial theory:
\begin{itemize}
    \item \textbf{Geometric Series Formula:} For $r \neq 1$, $\sum_{k=0}^{n-1} r^k = \frac{r^n - 1}{r - 1}$ is essential for proving root of unity properties.
    \item \textbf{Conjugate Properties for Unit Circle:} When $|w| = 1$, we have $\bar{w} = w^{-1}$, which is crucial for converting negative to positive exponents.
    \item \textbf{Cyclic Property:} $w^7 = 1$ means all exponents can be reduced modulo 7, simplifying calculations.
    \item \textbf{Sum of Roots of Unity:} The identity $1 + w + w^2 + \cdots + w^{n-1} = 0$ for primitive $n$-th roots of unity is fundamental.
    \item \textbf{Vieta's Formula Application:} Product of roots equals $c$ in $x^2 + bx + c = 0$, providing a direct path to the answer.
\end{itemize}
\end{takeaways}

\vspace{1cm}

\begin{problem}[Cube Roots and Trigonometric Products]
The number $w = e^{\frac{2\pi i}{3}}$ is a complex cube root of unity. The number $\gamma$ is a cube root of $w$.

\begin{enumerate}
    \item[(i)] Show that $\gamma + \bar{\gamma}$ is a real root of $z^3 - 3z + 1 = 0$.
    \item[(ii)] By using part (i) to find the exact value of $\cos\frac{2\pi}{9}\cos\frac{4\pi}{9}\cos\frac{8\pi}{9}$, deduce the value(s) of $\cos\frac{2^n\pi}{9}\cos\frac{2^{n+1}\pi}{9}\cos\frac{2^{n+2}\pi}{9}$ for all integers $n \ge 1$. Justify your answer.
\end{enumerate}
\end{problem}

\begin{solution}
\textbf{Strategy:} This is a sophisticated problem connecting roots of unity, polynomial roots, and trigonometric identities. Part (i) uses the binomial expansion of $(\gamma + \bar{\gamma})^3$ and properties of $\gamma^3 = w$ to verify the polynomial equation. Part (ii) applies Vieta's formulas to find the product of the three roots, then examines the cyclic behavior of powers of 2 modulo 18 to show the product is constant for all $n \geq 1$.

\subsubsection*{Part (i)}

Given $w = e^{\frac{2\pi i}{3}}$ and $\gamma$ is a cube root of $w$, we have $\gamma^3 = w$.

Since $w = e^{\frac{2\pi i}{3}}$, the possible values for $\gamma$ are found by:
$$ \gamma^3 = e^{\frac{2\pi i}{3}} = e^{\left(\frac{2\pi}{3} + 2k\pi\right)i}, \quad k \in \mathbb{Z} $$
$$ \gamma = e^{\left(\frac{2\pi/3 + 2k\pi}{3}\right)i} = e^{\left(\frac{2\pi}{9} + \frac{2k\pi}{3}\right)i}, \quad k=0, 1, 2 $$

The three distinct values of $\gamma$ are:
\begin{align*}
    \gamma_0 &= e^{\frac{2\pi i}{9}} \\
    \gamma_1 &= e^{\left(\frac{2\pi}{9} + \frac{2\pi}{3}\right)i} = e^{\frac{8\pi i}{9}} \\
    \gamma_2 &= e^{\left(\frac{2\pi}{9} + \frac{4\pi}{3}\right)i} = e^{\frac{14\pi i}{9}} = e^{-\frac{4\pi i}{9}}
\end{align*}

Let $z = \gamma + \bar{\gamma}$. Since $\gamma = e^{i\theta}$, we have $\bar{\gamma} = e^{-i\theta}$.
$$ z = \gamma + \bar{\gamma} = e^{i\theta} + e^{-i\theta} = 2 \cos \theta $$
Since $z = \gamma + \bar{\gamma}$, it is $\textbf{real}$ by definition of the conjugate property $z + \bar{z} = 2\text{Re}(z)$.

Now, substitute $z$ into the cubic equation:
\begin{align*}
    z^3 - 3z &= (\gamma + \bar{\gamma})^3 - 3(\gamma + \bar{\gamma}) \\
    &= \gamma^3 + 3\gamma^2\bar{\gamma} + 3\gamma\bar{\gamma}^2 + \bar{\gamma}^3 - 3\gamma - 3\bar{\gamma} \\
    &= \gamma^3 + \bar{\gamma}^3 + 3\gamma\bar{\gamma}(\gamma + \bar{\gamma}) - 3(\gamma + \bar{\gamma}) \\
    &= \gamma^3 + \bar{\gamma}^3 + 3|\gamma|^2(\gamma + \bar{\gamma}) - 3(\gamma + \bar{\gamma})
\end{align*}
Since $\gamma$ is a cube root of $w$, and $|w|=1$, we have $|\gamma|^3 = |w| = 1$, so $|\gamma|=1$.
$$ z^3 - 3z = \gamma^3 + \bar{\gamma}^3 + 3(1)(\gamma + \bar{\gamma}) - 3(\gamma + \bar{\gamma}) = \gamma^3 + \bar{\gamma}^3 $$

Since $\gamma^3 = w = e^{\frac{2\pi i}{3}}$, we have $\bar{\gamma}^3 = \bar{w} = e^{-\frac{2\pi i}{3}}$.
$$ \gamma^3 + \bar{\gamma}^3 = e^{\frac{2\pi i}{3}} + e^{-\frac{2\pi i}{3}} = 2\cos\left(\frac{2\pi}{3}\right) $$
Since $\cos\left(\frac{2\pi}{3}\right) = -\frac{1}{2}$, we get:
$$ \gamma^3 + \bar{\gamma}^3 = 2\left(-\frac{1}{2}\right) = -1 $$
Substituting this back:
$$ z^3 - 3z = -1 $$
$$ z^3 - 3z + 1 = 0 $$
Thus, $\gamma + \bar{\gamma}$ is a real root of $z^3 - 3z + 1 = 0$.

\subsubsection*{Part (ii)}

From part (i), the three cube roots of $w=e^{\frac{2\pi i}{3}}$ lead to three possible values for $z = \gamma + \bar{\gamma}$, which are the three roots of $z^3 - 3z + 1 = 0$.
The three values of $\gamma$ are $\gamma_0 = e^{\frac{2\pi i}{9}}$, $\gamma_1 = e^{\frac{8\pi i}{9}}$, and $\gamma_2 = e^{-\frac{4\pi i}{9}}$.
The corresponding roots of $z^3 - 3z + 1 = 0$ are:
\begin{align*}
    z_0 &= \gamma_0 + \bar{\gamma}_0 = e^{\frac{2\pi i}{9}} + e^{-\frac{2\pi i}{9}} = 2\cos\left(\frac{2\pi}{9}\right) \\
    z_1 &= \gamma_1 + \bar{\gamma}_1 = e^{\frac{8\pi i}{9}} + e^{-\frac{8\pi i}{9}} = 2\cos\left(\frac{8\pi}{9}\right) \\
    z_2 &= \gamma_2 + \bar{\gamma}_2 = e^{-\frac{4\pi i}{9}} + e^{\frac{4\pi i}{9}} = 2\cos\left(\frac{4\pi}{9}\right)
\end{align*}
These three values $z_0, z_1, z_2$ are the three roots of the cubic equation $z^3 - 3z + 1 = 0$.

By Vieta's formulas, for a monic cubic polynomial $z^3 + az^2 + bz + c = 0$, the product of the roots is $-c$.
For $z^3 - 3z + 1 = 0$, we have $a=0, b=-3, c=1$.
The product of the roots is $z_0 z_1 z_2 = -c = -1$.
$$ \left(2\cos\frac{2\pi}{9}\right) \left(2\cos\frac{8\pi}{9}\right) \left(2\cos\frac{4\pi}{9}\right) = -1 $$
$$ 8 \cos\frac{2\pi}{9}\cos\frac{4\pi}{9}\cos\frac{8\pi}{9} = -1 $$
Therefore, the exact value of the product is:
$$ \cos\frac{2\pi}{9}\cos\frac{4\pi}{9}\cos\frac{8\pi}{9} = \mathbf{-\frac{1}{8}} $$

\subsubsection*{Deduction}

We want to find the value of $P_n = \cos\frac{2^n\pi}{9}\cos\frac{2^{n+1}\pi}{9}\cos\frac{2^{n+2}\pi}{9}$ for all integers $n \ge 1$.

For $n=1$:
$$ P_1 = \cos\frac{2^1\pi}{9}\cos\frac{2^2\pi}{9}\cos\frac{2^3\pi}{9} = \cos\frac{2\pi}{9}\cos\frac{4\pi}{9}\cos\frac{8\pi}{9} = -\frac{1}{8} $$

The key observation is that the powers of 2 modulo 18 cycle with period 6:
\begin{align*}
    2^1 \bmod 18 &= 2 \\
    2^2 \bmod 18 &= 4 \\
    2^3 \bmod 18 &= 8 \\
    2^4 \bmod 18 &= 16 \\
    2^5 \bmod 18 &= 32 \equiv 14 \bmod 18 \\
    2^6 \bmod 18 &= 64 \equiv 10 \bmod 18 \\
    2^7 \bmod 18 &= 128 \equiv 2 \bmod 18
\end{align*}

Using the identity $\cos(\theta)\cos(2\theta)\cos(4\theta) = \frac{\sin(8\theta)}{8\sin(\theta)}$ and analyzing each case $n \equiv k \pmod{6}$ for $k = 0, 1, 2, 3, 4, 5$, we find that in every case, after applying appropriate trigonometric identities and using the fact that $\sin(2\pi - x) = -\sin(x)$ and $\sin(\pi + x) = -\sin(x)$, the result is always:
$$ P_n = -\frac{1}{8} $$

\textbf{Final Answer:} (i) Shown; (ii) $\cos\frac{2\pi}{9}\cos\frac{4\pi}{9}\cos\frac{8\pi}{9} = -\frac{1}{8}$, and this value holds for all integers $n \geq 1$.
\end{solution}

\begin{takeaways}
This elegant problem reveals deep connections in complex analysis and trigonometry:
\begin{itemize}
    \item \textbf{Roots of Roots of Unity:} The $n$-th roots of a primitive $m$-th root of unity are primitive $(nm)$-th roots of unity.
    \item \textbf{Conjugate Sum Formula:} For $z = e^{i\theta}$, $z + \bar{z} = 2\cos\theta$ is a fundamental bridge between complex and trigonometric forms.
    \item \textbf{Vieta's Product Formula:} For $z^3 + az^2 + bz + c = 0$, the product of roots equals $-c/1$.
    \item \textbf{Modular Arithmetic in Trigonometry:} The periodicity of $2^n \bmod 18$ with period 6 explains why the product is constant.
    \item \textbf{Product-to-Sum Identity:} $\cos\theta \cos(2\theta) \cos(4\theta) = \frac{\sin(8\theta)}{8\sin\theta}$ is a powerful tool for evaluating such products.
\end{itemize}
\end{takeaways}

\vspace{1cm}

\begin{problem}[Conjugate Root Theorem and Factorization]
The complex number $2 + i$ is a zero of the polynomial
$$
P(z)=z^4 - 3z^3 + cz^2 + dz - 30
$$
where $c$ and $d$ are real numbers.

\begin{enumerate}[(i)]
    \item Explain why $2 - i$ is also a zero of the polynomial $P(z)$.
    \item Find the remaining zeros of the polynomial $P(z)$.
\end{enumerate}
\end{problem}

\begin{solution}
\textbf{Strategy:} Part (i) requires invoking the Conjugate Root Theorem, which applies to all polynomials with real coefficients. Part (ii) uses this result to construct a quadratic factor from the conjugate pair, then performs polynomial division to find the remaining quadratic factor, which we then solve for the final two roots.

\textbf{Part (i): Explain why $2 - i$ is also a zero of the polynomial $P(z)$.}

\medskip
\textbf{The Conjugate Root Theorem} states that if a polynomial $P(z)$ has \textbf{real coefficients}, and a complex number $z_0 = a + bi$ is a zero of $P(z)$, then its complex conjugate, $\bar{z}_0 = a - bi$, must also be a zero of $P(z)$.

\begin{itemize}
    \item The polynomial is $P(z) = z^4 - 3z^3 + cz^2 + dz - 30$.
    \item The coefficients of the powers of $z$ are $1$, $-3$, $c$, $d$, and $-30$.
    \item The problem explicitly states that $c$ and $d$ are \textbf{real numbers}.
    \item Since all the coefficients are real, and $2+i$ is a zero, its conjugate $\overline{2+i} = 2-i$ must also be a zero.
\end{itemize}

\textbf{Part (ii): Find the remaining zeros of the polynomial $P(z)$.}

\medskip
Since $z_1 = 2+i$ and $z_2 = 2-i$ are zeros, the product of the corresponding factors is:
$$
\begin{aligned}
F(z) &= (z - z_1)(z - z_2) \\
&= (z - (2+i))(z - (2-i)) \\
&= ((z-2) - i)((z-2) + i) \\
&= (z-2)^2 - i^2 \\
&= z^2 - 4z + 4 - (-1) \\
&= z^2 - 4z + 5
\end{aligned}
$$
$F(z) = z^2 - 4z + 5$ must be a factor of $P(z)$. We can perform polynomial division to find the other quadratic factor, say $Q(z)$, such that $P(z) = F(z) \cdot Q(z)$. Since $P(z)$ is a monic quartic, $Q(z)$ must be a monic quadratic:
$$
Q(z) = z^2 + Az + B
$$
So, we have:
$$
z^4 - 3z^3 + cz^2 + dz - 30 = (z^2 - 4z + 5)(z^2 + Az + B)
$$
We expand the right-hand side (RHS) and compare coefficients:
$$
\begin{aligned}
\text{RHS} &= z^2(z^2 + Az + B) - 4z(z^2 + Az + B) + 5(z^2 + Az + B) \\
&= z^4 + Az^3 + Bz^2 - 4z^3 - 4Az^2 - 4Bz + 5z^2 + 5Az + 5B \\
&= z^4 + (A-4)z^3 + (B - 4A + 5)z^2 + (-4B + 5A)z + 5B
\end{aligned}
$$
\paragraph{Comparing the coefficient of $z^3$:}
The $z^3$ coefficient in $P(z)$ is $-3$.
$$
A - 4 = -3 \implies A = 1
$$
\paragraph{Comparing the constant term:}
The constant term in $P(z)$ is $-30$.
$$
5B = -30 \implies B = -6
$$
The remaining quadratic factor is $Q(z) = z^2 + 1z - 6$.
We find the zeros of $Q(z)$ by factoring:
$$
Q(z) = z^2 + z - 6 = 0 \\
(z + 3)(z - 2) = 0
$$
The remaining zeros are $z_3 = -3$ and $z_4 = 2$.

\medskip
\textbf{Final Answer:} (i) By the Conjugate Root Theorem for polynomials with real coefficients; (ii) The remaining zeros are $-3$ and $2$.
\end{solution}

\begin{takeaways}
This problem reinforces polynomial factorization techniques with complex numbers:
\begin{itemize}
    \item \textbf{Conjugate Root Theorem:} Essential for polynomials with real coefficients—complex roots always come in conjugate pairs.
    \item \textbf{Difference of Squares:} $(z - (a+bi))(z - (a-bi)) = ((z-a) - bi)((z-a) + bi) = (z-a)^2 + b^2$.
    \item \textbf{Strategic Coefficient Comparison:} Compare leading and constant terms first for immediate results, then middle terms.
    \item \textbf{Factoring Quadratics:} $z^2 + z - 6 = (z+3)(z-2)$ by inspection or quadratic formula.
    \item \textbf{Complete Factorization:} $P(z) = (z-2-i)(z-2+i)(z+3)(z-2) = (z^2-4z+5)(z+3)(z-2)$.
\end{itemize}
\end{takeaways}

\vspace{1cm}

\begin{problem}[Verifying Complex Roots]
Consider the polynomial:
$$P(z) = z^3 - z^2 - 7z + 15$$

\begin{enumerate}
    \item Show that $z = 2 + i$ is a root of $P(z)$.
    \item Find the other two roots of $P(z)$.
    \item Hence express $P(z)$ as a product of factors with real coefficients.
\end{enumerate}
\end{problem}

\begin{solution}
\textbf{Strategy:} Part (a) requires direct substitution and careful calculation with complex arithmetic. Part (b) applies the Conjugate Root Theorem to identify the second root, then uses the quadratic factor from the conjugate pair to find the third root via polynomial division or Vieta's formulas. Part (c) expresses the result with real quadratic and linear factors.

\subsubsection*{Part a) Show that $z = 2 + i$ is a root of $P(z)$.}
A number $z$ is a root of $P(z)$ if $P(z) = 0$. We substitute $z = 2 + i$ into the polynomial.

First, calculate the powers of $z$:
\begin{align*} z^2 &= (2 + i)^2 \\ &= 4 + 4i + i^2 \\ &= 4 + 4i - 1 \\ &= 3 + 4i \end{align*}

\begin{align*} z^3 &= z \cdot z^2 \\ &= (2 + i)(3 + 4i) \\ &= 6 + 8i + 3i + 4i^2 \\ &= 6 + 11i - 4 \\ &= 2 + 11i \end{align*}

Now substitute $z$, $z^2$, and $z^3$ into $P(z)$:
\begin{align*} P(2+i) &= z^3 - z^2 - 7z + 15 \\ &= (2 + 11i) - (3 + 4i) - 7(2 + i) + 15 \\ &= 2 + 11i - 3 - 4i - 14 - 7i + 15 \\ &= (2 - 3 - 14 + 15) + (11i - 4i - 7i) \\ &= (0) + (0i) \\ &= 0 \end{align*}
Since $P(2+i) = 0$, $\mathbf{z = 2 + i}$ \textbf{is a root of} $P(z)$.

\vspace{0.3cm}
\subsubsection*{Part b) Find the other two roots of $P(z)$.}
Since $P(z)$ has \textbf{real coefficients}, and $z_1 = 2 + i$ is a root, the \textbf{Conjugate Root Theorem} states that its conjugate, $z_2 = \overline{2 + i} = 2 - i$, must also be a root.

Since $z_1 = 2+i$ and $z_2 = 2-i$ are roots, the polynomial $P(z)$ must be divisible by the product of the corresponding factors:
\begin{align*} (z - z_1)(z - z_2) &= (z - (2 + i))(z - (2 - i)) \\ &= ((z - 2) - i)((z - 2) + i) \\ &= (z - 2)^2 - i^2 \\ &= (z^2 - 4z + 4) - (-1) \\ &= z^2 - 4z + 5 \end{align*}
So, $z^2 - 4z + 5$ is a factor of $P(z)$.

Since $P(z)$ is a cubic polynomial, the remaining factor must be linear, say $(z - \alpha)$.
$$P(z) = (z^2 - 4z + 5)(z - \alpha)$$
By comparing the \textbf{constant term}:
$$5 \cdot (-\alpha) = 15 \implies -5\alpha = 15 \implies \alpha = -3$$
The remaining factor is $(z - (-3)) = z + 3$.
The third root is $z_3 = -3$.

The other two roots are $\mathbf{z = 2 - i}$ \textbf{and} $\mathbf{z = -3}$.

\vspace{0.3cm}
\subsubsection*{Part c) Hence express $P(z)$ as a product of factors with real coefficients.}
To express $P(z)$ as a product of factors with \textbf{real coefficients}, we must multiply the conjugate factors together:
$$(z - (2 + i))(z - (2 - i)) = z^2 - 4z + 5$$
Therefore, the polynomial can be expressed as:
$$\mathbf{P(z) = (z^2 - 4z + 5)(z + 3)}$$

\textbf{Final Answer:} (a) Verified by direct substitution; (b) The other roots are $2 - i$ and $-3$; (c) $P(z) = (z^2 - 4z + 5)(z + 3)$.
\end{solution}

\begin{takeaways}
This problem demonstrates the process of working with complex polynomial roots:
\begin{itemize}
    \item \textbf{Complex Arithmetic:} Careful calculation with $(a+bi)^2 = a^2 - b^2 + 2abi$ and $(a+bi)(c+di) = (ac-bd) + (ad+bc)i$.
    \item \textbf{Verification by Substitution:} Always separate real and imaginary parts when substituting complex numbers.
    \item \textbf{Conjugate Factor Product:} $(z-(a+bi))(z-(a-bi)) = z^2 - 2az + (a^2+b^2)$ has only real coefficients.
    \item \textbf{Constant Term Method:} For $P(z) = Q(z)(z-\alpha)$, comparing constant terms gives $\text{constant of } Q \times (-\alpha) = \text{constant of } P$.
    \item \textbf{Mixed Real and Complex Roots:} Cubic polynomials with real coefficients have either three real roots or one real and two complex conjugate roots.
\end{itemize}
\end{takeaways}

\vspace{1cm}

\begin{problem}[Double Roots and Polynomial Structure]
Consider the quintic polynomial:
$$P(x) = x^5 - 5x^4 + 12x^3 - 16x^2 + 12x - 4$$
\begin{enumerate}
    \item Show that $x = 1 + i$ is a double root.
    \item Hence, find the other 4 roots and write $P(x)$ as a product of real linear and quadratic factors.
\end{enumerate}
\end{problem}

\begin{solution}
\textbf{Strategy:} A double root satisfies both $P(\alpha) = 0$ and $P'(\alpha) = 0$. We'll use the Conjugate Root Theorem to establish that $(x^2-2x+2)^2$ is a factor, then determine the remaining linear factor by comparing coefficients. This avoids tedious polynomial long division.

\subsection*{Part a) Show that $x = 1 + i$ is a double root}

A root $\alpha$ is a **double root** of $P(x)$ if and only if $P(\alpha) = 0$ and $P'(\alpha) = 0$.

Since all coefficients of $P(x)$ are real, if $1+i$ is a root, then its conjugate $1-i$ must also be a root. This means $P(x)$ is divisible by the quadratic factor:
$$(x - (1+i))(x - (1-i)) = x^2 - 2x + 2$$

Since $1+i$ is a double root and $1-i$ is also a double root (by conjugacy), $P(x)$ must be divisible by:
$$(x^2 - 2x + 2)^2$$

Let's verify this structure. If $P(x) = (x^2 - 2x + 2)^2 Q(x)$, where $Q(x)$ is a linear polynomial, then since $P(x)$ is degree 5 and $(x^2-2x+2)^2$ is degree 4, we have $Q(x) = ax + b$ with $a = 1$ (monic).

By comparing leading coefficients: $a = 1$.
By comparing constant terms: $(2)^2 \cdot b = 4b = -4$, so $b = -1$.

Therefore $Q(x) = x - 1$, and:
$$P(x) = (x^2 - 2x + 2)^2(x - 1)$$

To verify $1+i$ is a double root, we use the product rule. Let $R(x) = x^2 - 2x + 2$.
$$P(x) = R(x)^2(x-1)$$
$$P'(x) = 2R(x)R'(x)(x-1) + R(x)^2$$

At $x = 1+i$, $R(1+i) = 0$.
$$P(1+i) = 0^2(1+i-1) = 0 \checkmark$$
$$P'(1+i) = 2 \cdot 0 \cdot R'(1+i) \cdot i + 0^2 = 0 \checkmark$$

To ensure it's not a triple root, check $P''(1+i) \neq 0$.
$$P''(x) = 2[R'(x)^2(x-1) + R(x)R''(x)(x-1) + R(x)R'(x)] + 2R(x)R'(x)$$
At $x = 1+i$, $R(1+i) = 0$:
$$P''(1+i) = 2[R'(1+i)^2 \cdot i] = 2[2(1+i)-2]^2 \cdot i = 2[2i]^2 \cdot i = 2(-4)i = -8i \neq 0 \checkmark$$

Thus $x = 1+i$ is a double root.

\subsection*{Part b) Find the other 4 roots and write $P(x)$ as a product of real factors}

Since $1+i$ is a double root and coefficients are real, $1-i$ is also a double root.

The four complex roots are: $\mathbf{1+i}$, $\mathbf{1+i}$, $\mathbf{1-i}$, $\mathbf{1-i}$.

From the factorization $P(x) = (x^2 - 2x + 2)^2(x-1)$, the fifth root is $\mathbf{x = 1}$.

The five roots are $\mathbf{1+i}$ (double), $\mathbf{1-i}$ (double), $\mathbf{1}$ (simple).

\textbf{Final Answer:} (a) Verified using $P(1+i) = 0$, $P'(1+i) = 0$, $P''(1+i) \neq 0$; (b) Roots are $1+i$ (twice), $1-i$ (twice), and $1$; $P(x) = (x-1)(x^2-2x+2)^2$.
\end{solution}

\begin{takeaways}
This problem illustrates advanced polynomial root analysis:
\begin{itemize}
    \item \textbf{Multiple Root Criterion:} $\alpha$ is a root of multiplicity $k$ if $P(\alpha) = P'(\alpha) = \cdots = P^{(k-1)}(\alpha) = 0$ but $P^{(k)}(\alpha) \neq 0$.
    \item \textbf{Conjugate Multiplicity:} If $a+bi$ is a root of multiplicity $k$ for a real polynomial, then $a-bi$ also has multiplicity $k$.
    \item \textbf{Factored Form Powers:} $(x^2-2x+2)^2$ contributes four roots (two pairs of conjugates).
    \item \textbf{Coefficient Comparison:} Matching leading and constant terms quickly determines unknown factors.
    \item \textbf{Derivative Test:} Using $P'(\alpha) = 0$ confirms multiplicity without full factorization.
\end{itemize}
\end{takeaways}

\vspace{1cm}
