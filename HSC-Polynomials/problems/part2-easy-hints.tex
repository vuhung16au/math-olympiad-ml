% Part 2: Easy Problems (5 problems with hints - upside-down format)
% Students attempt first, then rotate page 180° to read hint

% =============================================================================
% PROBLEM 23: Binomial Theorem Statement
% =============================================================================
\begin{problem}
State the Binomial Theorem for the expansion of $(x+a)^n$, where $n$ is a positive integer. Define the binomial coefficient used in the expansion.
\end{problem}

\vspace{0.5cm}
\begin{hintbox}
The theorem expresses $(x+a)^n$ as a sum involving binomial coefficients $\binom{n}{r} = \frac{n!}{r!(n-r)!}$.
\end{hintbox}

\begin{solution}
The Binomial Theorem states:
$$
(x+a)^n = \sum_{r=0}^{n} \binom{n}{r} x^{n-r} a^{r}
$$
or equivalently:
$$
(x+a)^n = \binom{n}{0} x^n + \binom{n}{1} x^{n-1} a + \binom{n}{2} x^{n-2} a^2 + \dots + \binom{n}{n} a^n
$$

The binomial coefficient is defined as:
$$
\binom{n}{r} = \frac{n!}{r!(n-r)!}
$$
where $n! = n \times (n-1) \times \dots \times 2 \times 1$ and $0! = 1$.

\textbf{Answer:} Binomial Theorem with coefficients $\binom{n}{r}$.
\end{solution}

\begin{takeaways}
Binomial coefficients count combinations; central to polynomial expansions.
\end{takeaways}

\vspace{1cm}

% =============================================================================
% PROBLEM 39: Integration Problem - Partial Fraction Decomposition
% =============================================================================
\begin{problem}
Consider the integral:
$$
I = \int \frac{1}{x^6 - x^4} \, dx
$$
\begin{enumerate}
    \item[(i)] Factorise the polynomial $P(x) = x^6 - x^4$ completely.
    \item[(ii)] Decompose the integrand into partial fractions in simplest form.
    $$
    \frac{1}{x^6 - x^4} = A \frac{1}{x^2} + B \frac{1}{x^4} + \frac{C}{x^2 - 1}
    $$
    \item[(iii)] Hence, evaluate the integral $I$.
\end{enumerate}
\end{problem}

\vspace{0.5cm}
\begin{hintbox}
Factor out the greatest common factor and apply difference of squares. For the decomposition, set $u=x^2$ and perform partial fractions in $u$. Use standard antiderivatives for power functions and the logarithmic form for $1/(x^2-1)$.
\end{hintbox}

\begin{solution}
Factor: $x^6-x^4=x^4(x^2-1)=x^4(x-1)(x+1)$. Let $u=x^2$. Decompose:
$$
\frac{1}{u^2(u-1)}=\frac{A}{u}+\frac{B}{u^2}+\frac{C}{u-1}
$$
Solving gives $A=-1,B=-1,C=1$, so returning to $x$:
$$
\frac{1}{x^6-x^4} = -\frac{1}{x^2} - \frac{1}{x^4} + \frac{1}{x^2-1}
$$
Integrating termwise:
$$
I = \int\left(-x^{-2}-x^{-4}+\frac{1}{x^2-1}\right)dx = \frac{1}{x}+\frac{1}{3x^3}+\frac{1}{2}\ln\left|\frac{x-1}{x+1}\right| + C
$$
\end{solution}

\begin{takeaways}
Partial-fraction substitution can simplify high-degree rational integrands; using $u=x^2$ is a useful trick when even powers appear. Recognise standard antiderivatives for quick evaluation.
\end{takeaways}

\vspace{1cm}

% =============================================================================
% PROBLEM 25: System of equations (symmetric polynomials)
% =============================================================================
\begin{problem}
Solve for $p$, $q$, $r$ over the complex numbers, given:
$$
\begin{aligned}
p + q + r &= 1 \\
pq + pr + qr &= 9 \\
pqr &= 9
\end{aligned}
$$
\end{problem}

\vspace{0.5cm}
\begin{hintbox}
These are elementary symmetric polynomials. Construct cubic $P(x) = x^3 - x^2 + 9x - 9$ with roots $p, q, r$. Factor by grouping.
\end{hintbox}

\begin{solution}
The cubic polynomial with roots $p, q, r$ is:
$$
P(x) = x^3 - (p+q+r)x^2 + (pq+pr+qr)x - pqr = x^3 - x^2 + 9x - 9
$$

Factor by grouping:
$$
x^3 - x^2 + 9x - 9 = x^2(x-1) + 9(x-1) = (x^2+9)(x-1) = 0
$$

From $x-1=0$: $x=1$

From $x^2+9=0$: $x^2 = -9 \implies x = \pm 3i$

\textbf{Answer:} $\{p,q,r\} = \{1, 3i, -3i\}$ (in any order).
\end{solution}

\begin{takeaways}
Vieta's formulas connect roots to coefficients; factorization reveals complex roots.
\end{takeaways}

\vspace{1cm}

% =============================================================================
% PROBLEM 27: Complex roots of quadratic
% =============================================================================
\begin{problem}
The complex roots of $iz^2 + \sqrt{3}z - 1 = 0$ are $\alpha$ and $\beta$.
\begin{enumerate}
    \item Find $\alpha$ and $\beta$ in Cartesian form.
    \item Show that $\alpha^2\beta^2 + 1 = 0$.
\end{enumerate}
\end{problem}

\vspace{0.5cm}
\begin{hintbox}
Use quadratic formula with $a=i$. Find $\sqrt{3-4i}$ by setting $\sqrt{3-4i}=x+iy$ and solving. Part (b): Use Vieta's formula $\alpha\beta = -1/i = i$.
\end{hintbox}

\begin{solution}
\textbf{(a)} Using quadratic formula with $a=i, b=\sqrt{3}, c=-1$:
$$
z = \frac{-\sqrt{3} \pm \sqrt{3-4i}}{2i}
$$

To find $\sqrt{3-4i}$, let $x+iy = \sqrt{3-4i}$. Then $(x+iy)^2 = 3-4i$, giving:
$$
x^2-y^2 = 3, \quad 2xy = -4 \implies y = -2/x
$$

Substituting: $x^2 - 4/x^2 = 3 \implies x^4 - 3x^2 - 4 = 0 \implies (x^2-4)(x^2+1)=0$

Thus $x=2, y=-1$, so $\sqrt{3-4i} = \pm(2-i)$.

Computing: $\alpha = -\frac{1}{2} + \frac{\sqrt{3}-2}{2}i$ and $\beta = \frac{1}{2} + \frac{\sqrt{3}+2}{2}i$

\textbf{(b)} By Vieta's formulas: $\alpha\beta = \frac{-1}{i} = i$

Therefore: $\alpha^2\beta^2 = (\alpha\beta)^2 = i^2 = -1$, so $\alpha^2\beta^2 + 1 = 0$.

% \textbf{Answer:} (a) As shown above; (b) Verified.
\end{solution}

\begin{takeaways}
Finding square roots of complex numbers requires solving simultaneous equations; Vieta's formulas simplify products.
\end{takeaways}

\vspace{1cm}

% =============================================================================
% PROBLEM 37: Integer solution to polynomial equation
% =============================================================================
\begin{problem}
Prove that the only integer solution to
$$ (x-a)(x-b)(x-c)(x-d) - 4 = 0 $$
is $x = \frac{a+b+c+d}{4}$, where $a, b, c, d$ are unique integers.
\end{problem}

\vspace{0.5cm}
\begin{hintbox}
The product equals 4. For distinct integer factors, the only way is $\{1, 2, -1, -2\}$ with product 4. Their sum is 0.
\end{hintbox}

\begin{solution}
Let $y_i = x-i$ for $i \in \{a,b,c,d\}$. Then $y_a y_b y_c y_d = 4$.

Since $a, b, c, d$ are unique integers, the $y_i$ are four distinct integers.

The only way to factor 4 into four distinct integers is $\{-2, -1, 1, 2\}$, since their product is $(-2)(-1)(1)(2) = 4$.

Therefore: $\{x-a, x-b, x-c, x-d\} = \{-2, -1, 1, 2\}$

Summing: $(x-a)+(x-b)+(x-c)+(x-d) = -2-1+1+2 = 0$

Thus: $4x - (a+b+c+d) = 0 \implies x = \frac{a+b+c+d}{4}$

\textbf{Answer:} $x = \frac{a+b+c+d}{4}$ is the unique integer solution.
\end{solution}

\begin{takeaways}
Integer factorization constraints severely limit solutions; summing symmetric expressions reveals structure.
\end{takeaways}

\vspace{1cm}

% =============================================================================
% PROBLEM 38: No rational roots (parity argument)
% =============================================================================
\begin{problem}
Without using the rational roots theorem, prove that there is no rational solution to the equation $x^3 + x + 1 = 0$. Hint: Assume there exists a rational root and consider whether the LHS is odd or even.
\end{problem}

\vspace{0.5cm}
\begin{hintbox}
If $x = p/q$ (coprime), then $p^3 + pq^2 + q^3 = 0$. Check all parity cases: (odd,odd), (odd,even), (even,odd). All lead to odd=even contradiction.
\end{hintbox}

\begin{solution}
Assume $x = \frac{p}{q}$ where $\gcd(p,q)=1$. Substituting into $x^3+x+1=0$ and multiplying by $q^3$:
$$
p^3 + pq^2 + q^3 = 0
$$

\textbf{Case 1:} $p$ odd, $q$ odd $\implies p^3 + pq^2 + q^3 = \text{odd + odd + odd = odd} \neq 0$ (even). Contradiction.

\textbf{Case 2:} $p$ odd, $q$ even $\implies p^3 + pq^2 + q^3 = \text{odd + even + even = odd} \neq 0$. Contradiction.

\textbf{Case 3:} $p$ even, $q$ odd $\implies p^3 + pq^2 + q^3 = \text{even + even + odd = odd} \neq 0$. Contradiction.

All cases lead to contradiction.

\textbf{Answer:} No rational solution exists.
\end{solution}

\begin{takeaways}
Parity arguments provide elegant proofs without explicit factorization or rational root tests.
\end{takeaways}

\vspace{1cm}
