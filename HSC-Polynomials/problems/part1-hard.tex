% Part 1 Hard - HSC Polynomials
% Problems: 03, 07, 08, 14, 15

\begin{problem}[Complex Solutions with Triangle Inequality]
Consider the equation
$$z^n \cos(n\theta) + z^{n-1} \cos((n-1)\theta) + z^{n-2} \cos((n-2)\theta) + \cdots + z \cos(\theta) = 1$$
where $z \in \mathbb{C}$, $\theta \in \mathbb{R}$, and $n$ is a positive integer.

Using a proof by contradiction and the triangle inequality, or otherwise, prove that all the solutions to the equation lie outside the circle $|z| = \frac{1}{2}$ on the complex plane.
\end{problem}

\begin{solution}
\textbf{Strategy:} Use proof by contradiction: assume $|z_0| \leq \frac{1}{2}$ for a solution, apply triangle inequality to bound $|E|$, then show this contradicts $|E| = 1$.

\textbf{Proof by Contradiction:} Assume there exists a solution $z_0$ with $|z_0| \leq \frac{1}{2}$.

Let $E = \sum_{k=1}^n z^k \cos(k\theta)$. Since $z_0$ is a solution, $|E| = 1$.

By the triangle inequality:
$$|E| = \left| \sum_{k=1}^n z^k \cos(k\theta) \right| \leq \sum_{k=1}^n |z|^k |\cos(k\theta)| \leq \sum_{k=1}^n |z|^k$$

Using $|z_0| \leq \frac{1}{2}$ and $|\cos(k\theta)| \leq 1$:
$$|E| \leq \sum_{k=1}^n \left(\frac{1}{2}\right)^k = \frac{\frac{1}{2}(1 - (1/2)^n)}{1 - \frac{1}{2}} = 1 - \left(\frac{1}{2}\right)^n < 1$$

This gives $|E| < 1$, contradicting $|E| = 1$.

Therefore, all solutions satisfy $|z| > \frac{1}{2}$.

\textbf{Final Answer:} Proven by contradiction via triangle inequality.
\end{solution}

\begin{takeaways}
This problem showcases sophisticated proof techniques in complex analysis:
\begin{itemize}
    \item \textbf{Triangle Inequality:} $|z_1 + z_2 + \cdots + z_n| \leq |z_1| + |z_2| + \cdots + |z_n|$ is fundamental for bounding complex sums.
    \item \textbf{Proof by Contradiction:} Assume the negation, derive a logical impossibility, conclude the original statement is true.
    \item \textbf{Modulus Properties:} $|z^k| = |z|^k$ and $|z_1 z_2| = |z_1||z_2|$ simplify modulus calculations.
    \item \textbf{Bounded Trigonometric Functions:} $|\cos\theta| \leq 1$ for all real $\theta$ provides crucial bounds.
    \item \textbf{Geometric Series:} The formula $\sum_{k=1}^n r^k = r\frac{1-r^n}{1-r}$ is essential for summing powers.
    \item \textbf{Strict Inequality:} The key insight is that $\sum_{k=1}^n (1/2)^k < 1$ for all finite $n$, creating the necessary contradiction.
\end{itemize}
\end{takeaways}

\vspace{1cm}

\begin{problem}[Equilateral Triangle in Complex Plane]
Let $w$ be the complex number $w = e^{\frac{2\pi i}{3}}$.

\begin{enumerate}[label=(\roman*)]
    \item Show that $1 + w + w^2 = 0$.

    \medskip
    \noindent Three complex numbers $a$, $b$ and $c$ are represented in the complex plane by points $A$, $B$ and $C$ respectively.

    \item Show that if triangle $ABC$ is anticlockwise and equilateral, then $a + b w + c w^2 = 0$.

    \item It can be shown that if triangle $ABC$ is clockwise and equilateral, then $a + b w^2 + c w = 0$. (Do NOT prove this.)

    Show that if $ABC$ is an equilateral triangle, then
    $$a^2 + b^2 + c^2 = ab + bc + ca.$$
\end{enumerate}
\end{problem}

\begin{solution}
\textbf{Strategy:} Part (i) uses geometric series or factorization of $z^3 - 1$. Part (ii) exploits rotation property: $120^\circ$ rotation in complex plane gives $a + bw + cw^2 = 0$. Part (iii) multiplies anticlockwise and clockwise conditions, then applies $w + w^2 = -1$.

Let $w = e^{\frac{2\pi i}{3}}$.

\begin{enumerate}[label=(\roman*)]
    \item Since $w^3 = e^{2\pi i} = 1$, we have $1 + w + w^2 = \frac{w^3 - 1}{w - 1} = 0$. Alternatively, $w$ satisfies $z^3 - 1 = (z-1)(z^2 + z + 1) = 0$, so $w^2 + w + 1 = 0$.

    \item Rotating $\vec{BC}$ by $120^\circ$ anticlockwise gives $\vec{CA}$: $a - c = (c - b)w$. Rearranging: $a + bw - c(1 + w) = 0$. Since $1 + w = -w^2$ from (i), we get $a + bw + cw^2 = 0$.

    \item For anticlockwise: $a + bw + cw^2 = 0$. For clockwise: $a + bw^2 + cw = 0$. Multiplying:
    $$(a + bw + cw^2)(a + bw^2 + cw) = a^2 + b^2w^3 + c^2w^3 + ab(w + w^2) + ac(w + w^2) + bc(w + w^2)$$
    Using $w^3 = 1$ and $w + w^2 = -1$:
    $$a^2 + b^2 + c^2 - ab - ac - bc = 0$$
    Therefore, $a^2 + b^2 + c^2 = ab + bc + ca$.
\end{enumerate}

\textbf{Final Answer:} (i) $1 + w + w^2 = 0$ by geometric series or roots of unity; (ii) Rotation property gives $a + bw + cw^2 = 0$; (iii) $a^2 + b^2 + c^2 = ab + bc + ca$.
\end{solution}

\begin{takeaways}
This beautiful problem connects complex numbers with geometry:
\begin{itemize}
    \item \textbf{Roots of Unity Properties:} For $w = e^{2\pi i/3}$, we have $w^3 = 1$ and $1 + w + w^2 = 0$.
    \item \textbf{Rotation in Complex Plane:} Multiplying by $e^{i\theta}$ rotates a complex number by angle $\theta$ counterclockwise.
    \item \textbf{Equilateral Triangle Condition:} The relation $a + bw + cw^2 = 0$ (or its variant) characterizes equilateral triangles.
    \item \textbf{Algebraic Identity:} $w + w^2 = -1$ is a key simplification that appears repeatedly.
    \item \textbf{Product of Conditions:} Multiplying the anticlockwise and clockwise conditions eliminates the orientation dependence.
    \item \textbf{Symmetric Functions:} The identity $a^2 + b^2 + c^2 = ab + bc + ca$ is a beautiful symmetric relation for equilateral triangles.
\end{itemize}
\end{takeaways}

\vspace{1cm}

\begin{problem}[Fifth Roots of $-1$ and Trigonometric Values]
Consider the equation $z^5 + 1 = 0$, where $z$ is a complex number.

\begin{enumerate}
    \item Solve the equation $z^5 + 1 = 0$ by finding the $5^\text{th}$ roots of $-1$.
    \item Show that if $z$ is a solution of $z^5 + 1 = 0$ and $z \neq -1$, then $u = z + \frac{1}{z}$ is a solution of $u^2 - u - 1 = 0$.
    \item Hence find the exact value of $\cos \frac{3\pi}{5}$.
\end{enumerate}
\end{problem}

\begin{solution}
\textbf{Strategy:} Part (1) applies de Moivre's theorem for fifth roots. Part (2) divides by $z^2$ and substitutes $u = z + 1/z$ to derive a quadratic. Part (3) uses $u = 2\cos\theta$ and quadrant analysis to select the correct root.

\begin{enumerate}
    \item Solving $z^5 = -1$ with $-1 = e^{i\pi}$, the fifth roots are $z_k = e^{i(2k+1)\pi/5}$ for $k=0,1,2,3,4$, giving $\{e^{i\pi/5}, e^{i3\pi/5}, -1, e^{i7\pi/5}, e^{i9\pi/5}\}$.

    \item For $z \neq -1$, factor $z^5 + 1 = (z+1)(z^4 - z^3 + z^2 - z + 1) = 0$, so $z^4 - z^3 + z^2 - z + 1 = 0$. Dividing by $z^2$:
    $$z^2 - z + 1 - \frac{1}{z} + \frac{1}{z^2} = 0 \implies \left(z^2 + \frac{1}{z^2}\right) - \left(z + \frac{1}{z}\right) + 1 = 0$$
    Let $u = z + \frac{1}{z}$. Then $u^2 = z^2 + 2 + \frac{1}{z^2}$, so $z^2 + \frac{1}{z^2} = u^2 - 2$. Substituting: $(u^2 - 2) - u + 1 = 0$, giving $u^2 - u - 1 = 0$.

    \item For $z_1 = e^{i3\pi/5}$, we have $u = e^{i3\pi/5} + e^{-i3\pi/5} = 2\cos\frac{3\pi}{5}$. From part (ii), $u^2 - u - 1 = 0$ gives $u = \frac{1 \pm \sqrt{5}}{2}$. Since $\frac{3\pi}{5}$ is in the second quadrant, $\cos\frac{3\pi}{5} < 0$. Thus $2\cos\frac{3\pi}{5} = \frac{1 - \sqrt{5}}{2}$, so $\cos\frac{3\pi}{5} = \frac{1 - \sqrt{5}}{4}$.
\end{enumerate}

\textbf{Final Answer:} (1) $z_k = e^{i(2k+1)\pi/5}$ for $k=0,1,2,3,4$; (2) Shown by polynomial division and substitution; (3) $\cos \frac{3\pi}{5} = \frac{1 - \sqrt{5}}{4}$.
\end{solution}

\begin{takeaways}
This problem beautifully connects roots of unity with exact trigonometric values:
\begin{itemize}
    \item \textbf{de Moivre's Theorem:} For finding $n$-th roots, use $z^n = r e^{i\theta} \implies z = r^{1/n} e^{i(\theta + 2k\pi)/n}$.
    \item \textbf{Polynomial Factorization:} $z^5 + 1 = (z+1)(z^4 - z^3 + z^2 - z + 1)$ separates the real root.
    \item \textbf{Clever Substitution:} Setting $u = z + 1/z$ reduces a quartic to a quadratic, a powerful technique.
    \item \textbf{Euler's Formula Bridge:} $z + \bar{z} = 2\cos\theta$ connects complex and trigonometric forms.
    \item \textbf{Quadrant Analysis:} Determining the sign of $\cos\theta$ from the quadrant is essential for selecting the correct root.
    \item \textbf{Golden Ratio Connection:} $(1+\sqrt{5})/2$ is the golden ratio $\phi$, appearing naturally in pentagon geometry.
\end{itemize}
\end{takeaways}

\vspace{1cm}

\begin{problem}[De Moivre's Theorem and Secant Value]
\begin{enumerate}
    \item Solve $z^5 + 1 = 0$ by de Moivre's theorem, leaving your solutions in modulus-argument form.
    \item Prove that the solutions of $z^4 - z^3 + z^2 - z + 1 = 0$ are the non-real solutions of $z^5 + 1 = 0$.
    \item Show that if $z^4 - z^3 + z^2 - z + 1 = 0$ where $z = \text{cis } \theta$ then $4 \cos^2 \theta - 2 \cos \theta - 1 = 0$.
    \item Hence find the exact value of $\sec \frac{3\pi}{5}$.
\end{enumerate}
\end{problem}

\begin{solution}
\textbf{Strategy:} Part (i) applies de Moivre's theorem. Part (ii) uses factorization: $z^5 + 1 = (z+1)(z^4 - z^3 + z^2 - z + 1)$. Part (iii) divides by $z^2$ and substitutes $z + 1/z = 2\cos\theta$. Part (iv) solves the quadratic and rationalizes.

\begin{enumerate}
    \item Solving $z^5 = -1$ with $-1 = \text{cis}(\pi)$ gives $z_k = \text{cis}((2k+1)\pi/5)$ for $k=0,1,2,3,4$, yielding $\{\text{cis}(\pi/5), \text{cis}(3\pi/5), -1, \text{cis}(7\pi/5), \text{cis}(9\pi/5)\}$.

    \item Factor $z^5 + 1 = (z+1)(z^4 - z^3 + z^2 - z + 1)$. The root $z=-1$ is real. The quartic factor gives the four non-real solutions $z_0, z_1, z_3, z_4$ with modulus 1 and non-zero arguments.

    \item For $z^4 - z^3 + z^2 - z + 1 = 0$, divide by $z^2$: $z^2 - z + 1 - \frac{1}{z} + \frac{1}{z^2} = 0$. Group: $(z^2 + \frac{1}{z^2}) - (z + \frac{1}{z}) + 1 = 0$. For $z = \text{cis }\theta$, use $z + \frac{1}{z} = 2\cos\theta$ and $z^2 + \frac{1}{z^2} = (z + \frac{1}{z})^2 - 2 = 4\cos^2\theta - 2$. Substituting: $(4\cos^2\theta - 2) - 2\cos\theta + 1 = 0$, giving $4\cos^2\theta - 2\cos\theta - 1 = 0$.

    \item The quadratic gives $\cos\theta = \frac{2 \pm 2\sqrt{5}}{8} = \frac{1 \pm \sqrt{5}}{4}$. Since $\frac{3\pi}{5}$ is in the second quadrant, $\cos\frac{3\pi}{5} = \frac{1 - \sqrt{5}}{4}$. Thus $\sec\frac{3\pi}{5} = \frac{4}{1 - \sqrt{5}} = \frac{4(1 + \sqrt{5})}{-4} = -(1 + \sqrt{5})$.
\end{enumerate}

\textbf{Final Answer:} (i) $z_k = \text{cis}((2k+1)\pi/5)$ for $k=0,1,2,3,4$; (ii) Shown by factorization; (iii) $4\cos^2\theta - 2\cos\theta - 1 = 0$; (iv) $\sec \frac{3\pi}{5} = -(1 + \sqrt{5})$.
\end{solution}

\begin{takeaways}
This comprehensive problem ties together multiple advanced concepts:
\begin{itemize}
    \item \textbf{Cis Notation:} $\text{cis } \theta = \cos\theta + i\sin\theta = e^{i\theta}$ is a compact notation for complex exponentials.
    \item \textbf{Polynomial Factorization:} Separating real from non-real roots via $(z+1)$ factor.
    \item \textbf{Trigonometric Substitution:} For $|z| = 1$, the substitution $z + 1/z = 2\cos\theta$ is fundamental.
    \item \textbf{Double Angle Formula:} $\cos(2\theta) = 2\cos^2\theta - 1$ can derive $z^2 + 1/z^2$.
    \item \textbf{Rationalizing Denominators:} Multiply by conjugate: $\frac{1}{1-\sqrt{5}} \cdot \frac{1+\sqrt{5}}{1+\sqrt{5}} = \frac{1+\sqrt{5}}{-4}$.
    \item \textbf{Sign Determination:} Quadrant analysis is crucial for selecting correct values from quadratic formula.
\end{itemize}
\end{takeaways}

\vspace{1cm}

\begin{problem}[Tangent Function and Product Identity]
\textbf{(i}) Use De Moivre's theorem to express $\tan 5\theta$ in terms of powers of $\tan \theta$.

\textbf{(ii)} Hence show that $x^4 - 10x^2 + 5 = 0$ has roots $\pm \tan \frac{\pi}{5}$ and $\pm \tan \frac{2\pi}{5}$.

\textbf{(iii)} Deduce that $\tan \frac{\pi}{5} \cdot \tan \frac{2\pi}{5} \cdot \tan \frac{3\pi}{5} \cdot \tan \frac{4\pi}{5} = 5$.
\end{problem}

\begin{solution}
\textbf{Strategy:} Part (i) expands $(\cos\theta + i\sin\theta)^5$ via binomial theorem and divides by $\cos^5\theta$. Part (ii) sets $\tan 5\theta = 0$ to find angles $\theta = k\pi/5$, yielding the quartic. Part (iii) applies Vieta's formula for product of roots.

\begin{enumerate}[label=(\roman*)]
    \item By de Moivre's theorem, $(\cos\theta + i\sin\theta)^5 = \cos 5\theta + i\sin 5\theta$. Expanding and separating imaginary/real parts:
    $$\sin 5\theta = 5\cos^4\theta\sin\theta - 10\cos^2\theta\sin^3\theta + \sin^5\theta$$
    $$\cos 5\theta = \cos^5\theta - 10\cos^3\theta\sin^2\theta + 5\cos\theta\sin^4\theta$$
    Dividing numerator and denominator by $\cos^5\theta$:
    $$\tan 5\theta = \frac{5t - 10t^3 + t^5}{1 - 10t^2 + 5t^4}, \quad \text{where } t = \tan\theta$$

    \item Setting $\tan 5\theta = 0$ gives $5\theta = k\pi$, so $\theta = \frac{k\pi}{5}$. For $0 < \theta < \pi$: $\theta = \frac{\pi}{5}, \frac{2\pi}{5}, \frac{3\pi}{5}, \frac{4\pi}{5}$. The numerator vanishes: $t^5 - 10t^3 + 5t = 0$. Since $t \neq 0$, divide by $t$: $t^4 - 10t^2 + 5 = 0$. Using $\tan(\pi - A) = -\tan A$, the roots are $\pm\tan\frac{\pi}{5}$ and $\pm\tan\frac{2\pi}{5}$.

    \item By Vieta's formula, for $x^4 - 10x^2 + 5 = 0$, the product of roots is $5/1 = 5$. Therefore:
    $$\tan\frac{\pi}{5} \cdot \tan\frac{2\pi}{5} \cdot \tan\frac{3\pi}{5} \cdot \tan\frac{4\pi}{5} = 5$$
\end{enumerate}

\textbf{Final Answer:} (i) $\tan 5\theta = \frac{5t - 10t^3 + t^5}{1 - 10t^2 + 5t^4}$ where $t = \tan\theta$; (ii) Roots are $\pm\tan\frac{\pi}{5}, \pm\tan\frac{2\pi}{5}$; (iii) Product equals $5$.
\end{solution}

\begin{takeaways}
This elegant problem demonstrates the power of combining complex analysis with algebra:
\begin{itemize}
    \item \textbf{Binomial Expansion:} $(\cos\theta + i\sin\theta)^n = \sum_{k=0}^n \binom{n}{k} \cos^{n-k}\theta (i\sin\theta)^k$ generates multiple-angle formulas.
    \item \textbf{Separation Technique:} Dividing numerator and denominator by $\cos^n\theta$ converts to tangent form.
    \item \textbf{Zero Finding:} Setting $\tan n\theta = 0$ identifies specific angles whose tangents satisfy polynomial equations.
    \item \textbf{Supplementary Angle:} $\tan(\pi - \theta) = -\tan\theta$ explains why roots come in $\pm$ pairs.
    \item \textbf{Vieta's Product Formula:} For $a_nx^n + \cdots + a_0 = 0$, product of roots equals $(-1)^n a_0/a_n$.
    \item \textbf{Pentagon Connection:} These tangent values relate to regular pentagon geometry, where the number 5 appears naturally.
\end{itemize}
\end{takeaways}

\vspace{1cm}
