% Part 1 Easy - HSC Polynomials
% Problems: 01, 05, 10, 11, 21

\begin{problem}[Square Roots of Complex Numbers]
\begin{enumerate}[(i)]
    \item Find the two square roots of $-i$, giving the answers in the form $x + iy$, where $x$ and $y$ are real numbers.
    \item Hence, or otherwise, solve $z^2 + 2z + 1 + i = 0$ giving your solutions in the form $a + ib$ where $a$ and $b$ are real numbers.
\end{enumerate}
\end{problem}

\begin{solution}
\textbf{Strategy:} Set $z = x + iy$ and equate real and imaginary parts to find the square roots. Use completing the square in part (ii) to apply part (i)'s results.

\subsubsection*{(i) Finding the square roots of $-i$}

Let $z = x + iy$ where $x, y \in \mathbb{R}$. Then $z^2 = -i$ gives:
$$(x + iy)^2 = x^2 - y^2 + 2ixy = -i$$

Equating real and imaginary parts:
\begin{align}
    x^2 - y^2 &= 0 \label{eq:1a} \\
    2xy &= -1 \label{eq:2a}
\end{align}

From \eqref{eq:1a}: $y = \pm x$

\textbf{Case 1:} If $y = x$, then $2x^2 = -1 \implies x^2 = -\frac{1}{2}$ (no real solutions).

\textbf{Case 2:} If $y = -x$, then $-2x^2 = -1 \implies x^2 = \frac{1}{2} \implies x = \pm \frac{\sqrt{2}}{2}$

Therefore: $z_1 = \frac{\sqrt{2}}{2} - i \frac{\sqrt{2}}{2}$ and $z_2 = -\frac{\sqrt{2}}{2} + i \frac{\sqrt{2}}{2}$

\subsubsection*{(ii) Solving $z^2 + 2z + 1 + i = 0$}

Completing the square: $(z + 1)^2 = -i$

Let $w = z + 1$. From part (i), $w = \pm\left(\frac{\sqrt{2}}{2} - i \frac{\sqrt{2}}{2}\right)$

Since $z = w - 1$:
$$z_1 = \left(\frac{\sqrt{2}}{2} - 1\right) - i \frac{\sqrt{2}}{2}, \quad z_2 = \left(-\frac{\sqrt{2}}{2} - 1\right) + i \frac{\sqrt{2}}{2}$$
\end{solution}

\begin{takeaways}
This problem demonstrates several fundamental techniques in complex number algebra:
\begin{itemize}
    \item \textbf{Equating Real and Imaginary Parts:} When $(x + iy)^2 = a + ib$, we can separate into two real equations by equating coefficients, giving us a solvable system.
    \item \textbf{Completing the Square:} Recognizing $(z + 1)^2$ in the equation transforms a seemingly difficult problem into one we've already solved.
    \item \textbf{Alternative Method - Polar Form:} We could have found square roots using $-i = e^{i(-\pi/2 + 2k\pi)}$, then $\sqrt{-i} = e^{i(-\pi/4 + k\pi)}$ for $k=0,1$.
    \item \textbf{Conjugate Pairs:} Notice the two square roots are negatives of each other, which is always true for square roots of any complex number.
\end{itemize}
\end{takeaways}

\vspace{1cm}

\begin{problem}[Quadratic Equations with Complex Roots]
Solve the quadratic equation
$$z^2 - 3z + 4 = 0,$$
where $z$ is a complex number. Give your answers in \textbf{Cartesian form} ($x + iy$).
\end{problem}

\begin{solution}
\textbf{Strategy:} This is a straightforward application of the quadratic formula to a complex-valued equation. The discriminant is negative, indicating complex (non-real) conjugate roots. The key steps are: identify coefficients, calculate the discriminant, handle the negative square root using $i$, and simplify to Cartesian form.

The given quadratic equation is
$$z^2 - 3z + 4 = 0.$$
This is in the standard form $az^2 + bz + c = 0$, where $a=1$, $b=-3$, and $c=4$.

We use the \textbf{quadratic formula} to find the solutions for $z$:
$$z = \frac{-b \pm \sqrt{b^2 - 4ac}}{2a}$$

Substituting the values of $a$, $b$, and $c$:
\begin{align*} z &= \frac{-(-3) \pm \sqrt{(-3)^2 - 4(1)(4)}}{2(1)} \\ &= \frac{3 \pm \sqrt{9 - 16}}{2} \\ &= \frac{3 \pm \sqrt{-7}}{2}\end{align*}

Since $z$ is a complex number, we express $\sqrt{-7}$ using the imaginary unit $i$, where $i^2 = -1$:
$$\sqrt{-7} = \sqrt{7 \times (-1)} = \sqrt{7} \sqrt{-1} = i\sqrt{7}$$

Substituting this back into the expression for $z$:
$$z = \frac{3 \pm i\sqrt{7}}{2}$$

We can write the two solutions in the required Cartesian form, $x + iy$:
$$z_1 = \frac{3}{2} + i\frac{\sqrt{7}}{2}$$
$$z_2 = \frac{3}{2} - i\frac{\sqrt{7}}{2}$$

\textbf{Final Answer:} The two solutions are $z = \frac{3}{2} + i\frac{\sqrt{7}}{2}$ and $z = \frac{3}{2} - i\frac{\sqrt{7}}{2}$.
\end{solution}

\begin{takeaways}
This problem reinforces essential concepts in quadratic equations with complex numbers:
\begin{itemize}
    \item \textbf{Discriminant Analysis:} When $\Delta = b^2 - 4ac < 0$, the equation has two complex conjugate roots. Here, $\Delta = -7$.
    \item \textbf{Complex Conjugate Pairs:} For quadratic equations with real coefficients, complex roots always come in conjugate pairs: $a + bi$ and $a - bi$.
    \item \textbf{Imaginary Unit:} $\sqrt{-n} = i\sqrt{n}$ for positive real $n$ is a fundamental identity for working with complex numbers.
    \item \textbf{Geometric Interpretation:} These solutions are symmetric about the real axis in the complex plane, both at distance $\sqrt{(\frac{3}{2})^2 + (\frac{\sqrt{7}}{2})^2} = \sqrt{\frac{9+7}{4}} = 2$ from the origin.
\end{itemize}
\end{takeaways}

\vspace{1cm}

\begin{problem}[Polynomial with Given Factor]
Given that $(z + 2 - i)$ is a factor of $P(z) = z^4 + 4z^3 + 3z^2 - 8z - 10$, factorise $P(z)$ over the set of complex numbers.
\end{problem}

\begin{solution}
\textbf{Strategy:} Use the Conjugate Root Theorem to find the conjugate factor, multiply to get a real quadratic, divide into $P(z)$, then factorize the quotient.

Since $P(z)$ has real coefficients and $(z + 2 - i)$ is a factor, the conjugate root theorem implies $(z + 2 + i)$ is also a factor.

\textbf{Step 1: Find the quadratic from conjugate factors}

The product gives a quadratic with real coefficients:
$$Q(z) = (z + 2 - i)(z + 2 + i) = ((z + 2) - i)((z + 2) + i) = (z + 2)^2 + 1 = z^2 + 4z + 5$$

\textbf{Step 2: Perform polynomial division}

Write $P(z) = (z^2 + 4z + 5)(az^2 + bz + c)$. Comparing coefficients:
\begin{itemize}
    \item Leading term: $a = 1$
    \item Constant term: $5c = -10 \implies c = -2$
    \item Coefficient of $z^3$: $b + 4 = 4 \implies b = 0$
\end{itemize}
Thus $R(z) = z^2 - 2 = (z - \sqrt{2})(z + \sqrt{2})$.

\textbf{Final Answer:} $P(z) = (z + 2 - i)(z + 2 + i)(z - \sqrt{2})(z + \sqrt{2})$
\end{solution}

\begin{takeaways}
This problem illustrates several key polynomial factorization techniques:
\begin{itemize}
    \item \textbf{Conjugate Root Theorem:} For polynomials with real coefficients, complex roots always occur in conjugate pairs. If $\alpha$ is a root, so is $\bar{\alpha}$.
    \item \textbf{Product of Conjugate Factors:} $(z - (a + bi))(z - (a - bi)) = (z - a)^2 + b^2$ always gives a quadratic with real coefficients.
    \item \textbf{Polynomial Division Strategy:} Compare leading and constant coefficients first for quick results, then work through middle terms.
    \item \textbf{Complete Factorization:} Over $\mathbb{C}$, every polynomial factors completely into linear factors. Over $\mathbb{R}$, we can have irreducible quadratics.
    \item \textbf{Verification:} We can verify by expanding: $(z^2 + 4z + 5)(z^2 - 2)$ should give the original polynomial.
\end{itemize}
\end{takeaways}

\vspace{1cm}

\begin{problem}[Finding Polynomial Coefficients from Roots]
A cubic polynomial has the form
$$p(z) = z^3 + bz^2 + cz + d, \quad z \in \mathbb{C}, \quad \text{where } b, c, d \in \mathbb{R}.$$
Given that a solution of $p(z) = 0$ is $z_1 = 3 - 2i$ and that $p(-2) = 0$, find the values of $b, c$ and $d$.
\end{problem}

\begin{solution}
\textbf{Strategy:} Use the Conjugate Root Theorem to identify all three roots, then apply Vieta's formulas to relate roots to coefficients directly.

Since the coefficients are real, if $z_1 = 3 - 2i$ is a root, then $z_2 = 3 + 2i$ (conjugate) is also a root. Given $p(-2) = 0$, the three roots are: $z_1 = 3 - 2i$, $z_2 = 3 + 2i$, $z_3 = -2$.

\textbf{Applying Vieta's formulas:}

\textbf{Finding $b$:} $-b = z_1 + z_2 + z_3 = (3-2i) + (3+2i) + (-2) = 4 \implies b = -4$

\textbf{Finding $c$:} $c = z_1z_2 + z_1z_3 + z_2z_3$

Note $z_1z_2 = (3-2i)(3+2i) = 9 + 4 = 13$, so:
$$c = 13 + (3-2i)(-2) + (3+2i)(-2) = 13 - 12 = 1$$

\textbf{Finding $d$:} $-d = z_1z_2z_3 = 13 \cdot (-2) = -26 \implies d = 26$

\textbf{Final Answer:} $b = -4$, $c = 1$, $d = 26$. The polynomial is $p(z) = z^3 - 4z^2 + z + 26$.
\end{solution}

\begin{takeaways}
This problem showcases efficient polynomial reconstruction techniques:
\begin{itemize}
    \item \textbf{Vieta's Formulas:} These provide direct relationships between roots and coefficients, eliminating the need for expansion or division.
    \item \textbf{Product of Conjugates:} $(a + bi)(a - bi) = a^2 + b^2$ is a key simplification. Here, $(3-2i)(3+2i) = 9 + 4 = 13$.
    \item \textbf{Imaginary Parts Cancel:} When adding conjugate pairs, imaginary parts always cancel: $(3-2i) + (3+2i) = 6$.
    \item \textbf{Efficient Calculation:} Notice how $z_1z_2 + z_1z_3 + z_2z_3 = 13 + (z_1 + z_2)(z_3) = 13 + 6(-2) = 1$.
    \item \textbf{Verification Method:} We can verify by substituting back: $p(3-2i)$ should equal zero.
\end{itemize}
\end{takeaways}

\vspace{1cm}

\begin{problem}[Polynomial with Real Parameter]
Given that $w$ is a root of the cubic equation $z^3 + iz^2 + ikz + 2i = 0$, where $k$ is real, and $(1 - i)w$ is real, find the possible value of $k$.
\end{problem}

\begin{solution}
\textbf{Strategy:} Use the constraint that $(1-i)w$ is real to find the form of $w$, substitute into the cubic equation, then separate real and imaginary parts to solve for $k$.

\textbf{Step 1: Determine the form of $w$}

Let $w = x + iy$ where $x, y \in \mathbb{R}$. Since $(1-i)w = (1-i)(x+iy) = x+y + i(y-x)$ is real:
$$y - x = 0 \implies y = x \implies w = x(1+i), \quad x \neq 0$$

(Note: $x \neq 0$ since $w = 0$ gives $2i = 0$, a contradiction)

\textbf{Step 2: Calculate powers and substitute}

For $w = x(1+i)$:
$$w^2 = x^2(1+i)^2 = x^2(2i) = 2ix^2$$
$$w^3 = w \cdot w^2 = x(1+i) \cdot 2ix^2 = 2ix^3(1+i) = -2x^3 + 2ix^3$$

Substituting into $z^3 + iz^2 + ikz + 2i = 0$:
$$(-2x^3 + 2ix^3) + i(2ix^2) + ikx(1+i) + 2i = 0$$
$$(-2x^3 - 2x^2 - kx) + i(2x^3 + kx + 2) = 0$$

\textbf{Step 3: Solve the system}

Equating real and imaginary parts to zero:
\begin{align}
-2x^3 - 2x^2 - kx &= 0 \label{eq:1c} \\
2x^3 + kx + 2 &= 0 \label{eq:2c}
\end{align}

From \eqref{eq:1c}, divide by $-x$: $k = -2x^2 - 2x$

Substitute into \eqref{eq:2c}: $2x^3 + (-2x^2-2x)x + 2 = 0 \implies -2x^2 + 2 = 0 \implies x = \pm 1$

\textbf{Step 4: Find values of $k$}

For $x = 1$: $k = -2(1)^2 - 2(1) = -4$ (root: $w = 1+i$)

For $x = -1$: $k = -2(-1)^2 - 2(-1) = 0$ (root: $w = -1-i$)

\textbf{Final Answer:} $k = -4$ or $k = 0$
\end{solution}

\begin{takeaways}
This problem combines constraint analysis with polynomial root theory:
\begin{itemize}
    \item \textbf{Complex Constraint Analysis:} The condition "$(1-i)w$ is real" translates to requiring the imaginary part to vanish, giving us $y = x$.
    \item \textbf{Parametric Form:} Expressing $w = x(1+i)$ reduces the problem from two unknowns $(x,y)$ to one unknown $(x)$.
    \item \textbf{Simultaneous Equations:} Separating complex equations into real and imaginary parts always yields a system of real equations.
    \item \textbf{Strategic Elimination:} Dividing the first equation by $x$ (since $x \neq 0$) allows us to express $k$ in terms of $x$, which we then substitute into the second equation.
    \item \textbf{Multiple Solutions:} The problem allows two values of $k$ because different values of $x$ satisfy the constraints.
\end{itemize}
\end{takeaways}

\vspace{1cm}
