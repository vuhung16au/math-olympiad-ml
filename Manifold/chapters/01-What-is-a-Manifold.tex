\chapter{What is a Manifold?}
A \textbf{manifold} is a space that \textbf{locally looks like flat Euclidean space}, even if the overall shape is curved or complex. The idea is that if you zoom in very closely on any small neighborhood in a manifold, it looks like regular n-dimensional space.

Imagine standing on Earth. To you, the ground seems flat locally (like a plane), but we know Earth is actually a curved sphere globally. The small patch you see is like a flat 2D plane, even though the whole planet curves.

\section{The Key Concept: Local Mapping to Flat Space}

The fundamental property of a manifold is that every point has a neighborhood that can be mapped to a flat Euclidean space. This mapping, called a \textbf{chart} or \textbf{coordinate chart}, allows us to work with the manifold locally as if it were flat.

\begin{figure}[htbp]
\centering
\begin{tikzpicture}[scale=1.0]
  % Draw coordinate axes for the manifold (curved space)
  \draw[->, bookpurple, thick] (0, 0) -- ++(0, 1.8);
  \draw[->, bookpurple, thick] (0, 0) -- ++(2.3, 0.5);
  \draw[->, bookpurple, thick] (0, 0) -- ++(2.8, 0) node[midway, below, yshift=-0.5em]
      {Manifold $\mathcal{M}$};

  % Draw the curved manifold surface (like a sphere or curved surface)
  \draw[fill=bookpurple!50, draw=none, shift={(0.2, 0.7)}, scale=0.5]
    (0, 0) to[out=20, in=140] (1.5, -0.2) to [out=60, in=160]
    (5, 0.5) to[out=130, in=60] cycle;

  \shade[left color=bookpurple!10, right color=bookpurple!50, draw=none,
    shift={(0.2, 0.7)}, scale=0.5]
    (0, 0) to[out=10, in=140] (3.3, -0.8) to [out=60, in=190] (5, 0.5)
      to[out=130, in=60] cycle;

  % Highlight a local patch on the manifold
  \draw[fill=bookred!50, opacity=0.8, draw=bookred, thick, shift={(1.5, 1.3)}, scale=0.3]
    (0, 0) to[out=20, in=140] (1.5, -0.2) to [out=60, in=160]
    (5, 0.5) to[out=130, in=60] cycle;
  \node[above, bookred] at (1.5, 1.5) {\small Local patch};

  % Draw coordinate axes for the flat/Euclidean space
  \draw[->, bookpurple, thick] (4.8, 0.8) -- ++(0, 1.8);
  \draw[->, bookpurple, thick] (4.8, 0.8) -- ++(1.8, 0) node[midway, below, yshift=-0.5em]
      {Euclidean space $\mathbb{R}^n$};

  % Draw the flat rectangular region (chart)
  \draw[thin, fill=bookpurple!30, draw=bookpurple, thick, shift={(5.4, 1.1)}, rotate=20]
    (0, 0) -- (1, 0) -- (1, 1) -- (0, 1) -- cycle;
  \node[above, bookpurple] at (6.2, 2.3) {\small Chart (flat region)};

  % Draw the mapping function f from manifold to flat space
  \draw[thick, ->, bookred, line width=2pt]
    (1.5, 1.3) to [out=55, in=150] node[midway, above, xshift=6pt, yshift=2pt, bookred]
    {$f$ (chart)} (5.7, 2);

  % Draw the inverse mapping g from flat space back to manifold
  \draw[thick, ->, bookpurple!70, line width=2pt] (1.5, 1.3) ++(4.03, 0.3) to [out=150, in=55]
    node[midway, below, xshift=2pt, yshift=-2pt, bookpurple!80] {$f^{-1}$ (inverse)} ++(-3.6, -0.5);

  % Add annotation
  \node[below, align=center] at (3.5, -0.5) {\small Each point on the manifold has a neighborhood\\ that maps to a flat Euclidean space};
\end{tikzpicture}
\caption{The fundamental concept of a manifold: every point has a local neighborhood (patch) that can be mapped to flat Euclidean space via a chart $f$. The inverse map $f^{-1}$ allows us to work with coordinates on the flat space and transfer them back to the manifold. This local flatness property holds even though the manifold may be globally curved.}
\label{fig:manifold-mapping}
\end{figure}

This mapping property is what makes manifolds so useful: we can use familiar flat space mathematics (like calculus, geometry, and coordinates) locally on the manifold, even though the manifold itself is curved globally.

\section{Examples of Manifolds}

To better understand manifolds, let's visualize some key examples:

\begin{figure}[htbp]
\centering
\begin{minipage}{0.48\textwidth}
\centering
\begin{tikzpicture}[scale=1.2]
  % 1D plane/manifold - a curved line
  \draw[thick, bookpurple, line width=2pt] (0,0.5) .. controls (1,1) and (2,0.3) .. (3,0.5);
  % Add a small patch showing local flatness (a small line segment)
  \draw[fill=bookred!40, opacity=0.7, line width=4pt] (1.5,0.45) -- (1.8,0.55);
  \node[above] at (1.65, 0.6) {\small 1D patch};
  \node[above] at (1.5, 1.5) {\textbf{(a) 1D Manifold}};
  \node[below] at (1.5, -0.3) {\small (Curved line)};
\end{tikzpicture}
\end{minipage}
\hfill
\begin{minipage}{0.48\textwidth}
\centering
\begin{tikzpicture}[scale=1.2]
  % 2D plane - flat surface
  \draw[fill=bookpurple!20, draw=bookpurple, thick] (0,0) rectangle (3,2);
  \draw[fill=bookred!30, opacity=0.6] (1,0.7) rectangle (1.5,1.2);
  \node[above] at (1.25, 1.25) {\small 2D patch};
  \node[above] at (1.5, 2.5) {\textbf{(b) 2D Plane}};
  \node[below] at (1.5, -0.3) {\small (Flat surface)};
  % Add grid lines to show it's flat
  \draw[bookpurple!40, thin] (0,0.67) -- (3,0.67);
  \draw[bookpurple!40, thin] (0,1.33) -- (3,1.33);
  \draw[bookpurple!40, thin] (1,0) -- (1,2);
  \draw[bookpurple!40, thin] (2,0) -- (2,2);
\end{tikzpicture}
\end{minipage}

\vspace{0.5cm}

\begin{minipage}{0.48\textwidth}
\centering
\begin{tikzpicture}[scale=1.2]
  % 3D sphere - 2D sphere surface (the surface of a 3D sphere)
  % Draw sphere with shading
  \shade[ball color=bookpurple!40] (1.5,1) circle (0.9cm);
  \draw[bookpurple, thick] (1.5,1) circle (0.9cm);
  % Draw latitude lines for 3D effect
  \draw[bookpurple!60, dashed, very thin] (0.6,1) arc (180:0:0.9);
  \draw[bookpurple!60, dashed, very thin] (0.6,0.7) arc (180:0:0.7);
  \draw[bookpurple!60, dashed, very thin] (0.6,1.3) arc (180:0:0.7);
  % Draw a local patch on the sphere
  \draw[fill=bookred!50, opacity=0.7] (1.9,1.2) arc (0:45:0.3) arc (45:0:0.25);
  \node[right] at (2.1, 1.3) {\tiny patch};
  \node[above] at (1.5, 2.3) {\textbf{(c) 2D Sphere}};
  \node[below] at (1.5, -0.3) {\small (Surface of 3D sphere)};
\end{tikzpicture}
\end{minipage}
\hfill
\begin{minipage}{0.48\textwidth}
\centering
\begin{tikzpicture}[scale=1.2]
  % 2D torus
  % Draw the torus as a donut shape
  \begin{scope}
    \clip (0,0) rectangle (3,2.5);
    % Outer ellipse
    \draw[bookpurple, thick] (1.5,1.25) ellipse (1.2cm and 0.6cm);
    % Inner ellipse (hole)
    \draw[bookpurple, thick] (1.5,1.25) ellipse (0.5cm and 0.25cm);
    % Draw the torus surface with shading
    \shade[left color=bookpurple!40, right color=bookpurple!20] 
      (1.5,1.25) ellipse (1.2cm and 0.6cm);
    \shade[left color=bookwhite, right color=bookwhite] 
      (1.5,1.25) ellipse (0.5cm and 0.25cm);
    % Add a local patch
    \draw[fill=bookred!40, opacity=0.6] (2.2,1.1) -- (2.4,1.15) -- (2.35,1.3) -- (2.15,1.25) -- cycle;
    \node[right] at (2.4, 1.225) {\tiny patch};
  \end{scope}
  \node[above] at (1.5, 2.5) {\textbf{(d) 2D Torus}};
  \node[below] at (1.5, -0.3) {\small (Donut surface)};
\end{tikzpicture}
\end{minipage}
\caption{Examples of manifolds: (a) a 1D manifold (curved line), (b) a 2D plane, (c) a 2D sphere surface, and (d) a 2D torus. Each manifold locally looks like flat Euclidean space of the appropriate dimension, as indicated by the highlighted patches.}
\label{fig:manifold-examples}
\end{figure}

These visualizations illustrate the key concept: each manifold, regardless of its global shape, has the property that any small neighborhood (patch) looks like flat Euclidean space of the corresponding dimension. The 1D manifold looks like a line locally, the 2D manifolds look like a plane locally, even though globally they may be curved or have complex topology.
