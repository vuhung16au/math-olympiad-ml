\chapter{Open n-Ball}

In Chapter 1, we learned that manifolds are locally homeomorphic to Euclidean space. But what exactly does "Euclidean space" mean locally? The answer is: \textbf{open n-balls}. These are the fundamental building blocks that define what it means for a space to "look like" Euclidean space locally. This chapter explores open n-balls in detail, showing why they're essential to the manifold definition.

\section{Introduction: Why Open n-Balls Matter}

Recall from Chapter 1 that a manifold is a space where every point has a neighborhood that looks like Euclidean space. More precisely, every point has a neighborhood that is \textbf{homeomorphic} to an open ball in $\mathbb{R}^n$.

But what is an open ball? And why is it the "standard" representation of Euclidean space locally? This chapter answers these questions by:

\begin{itemize}
\item Defining open n-balls formally
\item Showing examples in different dimensions
\item Explaining why "open" matters (no boundary)
\item Connecting to the manifold definition
\item Exploring properties and applications
\end{itemize}

Understanding open n-balls is crucial because they're what we map to when we say a manifold "locally looks like Euclidean space."

\section{Definition of an Open n-Ball}

\subsection{Formal Definition}

An \textbf{open n-ball} (also called an \textbf{open ball} in $\mathbb{R}^n$) is the set of all points within a certain distance $r$ (the radius) from a center point $\mathbf{p}$ in $n$-dimensional Euclidean space.

\textbf{Mathematical definition}:

\[
B_r(\mathbf{p}) = \{\mathbf{x} \in \mathbb{R}^n : \|\mathbf{x} - \mathbf{p}\| < r\}
\]

where $\|\mathbf{x} - \mathbf{p}\|$ is the Euclidean distance between $\mathbf{x}$ and $\mathbf{p}$.

\textbf{In coordinates}: If $\mathbf{p} = (p_1, p_2, \ldots, p_n)$ and $\mathbf{x} = (x_1, x_2, \ldots, x_n)$, then:

\[
B_r(\mathbf{p}) = \left\{(x_1, \ldots, x_n) \in \mathbb{R}^n : \sum_{i=1}^n (x_i - p_i)^2 < r^2\right\}
\]

\textbf{Components}:
\begin{itemize}
\item \textbf{Center}: The point $\mathbf{p}$ around which the ball is centered
\item \textbf{Radius}: The positive real number $r > 0$ that determines the size
\item \textbf{Strict inequality}: The use of $<$ (not $\leq$) means the boundary is \textit{not} included
\end{itemize}

\subsection{Why "Open"?}

The term "open" is crucial and has a specific mathematical meaning:

\begin{itemize}
\item \textbf{Open ball}: $\|\mathbf{x} - \mathbf{p}\| < r$ (boundary \textit{not} included)
\item \textbf{Closed ball}: $\|\mathbf{x} - \mathbf{p}\| \leq r$ (boundary \textit{included})
\end{itemize}

\textbf{Why it matters for manifolds}: When we say a manifold is locally homeomorphic to Euclidean space, we need neighborhoods without boundaries. Including boundaries would create problems:
\begin{itemize}
\item Boundaries are "special" points that don't have full neighborhoods
\item Homeomorphisms would need to map boundaries, which is restrictive
\item The local structure should be "smooth" without edge effects
\end{itemize}

For example, if you're standing on Earth, your local neighborhood (the patch of ground you see) doesn't include a boundary—you can move in any direction. This is what "open" captures.

\section{Visual Examples in Different Dimensions}

\subsection{1D Open Ball (Open Interval)}

In one dimension, an open ball is simply an open interval on the real line.

\textbf{Definition}: $B_r(p) = \{x \in \mathbb{R} : |x - p| < r\} = (p-r, p+r)$

\begin{figure}[htbp]
\centering
\begin{minipage}{0.48\textwidth}
\centering
\begin{tikzpicture}[scale=1.2]
  % Draw the real line
  \draw[bookpurple, thick, <->] (-2.5,0) -- (2.5,0);
  % Mark center
  \filldraw[bookpurple] (0,0) circle (2pt) node[below] {$p$};
  % Mark radius
  \draw[bookpurple!50, dashed] (0,0) -- (-1.5,0) node[midway, below] {$r$};
  \draw[bookpurple!50, dashed] (0,0) -- (1.5,0) node[midway, below] {$r$};
  % Draw open interval (ball)
  \draw[bookred, very thick] (-1.5,0) -- (1.5,0);
  % Mark endpoints (open, not included)
  \draw[bookred, fill=white, thick] (-1.5,0) circle (2.5pt);
  \draw[bookred, fill=white, thick] (1.5,0) circle (2.5pt);
  \node[above] at (-1.5, 0.1) {$p-r$};
  \node[above] at (1.5, 0.1) {$p+r$};
  \node[above] at (0, 0.5) {\small Open interval (1D ball)};
  \node[below] at (0, -0.5) {\small Boundary points not included};
\end{tikzpicture}
\end{minipage}
\hfill
\begin{minipage}{0.48\textwidth}
\centering
\begin{tikzpicture}[scale=1.2]
  % Draw the real line
  \draw[bookpurple, thick, <->] (-2.5,0) -- (2.5,0);
  % Mark center
  \filldraw[bookpurple] (0,0) circle (2pt) node[below] {$p$};
  % Mark radius
  \draw[bookpurple!50, dashed] (0,0) -- (-1.5,0);
  \draw[bookpurple!50, dashed] (0,0) -- (1.5,0);
  % Draw closed interval (for comparison)
  \draw[bookpurple!70, very thick] (-1.5,0) -- (1.5,0);
  % Mark endpoints (closed, included)
  \filldraw[bookpurple] (-1.5,0) circle (2.5pt);
  \filldraw[bookpurple] (1.5,0) circle (2.5pt);
  \node[above] at (-1.5, 0.1) {$p-r$};
  \node[above] at (1.5, 0.1) {$p+r$};
  \node[above] at (0, 0.5) {\small Closed interval};
  \node[below] at (0, -0.5) {\small Boundary points included};
\end{tikzpicture}
\end{minipage}
\caption{Comparison: (a) Open 1D ball (open interval) - boundary points not included. (b) Closed interval - boundary points included.}
\label{fig:1d-ball}
\end{figure}

\begin{example}
If $p = 0$ and $r = 2$, then $B_2(0) = (-2, 2)$ is the open interval from $-2$ to $2$, excluding the endpoints.
\end{example}

\subsection{2D Open Ball (Open Disk)}

In two dimensions, an open ball is a disk (circle's interior) without its boundary.

\textbf{Definition}: $B_r(\mathbf{p}) = \{(x,y) \in \mathbb{R}^2 : (x-p_x)^2 + (y-p_y)^2 < r^2\}$

\begin{figure}[htbp]
\centering
\begin{minipage}{0.48\textwidth}
\centering
\begin{tikzpicture}[scale=1.0]
  % Draw open disk (filled, no boundary)
  \filldraw[fill=bookpurple!30, draw=bookpurple, thick, dashed] (1.5,1) circle (1cm);
  % Mark center
  \filldraw[bookred] (1.5,1) circle (2pt) node[below] {$\mathbf{p}$};
  % Draw radius
  \draw[bookpurple!50, dashed] (1.5,1) -- (2.5,1) node[midway, above] {$r$};
  % Mark a point inside
  \filldraw[bookred] (2,1.5) circle (1.5pt);
  \node[right] at (2.1, 1.5) {\tiny Inside};
  % Mark boundary (dashed to show not included)
  \node[above] at (1.5, 2.2) {\small Open 2D ball};
  \node[below] at (1.5, -0.3) {\small (Disk without boundary)};
\end{tikzpicture}
\end{minipage}
\hfill
\begin{minipage}{0.48\textwidth}
\centering
\begin{tikzpicture}[scale=1.0]
  % Draw closed disk (with boundary)
  \filldraw[fill=bookpurple!30, draw=bookpurple, thick] (1.5,1) circle (1cm);
  % Mark center
  \filldraw[bookred] (1.5,1) circle (2pt) node[below] {$\mathbf{p}$};
  % Draw radius
  \draw[bookpurple!50, dashed] (1.5,1) -- (2.5,1) node[midway, above] {$r$};
  % Mark boundary point (included)
  \filldraw[bookpurple] (2.5,1) circle (2pt);
  \node[above] at (2.5, 1.1) {\tiny Boundary};
  \node[above] at (1.5, 2.2) {\small Closed 2D ball};
  \node[below] at (1.5, -0.3) {\small (Disk with boundary)};
\end{tikzpicture}
\end{minipage}
\caption{Comparison: (a) Open 2D ball (open disk) - boundary circle not included (shown dashed). (b) Closed disk - boundary included (solid circle).}
\label{fig:2d-ball}
\end{figure}

\begin{example}
The unit disk centered at the origin is $B_1(\mathbf{0}) = \{(x,y) : x^2 + y^2 < 1\}$.
\end{example}

\subsection{3D Open Ball (Open Sphere)}

In three dimensions, an open ball is the interior of a sphere (without its surface).

\textbf{Definition}: $B_r(\mathbf{p}) = \{(x,y,z) \in \mathbb{R}^3 : (x-p_x)^2 + (y-p_y)^2 + (z-p_z)^2 < r^2\}$

\begin{figure}[htbp]
\centering
\begin{tikzpicture}[scale=1.0]
  % Draw 3D sphere (shaded ball)
  \shade[ball color=bookpurple!40, draw=bookpurple, thick, dashed] (1.5,1) circle (1cm);
  % Mark center
  \filldraw[bookred] (1.5,1) circle (2pt) node[below] {$\mathbf{p}$};
  % Draw radius
  \draw[bookpurple!50, dashed] (1.5,1) -- (2.5,1) node[midway, above] {$r$};
  % Mark a point inside
  \filldraw[bookred] (2,1.3) circle (1.5pt);
  \node[right] at (2.1, 1.3) {\tiny Inside};
  \node[above] at (1.5, 2.3) {\small Open 3D ball};
  \node[below] at (1.5, -0.3) {\small (Sphere interior, surface not included)};
\end{tikzpicture}
\caption{Open 3D ball: the interior of a sphere. The surface (boundary) is not included, shown here with a dashed outline.}
\label{fig:3d-ball}
\end{figure}

\begin{example}
The unit ball centered at the origin is $B_1(\mathbf{0}) = \{(x,y,z) : x^2 + y^2 + z^2 < 1\}$.
\end{example}

\subsection{n-Dimensional Open Ball}

The concept extends naturally to any dimension $n$:

\[
B_r(\mathbf{p}) = \left\{(x_1, x_2, \ldots, x_n) \in \mathbb{R}^n : \sum_{i=1}^n (x_i - p_i)^2 < r^2\right\}
\]

\textbf{Key insight}: Even though we can't visualize dimensions beyond 3, the mathematical structure is the same: all points within distance $r$ of the center, with the boundary excluded.

\textbf{Notation}: When the center is the origin $\mathbf{0}$, we often write simply $B_r$ instead of $B_r(\mathbf{0})$.

\section{Connection to Manifold Definition}

\subsection{The Role of Open Balls in Manifolds}

The formal definition of a manifold requires that every point has a neighborhood that is homeomorphic to an open ball in $\mathbb{R}^n$. This is what we mean when we say a manifold "locally looks like Euclidean space."

\textbf{Formal statement}: A space $M$ is an $n$-dimensional manifold if for every point $p \in M$, there exists:
\begin{itemize}
\item An open neighborhood $U$ of $p$ in $M$
\item A homeomorphism $\phi: U \to B_r(\mathbf{0}) \subset \mathbb{R}^n$ (an open ball)
\end{itemize}

The pair $(U, \phi)$ is called a \textbf{chart}, and $\phi$ maps the local neighborhood on the manifold to an open ball in Euclidean space.

\subsection{Visual Connection}

\begin{figure}[htbp]
\centering
\begin{tikzpicture}[scale=1.0]
  % Draw a curved manifold (2D surface)
  \draw[fill=bookpurple!30, draw=bookpurple, thick] 
    (0,0) .. controls (1,0.5) and (2,0.3) .. (3,0)
    .. controls (3.2,0.2) and (3,0.5) .. (2.5,0.8)
    .. controls (1.5,1) and (0.5,0.8) .. (0,0.5)
    .. controls (-0.2,0.3) and (0,0.1) .. (0,0);
  
  % Mark a point on the manifold
  \filldraw[bookred] (1.5,0.4) coordinate (P) circle (2.5pt) node[above left] {$p$};
  
  % Draw a local neighborhood (patch) on the manifold
  \draw[fill=bookred!40, opacity=0.6, draw=bookred, thick] 
    (1.2,0.3) .. controls (1.3,0.35) .. (1.7,0.35)
    .. controls (1.8,0.4) .. (1.7,0.5)
    .. controls (1.6,0.55) .. (1.3,0.5)
    .. controls (1.2,0.45) .. (1.2,0.3);
  \node[above, bookred] at (1.45, 0.55) {\small Neighborhood $U$};
  
  % Draw arrow to flat space
  \draw[thick, ->, bookred, line width=2pt] (2.5, 0.5) to [out=0, in=180] 
    node[midway, above, bookred] {$\phi$} (4, 1.5);
  
  % Draw the open ball in Euclidean space
  \filldraw[fill=bookpurple!30, draw=bookpurple, thick, dashed] (5, 1.5) circle (0.8cm);
  \node[above, bookpurple] at (5, 2.5) {\small Open ball $B_r(\mathbf{0})$};
  \node[below] at (5, 0.5) {\small in $\mathbb{R}^2$};
  
  % Label the manifold
  \node[below] at (1.5, -0.3) {\small 2D Manifold $M$};
\end{tikzpicture}
\caption{The manifold definition: Every point $p$ on a manifold $M$ has a neighborhood $U$ that maps via a homeomorphism $\phi$ to an open ball in $\mathbb{R}^n$. This is what "locally looks like Euclidean space" means precisely.}
\label{fig:manifold-ball-connection}
\end{figure}

\subsection{Dimension Matching}

Crucially, the dimension of the open ball must match the dimension of the manifold:
\begin{itemize}
\item A 1D manifold (like a curve) requires neighborhoods homeomorphic to open 1D balls (intervals).
\item A 2D manifold (like a sphere's surface) requires neighborhoods homeomorphic to open 2D balls (disks).
\item An $n$-dimensional manifold requires neighborhoods homeomorphic to open $n$-dimensional balls.
\end{itemize}

This dimension matching is essential: you can't map a 2D surface patch to a 1D interval without losing information.

\section{Distance Calculations in Open n-Balls}

\subsection{Euclidean Distance Formula}

Within an open n-ball, we use the standard Euclidean distance formula. For two points $\mathbf{p} = (p_1, p_2, \ldots, p_n)$ and $\mathbf{q} = (q_1, q_2, \ldots, q_n)$ in $\mathbb{R}^n$:

\[
d(\mathbf{p}, \mathbf{q}) = \sqrt{\sum_{i=1}^n (p_i - q_i)^2} = \|\mathbf{p} - \mathbf{q}\|
\]

This is the same formula we've used in previous chapters for distances on planes and in Euclidean space.

\subsection{Examples in Different Dimensions}

\subsubsection{1D Distance}

In a 1D open ball (interval), the distance is simply:

\[
d(p, q) = |p - q|
\]

\begin{example}
In the open interval $(-2, 2)$, the distance between $p = -1$ and $q = 1.5$ is:
\[
d(-1, 1.5) = |1.5 - (-1)| = 2.5
\]
\end{example}

\subsubsection{2D Distance}

In a 2D open ball (disk), the distance is:

\[
d(\mathbf{p}, \mathbf{q}) = \sqrt{(p_x - q_x)^2 + (p_y - q_y)^2}
\]

\begin{example}
In the unit disk centered at the origin, for points $\mathbf{p} = (0.3, 0.4)$ and $\mathbf{q} = (0.7, 0.2)$:
\[
d = \sqrt{(0.7-0.3)^2 + (0.2-0.4)^2} = \sqrt{0.16 + 0.04} = \sqrt{0.2} \approx 0.447
\]
\end{example}

\subsubsection{3D Distance}

In a 3D open ball, the distance is:

\[
d(\mathbf{p}, \mathbf{q}) = \sqrt{(p_x - q_x)^2 + (p_y - q_y)^2 + (p_z - q_z)^2}
\]

\begin{example}
In the unit ball, for points $\mathbf{p} = (0.5, 0.5, 0.5)$ and $\mathbf{q} = (0.8, 0.3, 0.6)$:
\[
d = \sqrt{(0.3)^2 + (-0.2)^2 + (0.1)^2} = \sqrt{0.09 + 0.04 + 0.01} = \sqrt{0.14} \approx 0.374
\]
\end{example}

\subsubsection{n-Dimensional Distance}

The general formula works for any dimension:

\[
d(\mathbf{p}, \mathbf{q}) = \sqrt{\sum_{i=1}^n (p_i - q_i)^2}
\]

\subsection{Properties of Distance in Open Balls}

\subsubsection{Maximum Distance Within a Ball}

The maximum distance between any two points in an open ball $B_r(\mathbf{p})$ is less than $2r$ (the diameter). Since the ball is open, we can approach but never reach $2r$:

\[
\sup\{d(\mathbf{x}, \mathbf{y}) : \mathbf{x}, \mathbf{y} \in B_r(\mathbf{p})\} = 2r
\]

\textbf{Note}: The supremum is $2r$, but this maximum is never actually achieved because the ball is open (boundary excluded).

\subsubsection{Triangle Inequality}

For any three points $\mathbf{p}$, $\mathbf{q}$, $\mathbf{r}$ in an open ball:

\[
d(\mathbf{p}, \mathbf{r}) \leq d(\mathbf{p}, \mathbf{q}) + d(\mathbf{q}, \mathbf{r})
\]

This is a fundamental property of Euclidean distance that holds in any dimension.

\section{Properties of Open n-Balls}

\subsection{Topological Properties}

Open n-balls have several important topological properties:

\subsubsection{Open Set}

An open ball is itself an \textbf{open set}: every point in the ball has a neighborhood entirely contained within the ball. This means:
\begin{itemize}
\item For any point $\mathbf{q} \in B_r(\mathbf{p})$, there exists $\epsilon > 0$ such that $B_\epsilon(\mathbf{q}) \subset B_r(\mathbf{p})$.
\item You can always "shrink" around any interior point and stay inside.
\end{itemize}

\subsubsection{Connected}

An open ball is \textbf{connected}: any two points in the ball can be connected by a continuous path that stays entirely within the ball. This makes sense intuitively—you can always draw a line (or curve) between any two points inside a ball without leaving it.

\subsubsection{Simply Connected}

An open ball is \textbf{simply connected}: any loop (closed path) can be continuously shrunk to a point while staying in the ball. There are no "holes" in an open ball.

\subsection{Geometric Properties}

\subsubsection{Convex}

An open ball is \textbf{convex}: the line segment between any two points in the ball is entirely contained in the ball. In other words, if $\mathbf{p}, \mathbf{q} \in B_r(\mathbf{c})$, then for all $t \in [0,1]$:

\[
(1-t)\mathbf{p} + t\mathbf{q} \in B_r(\mathbf{c})
\]

This is a key geometric property that makes balls "round" and symmetric.

\subsubsection{Bounded}

An open ball is \textbf{bounded}: all points are within a finite distance from the center. Specifically, every point $\mathbf{x} \in B_r(\mathbf{p})$ satisfies:

\[
\|\mathbf{x} - \mathbf{p}\| < r < \infty
\]

\subsubsection{Symmetric}

An open ball is \textbf{symmetric}: it looks the same in all directions from the center. This symmetry is a consequence of the Euclidean metric, which treats all directions equally.

\subsection{Comparison with Closed Balls}

It's instructive to compare open and closed balls:

\begin{table}[htbp]
\centering
\begin{tabular}{lcc}
\toprule
\textbf{Property} & \textbf{Open Ball} & \textbf{Closed Ball} \\
\midrule
Boundary & Not included & Included \\
Definition & $\|\mathbf{x} - \mathbf{p}\| < r$ & $\|\mathbf{x} - \mathbf{p}\| \leq r$ \\
Open set? & Yes & No \\
For manifolds & Required & Not suitable \\
\bottomrule
\end{tabular}
\caption{Comparison of open and closed balls.}
\label{tab:open-vs-closed}
\end{table}

\textbf{Why open balls for manifolds?}: The manifold definition requires neighborhoods without boundaries because:
\begin{itemize}
\item Boundaries create "edge effects" that complicate local structure
\item Homeomorphisms are cleaner without boundary constraints
\item The local structure should be "smooth" and unrestricted
\end{itemize}

\section{Open Balls as Neighborhoods}

\subsection{Neighborhood Concept}

A \textbf{neighborhood} of a point $\mathbf{p}$ is any open set containing $\mathbf{p}$. Open balls are the most fundamental type of neighborhood.

\textbf{Key fact}: Any neighborhood of a point contains an open ball around that point. This means open balls are the "smallest" or "most basic" neighborhoods we can use.

\subsection{Why This Matters for Manifolds}

The manifold definition states: "Every point has a neighborhood homeomorphic to an open ball." This is equivalent to saying "locally looks like Euclidean space" because:

\begin{itemize}
\item Open balls are the standard neighborhoods in Euclidean space
\item Any neighborhood in Euclidean space contains an open ball
\item So being homeomorphic to a neighborhood is equivalent to being homeomorphic to an open ball
\end{itemize}

This is why open balls are so fundamental—they're what we mean by "Euclidean space locally."

\subsection{Visual: Zooming In}

\begin{figure}[htbp]
\centering
\begin{tikzpicture}[scale=1.0]
  % Draw a curved manifold (sphere surface)
  \shade[ball color=bookpurple!30] (1.5,1) circle (1.2cm);
  \draw[bookpurple, thick] (1.5,1) circle (1.2cm);
  
  % Mark a point
  \filldraw[bookred] (2,1.5) coordinate (P) circle (2.5pt) node[right] {$p$};
  
  % Draw a small neighborhood (zoom in)
  \draw[bookred, dashed, very thick] (P) circle (0.3cm);
  
  % Arrow showing zoom
  \draw[thick, ->, bookred, line width=2pt] (3.5, 1.5) -- (4.5, 1.5);
  \node[above, bookred] at (4, 1.8) {\small Zoom in};
  
  % Draw zoomed view (flat disk)
  \filldraw[fill=bookpurple!30, draw=bookpurple, thick, dashed] (6, 1.5) circle (1cm);
  \filldraw[bookred] (6, 1.5) circle (2pt) node[below] {$p$};
  \node[above, bookpurple] at (6, 2.7) {\small Looks like open disk};
  \node[below] at (6, 0.3) {\small (2D open ball)};
\end{tikzpicture}
\caption{Zooming in on a manifold: As you zoom in on any point, the local neighborhood looks more and more like an open ball in Euclidean space. This is the essence of local flatness.}
\label{fig:zooming-in}
\end{figure}

This visualization captures the intuitive idea: no matter where you are on a manifold, if you zoom in enough, your local view looks like a flat open ball.

\section{Special Cases and Examples}

\subsection{Unit Ball}

The \textbf{unit ball} is the open ball of radius 1 centered at the origin:

\[
B_1(\mathbf{0}) = \{\mathbf{x} \in \mathbb{R}^n : \|\mathbf{x}\| < 1\}
\]

In different dimensions:
\begin{itemize}
\item \textbf{1D}: $(-1, 1)$ - the open interval from $-1$ to $1$
\item \textbf{2D}: $\{(x,y) : x^2 + y^2 < 1\}$ - the unit disk
\item \textbf{3D}: $\{(x,y,z) : x^2 + y^2 + z^2 < 1\}$ - the unit sphere (interior)
\item \textbf{nD}: $\{\mathbf{x} : \|\mathbf{x}\| < 1\}$ - the n-dimensional unit ball
\end{itemize}

The unit ball is often used as a standard reference because it's simple and symmetric.

\subsection{Balls at Different Centers}

Any open ball $B_r(\mathbf{p})$ is just a translation of the unit ball:

\[
B_r(\mathbf{p}) = \mathbf{p} + r \cdot B_1(\mathbf{0})
\]

This means all open balls are geometrically equivalent (homeomorphic) regardless of their center or radius.

\subsection{Relationship to Other Shapes}

Open balls are not the only open sets that could work for manifolds, but they're the standard choice:

\begin{itemize}
\item \textbf{Open ball}: Round, symmetric, standard choice
\item \textbf{Open rectangle/cube}: Also works, but not symmetric
\item \textbf{Open ellipse}: Also works, but more complex
\end{itemize}

\textbf{Why balls?}: 
\begin{itemize}
\item Symmetry makes them natural
\item Simplicity in calculations
\item Standard in mathematics
\item All open neighborhoods contain balls anyway
\end{itemize}

\section{Worked Examples}

\subsection{Example 1: Is a Point in an Open Ball?}

\begin{example}
\textbf{Question}: Is the point $(2, 3)$ in the open ball centered at $(1, 1)$ with radius $r = 2.5$?

\textbf{Solution}: Check if the distance is less than the radius:

\[
d((2,3), (1,1)) = \sqrt{(2-1)^2 + (3-1)^2} = \sqrt{1 + 4} = \sqrt{5} \approx 2.236
\]

Since $2.236 < 2.5$, the point $(2, 3)$ is inside the open ball.
\end{example}

\subsection{Example 2: Distance Between Two Points in a Ball}

\begin{example}
\textbf{Question}: Calculate the distance between points $(0.5, 0.5)$ and $(0.8, 0.7)$ in the unit ball $B_1(\mathbf{0})$.

\textbf{Solution}: First, verify both points are in the unit ball:
\begin{itemize}
\item For $(0.5, 0.5)$: $0.5^2 + 0.5^2 = 0.5 < 1$ (inside)
\item For $(0.8, 0.7)$: $0.8^2 + 0.7^2 = 0.64 + 0.49 = 1.13 > 1$ (outside!)
\end{itemize}

Since $(0.8, 0.7)$ is outside the unit ball, let's use $(0.6, 0.6)$ instead, which is inside ($0.6^2 + 0.6^2 = 0.72 < 1$).

Now calculate the distance between $(0.5, 0.5)$ and $(0.6, 0.6)$:

\[
d = \sqrt{(0.6-0.5)^2 + (0.6-0.5)^2} = \sqrt{0.01 + 0.01} = \sqrt{0.02} \approx 0.141
\]
\end{example}

Both points are in the unit ball, and their distance is approximately $0.141$.

\subsection{Example 3: Finding the Largest Contained Ball}

\begin{example}
\textbf{Question}: Given a rectangular region $R = \{(x,y) : 0 < x < 3, 0 < y < 2\}$, what's the largest open ball that fits entirely in $R$?

\textbf{Solution}: The largest ball will be limited by the closest boundary. The center should be at the center of the rectangle: $(1.5, 1.0)$. The distance to the nearest boundary is $\min(1.5, 1.0, 1.5, 1.0) = 1.0$.

So the largest open ball is $B_1((1.5, 1.0))$ with radius $r = 1.0$.
\end{example}

\begin{keytakeaways}
\begin{itemize}
\item \textbf{Definition}: An open n-ball $B_r(\mathbf{p})$ is the set of points within distance $r$ of center $\mathbf{p}$, with the boundary excluded.

\item \textbf{Role in manifolds}: Manifolds are locally homeomorphic to open n-balls—this is the precise meaning of "locally looks like Euclidean space."

\item \textbf{Dimension matching}: An n-dimensional manifold requires neighborhoods homeomorphic to n-dimensional open balls.

\item \textbf{Why "open"?}: Excluding the boundary is essential for manifolds—it ensures smooth local structure without edge effects.

\item \textbf{Distance}: Within open balls, we use standard Euclidean distance: $d(\mathbf{p}, \mathbf{q}) = \sqrt{\sum (p_i - q_i)^2}$.

\item \textbf{Properties}: Open balls are open sets, connected, convex, bounded, and symmetric.

\item \textbf{Standard neighborhoods}: Open balls are the fundamental neighborhoods in Euclidean space, making them the natural choice for manifold definitions.
\end{itemize}
\end{keytakeaways}

\section{What's Next?}

Understanding open n-balls provides the foundation for:
\begin{itemize}
\item \textbf{Charts and atlases}: Collections of open ball mappings that cover a manifold
\item \textbf{Smooth manifolds}: Manifolds with differentiable structure
\item \textbf{Differential geometry}: The study of geometric structures on manifolds
\item \textbf{Advanced topics}: Riemannian manifolds, Lie groups, and more
\end{itemize}

Open n-balls are the building blocks that make the manifold concept precise and mathematically rigorous.
