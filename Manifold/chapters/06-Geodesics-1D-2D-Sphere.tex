\chapter{Geodesics in 1D, 2D, and on a Sphere}

In previous chapters, we calculated distances on manifolds and learned that these distances equal the lengths of geodesics. This chapter goes deeper: we'll derive geodesics systematically, solve the geodesic equations, and explore their properties. Instead of just stating what geodesics are, we'll learn \textit{how} to find them and \textit{why} they have their particular forms.

\section{Introduction: From Distances to Paths}

We've already established that:
\begin{itemize}
\item The distance between two points on a manifold equals the length of the geodesic connecting them.
\item On a line, geodesics are straight segments.
\item On a circle, geodesics are arcs.
\item On a plane, geodesics are straight lines.
\item On a sphere, geodesics are great circle arcs.
\end{itemize}

But \textit{how} do we know these are the geodesics? And \textit{how} can we find geodesics on more complex manifolds? This chapter answers these questions by introducing systematic methods for deriving geodesics.

\section{Deriving Geodesics: The Variational Approach}

\subsection{The Path Length Functional}

A fundamental principle is that geodesics minimize path length. Mathematically, if we have a path $\gamma(t)$ between two points $P_1$ and $P_2$ on a manifold, its length is given by:

\[
L[\gamma] = \int_{t_1}^{t_2} \sqrt{g_{ij}\frac{dx^i}{dt}\frac{dx^j}{dt}} \, dt
\]

where $g_{ij}$ is the metric tensor encoding the geometry, and $x^i(t)$ are the coordinates of the path. A geodesic is a path that minimizes this functional.

\subsection{The Euler-Lagrange Equations}

To find paths that minimize $L[\gamma]$, we use the calculus of variations. The Euler-Lagrange equations tell us that if a path minimizes an integral $\int L(x, \dot{x}, t) \, dt$, then:

\[
\frac{d}{dt}\left(\frac{\partial L}{\partial \dot{x}^i}\right) - \frac{\partial L}{\partial x^i} = 0
\]

For the path length functional, this leads to the geodesic equation. Let's see how this works for simple manifolds.

\subsection{Applying Variational Calculus: 1D Line}

On a 1D line, the metric is simply $g_{11} = 1$, so the path length is:

\[
L = \int_{t_1}^{t_2} \left|\frac{dx}{dt}\right| \, dt = \int_{t_1}^{t_2} \sqrt{\left(\frac{dx}{dt}\right)^2} \, dt
\]

The Lagrangian is $L = \sqrt{(\dot{x})^2} = |\dot{x}|$. For a smooth path, we can assume $\dot{x} > 0$ (or reverse the parameterization), so $L = \dot{x}$.

The Euler-Lagrange equation gives:
\[
\frac{d}{dt}\left(\frac{\partial L}{\partial \dot{x}}\right) - \frac{\partial L}{\partial x} = \frac{d}{dt}(1) - 0 = 0
\]

This is automatically satisfied! But we need to consider the full functional. Actually, for the path length, we should use $L = (\dot{x})^2$ (minimizing squared length is equivalent), which gives:

\[
\frac{d}{dt}(2\dot{x}) = 0 \quad \Rightarrow \quad \ddot{x} = 0
\]

This means $\dot{x}$ is constant, so $x(t) = at + b$—a straight line! This confirms that geodesics on a line are straight segments.

\subsection{Applying Variational Calculus: 2D Plane}

On a 2D plane with Cartesian coordinates, the metric is $g_{ij} = \delta_{ij}$ (the identity matrix), so:

\[
L = \int \sqrt{(\dot{x})^2 + (\dot{y})^2} \, dt
\]

Using the squared length approach (minimizing $(\dot{x})^2 + (\dot{y})^2$), the Euler-Lagrange equations give:

\[
\ddot{x} = 0, \quad \ddot{y} = 0
\]

This means both $x(t)$ and $y(t)$ are linear functions, so the path is a straight line: $\mathbf{r}(t) = \mathbf{p}_1 + t(\mathbf{p}_2 - \mathbf{p}_1)$.

\section{Solving the Geodesic Equation}

\subsection{The Geodesic Equation in Detail}

The geodesic equation is:

\[
\frac{d^2 x^i}{dt^2} + \Gamma^i_{jk}\frac{dx^j}{dt}\frac{dx^k}{dt} = 0
\]

where $\Gamma^i_{jk}$ are the \textbf{Christoffel symbols}, which encode how the metric changes with position. They are given by:

\[
\Gamma^i_{jk} = \frac{1}{2}g^{il}\left(\frac{\partial g_{jl}}{\partial x^k} + \frac{\partial g_{kl}}{\partial x^j} - \frac{\partial g_{jk}}{\partial x^l}\right)
\]

where $g^{il}$ is the inverse of the metric tensor.

\subsection{Solving for a 1D Line}

On a line, the metric is constant: $g_{11} = 1$. Therefore, all partial derivatives vanish, so $\Gamma^1_{11} = 0$.

The geodesic equation becomes:
\[
\frac{d^2 x}{dt^2} = 0
\]

Solving this: $\dot{x} = \text{constant}$, so $x(t) = x_0 + vt$. This is a straight line with constant velocity, confirming our earlier result.

\subsection{Solving for a 2D Plane}

In Cartesian coordinates, the metric is $g_{ij} = \delta_{ij}$, which is constant. Therefore, all Christoffel symbols vanish: $\Gamma^i_{jk} = 0$.

The geodesic equations become:
\[
\frac{d^2 x}{dt^2} = 0, \quad \frac{d^2 y}{dt^2} = 0
\]

Solutions: $x(t) = x_0 + v_x t$, $y(t) = y_0 + v_y t$. This describes a straight line in the plane.

\subsection{Solving for a 1D Circle}

A circle of radius $r$ can be parameterized by angle $\theta$. The metric is $g_{\theta\theta} = r^2$ (constant), so $\Gamma^\theta_{\theta\theta} = 0$.

The geodesic equation becomes:
\[
\frac{d^2 \theta}{dt^2} = 0
\]

Solution: $\theta(t) = \theta_0 + \omega t$, where $\omega$ is constant. This describes uniform motion along the circle—a geodesic arc.

\subsection{Solving for a 2D Sphere}

On a sphere of radius $R$, using spherical coordinates $(\phi, \lambda)$ (latitude, longitude), the metric is:

\[
ds^2 = R^2(d\phi^2 + \sin^2\phi \, d\lambda^2)
\]

The Christoffel symbols are non-zero. The geodesic equations become:

\begin{align}
\frac{d^2\phi}{dt^2} - \sin\phi\cos\phi\left(\frac{d\lambda}{dt}\right)^2 &= 0 \\
\frac{d^2\lambda}{dt^2} + 2\cot\phi\frac{d\phi}{dt}\frac{d\lambda}{dt} &= 0
\end{align}

These equations are more complex, but their solutions describe great circles. One can verify that paths with constant $\lambda$ (lines of longitude) and the equator ($\phi = 0$) are solutions, representing great circles.

A more elegant approach uses the fact that great circles are intersections of the sphere with planes through the center, which we'll explore in the next section.

\section{Geometric Construction of Geodesics}

\subsection{Construction Methods}

Sometimes it's easier to construct geodesics geometrically rather than solving differential equations. Let's see how this works for each manifold.

\subsubsection{1D Line}

Trivial: the geodesic is simply the straight line segment connecting the two points.

\subsubsection{1D Circle}

Given two points on a circle, there are two arcs connecting them. The geodesic is the shorter arc. If the points are antipodal, both arcs are geodesics of equal length.

\subsubsection{2D Plane}

The geodesic is the straight line segment. This can be constructed using a ruler or by drawing the line through the two points.

\subsubsection{2D Sphere: Great Circle Construction}

For a sphere, geodesics are great circles. Here's how to construct the great circle through two points $P_1$ and $P_2$:

\begin{enumerate}
\item Consider the sphere centered at the origin.
\item The two points define vectors $\mathbf{p}_1$ and $\mathbf{p}_2$ from the center.
\item The great circle lies in the plane containing the origin, $P_1$, and $P_2$.
\item This plane is perpendicular to the normal vector $\mathbf{n} = \mathbf{p}_1 \times \mathbf{p}_2$.
\item The intersection of this plane with the sphere is the great circle.
\item The geodesic is the shorter arc of this great circle connecting $P_1$ and $P_2$.
\end{enumerate}

\begin{figure}[htbp]
\centering
\begin{minipage}{0.48\textwidth}
\centering
\begin{tikzpicture}[scale=1.2]
  % Draw sphere
  \shade[ball color=bookpurple!30] (1.5,1) circle (1cm);
  \draw[bookpurple, thick] (1.5,1) circle (1cm);
  % Mark center
  \filldraw[bookpurple] (1.5,1) circle (1.5pt) node[below] {Center};
  % Mark two points
  \filldraw[bookred] (0.8,0.7) coordinate (P1) circle (2pt) node[below left] {$P_1$};
  \filldraw[bookred] (2.2,1.3) coordinate (P2) circle (2pt) node[above right] {$P_2$};
  % Draw vectors from center
  \draw[bookpurple!50, dashed] (1.5,1) -- (P1);
  \draw[bookpurple!50, dashed] (1.5,1) -- (P2);
  % Draw great circle arc (geodesic)
  \draw[bookred, very thick] (P1) arc (225:45:1.5);
  % Show the plane (as a line through center)
  \draw[bookpurple!30, thick, dashed] (0.3,0.5) -- (2.7,1.5);
  \node[above] at (1.5, 2.2) {\textbf{(a) Great Circle Construction}};
\end{tikzpicture}
\end{minipage}
\hfill
\begin{minipage}{0.48\textwidth}
\centering
\begin{tikzpicture}[scale=1.2]
  % Show cross-section: plane cutting sphere
  \draw[bookpurple, thick] (1.5,0.5) arc (180:0:1.5 and 0.5);
  \draw[bookpurple, thick] (1.5,1.5) arc (180:0:1.5 and 0.5);
  \draw[bookpurple, thick] (0,1) -- (3,1);
  % Draw the cutting plane
  \draw[bookpurple!40, fill=bookpurple!20, opacity=0.5] (0,0.5) -- (3,1.5) -- (3,0.5) -- (0,1.5) -- cycle;
  % Mark intersection (great circle)
  \draw[bookred, very thick] (0.5,0.75) arc (180:0:1 and 0.25);
  % Mark two points
  \filldraw[bookred] (0.5,0.75) circle (2pt) node[left] {$P_1$};
  \filldraw[bookred] (2.5,1.25) circle (2pt) node[right] {$P_2$};
  % Mark center
  \filldraw[bookpurple] (1.5,1) circle (1.5pt) node[above] {Center};
  \node[above] at (1.5, 2.2) {\textbf{(b) Cross-Section View}};
  \node[below] at (1.5, -0.3) {\small Plane through center};
\end{tikzpicture}
\end{minipage}
\caption{Constructing a great circle geodesic on a sphere: (a) The great circle is the intersection of the sphere with a plane through the center and both points. (b) Cross-section showing how the plane cuts the sphere.}
\label{fig:great-circle-construction}
\end{figure}

This geometric construction is often more intuitive than solving differential equations and directly shows why great circles are geodesics.

\subsection{Coordinate-Free Description}

Geodesics are \textbf{intrinsic} to the manifold—they depend only on the metric, not on how the manifold is embedded in higher-dimensional space. For example, great circles on a sphere are geodesics regardless of how we view the sphere in 3D space. This is a fundamental property of Riemannian geometry.

\section{Multiple Geodesics and Special Cases}

\subsection{When Are There Multiple Geodesics?}

Not all pairs of points have a unique geodesic. Let's examine when multiple geodesics exist:

\begin{itemize}
\item \textbf{1D Line}: Always exactly one geodesic (the straight line segment).

\item \textbf{1D Circle}: 
  \begin{itemize}
  \item Usually one geodesic (the shorter arc).
  \item If points are antipodal: two geodesics of equal length (both semicircles).
  \end{itemize}

\item \textbf{2D Plane}: Always exactly one geodesic (the straight line).

\item \textbf{2D Sphere}:
  \begin{itemize}
  \item Usually one shortest geodesic (the shorter great circle arc).
  \item If points are antipodal: infinitely many geodesics (all great circles through them have the same length: $\pi R$).
  \item The longer great circle arc is also a geodesic, but not the shortest one.
  \end{itemize}
\end{itemize}

\begin{figure}[htbp]
\centering
\begin{minipage}{0.48\textwidth}
\centering
\begin{tikzpicture}[scale=1.0]
  % Circle with antipodal points
  \draw[bookpurple, thick] (0,0) circle (1.2cm);
  % Mark antipodal points
  \filldraw[bookred] (1.2,0) coordinate (P1) circle (2pt) node[right] {$P_1$};
  \filldraw[bookred] (-1.2,0) coordinate (P2) circle (2pt) node[left] {$P_2$};
  % Draw two geodesics (semicircles)
  \draw[bookred, very thick] (P1) arc (0:180:1.2);
  \draw[bookred!50, very thick, dashed] (P1) arc (0:-180:1.2);
  \node[above] at (0, 1.5) {\textbf{(a) Antipodal on Circle}};
  \node[below] at (0, -1.7) {\small Two equal geodesics};
\end{tikzpicture}
\end{minipage}
\hfill
\begin{minipage}{0.48\textwidth}
\centering
\begin{tikzpicture}[scale=1.0]
  % Sphere with antipodal points
  \shade[ball color=bookpurple!30] (1.5,1) circle (1cm);
  \draw[bookpurple, thick] (1.5,1) circle (1cm);
  % Mark antipodal points (top and bottom)
  \filldraw[bookred] (1.5,2) circle (2pt) node[above] {$P_1$};
  \filldraw[bookred] (1.5,0) circle (2pt) node[below] {$P_2$};
  % Draw several great circles through them
  \draw[bookred, very thick] (1.5,2) arc (90:270:1.5);
  \draw[bookred!70, very thick] (0.5,1) arc (180:360:1.5);
  \draw[bookred!40, very thick, dashed] (2.5,1) arc (0:180:1.5);
  \node[above] at (1.5, 2.5) {\textbf{(b) Antipodal on Sphere}};
  \node[below] at (1.5, -0.5) {\small Infinitely many geodesics};
\end{tikzpicture}
\end{minipage}
\caption{Multiple geodesics: (a) On a circle, antipodal points have two geodesics of equal length. (b) On a sphere, antipodal points have infinitely many geodesics (all great circles through them).}
\label{fig:multiple-geodesics}
\end{figure}

\subsection{Conjugate Points and Cut Locus}

The set of points where geodesics cease to be unique is called the \textbf{cut locus}. For example:
\begin{itemize}
\item On a circle, the cut locus of a point is its antipodal point.
\item On a sphere, the cut locus of a point is its antipodal point.
\end{itemize}

At these special points, the geometry becomes degenerate in the sense that multiple geodesics have the same length.

\section{Geodesics in Different Coordinate Systems}

\subsection{Coordinate Transformations}

Geodesics are geometric objects—they exist independently of coordinate systems. However, their descriptions change with coordinates. Let's see how the same geodesic looks in different coordinate systems.

\subsubsection{Example: Straight Line in Different Coordinates}

Consider a straight line in the plane. In Cartesian coordinates $(x, y)$, it's simply:
\[
y = mx + b
\]

In polar coordinates $(r, \theta)$, the same line becomes:
\[
r = \frac{b}{\sin\theta - m\cos\theta}
\]

This is more complex, but it's the same geometric object.

\subsubsection{Example: Great Circle on Sphere}

A great circle on a sphere can be described in:
\begin{itemize}
\item \textbf{Cartesian coordinates}: $ax + by + cz = 0$ (plane through origin)
\item \textbf{Spherical coordinates}: More complex, but the geodesic equation we solved earlier describes it.
\end{itemize}

The key insight: the geodesic equation transforms covariantly, so solving it in any coordinate system gives the same geometric geodesic.

\subsection{Natural Parameterization}

A particularly useful parameterization is the \textbf{arc length parameter} $s$. In this parameterization, the speed along the geodesic is constant (equal to 1), so:

\[
\left|\frac{d\mathbf{r}}{ds}\right| = 1
\]

For example:
\begin{itemize}
\item On a line: $x(s) = x_0 + s$ (if we choose appropriate orientation).
\item On a circle: $\theta(s) = \theta_0 + s/r$ (uniform angular speed).
\item On a plane: $\mathbf{r}(s) = \mathbf{r}_0 + s\mathbf{u}$ where $|\mathbf{u}| = 1$.
\end{itemize}

This parameterization simplifies many calculations and is often used in differential geometry.

\section{Properties and Characteristics of Geodesics}

\subsection{Local vs Global Properties}

\subsubsection{Local Minimality}

Geodesics are \textbf{locally shortest}: if you take a small enough neighborhood around any point on a geodesic, it's the shortest path between its endpoints in that neighborhood.

\subsubsection{Global Minimality}

On simply connected manifolds (like a plane or sphere), geodesics are also \textbf{globally shortest}—they minimize distance among all possible paths. However, there can be exceptions:

\begin{itemize}
\item On a sphere, the longer great circle arc is a geodesic but not the shortest path.
\item On a torus, there can be multiple geodesics between two points, and the shortest one depends on how the geodesic wraps around.
\end{itemize}

\subsection{Geodesic Completeness}

A manifold is \textbf{geodesically complete} if every geodesic can be extended indefinitely (in both directions). Examples:

\begin{itemize}
\item \textbf{Line}: Complete—geodesics extend to $\pm\infty$.
\item \textbf{Circle}: Complete—geodesics are closed (they loop around).
\item \textbf{Plane}: Complete—geodesics extend infinitely.
\item \textbf{Sphere}: Complete—geodesics are closed (great circles).
\end{itemize}

Geodesic completeness is an important property that relates to the global structure of the manifold.

\subsection{Geodesic Curvature}

Geodesics have \textbf{zero geodesic curvature}—they are "as straight as possible" on the manifold. This is an intrinsic property. However, they may appear curved when viewed from an embedding space:

\begin{itemize}
\item A great circle on a sphere has zero geodesic curvature (intrinsic).
\item But it appears curved when viewed in 3D space (extrinsic curvature).
\end{itemize}

This distinction between intrinsic and extrinsic geometry is fundamental to understanding manifolds.

\section{Worked Examples: Systematic Derivation}

\subsection{Example 1: Circle Geodesic via Variational Method}

Let's derive that arcs are geodesics on a circle using the variational approach.

\textbf{Setup}: Circle of radius $r$, parameterized by angle $\theta$. The metric is $g_{\theta\theta} = r^2$.

\textbf{Path length functional}:
\[
L = \int \sqrt{r^2\left(\frac{d\theta}{dt}\right)^2} \, dt = r\int \left|\frac{d\theta}{dt}\right| \, dt
\]

For smooth paths, we can use $L = r\int (\dot{\theta})^2 \, dt$ (minimizing squared length).

\textbf{Euler-Lagrange equation}:
\[
\frac{d}{dt}\left(\frac{\partial L}{\partial \dot{\theta}}\right) - \frac{\partial L}{\partial \theta} = \frac{d}{dt}(2r^2\dot{\theta}) = 0
\]

This gives $\ddot{\theta} = 0$, so $\theta(t) = \theta_0 + \omega t$—uniform motion along the circle, which describes an arc. This confirms that arcs are geodesics on a circle.

\subsection{Example 2: Sphere Geodesic via Geometric Construction}

Let's construct the great circle through two points on a sphere.

\textbf{Setup}: Sphere of radius $R$, points $P_1$ and $P_2$ with position vectors $\mathbf{p}_1$ and $\mathbf{p}_2$.

\textbf{Construction}:
\begin{enumerate}
\item The normal to the plane containing the great circle is $\mathbf{n} = \mathbf{p}_1 \times \mathbf{p}_2$.
\item The plane equation is $\mathbf{n} \cdot \mathbf{r} = 0$ (passes through origin).
\item The intersection of this plane with the sphere $|\mathbf{r}| = R$ is the great circle.
\item The geodesic is the shorter arc of this great circle.
\end{enumerate}

This geometric construction is often simpler than solving differential equations and directly shows why great circles are geodesics.

\subsection{Example 3: Coordinate Transformation}

Let's verify that a straight line geodesic on a plane transforms correctly under coordinate changes.

\textbf{In Cartesian coordinates}: Geodesic is $y = 2x + 1$ (straight line).

\textbf{In polar coordinates}: $x = r\cos\theta$, $y = r\sin\theta$. Substituting:
\[
r\sin\theta = 2r\cos\theta + 1
\]

Solving for $r$:
\[
r = \frac{1}{\sin\theta - 2\cos\theta}
\]

This is the same geodesic, just described in different coordinates. The geodesic equation, when transformed to polar coordinates, will give this same result.

\section{Comparison Across Manifolds: Deeper Insights}

\subsection{Systematic Comparison}

\begin{table}[htbp]
\centering
\begin{tabular}{lcccc}
\toprule
\textbf{Manifold} & \textbf{Curvature} & \textbf{Geodesic Type} & \textbf{Uniqueness} & \textbf{Completeness} \\
\midrule
1D Line & Zero & Straight line & Always unique & Complete \\
1D Circle & Positive & Arc & Unique (except antipodal) & Complete \\
2D Plane & Zero & Straight line & Always unique & Complete \\
2D Sphere & Positive & Great circle arc & Unique shortest & Complete \\
\bottomrule
\end{tabular}
\caption{Comparison of geodesic properties across different manifolds.}
\label{tab:geodesic-comparison}
\end{table}

\subsection{Unifying Principles}

Despite their differences, all geodesics share fundamental properties:

\begin{enumerate}
\item \textbf{Minimization}: They minimize path length (or are critical points of the length functional).

\item \textbf{Differential equation}: They satisfy the geodesic equation $\ddot{x}^i + \Gamma^i_{jk}\dot{x}^j\dot{x}^k = 0$.

\item \textbf{Local straightness}: They have zero geodesic curvature—they're "as straight as possible" on the manifold.

\item \textbf{Coordinate independence}: They are geometric objects, independent of coordinate choices.

\item \textbf{Intrinsic nature}: They depend only on the metric (intrinsic geometry), not on embeddings.
\end{enumerate}

These unifying principles allow us to study geodesics on any manifold, not just the simple ones we've explored here.

\section{Key Takeaways}

\begin{itemize}
\item \textbf{Systematic derivation}: Geodesics can be derived using variational calculus (minimizing path length) or by solving the geodesic differential equation.

\item \textbf{Variational approach}: The Euler-Lagrange equations applied to the path length functional yield the geodesic equation.

\item \textbf{Geodesic equation}: $\ddot{x}^i + \Gamma^i_{jk}\dot{x}^j\dot{x}^k = 0$ describes geodesics, where Christoffel symbols $\Gamma^i_{jk}$ encode the geometry.

\item \textbf{Geometric construction}: On spheres, great circles can be constructed as intersections with planes through the center—often simpler than solving equations.

\item \textbf{Multiple geodesics}: Some manifolds (circle, sphere) have multiple geodesics between certain point pairs (antipodal points).

\item \textbf{Coordinate independence}: Geodesics are geometric objects—their descriptions change with coordinates, but the geodesics themselves are invariant.

\item \textbf{Unifying principles}: All geodesics minimize length, satisfy the geodesic equation, have zero geodesic curvature, and are intrinsic to the manifold.
\end{itemize}

\section{What's Next?}

The methods we've developed here extend to more complex manifolds:
\begin{itemize}
\item \textbf{Torus}: Geodesics can wrap around in multiple ways, leading to rich behavior.
\item \textbf{General surfaces}: Numerical methods are often needed to find geodesics.
\item \textbf{Riemannian geometry}: The general theory provides tools for any manifold with a metric.
\item \textbf{Applications}: Geodesics appear in general relativity (spacetime paths), robotics (path planning), and computer graphics (surface rendering).
\end{itemize}

Understanding geodesics on simple manifolds provides the foundation for these advanced topics.
