\chapter{Summary Table}

A comprehensive summary table comparing key properties, formulas, and characteristics across different manifolds and concepts explored in this book.

\section{Manifold Comparison}

\begin{table}[htbp]
\centering
\small
\begin{tabular}{lcccc}
\toprule
\textbf{Manifold} & \textbf{Dimension} & \textbf{Curvature} & \textbf{Type} & \textbf{Key Property} \\
\midrule
1D Line & 1 & Zero & Open & Infinitely extendable \\
1D Circle & 1 & Positive & Closed & Loops back on itself \\
2D Plane & 2 & Zero & Open & Flat everywhere \\
2D Sphere & 2 & Positive & Closed & Constant positive curvature \\
2D Torus & 2 & Mixed & Closed & Positive and negative regions \\
\bottomrule
\end{tabular}
\caption{Comparison of basic manifold properties.}
\label{tab:manifold-properties}
\end{table}

\section{Distance Formulas}

\begin{table}[htbp]
\centering
\small
\begin{tabular}{lc}
\toprule
\textbf{Manifold} & \textbf{Distance Formula} \\
\midrule
1D Line & $d = |x_2 - x_1|$ \\
1D Circle & $d = r \cdot \theta$ \\
2D Plane & $d = \sqrt{(x_2-x_1)^2 + (y_2-y_1)^2}$ \\
2D Sphere & $d = R \cdot \arccos(\vec{v}_1 \cdot \vec{v}_2)$ \\
2D Sphere (coords) & $d = R \cdot \arccos(\sin\phi_1\sin\phi_2 + \cos\phi_1\cos\phi_2\cos(\Delta\lambda))$ \\
2D Sphere (haversine) & $d = 2R \cdot \arcsin\left(\sqrt{\sin^2(\Delta\phi/2) + \cos\phi_1\cos\phi_2\sin^2(\Delta\lambda/2)}\right)$ \\
2D Torus & $d \approx \sqrt{(R + r\cos\theta)^2(\Delta\phi)^2 + r^2(\Delta\theta)^2}$ \\
\bottomrule
\end{tabular}
\caption{Distance formulas for different manifolds.}
\label{tab:distance-formulas}
\end{table}

\section{Geodesic Properties}

\begin{table}[!htbp]
\centering
\small
\begin{tabular}{lcccc}
\toprule
\textbf{Manifold} & \textbf{Geodesic Type} & \textbf{Uniqueness} & \textbf{Completeness} & \textbf{Construction} \\
\midrule
1D Line & Straight line & Always unique & Complete & Direct \\
1D Circle & Arc & Unique (except antipodal) & Complete & Direct \\
2D Plane & Straight line & Always unique & Complete & Direct \\
2D Sphere & Great circle arc & Unique shortest & Complete & Plane intersection \\
2D Torus & Complex paths & Multiple possible & Complete & Numerical methods \\
\bottomrule
\end{tabular}
\caption{Comparison of geodesic properties across different manifolds.}
\label{tab:geodesic-properties}
\end{table}

\section{Geodesic Equations and Methods}

\begin{table}[htbp]
\centering
\small
\begin{tabular}{lcc}
\toprule
\textbf{Manifold} & \textbf{Geodesic Equation} & \textbf{Solution Method} \\
\midrule
1D Line & $\ddot{x} = 0$ & Direct integration \\
1D Circle & $\ddot{\theta} = 0$ & Direct integration \\
2D Plane & $\ddot{x} = 0$, $\ddot{y} = 0$ & Direct integration \\
2D Sphere & $\ddot{\phi} - \sin\phi\cos\phi\dot{\lambda}^2 = 0$ & Geometric construction \\
& $\ddot{\lambda} + 2\cot\phi\dot{\phi}\dot{\lambda} = 0$ & or numerical methods \\
\bottomrule
\end{tabular}
\caption{Geodesic equations and solution methods.}
\label{tab:geodesic-equations}
\end{table}

\section{Open n-Ball Properties}

\begin{table}[htbp]
\centering
\small
\begin{tabular}{lcc}
\toprule
\textbf{Dimension} & \textbf{Name} & \textbf{Definition} \\
\midrule
$n=1$ & Open interval & $B_r(p) = \{x : |x-p| < r\}$ \\
$n=2$ & Open disk & $B_r(\mathbf{p}) = \{(x,y) : (x-p_x)^2 + (y-p_y)^2 < r^2\}$ \\
$n=3$ & Open ball & $B_r(\mathbf{p}) = \{(x,y,z) : (x-p_x)^2 + (y-p_y)^2 + (z-p_z)^2 < r^2\}$ \\
$n$ & Open n-ball & $B_r(\mathbf{p}) = \{\mathbf{x} : \|\mathbf{x}-\mathbf{p}\| < r\}$ \\
\bottomrule
\end{tabular}
\caption{Open n-balls in different dimensions.}
\label{tab:open-n-balls}
\end{table}

\section{Generative AI Models and Manifolds}

\begin{table}[!htbp]
\centering
\small
\begin{tabular}{lcc}
\toprule
\textbf{Model} & \textbf{Manifold Connection} & \textbf{Key Component} \\
\midrule
VAE & Encoder/decoder as charts & Maps data $\leftrightarrow$ latent space \\
GAN & Generator defines manifold & $\mathcal{M}_G = \{G(\mathbf{z}) : \mathbf{z} \in \mathcal{Z}\}$ \\
Diffusion & Forward/reverse on manifold & Data $\to$ noise $\to$ data \\
Normalizing Flow & Manifold-to-manifold maps & Preserves volume/structure \\
\bottomrule
\end{tabular}
\caption{How generative models relate to manifolds.}
\label{tab:ai-manifolds}
\end{table}

\section{Curvature Effects}

\begin{table}[!htbp]
\centering
\small
\begin{tabular}{lccc}
\toprule
\textbf{Curvature Type} & \textbf{Geodesic Behavior} & \textbf{Example} & \textbf{Visual} \\
\midrule
Zero & Parallel geodesics stay parallel & Plane, line & Parallel lines \\
Positive & Geodesics converge & Sphere & Meridians meet at poles \\
Negative & Geodesics diverge & Saddle & Hyperbolic geometry \\
Mixed & Varies by region & Torus & Combination \\
\bottomrule
\end{tabular}
\caption{How curvature affects geodesic behavior.}
\label{tab:curvature-effects}
\end{table}

\section{Special Cases: Sphere Distance}

\begin{table}[!htbp]
\centering
\small
\begin{tabular}{lc}
\toprule
\textbf{Special Case} & \textbf{Simplified Formula} \\
\midrule
Same meridian & $d = R \cdot |\phi_2 - \phi_1|$ \\
Same parallel & $d = R \cdot \cos\phi \cdot |\Delta\lambda|$ \\
Near poles & Use haversine formula \\
Antipodal points & $d = \pi R$ (half circumference) \\
Very small distance & Flat Earth: $d \approx R \sqrt{(\Delta\phi)^2 + (\cos\bar{\phi})^2(\Delta\lambda)^2}$ \\
\bottomrule
\end{tabular}
\caption{Special cases for sphere distance calculations.}
\label{tab:sphere-special-cases}
\end{table}

\section{Unifying Principles}

All geodesics share these fundamental properties:
\begin{enumerate}
\item \textbf{Minimization}: They minimize path length (or are critical points of the length functional)
\item \textbf{Differential equation}: They satisfy $\ddot{x}^i + \Gamma^i_{jk}\dot{x}^j\dot{x}^k = 0$
\item \textbf{Local straightness}: Zero geodesic curvature—"as straight as possible" on the manifold
\item \textbf{Coordinate independence}: Geometric objects, independent of coordinate choices
\item \textbf{Intrinsic nature}: Depend only on the metric (intrinsic geometry), not on embeddings
\end{enumerate}

\section{Key Relationships}

\begin{itemize}
\item \textbf{Distance = Geodesic Length}: The distance between two points equals the length of the geodesic connecting them
\item \textbf{Manifold = Locally Open n-Ball}: Every point has a neighborhood homeomorphic to an open n-ball
\item \textbf{Chart = Local Coordinate System}: Charts provide mappings from manifolds to flat spaces
\item \textbf{Geodesic = Minimized Path Length}: Geodesics minimize the path length functional
\item \textbf{Curvature = Geodesic Convergence}: Positive curvature causes convergence, negative causes divergence
\end{itemize}
