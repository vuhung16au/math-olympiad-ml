\begin{problem}[Exponential Bounds on Factorials]
\begin{enumerate}[label=(\roman*)]
    \item Prove that $x > \ln x$, for $x > 0$.
    \item Using part (i), or otherwise, prove that for all positive integers $n$,
    $$e^{n^2+n} > (n!)^2.$$
\end{enumerate}
\end{problem}

\begin{solution}
\textbf{Part (i):} Let $f(x) = x - \ln x$. Then $f'(x) = 1 - \frac{1}{x} = 0 \implies x = 1$.

Since $f''(x) = \frac{1}{x^2} > 0$, $f$ has a minimum at $x=1$ with $f(1) = 1 - 0 = 1 > 0$.

Therefore, $x > \ln x$ for all $x > 0$.

\textbf{Part (ii):} Apply $\ln$ to both sides: $e^{n^2+n} > (n!)^2 \iff n^2+n > 2\ln(n!) \iff \frac{n^2+n}{2} > \sum_{k=1}^{n} \ln k$

From part (i), $k > \ln k$ for all positive integers $k$. Summing from $k=1$ to $n$:
\[
\sum_{k=1}^{n} k > \sum_{k=1}^{n} \ln k \implies \frac{n(n+1)}{2} = \frac{n^2+n}{2} > \sum_{k=1}^{n} \ln k
\]
Exponentiating: $e^{n^2+n} > (n!)^2$
\end{solution}

\begin{takeaways}

    \item \textbf{Calculus technique:} Use first and second derivatives to find and classify critical points, then evaluate the function at the critical point to determine global behavior.
    \item \textbf{Summation strategy:} Apply a single inequality to multiple values, then sum all inequalities together to obtain a cumulative result.
    \item \textbf{Logarithmic transformation:} Convert multiplicative inequalities to additive ones using logarithms, which simplifies the analysis.
    \item \textbf{Building block approach:} Use the result from a simpler part to prove a more complex statement in subsequent parts.
    \item \textbf{Common pitfall:} Don't forget to verify that the critical point is indeed a minimum (not a maximum or inflection point) by checking the second derivative or analyzing the sign of the first derivative around the critical point.
    \item You can prove a stronger result: $$\left( \frac{n+1}{2} \right)^{2n} > (n!)^2$$ by applying AM-GM to $1, 2, \ldots, n$.

\end{takeaways}

\begin{problem}[Sphere Inequalities via Vector Methods]
The point $P(x, y, z)$ lies on the sphere of radius $1$ centred at the origin $O$.
\begin{enumerate}[label=(\roman*)]
    \item Using the position vector of $P$, $\vec{OP} = x\mathbf{i} + y\mathbf{j} + z\mathbf{k}$, and the triangle inequality, or otherwise, show that $|x| + |y| + |z| \ge 1$.
    
    \item Given the vectors $\mathbf{a} = \begin{pmatrix} a_1 \\ a_2 \\ a_3 \end{pmatrix}$ and $\mathbf{b} = \begin{pmatrix} b_1 \\ b_2 \\ b_3 \end{pmatrix}$, show that
    \[
    |a_1b_1 + a_2b_2 + a_3b_3| \le \sqrt{a_1^2 + a_2^2 + a_3^2} \sqrt{b_1^2 + b_2^2 + b_3^2}.
    \]
    
    \item Using part (ii), or otherwise, show that $|x| + |y| + |z| \le \sqrt{3}$.
\end{enumerate}
\end{problem}

\begin{solution}
\textbf{Part (i): Show that $|x| + |y| + |z| \ge 1$}

Since $P(x, y, z)$ lies on the sphere of radius 1 centred at the origin, the magnitude of the position vector $\vec{OP}$ is 1:
\[
|\vec{OP}| = |x\mathbf{i} + y\mathbf{j} + z\mathbf{k}| = 1
\]

Applying the Triangle Inequality for vectors $|\mathbf{u} + \mathbf{v}| \le |\mathbf{u}| + |\mathbf{v}|$ repeatedly:
\begin{align*}
    |x\mathbf{i} + y\mathbf{j} + z\mathbf{k}| &\le |x\mathbf{i}| + |y\mathbf{j}| + |z\mathbf{k}| \\
    1 &\le |x||\mathbf{i}| + |y||\mathbf{j}| + |z||\mathbf{k}|
\end{align*}

Since the unit vectors have magnitude 1: $|\mathbf{i}| = |\mathbf{j}| = |\mathbf{k}| = 1$:
\[
1 \le |x| + |y| + |z|
\]

Therefore, $|x| + |y| + |z| \ge 1$.

\medskip
\textbf{Part (ii): Prove the Cauchy-Schwarz inequality}

We use the definition of the scalar (dot) product:
\[
\mathbf{a} \cdot \mathbf{b} = |\mathbf{a}| |\mathbf{b}| \cos \theta
\]
where $\theta$ is the angle between the vectors.

Taking the absolute value of both sides:
\begin{align*}
    |\mathbf{a} \cdot \mathbf{b}| &= ||\mathbf{a}| |\mathbf{b}| \cos \theta| \\
    |\mathbf{a} \cdot \mathbf{b}| &= |\mathbf{a}| |\mathbf{b}| |\cos \theta|
\end{align*}

Since $-1 \le \cos \theta \le 1$, we know that $|\cos \theta| \le 1$.

Therefore:
\[
|\mathbf{a} \cdot \mathbf{b}| \le |\mathbf{a}| |\mathbf{b}|
\]

Substituting the component forms of the vectors:
\[
|a_1b_1 + a_2b_2 + a_3b_3| \le \sqrt{a_1^2 + a_2^2 + a_3^2} \sqrt{b_1^2 + b_2^2 + b_3^2}
\]

\medskip
\textbf{Part (iii): Show that $|x| + |y| + |z| \le \sqrt{3}$}

We define two specific vectors to apply the Cauchy-Schwarz inequality from Part (ii):

Let $\mathbf{a} = \begin{pmatrix} |x| \\ |y| \\ |z| \end{pmatrix}$ and $\mathbf{b} = \begin{pmatrix} 1 \\ 1 \\ 1 \end{pmatrix}$.

Applying the result from (ii):
\begin{align*}
    | (|x|)(1) + (|y|)(1) + (|z|)(1) | &\le \sqrt{|x|^2 + |y|^2 + |z|^2} \sqrt{1^2 + 1^2 + 1^2} \\
    |x| + |y| + |z| &\le \sqrt{x^2 + y^2 + z^2} \cdot \sqrt{3}
\end{align*}

From the problem statement, $P$ is on the unit sphere, so $x^2 + y^2 + z^2 = 1$.

Therefore:
\begin{align*}
    |x| + |y| + |z| &\le \sqrt{1} \cdot \sqrt{3} \\
    |x| + |y| + |z| &\le \sqrt{3}
\end{align*}
\end{solution}

\begin{takeaways}

    \item \textbf{Triangle inequality:} For vectors, $|\mathbf{u} + \mathbf{v}| \le |\mathbf{u}| + |\mathbf{v}|$ provides lower bounds on sums of absolute values.
    \item \textbf{Cauchy-Schwarz application:} This fundamental inequality relates dot products to vector magnitudes and provides upper bounds on sums.
    \item \textbf{Strategic vector choice:} Choose specific vectors (like the all-ones vector) to convert Cauchy-Schwarz into the desired form.
    \item \textbf{Multi-part coordination:} Each part builds toward the final result; part (i) establishes a lower bound, part (ii) proves a general tool, and part (iii) applies it for an upper bound.
    \item \textbf{Common pitfall:} Remember that $|x|^2 = x^2$, so the constraint $x^2 + y^2 + z^2 = 1$ directly gives $\sqrt{|x|^2 + |y|^2 + |z|^2} = 1$.

\end{takeaways}

\begin{problem}[Logarithmic Inequalities and the Limit Definition of $e$]
Suppose that $x \geq 0$ and $n$ is a positive integer.
\begin{enumerate}[label=(\roman*)]
    \item Show that
    \[
    1-x \leq \frac{1}{1+x} \leq 1.
    \]

    \item Hence, or otherwise, show that
    \[
    1-\frac{1}{2n} \leq n\ln \left(1+\frac{1}{n}\right) \leq 1.
    \]

    \item Hence, explain why
    \[
    \lim_{n\to\infty} \left(1+\frac{1}{n}\right)^n = e.
    \]
\end{enumerate}
\end{problem}

\begin{solution}
\textbf{Part (i):} \textbf{Right:} Since $x \geq 0$: $1+x \geq 1 \implies \frac{1}{1+x} \leq 1$.

\textbf{Left:} From $(1-x)(1+x) = 1-x^2 \leq 1$ and $1+x > 0$: $1-x \leq \frac{1}{1+x}$.

Thus: $1-x \leq \frac{1}{1+x} \leq 1$

\textbf{Part (ii):} Integrate the inequality from $0$ to $\frac{1}{n}$:
\[
\int_{0}^{1/n} (1-t) \, dt \leq \int_{0}^{1/n} \frac{1}{1+t} \, dt \leq \int_{0}^{1/n} 1 \, dt
\]
Evaluating: $\frac{1}{n} - \frac{1}{2n^2} \leq \ln\left(1+\frac{1}{n}\right) \leq \frac{1}{n}$

Multiply by $n$: $1 - \frac{1}{2n} \leq n\ln\left(1+\frac{1}{n}\right) \leq 1$

\textbf{Part (iii):} Taking limits: $\lim_{n\to\infty} \left( 1 - \frac{1}{2n} \right) = 1 = \lim_{n\to\infty} (1)$

By Squeeze Theorem: $\lim_{n\to\infty} n\ln\left(1+\frac{1}{n}\right) = 1 \implies \lim_{n\to\infty} \ln\left(1+\frac{1}{n}\right)^n = 1$

Exponentiating: $\lim_{n\to\infty} \left(1+\frac{1}{n}\right)^n = e$
\end{solution}

\begin{takeaways}

    \item \textbf{Integration of inequalities:} Integrating all parts of a valid inequality preserves the inequality relation and is a powerful technique for deriving new bounds.
    \item \textbf{Squeeze Theorem:} When a function is bounded above and below by functions that converge to the same limit, the middle function must also converge to that limit.
    \item \textbf{Logarithm-exponential interplay:} Use logarithms to convert powers to products, then exponentiate to recover the original form after taking limits.
    \item \textbf{Progressive refinement:} Each part provides a tool or bound that is then used in subsequent parts to build toward the final result.
    \item \textbf{Common pitfall:} When integrating, don't forget to evaluate the definite integral at both bounds and subtract correctly. Also, remember that $\ln\left(a^n\right) = n\ln(a)$ when moving between forms.

\end{takeaways}

\begin{problem}[Homogeneous Inequality via Substitution and AM-GM]
Prove
\[
\frac{x^2}{y^2} + \frac{y^2}{z^2} + \frac{z^2}{x^2} \geq \frac{1}{yz}\sqrt{x^2y^2 + z^4} + \frac{1}{xz}\sqrt{z^2y^2 + x^4} \quad \text{for } x, y, z > 0.
\]
\end{problem}

\begin{solution}
\textbf{Step 1:} Simplify RHS:
\[
\frac{1}{yz}\sqrt{x^2y^2 + z^4} = \sqrt{\frac{x^2}{z^2} + \frac{z^2}{y^2}}, \quad \frac{1}{xz}\sqrt{z^2y^2 + x^4} = \sqrt{\frac{y^2}{x^2} + \frac{x^2}{z^2}}
\]
Inequality becomes: $\frac{x^2}{y^2} + \frac{y^2}{z^2} + \frac{z^2}{x^2} \geq \sqrt{\frac{x^2}{z^2} + \frac{z^2}{y^2}} + \sqrt{\frac{y^2}{x^2} + \frac{x^2}{z^2}}$

\textbf{Step 2:} Let $u = \frac{x}{y}$, $v = \frac{y}{z}$, $w = \frac{z}{x}$ (note: $uvw = 1$). Then:
\[
L = u^2 + v^2 + w^2, \quad R = u\sqrt{v^2 + w^2} + v\sqrt{w^2 + u^2}
\]

\textbf{Step 3:} Apply AM-GM: $\sqrt{AB} \leq \frac{A+B}{2}$:
\[
u\sqrt{v^2 + w^2} \leq \frac{u^2 + (v^2 + w^2)}{2}, \quad v\sqrt{w^2 + u^2} \leq \frac{v^2 + (w^2 + u^2)}{2}
\]
Adding: $R \leq \frac{2(u^2 + v^2 + w^2)}{2} = u^2 + v^2 + w^2 = L$
\end{solution}

\begin{takeaways}
    \item Think of AM-GM inequalities when you see symetrical sums and products with equal weights.
    \item \textbf{Homogeneous substitution:} For homogeneous inequalities, substitute ratios of variables (like $u = x/y$) to reduce the number of variables and simplify the problem.
    \item \textbf{Algebraic simplification:} Move constants in and out of square roots systematically to reveal the underlying structure.
    \item \textbf{AM-GM strategy:} Use AM-GM on products under square roots: $\sqrt{AB} \leq \frac{A+B}{2}$ is particularly useful when the sum $A+B$ appears elsewhere.
    \item \textbf{Summation technique:} When applying AM-GM to multiple terms, add the resulting inequalities to obtain the final bound.
    \item \textbf{Common pitfall:} Verify that the substitution constraint (like $uvw = 1$) is satisfied; this ensures the substitution is valid and the problem hasn't been changed.

\end{takeaways}

\begin{problem}[Bernoulli's Inequality and Sequence Monotonicity]

\textbf{Bernoulli's Inequality:} For any real number $x > -1$ and integer $r \ge 0$,
\[(1 + x)^r \ge 1 + rx,
\]with strict inequality if $x \neq 0$ and $r > 1$. 
The equality holds if $x = 0$ or $r = 0$ or $r = 1$.
(do NOT prove this)

Using Bernoulli's Inequality, prove that the sequence $a_n = \left(1 + \frac{1}{n}\right)^n$ is strictly increasing for integers $n \ge 1$. Specifically, prove:
\[ \left(1 + \frac{1}{n}\right)^n < \left(1 + \frac{1}{n+1}\right)^{n+1} \]
\end{problem}

\begin{solution}
To prove the sequence is increasing, we show: $\frac{\left(1 + \frac{1}{n+1}\right)^{n+1}}{\left(1 + \frac{1}{n}\right)^n} > 1$

\textbf{Step 1:} Rewrite:
\[
\frac{\left(\frac{n+2}{n+1}\right)^{n+1}}{\left(\frac{n+1}{n}\right)^n} = \left( \frac{(n+2)n}{(n+1)^2} \right)^{n+1} \times \frac{n+1}{n}
\]
Simplify: $\frac{(n+2)n}{(n+1)^2} = \frac{n^2 + 2n}{n^2 + 2n + 1} = 1 - \frac{1}{(n+1)^2}$

\textbf{Step 2:} Apply Bernoulli's Inequality with $x = -\frac{1}{(n+1)^2}$ and $r = n+1$:

Since $x > -1$, $x \neq 0$, and $r > 1$:
\[
\left( 1 - \frac{1}{(n+1)^2} \right)^{n+1} > 1 - \frac{n+1}{(n+1)^2} = 1 - \frac{1}{n+1} = \frac{n}{n+1}
\]

\textbf{Step 3:} Therefore: Ratio $> \frac{n}{n+1} \times \frac{n+1}{n} = 1$

Thus $\left(1 + \frac{1}{n}\right)^n < \left(1 + \frac{1}{n+1}\right)^{n+1}$, proving strict monotonicity.
\end{solution}

\begin{takeaways}

    \item \textbf{Ratio test for monotonicity:} To prove $a_n < a_{n+1}$, show that $\frac{a_{n+1}}{a_n} > 1$. This often simplifies the algebra.
    \item \textbf{Bernoulli's Inequality application:} When you have $(1 + x)^r$ with small $x$ and large $r$, Bernoulli provides a useful linear lower bound.
    \item \textbf{Strategic algebraic manipulation:} Rewrite expressions to isolate a $(1 + x)^r$ term suitable for Bernoulli's Inequality.
    \item \textbf{Strict vs. non-strict inequalities:} Bernoulli's Inequality is strict when $x \neq 0$ and $r > 1$, which is crucial for proving strict monotonicity.
    \item \textbf{Common pitfall:} Verify that Bernoulli's Inequality applies: check that $x > -1$ and that the inequality is strict (not just $\geq$) when needed. Also, be careful with sign changes when $x$ is negative.

\end{takeaways}
