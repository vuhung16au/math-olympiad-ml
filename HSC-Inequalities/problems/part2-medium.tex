% ==========================================
% PART 2: MEDIUM PROBLEMS (Problems 24-33)
% ==========================================

\begin{problem}[Problem 24: AM-GM with Harmonic Constraint]
Let $a, b, c$ be positive real numbers such that $\frac{1}{a}+\frac{1}{b}+\frac{1}{c}=1$. 
It is known that for all positive real numbers $x, y$:
\[ x+y \geq 2\sqrt{xy} \]
Prove that:
\[ a\sqrt{bc} + b\sqrt{ac} + c\sqrt{ab} \leq abc \]
\end{problem}

\begin{hint}
Apply AM-GM then divide both sides by abc.
\end{hint}

\begin{solution}
Divide both sides by $abc$ (positive):
\begin{align*}
\frac{a\sqrt{bc}}{abc} + \frac{b\sqrt{ac}}{abc} + \frac{c\sqrt{ab}}{abc} &\leq 1 \\
\frac{1}{\sqrt{bc}} + \frac{1}{\sqrt{ac}} + \frac{1}{\sqrt{ab}} &\leq 1
\end{align*}

Apply AM-GM to pairs: $\frac{1}{\sqrt{xy}} \leq \frac{1}{2}\left(\frac{1}{x} + \frac{1}{y}\right)$

\begin{align*}
\frac{1}{\sqrt{bc}} &\leq \frac{1}{2}\left(\frac{1}{b} + \frac{1}{c}\right) \\
\frac{1}{\sqrt{ac}} &\leq \frac{1}{2}\left(\frac{1}{a} + \frac{1}{c}\right) \\
\frac{1}{\sqrt{ab}} &\leq \frac{1}{2}\left(\frac{1}{a} + \frac{1}{b}\right)
\end{align*}

Sum all three:
\begin{align*}
\frac{1}{\sqrt{bc}} + \frac{1}{\sqrt{ac}} + \frac{1}{\sqrt{ab}} &\leq \frac{1}{2} \cdot 2\left(\frac{1}{a} + \frac{1}{b} + \frac{1}{c}\right) \\
&= \frac{1}{a} + \frac{1}{b} + \frac{1}{c} = 1
\end{align*}
\end{solution}

\begin{takeaways}
\item Divide by positive terms to simplify before applying AM-GM
\item Use AM-GM on reciprocals: $\frac{1}{\sqrt{xy}} \leq \frac{1}{2}(\frac{1}{x} + \frac{1}{y})$
\end{takeaways}

% ==========================================

\begin{problem}[Problem 25: Binomial Inequality via Induction]
Use mathematical induction to prove that ${}^{2n}C_n < 2^{2n-2}$ for all integers $n \geq 5$.
\end{problem}

\begin{hint}
Use binomial formula, express k+1 case in terms of case k.
\end{hint}

\begin{solution}
\textbf{Base Case ($n=5$):}
\begin{align*}
{}^{10}C_5 &= \frac{10!}{5!5!} = 252 \\
2^{2(5)-2} &= 2^8 = 256
\end{align*}
Since $252 < 256$, base case holds.

\textbf{Inductive Hypothesis:} Assume ${}^{2k}C_k < 2^{2k-2}$ for some $k \geq 5$.

\textbf{Inductive Step:} Show ${}^{2(k+1)}C_{k+1} < 2^{2(k+1)-2}$, i.e., ${}^{2k+2}C_{k+1} < 2^{2k}$.

Express in terms of ${}^{2k}C_k$:
\begin{align*}
{}^{2k+2}C_{k+1} &= \frac{(2k+2)!}{(k+1)!(k+1)!} = \frac{2(2k+1)}{k+1} \cdot {}^{2k}C_k
\end{align*}

Since $\frac{2(2k+1)}{k+1} = 4 - \frac{2}{k+1} < 4$ for $k \geq 5$:
\begin{align*}
{}^{2k+2}C_{k+1} &< 4 \cdot {}^{2k}C_k < 4 \cdot 2^{2k-2} = 2^2 \cdot 2^{2k-2} = 2^{2k}
\end{align*}

By induction, the inequality holds for all $n \geq 5$.
\end{solution}

\begin{takeaways}
\item Express ${}^{2k+2}C_{k+1}$ as a multiple of ${}^{2k}C_k$ using binomial identities
\item Show the multiplier $\frac{2(2k+1)}{k+1} < 4$ when $k \geq 5$
\end{takeaways}

% ==========================================

\begin{problem}[Problem 26: Surface Area to Volume Optimization]
It is given that for positive numbers $x_1, x_2, x_3, \dots, x_n$ with arithmetic mean $A$:
\[
\frac{x_1 \times x_2 \times x_3 \times \dots \times x_n}{A^n} \le 1
\]
Suppose a rectangular prism has dimensions $a, b, c$ and surface area $S$.
\begin{enumerate}
    \item[(i)] Show that $abc \le \left(\frac{S}{6}\right)^{\frac{3}{2}}$.
    \item[(ii)] Using part (i), show that when the rectangular prism with surface area $S$ is a cube, it has maximum volume.
\end{enumerate}
\end{problem}

\begin{hint}
Let the numbers be the face areas bc, ca, ab.
\end{hint}

\begin{solution}
\textbf{Part (i):} Surface area $S = 2(ab + bc + ca)$. Let $x_1 = ab$, $x_2 = bc$, $x_3 = ca$.

Arithmetic mean:
\[ A = \frac{ab + bc + ca}{3} = \frac{S/2}{3} = \frac{S}{6} \]

Apply given inequality with $n=3$:
\begin{align*}
\frac{(ab)(bc)(ca)}{A^3} &\leq 1 \\
\frac{(abc)^2}{A^3} &\leq 1 \\
(abc)^2 &\leq A^3 = \left(\frac{S}{6}\right)^3
\end{align*}

Taking square root: $abc \leq \left(\frac{S}{6}\right)^{\frac{3}{2}}$.

\textbf{Part (ii):} Volume $V = abc \leq \left(\frac{S}{6}\right)^{\frac{3}{2}}$. 

Maximum occurs when equality holds, which requires $x_1 = x_2 = x_3$:
\[ ab = bc = ca \implies a = b = c \]

A rectangular prism with all equal dimensions is a cube.
\end{solution}

\begin{takeaways}
\item Apply AM-GM to face areas, not edge lengths
\item Equality in AM-GM occurs when all terms are equal ($ab = bc = ca \Rightarrow a = b = c$)
\end{takeaways}

\begin{remark}[Two Approaches to Proving AM-GM]
The AM-GM inequality states that for positive reals $x_1, x_2, \ldots, x_n$:
\[ \frac{x_1 + x_2 + \cdots + x_n}{n} \geq \sqrt[n]{x_1 x_2 \cdots x_n} \]

\textbf{(a) Proof by Induction on $n$:}
\begin{enumerate}
    \item \textit{Base case:} $n=2$ follows from $(\sqrt{x_1} - \sqrt{x_2})^2 \geq 0$
    \item \textit{Forward-backward step:} Prove $n=2^k \Rightarrow n=2^{k+1}$ by grouping pairs
    \item \textit{Backward step:} Show $n=k \Rightarrow n=k-1$ by setting $x_k = \frac{x_1 + \cdots + x_{k-1}}{k-1}$ and applying the $n=k$ case
\end{enumerate}

\textbf{(b) Proof using Convex Functions:}
\begin{enumerate}
    \item Consider $f(x) = -\ln(x)$, which is convex for $x > 0$ (since $f''(x) = \frac{1}{x^2} > 0$)
    \item By Jensen's inequality for convex functions:
    \[ f\left(\frac{x_1 + x_2 + \cdots + x_n}{n}\right) \leq \frac{f(x_1) + f(x_2) + \cdots + f(x_n)}{n} \]
    \item Substituting $f(x) = -\ln(x)$:
    \[ -\ln\left(\frac{x_1 + \cdots + x_n}{n}\right) \leq \frac{-\ln(x_1) - \cdots - \ln(x_n)}{n} = -\ln(\sqrt[n]{x_1 \cdots x_n}) \]
    \item Multiply by $-1$ and exponentiate to obtain AM-GM
\end{enumerate}
\end{remark}

% ==========================================

\begin{problem}[Problem 27: Cubic Sum Inequality]
Let $a, b > 0$. Prove that:
\[
a^3 + b^3 \geq \frac{(a + b)^3}{4}
\]
\end{problem}

\begin{hint}
Expand, cancel out, then factor into a product.
\end{hint}

\begin{solution}
Multiply both sides by 4:
\[ 4(a^3 + b^3) \geq (a + b)^3 \]

Expand RHS:
\[ 4a^3 + 4b^3 \geq a^3 + 3a^2b + 3ab^2 + b^3 \]

Rearrange:
\[ 3a^3 - 3a^2b - 3ab^2 + 3b^3 \geq 0 \]

Factor:
\begin{align*}
3(a^3 - a^2b - ab^2 + b^3) &\geq 0 \\
3[a^2(a - b) - b^2(a - b)] &\geq 0 \\
3(a - b)(a^2 - b^2) &\geq 0 \\
3(a - b)(a - b)(a + b) &\geq 0 \\
3(a - b)^2(a + b) &\geq 0
\end{align*}

Since $(a - b)^2 \geq 0$ and $(a + b) > 0$ for $a, b > 0$, the inequality holds.
\end{solution}

\begin{takeaways}
\item Clear denominators first, then expand and factor
\item Factor as $(a-b)^2(a+b) \geq 0$ where perfect square ensures non-negativity
\end{takeaways}

% ==========================================

\begin{problem}[Problem 28: Product of Sums via AM-GM]
Let $a, b, c > 0$. 
\begin{enumerate}
    \item[(a)] Prove that $a + b \ge 2\sqrt{ab}$.
    \item[(b)] Hence, or otherwise, show that $(a + b)(b + c)(a + c) \ge 8abc$.
\end{enumerate}
\end{problem}

\begin{hint}
Multiply three inequalities since all terms are positive.
\end{hint}

\begin{solution}
\textbf{Part (a):} Consider $(\sqrt{a} - \sqrt{b})^2 \geq 0$:
\begin{align*}
a - 2\sqrt{ab} + b &\geq 0 \\
a + b &\geq 2\sqrt{ab}
\end{align*}

\textbf{Part (b):} Apply part (a) to each pair:
\begin{align*}
a + b &\geq 2\sqrt{ab} \\
b + c &\geq 2\sqrt{bc} \\
a + c &\geq 2\sqrt{ac}
\end{align*}

Multiply all three inequalities (all terms positive):
\begin{align*}
(a + b)(b + c)(a + c) &\geq 2\sqrt{ab} \cdot 2\sqrt{bc} \cdot 2\sqrt{ac} \\
&= 8\sqrt{ab \cdot bc \cdot ac} \\
&= 8\sqrt{a^2b^2c^2} = 8abc
\end{align*}
\end{solution}

\begin{takeaways}
\item AM-GM for two variables: $(x-y)^2 \geq 0 \Rightarrow x + y \geq 2\sqrt{xy}$
\item Can multiply inequalities when all terms are positive
\end{takeaways}

% ==========================================

\begin{problem}[Problem 29: Nested Inequalities]
Let $a, b, c > 0$.
\begin{enumerate}
    \item[(i)] Show that $\frac{a}{b} + \frac{b}{a} \ge 2$.
    \item[(ii)] Show that $(a+b)\left(\frac{1}{a} + \frac{1}{b}\right) \ge 4$.
    \item[(iii)] Hence, or otherwise, show that $(a+b+c)\left(\frac{1}{a} + \frac{1}{b} + \frac{1}{c}\right) \ge 9$.
\end{enumerate}
\end{problem}

\begin{hint}
Expand, apply AM-GM to each part.
\end{hint}

\begin{solution}
\textbf{Part (i):} Apply AM-GM with $x = \frac{a}{b}$, $y = \frac{b}{a}$:
\[ \frac{a}{b} + \frac{b}{a} \geq 2\sqrt{\frac{a}{b} \cdot \frac{b}{a}} = 2 \]

\textbf{Part (ii):} Expand:
\[ (a+b)\left(\frac{1}{a} + \frac{1}{b}\right) = 1 + \frac{b}{a} + \frac{a}{b} + 1 = 2 + \left(\frac{a}{b} + \frac{b}{a}\right) \]
By part (i): $\geq 2 + 2 = 4$.

\textbf{Part (iii):} Expand:
\begin{align*}
(a+b+c)\left(\frac{1}{a} + \frac{1}{b} + \frac{1}{c}\right) &= 3 + \frac{a}{b} + \frac{b}{a} + \frac{b}{c} + \frac{c}{b} + \frac{a}{c} + \frac{c}{a}
\end{align*}
Apply part (i) to each pair:
\[ \geq 3 + 2 + 2 + 2 = 9 \]
\end{solution}

\begin{takeaways}
\item Expand products before applying AM-GM to identify reciprocal pairs
\item Each pair $\frac{x}{y} + \frac{y}{x} \geq 2$ contributes 2 to the bound
\end{takeaways}

% ==========================================

\begin{problem}[Problem 30: Cauchy-Schwarz Bound]
Let $x, y$ be real numbers such that $x^2 + y^2 \neq 0$. Prove that:
\[
\frac{(x+y)^2}{x^2 + y^2} \le 2
\]
\end{problem}

\begin{hint}
Start with $(x-y)^2 \ge 0$.
\end{hint}

\begin{solution}
Start with $(x - y)^2 \geq 0$:
\begin{align*}
x^2 - 2xy + y^2 &\geq 0 \\
x^2 + y^2 &\geq 2xy
\end{align*}

Add $x^2 + y^2$ to both sides:
\[ 2(x^2 + y^2) \geq x^2 + 2xy + y^2 = (x + y)^2 \]

Divide by $x^2 + y^2 > 0$:
\[ 2 \geq \frac{(x + y)^2}{x^2 + y^2} \]
\end{solution}

\begin{takeaways}
\item Use $(x-y)^2 \geq 0$ to establish $x^2 + y^2 \geq 2xy$
\item Add equal term to both sides to create perfect square on RHS
\end{takeaways}

% ==========================================

\begin{problem}[Problem 31: Cauchy-Schwarz with Constraint]
Let $x, y, z$ be real numbers such that $x^2 + y^2 + z^2 = 25$. Prove that:
\[
3x + 4y + 5z \le 25\sqrt{2}
\]
\end{problem}

\begin{hint}
Apply Cauchy-Schwarz to sequences (x,y,z) and (3,4,5).
\end{hint}

\begin{solution}
Apply Cauchy-Schwarz inequality to $(x, y, z)$ and $(3, 4, 5)$:
\[ (x^2 + y^2 + z^2)(3^2 + 4^2 + 5^2) \geq (3x + 4y + 5z)^2 \]

Substitute $x^2 + y^2 + z^2 = 25$:
\begin{align*}
(25)(9 + 16 + 25) &\geq (3x + 4y + 5z)^2 \\
(25)(50) &\geq (3x + 4y + 5z)^2 \\
1250 &\geq (3x + 4y + 5z)^2
\end{align*}

Take square root:
\[ \sqrt{1250} = \sqrt{625 \times 2} = 25\sqrt{2} \geq 3x + 4y + 5z \]
\end{solution}

\begin{takeaways}
\item Cauchy-Schwarz: $(a_1^2 + a_2^2 + a_3^2)(b_1^2 + b_2^2 + b_3^2) \geq (a_1b_1 + a_2b_2 + a_3b_3)^2$
\item Choose coefficient sequence to match linear form on RHS
\end{takeaways}

% ==========================================

\begin{problem}[Problem 32: Bernoulli's Inequality Application]
Without using a calculator, apply Bernoulli's Inequality to prove:
\[ (1.005)^{200} > 2 \]
\end{problem}

\begin{hint}
Write 1.005 as 1.00x, x = 0.005.
\end{hint}

\begin{solution}
Express $1.005 = 1 + 0.005$. Let $x = 0.005$ and $n = 200$.

Check conditions: $x > -1$ and $n$ is a positive integer, so Bernoulli's Inequality applies:
\[ (1 + x)^n \geq 1 + nx \]

Since $n > 1$ and $x \neq 0$, the inequality is strict:
\begin{align*}
(1 + 0.005)^{200} &> 1 + 200(0.005) \\
(1.005)^{200} &> 1 + 1 \\
(1.005)^{200} &> 2
\end{align*}
\end{solution}

\begin{takeaways}
\item Bernoulli: $(1+x)^n \geq 1 + nx$ for $x > -1$ and $n \in \mathbb{Z}^+$
\item Strict inequality when $n > 1$ and $x \neq 0$
\end{takeaways}

% ==========================================

\begin{problem}[Problem 33: Exponential Inequality via Induction]
\begin{enumerate}
    \item[(i)] By considering $f'(x)$ where $f(x) = e^x - x$, show that $e^x > x$ for $x \ge 0$.
    \item[(ii)] Hence, use Mathematical Induction to show that for $x \ge 0$, $e^x > \frac{x^n}{n!}$ for all positive integers $n \ge 1$.
\end{enumerate}
\end{problem}

\begin{hint}
The derivative of P(n+1) comes directly from P(n).
\end{hint}

\begin{solution}
\textbf{Part (i):} Let $f(x) = e^x - x$. Then $f'(x) = e^x - 1 > 0$ for $x > 0$.

Since $f(0) = 1 > 0$ and $f$ is increasing for $x \geq 0$, we have $f(x) \geq 1 > 0$, so $e^x > x$.

\textbf{Part (ii):} 
\textit{Base Case ($n=1$):} From part (i), $e^x > x = \frac{x^1}{1!}$.

\textit{Inductive Hypothesis:} Assume $e^x > \frac{x^k}{k!}$ for some $k \geq 1$.

\textit{Inductive Step:} Let $g(x) = e^x - \frac{x^{k+1}}{(k+1)!}$. Then:
\[ g'(x) = e^x - \frac{x^k}{k!} > 0 \]
by the inductive hypothesis.

Since $g(0) = 1 > 0$ and $g$ is increasing, $g(x) > 0$ for $x \geq 0$, so:
\[ e^x > \frac{x^{k+1}}{(k+1)!} \]

By induction, $e^x > \frac{x^n}{n!}$ for all $n \geq 1$ and $x \geq 0$.
\end{solution}

\begin{takeaways}
\item Use derivative to show function is increasing, combined with initial value
\item In induction step, derivative of $g(x)$ involves inductive hypothesis directly
\end{takeaways}
