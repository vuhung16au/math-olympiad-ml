\begin{problem}[Arithmetic Sequence of Reciprocals]
Positive real numbers $a, b, c$ and $d$ are chosen such that $\frac{1}{a}, \frac{1}{b}, \frac{1}{c}$ and $\frac{1}{d}$ are consecutive terms in an arithmetic sequence with common difference $k$, where $k \in \mathbb{R}$, $k > 0$.

Show that $b+c < a+d$.
\end{problem}

\begin{solution}
Since $\frac{1}{a}, \frac{1}{b}, \frac{1}{c}, \frac{1}{d}$ are consecutive terms in an arithmetic sequence with common difference $k > 0$, we have:
\begin{align*}
\frac{1}{b} &= \frac{1}{a} + k \quad \implies \quad k = \frac{1}{b} - \frac{1}{a} = \frac{a-b}{ab} \tag{1} \\
\frac{1}{c} &= \frac{1}{b} + k \quad \implies \quad k = \frac{1}{c} - \frac{1}{b} = \frac{b-c}{bc} \tag{2} \\
\frac{1}{d} &= \frac{1}{c} + k \quad \implies \quad k = \frac{1}{d} - \frac{1}{c} = \frac{c-d}{cd} \tag{3}
\end{align*}

Since $k > 0$ and $a, b, c, d$ are positive, the numerators must be positive, which implies:
\begin{align*}
&a-b > 0 \quad \implies \quad a > b \\
&b-c > 0 \quad \implies \quad b > c \\
&c-d > 0 \quad \implies \quad c > d
\end{align*}
Thus, $a > b > c > d$.

We want to show that $b+c < a+d$, which is equivalent to showing $0 < (a+d) - (b+c)$, or $0 < (a-b) - (c-d)$.

From equations (1) and (3), we can express the differences in terms of $k$:
\begin{align*}
a-b &= k \cdot ab \tag{4} \\
c-d &= k \cdot cd \tag{5}
\end{align*}

We need to compare $k \cdot ab$ and $k \cdot cd$. Since $k > 0$, the inequality $a-b > c-d$ is equivalent to showing:
\[ ab > cd \]

Since we established $a > b > c > d$, it is clear that $a > c$ and $b > d$. Since all are positive:
\begin{align*} 
a &> c > 0 \\ 
b &> d > 0 
\end{align*}

Multiplying these two inequalities:
\[ ab > cd \]

Multiplying by $k>0$:
\[ k \cdot ab > k \cdot cd \]

Substituting from equations (4) and (5):
\[ a-b > c-d \]

Rearranging the terms:
\[ a+d > b+c \]

Therefore, $b+c < a+d$ as required.
\end{solution}

\begin{takeaways}

\item \textbf{Technique:} Converting arithmetic sequence conditions into algebraic equations allows us to extract ordering information about the original terms.
\item \textbf{Strategy:} When dealing with reciprocals in arithmetic progression, recognize that the original terms form a decreasing sequence, and use this monotonicity to compare products.
\item \textbf{Key Insight:} Express target differences as products with a common positive factor ($k$), reducing the problem to comparing products of ordered terms.
\item \textbf{Pitfall:} Don't forget that $k > 0$ is crucial for establishing the ordering $a > b > c > d$; without this, the inequality direction could reverse.

\end{takeaways}

\begin{problem}[Cascading AM-GM Applications]
For all non-negative real numbers $x$ and $y$, $\sqrt{xy} \le \frac{x+y}{2}.$ (Do NOT prove this.)

\begin{enumerate}
    \item[(i)] Using this fact, show that for all non-negative real numbers $a, b$ and $c$,
    \[
    \sqrt{abc} \le \frac{a^2+b^2+2c}{4}.
    \]

    \item[(ii)] Using part (i), or otherwise, show that for all non-negative real numbers $a, b$ and $c$,
    \[
    \sqrt{abc} \le \frac{a^2+b^2+c^2+a+b+c}{6}.
    \]
\end{enumerate}
\end{problem}

\begin{solution}
\textbf{Part (i):} Apply AM-GM to $a^2, b^2$: $ab \le \frac{a^2+b^2}{2}$. Apply AM-GM to $ab, c$:
\[
\sqrt{abc} \le \frac{ab+c}{2} \le \frac{\frac{a^2+b^2}{2} + c}{2} = \frac{a^2+b^2+2c}{4}
\]

\textbf{Part (ii):} By cyclic permutation of (i):
\[
\sqrt{abc} \le \frac{a^2+b^2+2c}{4}, \quad \sqrt{abc} \le \frac{b^2+c^2+2a}{4}, \quad \sqrt{abc} \le \frac{c^2+a^2+2b}{4}
\]
Adding: $3\sqrt{abc} \le \frac{2(a^2 + b^2 + c^2 + a + b + c)}{4} = \frac{a^2 + b^2 + c^2 + a + b + c}{2}$.

Thus: $\sqrt{abc} \le \frac{a^2 + b^2 + c^2 + a + b + c}{6}$
\end{solution}

\begin{takeaways}

\item \textbf{Technique:} Cascade AM-GM applications by strategically choosing pairs of terms, then use the resulting inequality as input for another AM-GM application.
\item \textbf{Strategy:} When proving symmetric inequalities, exploit cyclic symmetry by generating multiple versions of an intermediate result and summing them.
\item \textbf{Key Insight:} The transition from part (i) to part (ii) demonstrates how adding symmetric inequalities can yield a stronger bound with better balance among variables.
\item \textbf{Common Pattern:} Notice that $\frac{a^2+b^2}{2} \ge ab$ is a specific application of AM-GM to squares, which often serves as a useful intermediate step.

\end{takeaways}

\begin{problem}[Inductive Sum of Squared Reciprocals]
Prove by mathematical induction that, for $n \ge 2$,
\[
\frac{1}{2^2} + \frac{1}{3^2} + \dots + \frac{1}{n^2} < \frac{n-1}{n}.
\]
\end{problem}

\begin{solution}
\textbf{Base Case ($n=2$):} LHS $= \frac{1}{4}$, RHS $= \frac{1}{2}$. Since $\frac{1}{4} < \frac{1}{2}$, true for $n=2$.

\textbf{Inductive Hypothesis:} Assume $\frac{1}{2^2} + \frac{1}{3^2} + \dots + \frac{1}{k^2} < \frac{k-1}{k}$ (*)

\textbf{Inductive Step:} Need to show $\frac{1}{2^2} + \dots + \frac{1}{k^2} + \frac{1}{(k+1)^2} < \frac{k}{k+1}$.

Using (*): LHS $< \frac{k-1}{k} + \frac{1}{(k+1)^2}$. 

The gap is: $\frac{k}{k+1} - \frac{k-1}{k} = \frac{1}{k(k+1)}$

Need: $\frac{1}{(k+1)^2} < \frac{1}{k(k+1)} \iff k(k+1) < (k+1)^2 \iff k < k+1$.

Thus: $\frac{k-1}{k} + \frac{1}{(k+1)^2} < \frac{k}{k+1}$

By induction, the inequality holds for all $n \ge 2$.
\end{solution}

\begin{takeaways}

\item \textbf{Technique:} In inductive proofs of inequalities, compute the ``gap'' between successive right-hand sides to determine what bound is needed on the new term.
\item \textbf{Strategy:} Show that the additional term $\frac{1}{(k+1)^2}$ is strictly less than the increase in the RHS from $\frac{k-1}{k}$ to $\frac{k}{k+1}$.
\item \textbf{Key Insight:} The inequality $k(k+1) < (k+1)^2$ (equivalently $k < k+1$) is the critical comparison that makes the inductive step work.
\item \textbf{Pitfall:} Don't assume the new term is small enough without verification; always explicitly show the required inequality between the new term and the gap in the RHS.

\end{takeaways}

\begin{problem}[Power Mean Inequality via QM-RMS]
For positive real numbers $x$ and $y$, $\sqrt{xy} \leq \frac{x+y}{2}$. (Do NOT prove this.)

\begin{enumerate}[label=(\roman*)]
    \item Prove $\sqrt{xy} \leq \sqrt{\frac{x^2 + y^2}{2}}$, for positive real numbers $x$ and $y$.
    \item Prove $\sqrt[4]{abcd} \leq \sqrt{\frac{a^2 + b^2 + c^2 + d^2}{4}}$, for positive real numbers $a, b, c$ and $d$.
\end{enumerate}
\end{problem}

\begin{solution}
\textbf{Part (i):} From $(x-y)^2 \geq 0$: $x^2 + y^2 \geq 2xy \implies \frac{x^2 + y^2}{2} \geq xy$. 

Taking square roots: $\sqrt{xy} \leq \sqrt{\frac{x^2 + y^2}{2}}$

\textbf{Part (ii):} Note $\sqrt[4]{abcd} = \sqrt{\sqrt{ab} \cdot \sqrt{cd}}$. Apply Part (i) to pairs $(a,b)$ and $(c,d)$:
\[
\sqrt{ab} \leq \sqrt{\frac{a^2+b^2}{2}}, \quad \sqrt{cd} \leq \sqrt{\frac{c^2+d^2}{2}}
\]
Let $X = \sqrt{\frac{a^2+b^2}{2}}$, $Y = \sqrt{\frac{c^2+d^2}{2}}$. Then $\sqrt{abcd} \leq XY$.

Apply Part (i) to $X, Y$: $XY \leq \sqrt{\frac{X^2 + Y^2}{2}} = \sqrt{\frac{\frac{a^2+b^2+c^2+d^2}{2}}{2}} = \sqrt{\frac{a^2+b^2+c^2+d^2}{4}}$

Therefore: $\sqrt[4]{abcd} \leq \sqrt{\frac{a^2+b^2+c^2+d^2}{4}}$
\end{solution}

\begin{takeaways}

\item \textbf{Technique:} The inequality $(x-y)^2 \geq 0$ is a fundamental tool for proving that the quadratic mean (RMS) dominates the geometric mean.
\item \textbf{Strategy:} Build up to higher-order inequalities by pairing terms and applying proven results recursively; here we go from 2 terms to 4 terms.
\item \textbf{Key Insight:} The relationship $\sqrt{xy} \leq \sqrt{\frac{x^2+y^2}{2}}$ (GM $\leq$ QM) serves as a bridge to extend AM-GM style inequalities to power means.
\item \textbf{Common Pattern:} When dealing with fourth roots of products, rewrite as square roots of square roots, then apply two-variable inequalities twice.

\end{takeaways}

\begin{problem}[Calculus and Induction for Harmonic Inequality]
\begin{enumerate}
    \item[(i)] Use calculus to show that $x > \ln(1+x)$ for all $x > 0$.
    \item[(ii)] Use the inequality in part (i) and the principle of mathematical induction to prove that
    \[ \frac{1}{1} + \frac{1}{2} + \frac{1}{3} + \dots + \frac{1}{n} > \ln(1+n) \]
    for all positive integers, $n$.
\end{enumerate}
\end{problem}

\begin{solution}
\textbf{Part (i):} Let $f(x) = x - \ln(1+x)$. Then $f'(x) = 1 - \frac{1}{1+x} = \frac{x}{1+x} > 0$ for $x > 0$.

Since $f(0) = 0$ and $f$ is strictly increasing for $x > 0$, we have $f(x) > 0$ for all $x > 0$.

Therefore, $x > \ln(1+x)$ for all $x > 0$.

\textbf{Part (ii):} \textbf{Base ($n=1$):} LHS $= 1$, RHS $= \ln(2) \approx 0.693$. Since $1 > \ln(2)$, true.

\textbf{Hypothesis:} Assume $\sum_{r=1}^{k} \frac{1}{r} > \ln(1+k)$ (*)

\textbf{Step:} Using (*): LHS $> \ln(k+1) + \frac{1}{k+1}$.

From Part (i) with $x = \frac{1}{k+1}$:
\[
\frac{1}{k+1} > \ln\left(1 + \frac{1}{k+1}\right) = \ln\left(\frac{k+2}{k+1}\right) = \ln(k+2) - \ln(k+1)
\]
Thus: LHS $> \ln(k+1) + [\ln(k+2) - \ln(k+1)] = \ln(k+2)$

By induction, the inequality holds for all positive integers $n$.
\end{solution}

\begin{takeaways}

\item \textbf{Technique:} Use calculus to establish a continuous inequality, then leverage it as a lemma in an inductive proof for a discrete sum.
\item \textbf{Strategy:} The key connection is recognizing that $\frac{1}{k+1} > \ln(k+2) - \ln(k+1)$ allows us to bridge from $\ln(k+1)$ to $\ln(k+2)$ in the inductive step.
\item \textbf{Key Insight:} The logarithm property $\ln(a) - \ln(b) = \ln(a/b)$ is crucial for converting the Part (i) inequality into a form usable in the induction.
\item \textbf{Common Pattern:} When proving inequalities involving harmonic sums and logarithms, calculus-based lemmas about $\ln(1+x)$ frequently serve as bridges between consecutive cases.
\item \textbf{Pitfall:} Don't forget to verify that the substitution $x = \frac{1}{k+1}$ satisfies the domain condition $x > 0$ required for Part (i) to apply.

\end{takeaways}
