% ==============================================================================
% PART 2: ADVANCED PROBLEMS (34-45)
% Hard Problems - HSC Extension 2 Level
% ==============================================================================

\begin{problem}[Bernoulli's Inequality - Weighted AM-GM]
Let $n$ be a positive integer and let $x$ be a positive real number.
\begin{enumerate}
    \item[(i)] Show that $x^n - 1 - n(x-1) = (x-1)(1+x+x^2+\dots+x^{n-1}-n)$.
    \item[(ii)] Hence show that $x^n \ge 1 + n(x-1)$.
    \item[(iii)] Deduce that for positive real numbers $a$ and $b$,
    \[ a^n b^{1-n} \ge na + (1-n)b. \]
\end{enumerate}
\end{problem}

\begin{hint}
Use the standard factorization for difference of powers.
\end{hint}

\begin{solution}
\textbf{Part (i):} Use standard factorization: $x^n - 1 = (x-1)(1 + x + \dots + x^{n-1})$. Then
\begin{align*}
x^n - 1 - n(x-1) &= (x-1)(1 + x + \dots + x^{n-1}) - n(x-1)\\
&= (x-1)(1 + x + \dots + x^{n-1} - n)
\end{align*}

\textbf{Part (ii):} Analyze sign of $(x-1)(1+x+\dots+x^{n-1}-n)$:
\begin{itemize}
\item If $x=1$: expression equals 0
\item If $x>1$: both factors positive $\implies$ product $> 0$
\item If $0<x<1$: both factors negative $\implies$ product $> 0$
\end{itemize}
Thus $x^n - 1 - n(x-1) \ge 0 \implies x^n \ge 1 + n(x-1)$.

\textbf{Part (iii):} Substitute $x = \frac{a}{b}$ into (ii):
\begin{align*}
\left(\frac{a}{b}\right)^n &\ge 1 + n\left(\frac{a}{b} - 1\right)\\
\frac{a^n}{b^n} &\ge 1 + \frac{na - nb}{b}
\end{align*}
Multiply by $b$: $\frac{a^n}{b^{n-1}} \ge na + b(1-n)$, giving $a^n b^{1-n} \ge na + (1-n)b$.
\end{solution}

\begin{takeaways}
\item Bernoulli's inequality extends to weighted AM-GM forms
\item Sign analysis crucial when $x$ varies around 1
\item The inequality can be generalized when $n$ is not an integer 
\end{takeaways}

% ------------------------------------------------------------------------------

\begin{problem}[Convexity and Product Constraints (Jensen)]

\begin{quote}
\textbf{Jensen's Inequality:}
Let $f(x)$ be a convex function on an interval $I$. For any $x_1, x_2, \dots, x_n \in I$:

\[ \frac{f(x_1) + f(x_2) + \dots + f(x_n)}{n} \geq f\left( \frac{x_1 + x_2 + \dots + x_n}{n} \right) \]

\textit{Note: You may use this inequality in the following problem without proof.}

\end{quote}

Let $a_1,\dots,a_n>0$ satisfy $\prod_{i=1}^n a_i = a_1 a_2 \cdots a_n = 1$ and set $x_i=\ln a_i$. 

Let $f(x)=\ln(1+e^x)$.

\begin{enumerate}
    \item[(i)] Show $f''(x)=\tfrac{e^x}{(1+e^x)^2}>0$, so $f$ is convex on $\mathbb{R}$.
    \item[(ii)] Using Jensen's inequality deduce
    \[ \prod_{i=1}^n(1+a_i)\ge 2^n. \]
    \item[(iii)] Let $g(k)=\sum_{i=1}^n f(kx_i)$ for $k\ge0$. Show $g'(0)=0$ and $g''(k)\ge0$.
    \item[(iv)] Conclude the cyclic inequality
    \[ \prod_{i=1}^n(1+a_i)\ge\prod_{i=1}^n\big(1+a_i^{1/n}\big). \]
\end{enumerate}
\end{problem}

\begin{hint}
For (ii) use $\sum x_i=0$ (since $\prod a_i=1$) so Jensen at the average gives $\sum f(x_i)\ge n f(0)=n\ln2$. For (iii) compute derivatives by the chain rule and use $f''>0$. For (iv) compare $g(1)$ and $g(1/n)$.
\end{hint}

\begin{solution}
(i) Differentiate: $f'(x)=\tfrac{e^x}{1+e^x}$ and $f''(x)=\tfrac{e^x}{(1+e^x)^2}>0$. 

(ii) Since $\sum x_i=0$, Jensen gives $\frac{1}{n}\sum f(x_i)\ge f(0)=\ln2$, so $\sum\ln(1+a_i)\ge n\ln2$ and exponentiating gives $\prod(1+a_i)\ge2^n$.

(iii) $g'(k)=\sum x_i f'(kx_i)$ so $g'(0)=f'(0)\sum x_i=\tfrac12\cdot0=0$. Also $g''(k)=\sum x_i^2 f''(kx_i)\ge0$ since each term is nonnegative.

(iv) Convexity of $g$ on $[0,\infty)$ (from $g''\ge0$) implies $g(1)\ge g(1/n)$, i.e. $\sum\ln(1+a_i)\ge\sum\ln(1+a_i^{1/n})$. Exponentiate to obtain the cyclic inequality.
\end{solution}

\begin{takeaways}
\item Log-space transformations turn products into sums that are amenable to Jensen and calculus.
% \item Convexity plus scaling gives monotonicity in the parameter; comparing parameters yields useful cyclic bounds.
\item Convexity is Key: In Extension 2, proving convexity ($f'' > 0$) is often the "hidden" first step to solving complex inequalities.
\item The Power of $k$: By introducing a scaling factor $k$, we can show that an inequality isn't just a "one-off" truth, but part of a continuous growth pattern.
\item Logarithmic Limits: This problem proves that the closer the $a_i$ values are to each other (and thus to 1), the smaller the product becomes, reaching its minimum at $2^n$.
\end{takeaways}


\begin{problem}[Cauchy-Schwarz and Sums]
Let $a, b, A$ and $B$ be positive numbers.
\begin{enumerate}[label=\textbf{(\roman*)}]
    \item Prove that
    \[ \frac{ab}{AB} \leq \frac{1}{2} \left( \frac{a^2}{A^2} + \frac{b^2}{B^2} \right) \]
    \item Let $A = \sqrt{\sum_{k=1}^n a_k^2}$ and $B = \sqrt{\sum_{k=1}^n b_k^2}$, where $a_k$ and $b_k$ are positive real numbers.
    Use (i) to prove that
    \[ \left( \sum_{k=1}^n a_k b_k \right)^2 \leq \left( \sum_{k=1}^n a_k^2 \right) \left( \sum_{k=1}^n b_k^2 \right) \]
    \item Let $S = x_1 + x_2 + x_3 + \dots + x_n$, where $x_k > 0$ for all $1 \leq k \leq n$. Use (ii) to prove that
    \[ \frac{S}{S-x_1} + \frac{S}{S-x_2} + \cdots + \frac{S}{S-x_n} \geq \frac{n^2}{n-1} \]
\end{enumerate}
\end{problem}

\begin{hint}
For (i), use $(x-y)^2 \geq 0$ with $x = \frac{a}{A}$, $y = \frac{b}{B}$. For (ii), sum the result from (i). For (iii), apply Cauchy-Schwarz with suitable choices.
\end{hint}

\begin{solution}
	\textbf{(i)} $(x-y)^2 \geq 0 \implies x^2 + y^2 \geq 2xy$. Let $x = \frac{a}{A}$, $y = \frac{b}{B}$:
\[ \frac{a^2}{A^2} + \frac{b^2}{B^2} \geq 2\frac{ab}{AB} \implies \frac{ab}{AB} \leq \frac{1}{2}\left(\frac{a^2}{A^2} + \frac{b^2}{B^2}\right) \]

	\textbf{(ii)} Apply (i) for each $k$:
\[ \frac{a_k b_k}{AB} \leq \frac{1}{2}\left(\frac{a_k^2}{A^2} + \frac{b_k^2}{B^2}\right) \]
Sum over $k$:
\[ \frac{1}{AB} \sum_{k=1}^n a_k b_k \leq 1 \implies \sum_{k=1}^n a_k b_k \leq AB \]
Square both sides:
\[ \left(\sum_{k=1}^n a_k b_k\right)^2 \leq \left(\sum_{k=1}^n a_k^2\right)\left(\sum_{k=1}^n b_k^2\right) \]

	\textbf{(iii)} Let $a_k = \sqrt{S-x_k}$, $b_k = \frac{1}{\sqrt{S-x_k}}$.
\[ \left(\sum_{k=1}^n 1\right)^2 = n^2 \leq (S(n-1)) \sum_{k=1}^n \frac{1}{S-x_k} \]
So 

$$\sum_{k=1}^n \frac{S}{S-x_k} \geq \frac{n^2}{n-1}$$.

\end{solution}

\begin{takeaways}
\item Cauchy-Schwarz can be derived from simple quadratic inequalities
\item Summing pairwise inequalities yields the general form
\item Clever substitutions can turn Cauchy-Schwarz into other inequalities
\item The similar-looking, Nesbitt’s Inequality, states that for positive numbers $a, b, c$, the following holds: 

$$\frac{a}{b+c} + \frac{b}{a+c} + \frac{c}{a+b} \geq \frac{3}{2}$$

which can be proved using AM-GM or Cauchy-Schwarz.
\end{takeaways}

% ------------------------------------------------------------------------------

\begin{problem}[Power Mean, Young's, and AM-GM]

\section*{Problem 3.26: Power Mean, Young's, and AM-GM}

Let $p, q,$ and $s$ be fixed positive real numbers with $p > 1$ and $q > 1$. \\
These constants satisfy the relationship $\frac{1}{p} + \frac{1}{q} = 1$.

\begin{enumerate}
    \item[(i)] Consider the function defined for $t > 0$:
    \[ f(t) = \frac{s^p}{p} + \frac{t^q}{q} - st \]
    Find the positive value of $t$ (in terms of $s$) that minimises $f(t)$.

    \item[(ii)] Hence, show that for all $t > 0$:
    \[ \frac{s^p}{p} + \frac{t^q}{q} \ge st \]
    \textit{(This result is known as Young's Inequality).}

    \item[(iii)] Prove by mathematical induction that:
    \[ (x_1 x_2 \cdots x_n)^{\frac{1}{n}} \le \frac{x_1 + x_2 + \cdots + x_n}{n} \]
    for all positive real numbers $x_1, x_2, \dots, x_n$.
    \textit{(This result is known as the Arithmetic Mean - Geometric Mean Inequality).}

    \item[(iv)] Using the result from part (iii), deduce that for all positive real numbers $y_1, y_2, \dots, y_n$:
    \[ \frac{y_1}{y_2} + \frac{y_2}{y_3} + \cdots + \frac{y_{n-1}}{y_n} + \frac{y_n}{y_1} \ge n \]
\end{enumerate}

% The numbers $p, q$ and $s$ are fixed and positive constants. 
% Also $p > 1, q > 1$ and  $1/p + 1/q = 1$.
% % $p = \frac{q}{q-1}$.
% \begin{enumerate}[label=(\roman*)]
%     \item What positive value of $t$ minimises the expression
%     \[ f(t) = \frac{s^p}{p} + \frac{t^q}{q} - st \ ? \]
%     \item Show that for all $t > 0$,
%     \[ \frac{s^p}{p} + \frac{t^q}{q} \geq st . \]
%     \item Prove by induction that
%     \[ (x_1 x_2 \cdots x_n)^{\frac{1}{n}} \leq \frac{x_1 + x_2 + \cdots + x_n}{n} \]
%     for all $x_1, \dots, x_n > 0$.
%     \item Deduce that, for all $y_1, y_2, \dots, y_n > 0$,
%     \[ \frac{y_1}{y_2} + \frac{y_2}{y_3} + \cdots + \frac{y_{n-1}}{y_n} + \frac{y_n}{y_1} \geq n . \]
% \end{enumerate}

\end{problem}

\begin{hint}
For (i), find the stationary point of $f(t)$. For (ii), evaluate $f$ at the minimum. For (iii), use induction on $n$ for AM-GM. For (iv), apply AM-GM to the cyclic ratios.
\end{hint}

\begin{solution}
	\textbf{(i)} $f'(t) = t^{q-1} - s = 0 \implies t = s^{1/(q-1)} = s^{p-1}$. $f''(t) > 0$ for $q > 1$, so this is a minimum.

	\textbf{(ii)} At $t = s^{p-1}$:
    \[ f(s^{p-1}) = \frac{s^p}{p} + \frac{s^p}{q} - s^p = s^p\left(\frac{1}{p} + \frac{1}{q} - 1\right) = 0 \]
    So $f(t) \geq 0$ for all $t > 0$.

	\textbf{(iii)} Induction for AM-GM: Base case $n=1$ is trivial. Assume for $k$, prove for $k+1$ by grouping and using the hypothesis.

	\textbf{(iv)} Apply AM-GM to $\frac{y_1}{y_2}, \dots, \frac{y_n}{y_1}$:
    \[ \sum_{cyc} \frac{y_i}{y_{i+1}} \geq n \left(1\right)^{1/n} = n \]
\end{solution}

\begin{takeaways}
\item Young's inequality generalizes AM-GM
\item Induction is a powerful tool for inequalities. And convexity applied for Inequalities. 
\item Cyclic sums often reduce to AM-GM
\item Pay close attention to how the product of cyclic terms ($\frac{y_1}{y_2} \dots \frac{y_n}{y_1}$) cancels to 1; this is a common technique in harder inequality proofs.
\item \textbf{{General Young's Inequality}:} 

Let $a_1, a_2, \dots, a_n$ be non-negative real numbers. Let $p_1, p_2, \dots, p_n$ be positive real numbers satisfying the condition:
\[
\sum_{i=1}^{n} \frac{1}{p_i} = 1
\]
Then the product of these numbers is bounded by the sum of their powers:
\[
\prod_{i=1}^{n} a_i \le \sum_{i=1}^{n} \frac{a_i^{p_i}}{p_i}
\]
Or written explicitly:
\[
a_1 a_2 \cdots a_n \le \frac{a_1^{p_1}}{p_1} + \frac{a_2^{p_2}}{p_2} + \cdots + \frac{a_n^{p_n}}{p_n}
\]
\textit{Equality holds if and only if } $a_1^{p_1} = a_2^{p_2} = \cdots = a_n^{p_n}$.

Link this inequality to weighted AM-GM inequalities.
\end{takeaways}

% ------------------------------------------------------------------------------

\begin{problem}[Inductive Proof of AM-GM]
The real numbers $a_1, a_2, \dots$ are all positive. For each positive $n$, $A_n$ and $G_n$ are defined by:
\[ A_n = \frac{a_1 + a_2 + \dots + a_n}{n} \quad \text{and} \quad G_n = (a_1 a_2 \dots a_n)^{1/n} \]
\begin{enumerate}[label=(\roman*)]
    \item Show that, for any positive integer $k$,
    \[ \text{if } (\lambda_k)^{k+1} - (k+1)\lambda_k + k \geq 0, \text{ where } \lambda_k = \left( \frac{a_{k+1}}{G_k} \right)^{1/(k+1)} \]
    \[ \text{then } (k+1)(A_{k+1} - G_{k+1}) \geq k(A_k - G_k) \]
    \item Let $f(x) = x^{k+1} - (k+1)x + k$, $x > 0$, $k \in \mathbb{Z}^+$. Show $f(x) \geq 0$.
    \item Hence prove by induction $A_n \geq G_n$ for all $n \in \mathbb{Z}^+$.
\end{enumerate}
\end{problem}

\begin{hint}
For (i), manipulate the given inequality and substitute $\lambda_k$. For (ii), analyze $f(x)$ using calculus. For (iii), use induction and results from (i) and (ii).
\end{hint}

\begin{solution}
	\textbf{(i)} Given $\lambda_k^{k+1} + k \geq (k+1)\lambda_k$. Multiply by $G_k > 0$ and substitute $\lambda_k$ to relate $A_{k+1}$ and $G_{k+1}$. (See sample for full algebraic steps.)

	\textbf{(ii)} $f'(x) = (k+1)x^k - (k+1)$. Minimum at $x=1$, $f(1)=0$. So $f(x) \geq 0$ for $x > 0$.

	\textbf{(iii)} Induction: Base case $n=1$ is trivial. Assume for $k$, use (i) and (ii) to show $A_{k+1} \geq G_{k+1}$.
\end{solution}

\begin{takeaways}
\item Inductive proofs can be structured using auxiliary inequalities
\item Calculus can establish non-negativity for all $x > 0$
\item AM-GM is a fundamental result for all $n$
\end{takeaways}

% ------------------------------------------------------------------------------
% ------------------------------------------------------------------------------

\begin{problem}[Reciprocal Polynomial with AM-GM]
Let $a, b, c$ be real numbers. Suppose that $P(x) = x^4 + ax^3 + bx^2 + cx + 1$ has roots $\alpha, \frac{1}{\alpha}, \beta, \frac{1}{\beta}$, where $\alpha > 0$ and $\beta > 0$.
\begin{enumerate}
    \item[(i)] Prove that $a = c$.
    \item[(ii)] Using the inequality, show that $b \ge 6$.
\end{enumerate}
\end{problem}

\begin{hint}
$P(x)$ is a reciprocal polynomial.
\end{hint}

\begin{solution}
\textbf{Part (i):} Consider $Q(x) = x^4 P(1/x) = x^4 + cx^3 + bx^2 + ax + 1$. The roots of $P(1/x)$ are reciprocals of roots of $P(x)$, which are $\frac{1}{\alpha}, \alpha, \frac{1}{\beta}, \beta$ - the same set. Since $P$ and $Q$ are monic with same roots, $P \equiv Q$. Comparing coefficients: $a = c$.

\textbf{Part (ii):} By Vieta's formulas, $b$ equals sum of products of roots taken two at a time:
\begin{align*}
b &= \alpha \cdot \frac{1}{\alpha} + \alpha\beta + \frac{\alpha}{\beta} + \frac{\beta}{\alpha} + \frac{1}{\alpha\beta} + \beta \cdot \frac{1}{\beta}\\
&= 2 + \left(\alpha\beta + \frac{1}{\alpha\beta}\right) + \left(\frac{\alpha}{\beta} + \frac{\beta}{\alpha}\right)
\end{align*}
Apply AM-GM: $\alpha\beta + \frac{1}{\alpha\beta} \ge 2$ and $\frac{\alpha}{\beta} + \frac{\beta}{\alpha} \ge 2$. Thus $b \ge 2 + 2 + 2 = 6$.
\end{solution}

\begin{takeaways}
\item Reciprocal polynomials satisfy $P(x) = x^n P(1/x)$
\item AM-GM applies to sum of reciprocals: $x + \frac{1}{x} \ge 2$
\end{takeaways}

% ------------------------------------------------------------------------------

\begin{problem}[Nested AM-GM Application]
Let $x, y, z, w$ be positive real numbers.
\begin{enumerate}
    \item[(i)] Given that $x>0$ and $y>0$, show that $x+y \ge 2\sqrt{xy}$.
    \item[(ii)] Hence show that for $x>0$, $y>0$, $z>0$ and $w>0$,
    \[ x+y+z+w \ge 4\sqrt[4]{xyzw}. \]
    \item[(iii)] Consider $x, y, z$ and $w = \frac{x+y+z}{3}$. Apply the result in (ii) to show that
    \[ \frac{x+y+z}{3} \ge \sqrt[3]{xyz}. \]
\end{enumerate}
\end{problem}

\begin{hint}
First apply to pairs. Do not apply AM-GM directly.
\end{hint}

\begin{solution}
\textbf{Part (i):} Standard AM-GM: $(\sqrt{x} - \sqrt{y})^2 \ge 0 \implies x - 2\sqrt{xy} + y \ge 0 \implies x+y \ge 2\sqrt{xy}$.

\textbf{Part (ii):} Apply (i) twice:
\begin{align*}
x+y &\ge 2\sqrt{xy}, \quad z+w \ge 2\sqrt{zw}\\
(x+y) + (z+w) &\ge 2\sqrt{xy} + 2\sqrt{zw}\\
x+y+z+w &\ge 2(\sqrt{xy} + \sqrt{zw}) \ge 2 \cdot 2\sqrt{\sqrt{xy} \cdot \sqrt{zw}} = 4\sqrt[4]{xyzw}
\end{align*}

\textbf{Part (iii):} Let $w = \frac{x+y+z}{3}$. Then:
\begin{align*}
x+y+z+w &\ge 4\sqrt[4]{xyzw}\\
\frac{4}{3}(x+y+z) &\ge 4\sqrt[4]{xyz \cdot \frac{x+y+z}{3}}\\
\frac{x+y+z}{3} &\ge \sqrt[4]{xyz \cdot \frac{x+y+z}{3}}
\end{align*}
Raise to 4th power: $\left(\frac{x+y+z}{3}\right)^4 \ge xyz \cdot \frac{x+y+z}{3}$. Divide by $\frac{x+y+z}{3}$: $\left(\frac{x+y+z}{3}\right)^3 \ge xyz$.
\end{solution}

\begin{takeaways}
\item AM-GM for $n=3$ derived from $n=2$ and $n=4$ cases
\item Nested application: pair terms, then apply again
\item Clever choice of variables can be used to solve the general case
\end{takeaways}

% ------------------------------------------------------------------------------

\begin{problem}[Triangle Inequality - Quadratic Forms]
Let $p, q, r$ be the lengths of the three sides of a triangle.
\begin{enumerate}
    \item[(a)] Show that: $p^2 + q^2 + r^2 \ge pq + pr + qr$
    \item[(b)] Show that: $3(pq + pr + qr) \le (p + q + r)^2 < 4(pq + pr + qr)$
\end{enumerate}
\end{problem}

\begin{hint}
Consider sum of squares of differences.
\end{hint}

\begin{solution}
\textbf{Part (a):} Consider $(p-q)^2 + (q-r)^2 + (p-r)^2 \ge 0$:
\begin{align*}
2p^2 + 2q^2 + 2r^2 - 2pq - 2qr - 2pr &\ge 0\\
p^2 + q^2 + r^2 &\ge pq + pr + qr
\end{align*}

\textbf{Part (b):} Expand $(p+q+r)^2 = p^2 + q^2 + r^2 + 2(pq+pr+qr)$. Using (a):
\begin{align*}
(p+q+r)^2 &\ge (pq+pr+qr) + 2(pq+pr+qr) = 3(pq+pr+qr)
\end{align*}
For the upper bound, use triangle inequalities $p < q+r$, $q < p+r$, $r < p+q$:
\begin{align*}
p^2 < pq + pr, \quad q^2 < pq + qr, \quad r^2 < pr + qr\\
p^2 + q^2 + r^2 &< 2(pq + pr + qr)
\end{align*}
Add $2(pq+pr+qr)$ to both sides: $(p+q+r)^2 < 4(pq+pr+qr)$.
\end{solution}

\begin{takeaways}
\item Sum of squares of differences always non-negative
\item Triangle inequality crucial for strict upper bound
\end{takeaways}

% ------------------------------------------------------------------------------

\begin{problem}[Complex Triangle Inequality]
Prove that for any two complex numbers $z_1, z_2 \in \mathbb{C}$:
\[ |z_1 + z_2| \le |z_1| + |z_2| \]
\end{problem}

\begin{hint}
Use the property that $|w| \ge \text{Re}(w)$.
\end{hint}

\begin{solution}
Square the modulus: $|z_1 + z_2|^2 = (z_1 + z_2)(\overline{z_1 + z_2})$:
\begin{align*}
|z_1 + z_2|^2 &= z_1\bar{z}_1 + z_1\bar{z}_2 + z_2\bar{z}_1 + z_2\bar{z}_2\\
&= |z_1|^2 + (z_1\bar{z}_2 + \overline{z_1\bar{z}_2}) + |z_2|^2\\
&= |z_1|^2 + 2\text{Re}(z_1\bar{z}_2) + |z_2|^2
\end{align*}
Use $\text{Re}(w) \le |w|$:
\begin{align*}
|z_1 + z_2|^2 &\le |z_1|^2 + 2|z_1\bar{z}_2| + |z_2|^2\\
&= |z_1|^2 + 2|z_1||z_2| + |z_2|^2 = (|z_1| + |z_2|)^2
\end{align*}
Taking square roots: $|z_1 + z_2| \le |z_1| + |z_2|$.
\end{solution}

\begin{takeaways}
\item Key identity: $z\bar{z} = |z|^2$ and $w + \bar{w} = 2\text{Re}(w)$
\item $\text{Re}(w) \le |w|$ fundamental for complex inequalities
\item This is the triangle inequality in the complex plane, analogous to the geometric triangle inequality.
\item The Inequality can be proven using other methods, such as geometric interpretations, where the modulus represents the distance from the origin in the complex plane.
\end{takeaways}

% ------------------------------------------------------------------------------

\begin{problem}[Complex Modulus with Constraint]
Given $|z| < \frac{1}{2}$, show that:
\[ |(1+i)z^3 + iz| < \frac{3}{4} \]
\end{problem}

\begin{hint}
Apply the complex triangle inequality.
\end{hint}

\begin{solution}
Apply triangle inequality:
\begin{align*}
|(1+i)z^3 + iz| &\le |(1+i)z^3| + |iz|\\
&= |1+i||z|^3 + |i||z|\\
&= \sqrt{2}|z|^3 + |z|
\end{align*}
Since $|z| < \frac{1}{2}$ and $f(x) = \sqrt{2}x^3 + x$ is increasing for $x > 0$:
\begin{align*}
|(1+i)z^3 + iz| &< \sqrt{2}\left(\frac{1}{2}\right)^3 + \frac{1}{2}\\
&= \frac{\sqrt{2}}{8} + \frac{4}{8} = \frac{4 + \sqrt{2}}{8}
\end{align*}
Since $\sqrt{2} < 2$: $\frac{4 + \sqrt{2}}{8} < \frac{6}{8} = \frac{3}{4}$.
\end{solution}

\begin{takeaways}
\item Triangle inequality: $|w_1 + w_2| \le |w_1| + |w_2|$
\item Evaluate at boundary of constraint for tight bounds
\end{takeaways}

% ------------------------------------------------------------------------------

\begin{problem}[Problem 40: Harmonic-Arithmetic Mean Inequality]
Let $a, b, c$ be positive real numbers such that $\frac{1}{a} + \frac{1}{b} + \frac{1}{c} = 1$.
Using the fact that $x+y \ge 2\sqrt{xy}$ for positive $x, y$, prove that:
\[ a\sqrt{bc} + b\sqrt{ac} + c\sqrt{ab} \le abc \]
\end{problem}

\begin{hint}
Use the given condition to show that $ab + bc + ac = abc$.
\end{hint}

\begin{solution}
Multiply given condition by $abc$: $bc + ac + ab = abc$.

Apply AM-GM to pairs:
\begin{align*}
ab + ac &\ge 2\sqrt{a^2bc} = 2a\sqrt{bc}\\
ab + bc &\ge 2\sqrt{ab^2c} = 2b\sqrt{ac}\\
ac + bc &\ge 2\sqrt{abc^2} = 2c\sqrt{ab}
\end{align*}
Sum the three inequalities:
\begin{align*}
2(ab + bc + ac) &\ge 2(a\sqrt{bc} + b\sqrt{ac} + c\sqrt{ab})\\
ab + bc + ac &\ge a\sqrt{bc} + b\sqrt{ac} + c\sqrt{ab}
\end{align*}
Substitute $ab + bc + ac = abc$: $abc \ge a\sqrt{bc} + b\sqrt{ac} + c\sqrt{ab}$.
\end{solution}

\begin{takeaways}
\item Convert harmonic condition to algebraic form first
\item Apply AM-GM systematically to all pairs
\end{takeaways}

% ------------------------------------------------------------------------------

\begin{problem}[Logarithmic Inequality with Factorial]
\begin{enumerate}
\item[(i)] Prove that $x > \ln(x)$ for all positive real numbers $x$.
\item[(ii)] Hence, show that for all positive integers $n$:
\[ e^{n^2+n} > (n!)^2 \]
\end{enumerate}
\end{problem}

\begin{hint}
Consider $f(x) = x - \ln(x)$.
\end{hint}

\begin{solution}
\textbf{Part (i):} Let $f(x) = x - \ln(x)$. Then $f'(x) = 1 - \frac{1}{x}$.
Setting $f'(x) = 0$ gives $x = 1$. Since $f''(x) = \frac{1}{x^2} > 0$, point $(1,1)$ is a minimum.
Thus $f(x) \ge f(1) = 1 - \ln(1) = 1 > 0$, so $x > \ln(x)$.

\textbf{Part (ii):} Apply (i) to $k = 1, 2, \ldots, n$ and sum:
\begin{align*}
\sum_{k=1}^{n} k &> \sum_{k=1}^{n} \ln(k)\\
\frac{n(n+1)}{2} &> \ln(n!)\\
\frac{n^2 + n}{2} &> \ln(n!)\\
n^2 + n &> 2\ln(n!) = \ln((n!)^2)
\end{align*}
Exponentiating: $e^{n^2+n} > (n!)^2$.
\end{solution}

\begin{takeaways}
\item Calculus proves $x > \ln(x)$ via minimization
\item Sum inequalities to relate arithmetic to logarithmic sums
\end{takeaways}

% ------------------------------------------------------------------------------

\begin{problem}[Cauchy-Schwarz with Homogenization]
Prove that for all positive real numbers $a, b, c$:
\[ \frac{a^2}{3a+2b} + \frac{b^2}{3b+2c} + \frac{c^2}{3c+2a} \ge \frac{a+b+c}{5} \]
\end{problem}

\begin{hint}
Apply Cauchy-Schwarz.
\end{hint}

\begin{solution}
Apply Cauchy-Schwarz: $\left(\sum \frac{u_i^2}{v_i}\right)\left(\sum v_i\right) \ge \left(\sum u_i\right)^2$.

Let $u_i = (a, b, c)$ and $v_i = (3a+2b, 3b+2c, 3c+2a)$:
\begin{align*}
\left(\frac{a^2}{3a+2b} + \frac{b^2}{3b+2c} + \frac{c^2}{3c+2a}\right) &\cdot [(3a+2b) + (3b+2c) + (3c+2a)]\\
&\ge (a+b+c)^2
\end{align*}
Simplify denominator sum:
\begin{align*}
(3a+2b) + (3b+2c) + (3c+2a) &= 3(a+b+c) + 2(a+b+c) = 5(a+b+c)
\end{align*}
Therefore:
\[ \text{LHS} \cdot 5(a+b+c) \ge (a+b+c)^2 \implies \text{LHS} \ge \frac{a+b+c}{5} \]
\end{solution}

\begin{takeaways}
\item Cauchy-Schwarz in Titu's Lemma form: $\sum \frac{x_i^2}{y_i} \ge \frac{(\sum x_i)^2}{\sum y_i}$
\item Check that denominators sum to simple multiple of numerator sum
\end{takeaways}

% ------------------------------------------------------------------------------

\begin{problem}[Bernoulli's Inequality - Power Form]
Prove the following inequality for all integers $n \ge 1$:
\[ \left(1 + \frac{1}{\sqrt{n}}\right)^n \ge 1 + \sqrt{n} \]
\end{problem}

\begin{hint}
Apply Bernoulli directly.
\end{hint}

\begin{solution}
Let $x = \frac{1}{\sqrt{n}}$. Since $n \ge 1$, we have $x > 0 > -1$.

Apply Bernoulli's inequality $(1+x)^n \ge 1+nx$:
\begin{align*}
\left(1 + \frac{1}{\sqrt{n}}\right)^n &\ge 1 + n \cdot \frac{1}{\sqrt{n}}\\
&= 1 + \frac{n}{\sqrt{n}}\\
&= 1 + \sqrt{n}
\end{align*}
\end{solution}

\begin{takeaways}
\item Bernoulli's inequality: $(1+x)^n \ge 1+nx$ for $x > -1$, $n \ge 1$
\item Choose substitution to match target form
\end{takeaways}

% ------------------------------------------------------------------------------

\begin{problem}[Strict Bernoulli via Induction]
\begin{enumerate}
    \item[(i)] Prove that $(1+x)^n > 1+nx$ for $n \ge 1$ and $x > -1$.
    \item[(ii)] Hence, deduce that $\left(1 - \frac{1}{2n}\right)^n > \frac{1}{2}$ for $n > 1$.
\end{enumerate}
\end{problem}

\begin{hint}
Prove by math induction.
\end{hint}

\begin{solution}
\textbf{Part (i):} By induction. Base case $n=1$: equality holds.

Inductive step: Assume $(1+x)^k \ge 1+kx$. Then:
\begin{align*}
(1+x)^{k+1} &= (1+x)(1+x)^k \ge (1+x)(1+kx)\\
&= 1 + kx + x + kx^2 = 1 + (k+1)x + kx^2
\end{align*}
Since $k \ge 1$ and $x^2 \ge 0$, we have $kx^2 \ge 0$, so $(1+x)^{k+1} \ge 1+(k+1)x$.
For $n > 1$ and $x \neq 0$, strict inequality holds since $kx^2 > 0$.

\textbf{Part (ii):} Let $x = -\frac{1}{2n}$. Check $x > -1$: $-\frac{1}{2n} > -1$ holds for $n > \frac{1}{2}$.
Apply (i):
\begin{align*}
\left(1 - \frac{1}{2n}\right)^n &> 1 + n\left(-\frac{1}{2n}\right) = 1 - \frac{1}{2} = \frac{1}{2}
\end{align*}
\end{solution}

\begin{takeaways}
\item Induction proves Bernoulli; strict inequality when $kx^2 > 0$
\item Negative substitutions require careful domain checking
\end{takeaways}

% ------------------------------------------------------------------------------

\begin{problem}[Summation Inequality via Induction]
Given that for $k > 0$, $2k + 3 > 2\sqrt{(k+1)(k+2)}$, prove that:
\[ \sum_{r=1}^{n} \frac{1}{\sqrt{r}} > 2\left(\sqrt{n+1} - 1\right) \]
for all positive integers $n$.
\end{problem}

\begin{hint}
Prove by induction. Use the given inequality.
\end{hint}

\begin{solution}
\textbf{Base case} ($n=1$): LHS $= 1$, RHS $= 2(\sqrt{2}-1) \approx 0.828$. True.

\textbf{Inductive step:} Assume $\sum_{r=1}^{k} \frac{1}{\sqrt{r}} > 2(\sqrt{k+1} - 1)$.
Then:
\begin{align*}
\sum_{r=1}^{k+1} \frac{1}{\sqrt{r}} &= \sum_{r=1}^{k} \frac{1}{\sqrt{r}} + \frac{1}{\sqrt{k+1}}\\
&> 2(\sqrt{k+1} - 1) + \frac{1}{\sqrt{k+1}}\\
&= 2\sqrt{k+1} + \frac{1}{\sqrt{k+1}} - 2\\
&= \frac{2(k+1) + 1}{\sqrt{k+1}} - 2 = \frac{2k+3}{\sqrt{k+1}} - 2
\end{align*}
Given $2k+3 > 2\sqrt{(k+1)(k+2)}$:
\begin{align*}
\frac{2k+3}{\sqrt{k+1}} - 2 &> \frac{2\sqrt{(k+1)(k+2)}}{\sqrt{k+1}} - 2\\
&= 2\sqrt{k+2} - 2 = 2(\sqrt{k+2} - 1)
\end{align*}
Thus the inequality holds for $n = k+1$.
\end{solution}

\begin{takeaways}
\item Use given auxiliary inequality in inductive step
\item Algebraic manipulation converts sum to target form
\end{takeaways}
